\begin{enunciado}{\ejExtra}
  Una aplicación para escuchar música estudiar un estraño comportamiento en las elecciones
  musicales de sus usuarios. Cada dos minutos cambian entre,
  \textit{Juan d'Arienzo},
  \textit{Carlos di Sarli},
  \textit{Aníbal Troilo}
  y
  \textit{Francisco Canaro} según el diagrama:
  $$
    \begin{tikzpicture}[
      scale=3,
      node distance=5cm,
      estado/.style={circle, draw, fill=violet!20!white, minimum size=2cm, inner sep=0pt},
      proba/.style={-{Latex[length=2mm]}},
      every node/.style={font={\normalsize}}
      ]
      \node[estado, align = center] (1) {Carlos \\ di Sarli};
      \node[estado, right of=1, align = center] (2) {Francisco \\ Canaro};
      \node[estado, below of=1, align = center] (3) {Aníbal \\ Troilo};
      \node[estado, below of=2, align = center] (4) {Juan \\ d'Arienzo};

      \draw[proba, bend right=20] (3) to node[right] {$0.5$} (1);
      \draw[proba, bend right=20] (1) to node[left] {$c$} (3);
      \draw[proba, bend right=20] (2) to node[above] {$0.5$} (1.east);
      \draw[proba, bend right=20] (1) to node[below] {$b$} (2.west);
      \draw[proba, bend right=5]  (1.south east) to node[left, below] {$d$}(4.north west);

      \draw[proba] (1) to [out=180, in=90, looseness=4]  node[left] {$a$} (1.north);
      \draw[proba] (2.east) to [out=0, in=90, looseness=4]  node[right] {$0.5$} (2.north);
      \draw[proba] (3.south) to [out=270, in=180, looseness=4]  node[left] {$0.5$} (3.west);
      \draw[proba] (4.south) to [out=270, in=0, looseness=4]  node[right] {$1$} (4.east);
    \end{tikzpicture}
  $$
  Por ejemplo, de quienes están escuchando a \textit{Pancho Canaro}, en dos minutos el 50\% pasará a
  escuchar a \textit{Carlos Di Sarli} y el otro 50\% seguirá escuchando a \textit{Francisco Canaro}.

  \bigskip

  Sabiendo que la matriz $P$ que representa el proceso de transición es de Markov y que
  $$v = \frac{1}{8}
    \matriz{c}{
      3\\
      2\\
      2\\
      1
    }
    \hspace{-5pt}
    \begin{array}{l}
      \to  \text{Carlos di Sarli}  \\
      \to  \text{Francisco Canaro} \\
      \to  \text{Aníbal Troilo}    \\
      \to  \text{Juan d'Arienzo}
    \end{array}
  $$
  es un estado de equilibrio.
  \begin{enumerate}[label=(\alph*)]
    \item (1 pt.) Dar la matriz de transición $P$ y determinar si existe $P^\infinito$.
    \item (1.5 pt.) Si en el instante inicial del estudio hay 300 usuarios escuchando a \textit{Carlos di Sarli},
          100 usuarios escuchando a \textit{Francisco Canaro} y 300 usuarios escuchando a \textit{Aníbal Troilo}.
          ¿Cuántos usuarios aproximandamente estarán escuchando a \textit{Aníbal Troilo} a los 20 minutos?
  \end{enumerate}
\end{enunciado}

\begin{enumerate}[label=(\alph*)]
  \item
        $$
          \begin{array}{rr}
            \text{Di Sarli}  & \to \\
            \text{Canaro}    & \to \\
            \text{Troilo}    & \to \\
            \text{d'Arienzo} & \to
          \end{array}
          \matriz{cccc}{
            a & 0.5 & 0.5 & 0 \\
            b & 0.5 & 0 & 0 \\
            c & 0 & 0.5 & 0 \\
            d & 0 & 0 & 1
          }
          = P
        $$
        Para que sea una matriz \textit{estocástica, de probabilidad o de Markov}, sus columnas tienen que sumar 1.
        Si $v$ es un estado de equilibrio:
        $$
          Pv = v
          \sii
          Pv =
          \frac{1}{8}
          \matriz{c}{
            3a + 2 \\
            3b + 1 \\
            3c + 1 \\
            3d + 1
          }
          =
          \frac{1}{8}
          \matriz{c}{
            3 \\
            2 \\
            2 \\
            1
          }
          \sisolosi
          \llave{rcl}{
            a & = & \frac{1}{3} \\
            b & = & \frac{1}{3} \\
            c & = & \frac{1}{3} \\
            d & = & 0 \\
          }
        $$
        Por lo tanto:
        $$
          P =
          \matriz{cccc}{
            \frac{1}{3} & 0.5 & 0.5 & 0 \\
            \frac{1}{3} & 0.5 & 0 & 0 \\
            \frac{1}{3} & 0 & 0.5 & 0 \\
            0 & 0 & 0 & 1
          }
        $$
        Para determinar la existencia de $P^\infinito$ necesito que los autovalores de la matriz de módulo 1 sean solo el 1:
        $$
          \text{Si } \paratodo \lambda \text{ tal que } |\lambda| = 1 \ytext \lambda = 1
          \entonces
          \existe P^\infinito
        $$
        Calculo autovalores:
        {\small
        $$
          |P - \lambda I_4| = 0
          \sii
          \deter{cccc}{
            \frac{1}{3} - \lambda & 0.5           & 0.5           & 0           \\
            \frac{1}{3}           & 0.5 - \lambda & 0             & 0           \\
            \frac{1}{3}           & 0             & 0.5 - \lambda & 0           \\
            0                     & 0             & 0             & 1 - \lambda
          }
          = 0
          \Sii{\red{!!}}
          \llave{rcccl}{
            \lambda_1 &\igual{\red{!}}& 1 & \to & E_{\lambda = 1} = \ket{(0,0,0,1), (\frac{3}{2},1,1,0)} \\
            \lambda_2 &=& \frac{1}{2} & \to & E_{\lambda = \frac{1}{2}} = \ket{(0,-1,1,0)} \\
            \lambda_3 &=& \frac{-1}{6} & \to & E_{\lambda = \frac{-1}{6}} = \ket{(-2,1,1,0)}
          }
        $$
        }
        Dado que los únicos autovalores de módulo uno son el 1, la matriz $P^\infinito$ existirá. En este caso
        el autovalor 1 es múltiple raíz del polinomio característico y su espacio asociado junto los otros autovectores
        generan una base de $\reales^4$. Esto me asegura la convergencia de cualquier estado inicial a un estado
        límite $v^{(0)} \to v^{\infinito}$ donde $ v^{(\infinito)} \en E_{\lambda = 1}$.

        Como la matriz es diagonalizable puedo encontrar $P^\infinito$.
        El límite:
        $$
          \limite{k}{\infinito} P^k =
          \limite{k}{\infinito} C_{\magenta{j}} D^k C_{\magenta{j}}^{-1} = P^\infinito
        $$
        existirá para cualquier matriz $C_{\magenta{j}}$, cuyas columnas son una base de los autovectores de $P$, que elija.
        El subíndice $\magenta{j}$ lo pongo para designar que habrá distintas $C$ según los autovectores que uno haya despejado
        en $E_{\lambda = 1}$.

        Como $P$ es diagonalizable,
        puedo encontrar una forma exacta de $P^\infinito$, \ul{si bien en el enunciado no piden el cálculo}.
        $$
          \begin{array}{rcl}
            \limite{k}{\infinito}
            P^k = C_{\magenta{j}} D^k C_{\magenta{j}}^{-1} & =            &
            \limite{k}{\infinito}
            \matriz{cccc}{
            0                                              & \frac{3}{2}  & 0               & -2               \\
            0                                              & 1            & -1              & 1                \\
            0                                              & 1            & 1               & 1                \\
            1                                              & 0            & 0               & 0
            }
            \matriz{cccc}{
            1                                              & 0            & 0               & 0                \\
            0                                              & 1            & 0               & 0                \\
            0                                              & 0            & (\frac{1}{2})^k & 0                \\
            0                                              & 0            & 0               & (\frac{-1}{6})^k
            }
            \matriz{cccc}{
            0                                              & \frac{3}{2}  & 0               & -2               \\
            0                                              & 1            & -1              & 1                \\
            0                                              & 1            & 1               & 1                \\
            1                                              & 0            & 0               & 0
            }^{-1}                                                                                             \\
                                                           & =            &
            \matriz{cccc}{
            0                                              & \frac{3}{2}  & 0               & -2               \\
            0                                              & 1            & -1              & 1                \\
            0                                              & 1            & 1               & 1                \\
            1                                              & 0            & 0               & 0
            }
            \matriz{cccc}{
            1                                              & 0            & 0               & 0                \\
            0                                              & 1            & 0               & 0                \\
            0                                              & 0            & 0               & 0                \\
            0                                              & 0            & 0               & 0
            }
            \matriz{cccc}{
            0                                              & 0            & 0               & 1                \\
            \frac{2}{7}                                    & \frac{2}{7}  & \frac{2}{7}     & 0                \\
            0                                              & \frac{-1}{2} & \frac{1}{2}     & 0                \\
            \frac{-2}{7}                                   & \frac{3}{14} & \frac{3}{14}    & 0
            }                                                                                                  \\
                                                           & =            &
            \matriz{cccc}{
            0                                              & \frac{3}{2}  & 0               & 0                \\
            0                                              & 1            & 0               & 0                \\
            0                                              & 1            & 0               & 0                \\
            1                                              & 0            & 0               & 0
            }
            \matriz{cccc}{
            0                                              & 0            & 0               & 1                \\
            \frac{2}{7}                                    & \frac{2}{7}  & \frac{2}{7}     & 0                \\
            0                                              & \frac{-1}{2} & \frac{1}{2}     & 0                \\
            \frac{-2}{7}                                   & \frac{3}{14} & \frac{3}{14}    & 0
            }                                                                                                  \\
                                                           & =            &
            \matriz{cccc}{
            \frac{3}{7}                                    & \frac{3}{7}  & \frac{3}{7}     & 0                \\
            \frac{2}{7}                                    & \frac{2}{7}  & \frac{2}{7}     & 0                \\
            \frac{2}{7}                                    & \frac{2}{7}  & \frac{2}{7}     & 0                \\
            0                                              & 0            & 0               & 1
            }
            = P^\infinito
          \end{array}
        $$
        \parrafoDestacado{
          A primera vista no parece obvio que al cambiar una $C_{\magenta{j}}$ por otra
          $C_{\magenta{i}}$ las cuentas den siempre el mismo resultado, pero esto debe ser así.
          El límite de $\limite{k}{\infinito}P^k$ existe y por definición de límite,
          no importa \textit{el camino usado} siempre se debe llegar al mismo resultado.
        }
  \item Hay que agarrar la calculadora y armarse de paciencia. Me voy a basar en las
        \textit{palabras del enunciado, sobre todo en la de aproximadamente}.

        Quieren que se calcule $v^{(10)} = P^{(10)}v^{(0)}$ para el estado inicial:
        $$
          v^{(0)}=
          \matriz{c}{
            300     \\
            100     \\
            300     \\
            0
          }
        $$
        Calculo:
        $$
          \begin{array}{rcccl}
            v^{(1)}  & =      & P v^{(0)} & =      & (300, 150, 250, 0)                                \\
            v^{(2)}  & =      & P v^{(1)} & =      & (300, 175, 225, 0)                                \\
            v^{(3)}  & =      & P v^{(2)} & =      & (300, 187.5, 212.5, 0)                            \\
            v^{(4)}  & =      & P v^{(3)} & =      & (300, 193.75, 206.25, 0)                          \\
            v^{(5)}  & =      & P v^{(4)} & =      & (300, 196.875, 203.125, 0)                        \\
            v^{(6)}  & =      & P v^{(5)} & =      & (300, 198.437, 201.563, 0)                        \\
            \vdots   & \vdots & \vdots    & \vdots & \vdots                                            \\
            v^{(10)} & =      & P v^{(9)} & =      & (300, 199.9, 200.1, 0) \approx (300, 200, 200, 0)
          \end{array}
        $$
        Así habrá aproximadamente:
        $$
          \cajaResultado{
            v^{(10)} \approx (300, 200, 200, 0)
          }
        $$
        O también está la versión con $P^{\red{10}}$:
        $$
          \begin{array}{rcl}
            P^{\red{10}} = C D^{\red{10}} C^{-1}
                         & =            &
            \matriz{cccc}{
            0            & \frac{3}{2}  & 0                        & -2                        \\
            0            & 1            & -1                       & 1                         \\
            0            & 1            & 1                        & 1                         \\
            1            & 0            & 0                        & 0
            }
            \matriz{cccc}{
            1            & 0            & 0                        & 0                         \\
            0            & 1            & 0                        & 0                         \\
            0            & 0            & (\frac{1}{2})^{\red{10}} & 0                         \\
            0            & 0            & 0                        & (\frac{-1}{6})^{\red{10}}
            }
            \matriz{cccc}{
            0            & 0            & 0                        & 1                         \\
            \frac{2}{7}  & \frac{2}{7}  & \frac{2}{7}              & 0                         \\
            0            & \frac{-1}{2} & \frac{1}{2}              & 0                         \\
            \frac{-2}{7} & \frac{3}{14} & \frac{3}{14}             & 0
            }                                                                                  \\
                         & \approx      &
            \matriz{cccc}{
            0            & \frac{3}{2}  & 0                        & 0                         \\
            0            & 1            & 0                        & 0                         \\
            0            & 1            & 0                        & 0                         \\
            1            & 0            & 0                        & 0
            }
            \matriz{cccc}{
            1            & 0            & 0                        & 0                         \\
            0            & 1            & 0                        & 0                         \\
            0            & 0            & 0.001                    & 0                         \\
            0            & 0            & 0                        & 0.00000002
            }
            \matriz{cccc}{
            0            & 0            & 0                        & 1                         \\
            \frac{2}{7}  & \frac{2}{7}  & \frac{2}{7}              & 0                         \\
            0            & \frac{-1}{2} & \frac{1}{2}              & 0                         \\
            \frac{-2}{7} & \frac{3}{14} & \frac{3}{14}             & 0
            }                                                                                  \\
                         & \approx      &
            P^\infinito
          \end{array}
        $$
        Por lo tanto, nuevamente, me voy a agarrar de las palabras del enunciado, y aproximadamente habrá:
        $$
          \cajaResultado{
          P^{(10)}v^{(0)}
          \approx
          P^\infinito v^{(0)} =
          \matriz{cccc}{
            \frac{3}{7}        & \frac{3}{7}  & \frac{3}{7}     & 0                \\
            \frac{2}{7}        & \frac{2}{7}  & \frac{2}{7}     & 0                \\
            \frac{2}{7}        & \frac{2}{7}  & \frac{2}{7}     & 0                \\
            0                  & 0            & 0               & 1
          }
          \matriz{c}{
            300     \\
            100     \\
            300     \\
            0
          }
          \approx
          \matriz{c}{
            300     \\
            200     \\
            200     \\
            0
          }
          }
        $$
        Por lo tanto de los 700 usuarios con un muy buen gusto musical, es decir no escuchan la \textit{basura} que se escucha hoy en día,
        estarían escuchando 300 a \textit{el señor del tango}, 200 a \textit{Pirincho} y otros 200 a \textit{Pichuco}.
\end{enumerate}

\begin{aportes}
  \item \aporte{\dirRepo}{naD GarRaz \github}
\end{aportes}
