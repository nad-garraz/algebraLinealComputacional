\begin{enunciado}{\ejExtra} \fechaEjercicio{final 24/02/25}
  Se dice que $A \en K^{n \times n}$ es semejante a $B \en K^{n \times n}$ si existe una matriz invertible $S \en K^{n \times n}$ tal que:
  $$
    SA(S)^{-1} = B
  $$
  \begin{enumerate}[label=\arabic*.]
    \item Demostrar que la relación de semejanza es una relación de equivalencia.
    \item Demostrar que si $A$ es semejante a $B$, entonces:
          $$
            \traza(A) = \traza(B)
          $$
          \textbf{Sugerencia:} Utilizar la propiedad $\traza(EC) = \traza(CE)$ para matrices $C$ y $E$

    \item Probar que si $A$ es diagonalizable (es decir, $A$ es semejante a una matriz diagonal $D$) y los valores propios de
          $A$ son 0 y 1, entonces:
          $$
            A^2 = A
          $$
  \end{enumerate}
\end{enunciado}

\begin{enumerate}[label=\arabic*.]
  \item Para demostrar que es una relación de equivalencia voy a demostrar que la relación es
        \textit{reflexiva}, \textit{simétrica} y \textit{transitiva}:

        \bigskip

        \textit{Reflexividad:} ¿Es $B$ semejante a $B$?
        $$
          B = I_n B I_n^{-1}
        $$
        Sí, la relación de semejanza es \textit{reflexiva}.

        \bigskip

        \textit{Simetría:} Si $B$ es semejante a $A$, ¿$A$ es semejante a $B$?
        $$
          B = S A S^{-1}
          \Sii{$S$ es}[invertible]
          S^{-1} B S = A
        $$
        Sí, la relación de semejanza es \textit{simétrica}.

        \bigskip

        \textit{Transitividad:} Si $B$ es semejante a $A$ y  $A$ es semejante a $C$ ¿$B$ es semejante a $C$?
        $$
          B = S \blue{A} S^{-1}
          \ytext
          \blue{A} = \blue{Q} \violet{C} \blue{Q^{-1}}
          \Sii{reemplazo}
          B = S \blue{Q} \violet{C} \blue{Q^{-1}} S^{-1}
          \sii
          B = (S \blue{Q}) \violet{C} (SQ)^{-1}
          \Sii{$P = SQ$}
          B = P \violet{C} P^{-1}
        $$
        Sí, la relación de semejanza es \textit{transitiva}.

        Así de muestra que la relación de semejanza es una relación de equivalencia.

  \item Usando la \textbf{sugerencia} y que $A = SBS^{-1}$:
        $$
          \traza(A) =
          \traza(SBS^{-1})
          =
          \traza((S) \cdot(BS^{-1}))
          \igual{\red{!}}[\textbf{sug.}]
          \traza((BS^{-1}) \cdot (S))
          =
          \traza(B)
        $$

  \item hola
        $$
          \begin{array}{rcl}
            A = CDC^{-1}
                        & \sii                                                                                                                                &
            A = C
            \matriz{ccc}{
            \lambda_1   & \cdots                                                                                                                              & 0           \\
            \vdots      & \ddots                                                                                                                              & \vdots      \\
            0           & \cdots                                                                                                                              & \lambda_n
            }
            C^{-1}                                                                                                                                                          \\\\
                        & \sii                                                                                                                                &
            A^2 = CDC^{-1} \cdot CDC^{-1}                                                                                                                                   \\
                        & \sii                                                                                                                                &
            A^2 = CD^2C^{-1}                                                                                                                                                \\
                        & \sii                                                                                                                                &
            A^2 = C
            \matriz{ccc}{
            \lambda_1^2 & \cdots                                                                                                                              & 0           \\
            \vdots      & \ddots                                                                                                                              & \vdots      \\
            0           & \cdots                                                                                                                              & \lambda_n^2
            }
            C^{-1}                                                                                                                                                          \\
                        & \Sii{$\scriptscriptstyle \lambda = 1 \Rightarrow \lambda_i^2 = 1$} [$\scriptscriptstyle \lambda_i = 0 \Rightarrow \lambda_i^2 = 0$] &
            \cajaResultado{
              A^2 =
              C
              \matriz{ccc}{
            \lambda_1   & \cdots                                                                                                                              & 0           \\
            \vdots      & \ddots                                                                                                                              & \vdots      \\
            0           & \cdots                                                                                                                              & \lambda_n
              }
              C^{-1}
              = A
            }
          \end{array}
        $$
\end{enumerate}

\begin{aportes}
  \item \aporte{\dirRepo}{naD GarRaz \github}
\end{aportes}
