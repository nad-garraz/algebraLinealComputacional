\begin{enunciado}{\ejExtra}
	Dadas las matrices $A, B \en \complejos^{n \times n}$ y un vector $v \en \complejos^n$, para cada una de las siguientes afirmaciones, determinar
	su validez. En caso de ser falsas, dar un contraejemplo, y en caso de ser verdaderas demostrarlas:
	\begin{enumerate}[label=(\alph*)]
		\item Si $v$ es un autovector de $A$, y $A$ es inversible, entonces $v$ es un autovector de $A^{-1}$.

		\item Si $A$ y $B$ son diagonalizables, $A+B$ también lo es.

		\item Si $A$ y $B$ son diagonalizables, entonces $AB$ es diagonalizable.

		\item Si $A$ o $B$ es inversible y $AB$ es diagonalizable entonces BA también es diagonalizables.
	\end{enumerate}
\end{enunciado}

\parrafoDestacado[\atencion]{
	Ejercicio de demostraciones. Dependiendo las horas que dormiste la noche anterior esto puede salir
	enseguida o en horas. La matriz que uso en los contraejemplos
	suele ser un \textit{caballito de batalla} para estos problemas, guardátela.
}

\begin{enumerate}[label=(\alph*)]
	\item Si $v$ es un autovector y además $\existe A^{-1}$ entonces:
	      $$
		      Av = \lambda v
		      \Sii{\red{!}}
		      A^{-1}Av = \lambda A^{-1}v
		      \Sii{\red{!}}
		      A^{-1}v = \frac{1}{\lambda} v
	      $$
	      Por lo tanto:
	      $$
		      \cajaResultado{
			      \text{resultó verdadera}
		      }
	      $$

	\item Si las matrices son diagonalizables, ¿La suma también lo es?:
	      $$
		      A =
		      \matriz{cc}{
			      1 & 1 \\
			      0 & 1
		      }
		      \ytext
		      B =
		      \matriz{cc}{
			      -1 & 1 \\
			      0 & 0
		      }
	      $$
	      Esas matrices son diagonalizables, porque cada una tiene todos sus autovalores distintos.
	      $$
		      A + B =
		      \matriz{cc}{
			      0 & 1 \\
			      0 & 0
		      }
	      $$
	      Matriz que no es diagonalizable, ya que tiene a $0$ como un autovalor doble, pero el autoespacio asociado es de dimensión 1:
	      $$
		      \chi(\lambda) = \lambda^2 = 0, \quad \text{luego } E_{\lambda = 0} = \set{(1,0)}
	      $$
	      Por lo tanto:
	      $$
		      \cajaResultado{
			      \text{resultó falsa}
		      }
	      $$

	\item Si las matrices son diagonalizables, ¿El producto también lo es?:
	      $$
		      A =
		      \matriz{cc}{
			      1 & -1 \\
			      0 & 0
		      }
		      \ytext
		      B =
		      \matriz{cc}{
			      0 & 0 \\
			      0 & -1
		      }
	      $$
	      Esas matrices son diagonalizables, porque cada una tiene todos sus autovalores distintos.
	      $$
		      A B =
		      \matriz{cc}{
			      0 & 1 \\
			      0 & 0
		      }
	      $$
	      Matriz que no es diagonalizable, ya que tiene a $0$ como un autovalor doble, pero el autoespacio asociado es de dimensión 1:
	      $$
		      \chi(\lambda) = \lambda^2 = 0, \quad \text{luego } E_{\lambda = 0} = \set{(1,0)}
	      $$
	      Por lo tanto:
	      $$
		      \cajaResultado{
			      \text{resultó falsa}
		      }
	      $$

	\item Alguna de las dos matrices es inversible y $AB$ es diagonalizable, entonces ¿$BA$ es diagonalizable también?

	      Supongo que $\existe A^{-1}$:
	      $$
		      AB = C D C^{-1}
		      \Sii{$ \to \times \blue{A^{-1}}$}[$ \ot \times \blue{A}$]
		      \blue{A^{-1}}AB\blue{A} = \blue{A^{-1}}C D C^{-1} \blue{A}
		      \sii
		      B\blue{A} = \blue{A^{-1}}C  D (\blue{A^{-1}}C)^{-1}
		      \Sii{$P = \blue{A^{-1}} C$}
		      B\blue{A} = P D P^{-1}
	      $$
	      La expresión de $BA$ resultó diagonalizable. La demostración con $B$ diagonal es análoga.
	      Por lo tanto:
	      $$
		      \cajaResultado{
			      \text{resultó verdadera}
		      }
	      $$



\end{enumerate}
