\begin{enunciado}{\ejExtra}
  Sea $A =
    \matriz{ccc}{
      r & s & t \\
      -12 & 6 & 16 \\
      0 & 0 & 2
    } \en \reales^{3 \times 3}
  $ una matriz tal que $v = (1,2,0)$, $w = (2, 6, 0)$ y $u = (-2, -2, -1)$ son autovectores de $A$.
  \begin{enumerate}[label=\alph*)]
    \item Probar que $A$ es diagonalizable.
    \item Calcular los autovalores de $A$ y determinar $r, s \ytext t$.
  \end{enumerate}
\end{enunciado}

\begin{enumerate}[label=\alph*)]
  \item Es diagonalizable porque estamos en $reales^{3 \times 3}$ y hay una base de dimensión 3 de autovectores:
        $$
          B = \set{(1,2,0), (2,6,0), (-2,-2,-1)},
        $$
        son autovectores de $A$.

  \item Los \textit{autovectores}, son vectores que cumplen la \textit{ecuación característica}:
        $$
          A \cdot v_\lambda = \lambda \cdot v_\lambda
        $$
        Es solo cuestión de pedirle a los autovectores del enunciado que cumplan esa ecuación y despejar.
        $$
          A \cdot
          \matriz{c}{
            1\\
            2\\
            0
          }
          =
          \lambda \cdot
          \matriz{c}{
            1\\
            2\\
            0
          }
          \flecha{de las cuentas}[sale que]
          \llave{rcc}{
            r & \igual{$\llamada1$} & -2s\\
            \lambda & = & 0
          }
        $$
        Siguiente autovector:
        $$
          A \cdot
          \matriz{c}{
            2\\
            6\\
            0
          }
          =
          \lambda \cdot
          \matriz{c}{
            2\\
            6\\
            0
          }
          \flecha{de las cuentas}[sale que]
          \llave{rcl}{
            s & = & 1 \entonces r \igual{$\llamada1$} -2\\
            \lambda & = & 2
          }
        $$
        Siguiente y último autovector
        $$
          A \cdot
          \matriz{c}{
            -2\\
            -2\\
            -1
          }
          =
          \lambda \cdot
          \matriz{c}{
            -2\\
            -2\\
            -1
          }
          \flecha{de las cuentas}[sale que]
          \llave{rcc}{
            t & = & 6\\
            \lambda & = & 2
          }
        $$

        Listo hay subespacios para justificar aún más la diagonabilidad de la matriz:
        $$
          E_{\lambda = 0} = \ket{1,2,0}
          \quad \ytext \quad
          E_{\lambda = 2} = \ket{(-2,-2,-1),(2,6,0)}
        $$
        La multiplicidad geométrica es igual a la multiplicidad aritmética:
        $$
          \multiGeo_A(\lambda = 2) = \multiAri_A(\lambda = 2) = 2
          \quad \ytext \quad
          \multiGeo_A(\lambda = 0) = \multiAri_A(\lambda = 0) = 1
        $$
        La matriz en forma diagonal:
        $$
          \matriz{ccc}{
            -2 & 1 & 6 \\
            -12 & 6 & 16 \\
            0 & 0 & 2
          }
          =
          \matriz{ccc}{
            1 & -2 & 2 \\
            2 & -2 & 6 \\
            0 & -1 & 0
          }
          \matriz{ccc}{
            0 & 0 & 0 \\
            0 & 2 & 0 \\
            0 & 0 & 2
          }
          \matriz{ccc}{
            1 & -2 & 2 \\
            2 & -2 & 6 \\
            0 & -1 & 0
          }^{-1}
        $$
\end{enumerate}

\begin{aportes}
  \item \aporte{\dirRepo}{naD GarRaz \github}
\end{aportes}
