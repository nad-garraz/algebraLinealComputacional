\begin{enunciado}{\ejExtra} \fechaEjercicio{final 7/8/25}
  \begin{enumerate}[label=\alph*)]
    \item Sea $P \en \reales^{n \times n}$ la matriz de un proceso de Markov en el que hay $k$ estados $i_1,\, i_2,\,\ldots,\,i_k$
          tales que la probabilidad de pasar de $i_j$  $i_{j+1}$ para $j = 1, \ldots, k-1$ y de $i_k$ a $i_1$
          es 1. Probar que existe un $\lambda \en \complejos$ autovalor de $P$ tal que $\lambda \distinto 1$, pero
          $|\lambda| = 1$.

    \item
          \begin{minipage}{0.7\textwidth}
            Considerar el proceso descripto por el grafo, donde las probabilidades de transición desde cada nodo se reparten en partes iguales
            entre todas la ramas salientes. Hallar un estado de equilibrio. ¿Es único? ¿Se alcanza este equilibrio desde cualquier estado inicial?
          \end{minipage}
          \begin{minipage}{0.2\textwidth}
            $$
              \begin{tikzpicture}[
                baseline= 3,
                scale=2,
                node distance=1.4cm,
                estado/.style={circle, draw, fill=white, minimum size=0.75cm, inner sep=0pt},
                proba/.style={-{latex[length=1pt]}},
                every node/.style={font={\bf\normalsize}}
                ]
                \node[estado, align = center] (1) {1};
                \node[estado, below right of=1, align = center] (4) {4};
                \node[estado, above right of=1, align = center] (2) {2};
                \node[estado, right of=2, align = center] (3) {3};
                \node[estado, right of=4, align = center] (5) {5};

                \draw[proba] (1) to node[left] {} (2);
                \draw[proba] (2) to node[above] {} (4);
                \draw[proba] (4) to node[] {} (1.south east);
                \draw[proba] (5) to node[] {} (4.east);
                \draw[proba, bend right = 10mm] (3.south) to node[] {} (5.north);
                \draw[proba, bend right = 10mm] (5.north) to node[] {} (3.south);

                \draw[proba] (3) to [out=30, in=90, looseness=6]  node[left] {} (3);
              \end{tikzpicture}
            $$
          \end{minipage}
  \end{enumerate}
\end{enunciado}

\begin{enumerate}[label=\alph*)]
  \item El grafo que describe el proceso de Markov es:
        $$
          \begin{tikzpicture}[
            node distance=2cm,
            estado/.style={circle, draw, minimum size=1cm, inner sep=0pt},
            proba/.style={-{Latex[length=2mm]}, thin, bend left=20}
            ]
            \node[estado] (1) {$i_1$};
            \node[estado, right of=1] (2) {$i_2$};
            \node[estado, right of=2] (3) {$i_3$};
            \node[estado, right of=3, dashed] (dots) {$\cdots$};
            \node[estado, right of=dots] (k-2) {$i_{k-2}$};
            \node[estado, right of=k-2] (k-1) {$i_{k-1}$};
            \node[estado, right of=k-1] (k) {$i_k$};

            \draw[proba] (1) to node[above] {$1$} (2);
            \draw[proba] (2) to node[above] {$1$} (3);
            \draw[proba] (3) to node[above] {$1$} (dots);
            \draw[proba] (dots.north east) to node[above] {$1$} (k-2);
            \draw[proba] (k-2) to node[above] {$1$} (k-1);
            \draw[proba] (k-1) to node[above] {$1$} (k);
            \draw[proba] (k) to node[below] {$1$} (1);

          \end{tikzpicture}
        $$
        Cuya matriz $M$:
        $$
          M =
          \matriz{c|c}{
            \bm{0}^t & 1          \\\hline
            I_{n-1}      & \bm{0}
          }
          =
          \begin{pmatrix}
            0      & 0      & \cdots & 0      & 1      \\
            1      & 0      & \cdots & 0      & 0      \\
            0      & 1      & \cdots & 0      & 0      \\
            \vdots & \ddots & \ddots & \vdots & \vdots \\
            0      & \cdots & 0      & 1      & 0
          \end{pmatrix}
        $$
        Los autovalores de $M$:
        $$
          \det(M - \lambda I) = 0
          \sii
          \begin{pmatrix}
            \lambda & 0       & \cdots & 0      & 1       \\
            1       & \lambda & \cdots & 0      & 0       \\
            0       & 1       & \cdots & 0      & 0       \\
            \vdots  & \ddots  & \ddots & \vdots & \vdots  \\
            0       & \cdots  & 0      & 1      & \lambda
          \end{pmatrix}
          \Sii{$F_1$}
          \lambda^k  - 1 = 0
          \sii
          \cajaResultado{
            \lambda \en G_k
          }
        $$
        Donde $G_k$ son la $k$ raíces $k-$ésimas de la unidad, \textit{esas de álgebra 1, sí, esas que no te acordas ni a palos}:
        $$
          G_{\blue{k}} = \set{e^{i \frac{2\pi}{\blue{k}} h}} \quad \text{con}\quad h \en \enteros{[0,\blue{k}-1]}
        $$
        Todos los elementos de $G_{\blue{k}}$ tienen módulo 1.

  \item
        \textit{Hallar un estado de equilibrio}.
        Únicamente viendo el grafo, los nodos 1, 2 y 4, tienen \blue{una arista de probabilidad \textbf{1}} de ir de uno a otro estado. Considero:
        $$
          v_{eq} = (1, 1, 0, 1, 0)^t
        $$
        Ese estado es de equilibrio porque en cierta forma se van a estar pasando la pelota entre los 3 nodos y nada va a cambiar.

        \bigskip

        \textit{¿Es único?} Sí, tiene pinta, pero no quiero hacer las cuentas.
        Habría que calcular la matriz del proceso y ver la multiplicidad aritmética de $\lambda = 1$.

        \bigskip

        \textit{¿Se alcanza este equilibrio desde cualquier estado inicial?} No. Por ejemplo:
        $$
          v = (1, 0, 0, 0, 0)^t
        $$
        Ese estado va a oscilar entre los estados:
        $$
          (1, 0, 0, 0, 0)^t
          \to
          (0, 1, 0, 0, 0)^t
          \to
          (0, 0, 0, 1, 0)^t
          \to
          (1, 0, 0, 0, 0)^t
        $$
        \textit{...and so on.}

\end{enumerate}

\begin{aportes}
  \item \aporte{\dirRepo}{naD GarRaz \github}
\end{aportes}

