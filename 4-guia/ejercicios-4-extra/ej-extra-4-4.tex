\begin{enunciado}{\ejExtra}
  {\tiny[Segundo cuatrimestre del 2023]}

  Supongamos que los resultados de las elecciones presidenciales del próximo 22 de octubre dependen únicamente de los votos de las primarias del
  13 de Agosto. Consideremos los tres candidatos más votados, denominados L, T y G.
  Los encuestadores nos dicen que:
  \begin{itemize}
    \item Para los votantes de L:
          \begin{itemize}
            \item 80\% mantiene su voto a L
            \item Ninguno cambiará su voto a G
          \end{itemize}

    \item Para los votantes de T:
          \begin{itemize}
            \item El porcentaje de gente que cambia su voto a G y el porcentaje de gente que cambia su voto a L es el mismo.
            \item El porcentaje de gente que cambia su voto es el mismo porcentaje de gente que cambia su voto para los
                  votantes de L.
          \end{itemize}

    \item Para los votantes de G:
          \begin{itemize}
            \item 40\% mantiene su voto a G
            \item El resto se divide equitativamente entre L y T.
          \end{itemize}
  \end{itemize}
  \begin{enumerate}[label=(\alph*)]
    \item Construir la matriz de transición $A$ del proceso.
    \item Si el 13 de Agosto la cantidad de votos para cada uno de los candidatos fue
          \begin{itemize}
            \item L: 30\%
            \item T: 34\%
            \item G: 36\%
          \end{itemize}
          determinar el porcentaje esperado para cada candidato en las elecciones del 22 de octubre.
    \item Asumiendo que el proceso seguirá a largo plazo para próximas elecciones
          (considerando como una unidad de tiempo el tiempo entre una elección y la siguiente),
          decidir si existe un estado límite para los datos iniciales datos, y calcular, si existe, $A^{(\infinito)}$
  \end{enumerate}
\end{enunciado}

\begin{enumerate}[label=(\alph*)]
  \item  Teniendo en cuenta la interpretación de los elementos de una matriz de $Markov$ , pensando
        que las columnas tienen que sumar 1, \hyperlink{teoria-4:markov}{mirá acá el resumen \click}:
        $$
          \matriz{ccc}{
            0.8 & 0.1 & 0.3 \\
            0.2 & 0.8 & 0.3 \\
            0 & 0.1 & 0.4
          }
        $$

  \item $$
          A \cdot
          \matriz{c}{
            0.3\\
            0.34\\
            0.36
          }
          =
          \matriz{c}{
            0.403\\
            0.4818\\
            0.1152
          }
        $$

  \item $A$ tiene un autovalor igual a 1 y ninguno igual a $-1$. Listo con eso sé que va a existir $A^{(\infinito)}$
        más aún sé que las columnas de $A^{(\infinito)}$ son el autovector $v_1$, autovector asociado a $\lambda = 1$
        $$
          A =
          \matriz{c|c|c}{
            &&  \\
            v_1 &\ldots&  v_1\\
            &&
          }
        $$
\end{enumerate}
