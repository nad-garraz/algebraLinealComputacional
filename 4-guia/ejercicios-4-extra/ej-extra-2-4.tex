\begin{enunciado}{\ejExtra}
  \begin{enumerate}[label=\alph*)]
    \item
          Sea $A \en \reales^{n \times n}$. Probar que si $A$ es inversible y diagonalizable, entonces $A^{-1}$
          y $A^k - kI_n$ son diagonalizables para cualquier $k \en \naturales$.
    \item
          Sea $J =
            \matriz{ccc}{
              2 & 0 & 0 \\
              1 & 3 & -1 \\
              -1 & -1 & 3
            } \en \reales^{3 \times 3}
          $.
          \begin{enumerate}[label=\roman*)]
            \item Probar que $J$ es una matriz diagonalizable.
            \item Calcular $J^5 - 5I_3$.
          \end{enumerate}
  \end{enumerate}
\end{enunciado}

\begin{enumerate}[label=\alph*)]
  \item  Truquito destacable: $I_n = PP^{1}$ para luego sacar factor común al calcular $A^k - kI_n$
        Por otro lado, la inversibilidad de una matriz diagonalizable asegura que los autovalores son distintos de cero:
        $$
          |A| = |P D P^{-1}| = |P| |D| |P^{-1}| \igual{\red{!}} |D| = \productoria{i=1}{n} \lambda_i
        $$
        Las matrices inversibles tienen $\det(A) \distinto 0$.

  \item
        \begin{enumerate}[label=\roman*)]
          \item Se calculan los autovectores y autovalores:
                $$
                  E_{\lambda = 2} = \set{(1,0,1), (-1,1,0)}
                  \ytext
                  E_{\lambda = 4} = \set{(0,1,1)}
                  \entonces
                  P =
                  \matriz{ccc}{
                    1 & -1 & 0 \\
                    0 & 1 & 1 \\
                    1 & 0 & 1
                  }
                  D =
                  \matriz{ccc}{
                    2 & 0 & 0 \\
                    0 & 2 & 0 \\
                    0 & 0 & 4
                  }
                $$
                Te debo la inversa por \textit{pajilla}.

          \item
                Sale combinando lo que se usó hasta ahora.
        \end{enumerate}
\end{enumerate}
