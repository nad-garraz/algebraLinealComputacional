\begin{enunciado}{\ejercicio}
  En el instante inicial 20 ratones se encuentran en el compartimiento $I$.
  Las puertas que separan los compartimientos permanecen cerradas salvo durante
  \tikzset{
    puerta/.pic = {
        \fill[scale=0.7] (0,0) circle (2pt);
        \draw[scale=0.7] (0,0) -- (0,\tunelWidth);
      }
  }
  \newcommand{\drawmaze}[2]{%
    \tikz[scale=0.7,every node/.style={font={\small}}]{
      \def\compwidth{2}
      \def\compheight{2}
      \pgfmathsetmacro{\paredAbiertaH}{0.375*\compheight}
      \def\paredAbiertaW{0.375*\compwidth}
      \def\tunelLength{1}
      \pgfmathsetmacro{\tunelWidth}{-\compheight + 2*\paredAbiertaH}

      \draw (0,0)
      -- ++(0,\compheight)
      -- ++(\compwidth,0)
      -- ++(0,-\paredAbiertaH)
      -- ++(\tunelLength,0)
      pic[#1, midway, yshift=#2] {puerta=I}
      -- ++(0,\paredAbiertaH)
      -- ++(\compwidth,0)
      -- ++(0,-\paredAbiertaH)
      -- ++(\tunelLength,0)
      pic[#1, midway, yshift=#2] {puerta=II}
      -- ++(0,\paredAbiertaH)
      -- ++(\compwidth,0)
      -- ++(0,-\compheight)
      -- ++(-\compwidth,0)
      -- ++(0,\paredAbiertaH)
      -- ++(-\tunelLength,0)
      -- ++(0,-\paredAbiertaH)
      -- ++(-\paredAbiertaW,0)
      -- ++(0,-\tunelLength)
      -- ++(\paredAbiertaW,0)
      -- ++(0,-\compheight)
      -- ++(-\compwidth,0)
      -- ++(0,\compheight)
      -- ++(\paredAbiertaW,0)
      -- ++(0,\tunelLength)
      pic[#1, midway, rotate=90, yshift=#2] {puerta=III}
      -- ++(-\paredAbiertaW,0)
      -- ++(0,\paredAbiertaH)
      -- ++(-\tunelLength,0)
      -- ++(0,-\paredAbiertaH)
      -- ++(-\compwidth,0);

      % Labels
      \node (III) at ({\compwidth * 0.5},{\compheight * 0.5}){III};
      \node (I) at ($(III) + ({\compwidth + \tunelLength},0)$) {I};
      \node (II) at ($(I) + ({\compwidth + \tunelLength},0)$) {II};
      \node (IV) at ($(I) + (0,{-\compheight - \tunelLength})$) {IV};
    }%
  }

  \begin{center}
    \begin{multicols}{2}
      \drawmaze{red}{0}
      \\
      \drawmaze{green}{10pt}
    \end{multicols}
  \end{center}

  un breve lapso cada hora, donde los ratones pueden pasar a un compartimiento adyacente o permanecer
  en el mismo. Se supone que nada distingue un compartimiento de otro, es decir que es igualmente probable
  que un ratón pase a cualquier de los adyacentes o se quede en el compartimiento en el que está. Se
  realizan observaciones cada hora y se registra el número de ratones en cada compartimiento.

  \begin{enumerate}[label=(\alph*)]
    \item Determinar la matriz de transición del proceso $P$.
    \item Determinar cuántos ratones habrá en cada celda al cabo de 4 horas.
    \item Decidir si existe o no un estado de equilibrio.
    \item Decidir si existe $P^\infinito$ y en tal caso calcularla. ¿Qué aspecto tiene? ¿Por qué?
  \end{enumerate}
\end{enunciado}
\hacer
