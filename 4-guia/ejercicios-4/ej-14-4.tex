\begin{enunciado}{\ejercicio}
  En el instante inicial 20 ratones se encuentran en el compartimiento I.
  Las puertas que separan los compartimientos permanecen cerradas salvo durante
  \tikzset{
    puerta/.pic = {
        \fill[scale=0.7] (0,0) circle (2pt);
        \draw[scale=0.7] (0,0) -- (0,\tunelWidth);
      }
  }
  \newcommand{\drawmaze}[2]{%
    \tikz[scale=0.7,every node/.style={font={\small}}]{
      \def\compwidth{2}
      \def\compheight{2}
      \pgfmathsetmacro{\paredAbiertaH}{0.375*\compheight}
      \def\paredAbiertaW{0.375*\compwidth}
      \def\tunelLength{1}
      \pgfmathsetmacro{\tunelWidth}{-\compheight + 2*\paredAbiertaH}

      \draw (0,0)
      -- ++(0,\compheight)
      -- ++(\compwidth,0)
      -- ++(0,-\paredAbiertaH)
      -- ++(\tunelLength,0)
      pic[#1, midway, yshift=#2] {puerta=I}
      -- ++(0,\paredAbiertaH)
      -- ++(\compwidth,0)
      -- ++(0,-\paredAbiertaH)
      -- ++(\tunelLength,0)
      pic[#1, midway, yshift=#2] {puerta=II}
      -- ++(0,\paredAbiertaH)
      -- ++(\compwidth,0)
      -- ++(0,-\compheight)
      -- ++(-\compwidth,0)
      -- ++(0,\paredAbiertaH)
      -- ++(-\tunelLength,0)
      -- ++(0,-\paredAbiertaH)
      -- ++(-\paredAbiertaW,0)
      -- ++(0,-\tunelLength)
      -- ++(\paredAbiertaW,0)
      -- ++(0,-\compheight)
      -- ++(-\compwidth,0)
      -- ++(0,\compheight)
      -- ++(\paredAbiertaW,0)
      -- ++(0,\tunelLength)
      pic[#1, midway, rotate=90, yshift=#2] {puerta=III}
      -- ++(-\paredAbiertaW,0)
      -- ++(0,\paredAbiertaH)
      -- ++(-\tunelLength,0)
      -- ++(0,-\paredAbiertaH)
      -- ++(-\compwidth,0);

      % Labels
      \node (III) at ({\compwidth * 0.5},{\compheight * 0.5}){III};
      \node (I) at ($(III) + ({\compwidth + \tunelLength},0)$) {I};
      \node (II) at ($(I) + ({\compwidth + \tunelLength},0)$) {II};
      \node (IV) at ($(I) + (0,{-\compheight - \tunelLength})$) {IV};
    }%
  }

  \begin{center}
    \begin{multicols}{2}
      \drawmaze{red}{0}
      \\
      \drawmaze{green}{10pt}
    \end{multicols}
  \end{center}

  un breve lapso cada hora, donde los ratones pueden pasar a un compartimiento adyacente o permanecer
  en el mismo. Se supone que nada distingue un compartimiento de otro, es decir que es igualmente probable
  que un ratón pase a cualquier de los adyacentes o se quede en el compartimiento en el que está. Se
  realizan observaciones cada hora y se registra el número de ratones en cada compartimiento.

  \begin{enumerate}[label=(\alph*)]
    \item Determinar la matriz de transición del proceso $P$.
    \item Determinar cuántos ratones habrá en cada celda al cabo de 4 horas.
    \item Decidir si existe o no un estado de equilibrio.
    \item Decidir si existe $P^\infinito$ y en tal caso calcularla. ¿Qué aspecto tiene? ¿Por qué?
  \end{enumerate}
\end{enunciado}

\begin{enumerate}[label=(\alph*)]
  \item Pensando que no hay ninguna preferencia por ir a un u otro compartimiento adyacente y dado que
        la matriz resultante tiene que ser de \textit{Markov}:
        $$
          P =
          \matriz{cccc}{
            \frac{1}{4} & \frac{1}{2} & \frac{1}{2} & \frac{1}{2}\\
            \frac{1}{4} & \frac{1}{2} & 0 & 0 \\
            \frac{1}{4} & 0 & \frac{1}{2} & 0 \\
            \frac{1}{4} & 0 & 0 & \frac{1}{2} \\
          }
        $$

  \item Comenzando el experimento tengo 20 ratones en el compartimiento I:
        $$
          \matriz{c}{
            \text{I}\\
            \text{II}\\
            \text{III}\\
            \text{IV}
          }^{(0)} =
          v^{(0)} =
          \matriz{c}{
            20\\
            0\\
            0\\
            0
          }
        $$
        Luego para conseguir el estado siguiente, $v^{(1)}$, multiplico por la matriz $P$:
        $$
          v^{(1)}=
          P v^{(0)} =
          P \matriz{c}{
            20\\
            0\\
            0\\
            0
          }
          =
          \matriz{c}{
            5\\
            5\\
            5\\
            5
          }
        $$
        Siguiendo para conseguir el estado siguiente, $v^{(2)}$, multiplico por la matriz $P$ nuevamente:
        $$
          v^{(2)}=
          P v^{(1)} =
          P \matriz{c}{
            5\\
            5\\
            5\\
            5
          }
          =
          \matriz{c}{
            \frac{35}{4}\\
            \frac{15}{4}\\
            \frac{15}{4}\\
            \frac{15}{4}
          }
        $$
        Siguiendo para conseguir el estado siguiente, $v^{(3)}$, multiplico por la matriz $P$ nuevamente:
        $$
          v^{(3)}=
          P v^{(2)} =
          P
          \matriz{c}{
            \frac{35}{4}\\
            \frac{15}{4}\\
            \frac{15}{4}\\
            \frac{15}{4}
          }
          =
          \matriz{c}{
            \frac{125}{16}\\
            \frac{65}{16}\\
            \frac{65}{16}\\
            \frac{65}{16}
          }
          =
          \matriz{c}{
            7.8125\\
            4.0625\\
            4.0625\\
            4.0625
          }
        $$
        Siguiendo para conseguir el estado siguiente, $v^{(4)}$, multiplico por la matriz $P$ nuevamente:
        $$
          v^{(4)}=
          P v^{(3)} =
          P
          \matriz{c}{
            \frac{125}{16}\\
            \frac{65}{16}\\
            \frac{65}{16}\\
            \frac{65}{16}
          }
          =
          \matriz{c}{
            \frac{514}{64}\\
            \frac{255}{64}\\
            \frac{255}{64}\\
            \frac{255}{64}
          }
          =
          \matriz{c}{
            8.046875\\
            3.984375\\
            3.984375\\
            3.984375
          }
        $$
        Se esperan entonce 8 ratones en el compratimiento I y 4 en los compartimientos II, III, IV.
          {\small
            \codigoPython{ej-14/codigo4-14-2.py}
          }

  \item
        Hay un equilibrio, parece ser que la sucesión converge al estado $v^{\text{eq}} = (8,4,4,4)^t$

        Calculo los autovalores y autovectores:
        \codigoPython{ej-14/codigo4-14-1.py}
        Dado que tengo un solo autovalor igual a 1, me agarro de eso para decir que va a haber un equilibrio estable
        hacia el autovector asociado:
        $$
          E_{\lambda = 1} =
          \ket{
            \matriz{c}{
              2\\
              1\\
              1\\
              1
            }
          }
        $$

  \item Tengo 4 autovectores que forman una base de $\reales^4$, por lo tanto la $P$ es diagonalizable:
        $$
          \begin{array}{rcc}
            P = C D C^{-1}
                        & \sii                             &
            P^{\red{n}} = C D^{\red{n}} C^{-1}                                                                 \\
                        & =                                &
            C
            \matriz{cccc}{
            1^{\red{n}} & 0                                & 0                       & 0                       \\
            0           & (-\frac{1}{4})^{\red{n}}         & 0                       & 0                       \\
            0           & 0                                & (\frac{1}{2})^{\red{n}} & 0                       \\
            0           & 0                                & 0                       & (\frac{1}{2})^{\red{n}}
            }
            C^{-1}                                                                                             \\
                        & \flecha{$\red{n} \to \infinito$} &
            C
            \matriz{cccc}{
            1           & 0                                & 0                       & 0                       \\
            0           & 0                                & 0                       & 0                       \\
            0           & 0                                & 0                       & 0                       \\
            0           & 0                                & 0                       & 0
            }
            C^{-1}
          \end{array}
        $$
        Entonces:
        $$
          \begin{array}{rcl}
            \limite{\red{n}}{\infinito} P^{\red{n}}
                        & =                 &
            \ub{
              \matriz{cccc}{
            2           & 0                 & 0           & 0           \\
            1           & 0                 & 0           & 0           \\
            1           & 0                 & 0           & 0           \\
            1           & 0                 & 0           & 0
              }
            }{
              C D^{\red{n}}
            }
            \ub{
              \matriz{cccc}{
            a           & b                 & c           & d           \\
            *           & *                 & *           & *           \\
            *           & *                 & *           & *           \\
            *           & *                 & *           & *           \\
              }
            }{
              C^{-1}
            }                                                           \\
                        & =                 &
            \matriz{c|c|c|c}{
              C D^{\red{n}} \
              \matriz{c}{
            a                                                           \\
            *                                                           \\
            *                                                           \\
                *
              }
                        &
              C D^{\red{n}}
              \matriz{c}{
            b                                                           \\
            *                                                           \\
            *                                                           \\
                *
              }
                        &
              C D^{\red{n}}
              \matriz{c}{
            c                                                           \\
            *                                                           \\
            *                                                           \\
                *
              }
                        &
              C D^{\red{n}}
              \matriz{c}{
            d                                                           \\
            *                                                           \\
            *                                                           \\
                *
              }
            }                                                           \\
                        & =                 &
            \matriz{c|c|c|c}{
              a\matriz{c}{
            2                                                           \\
            1                                                           \\
            1                                                           \\
                1
              }
                        &
              b\matriz{c}{
            2                                                           \\
            1                                                           \\
            1                                                           \\
                1
              }
                        &
              c\matriz{c}{
            2                                                           \\
            1                                                           \\
            1                                                           \\
                1
              }
                        &
              d\matriz{c}{
            2                                                           \\
            1                                                           \\
            1                                                           \\
                1
              }
            }                                                           \\
                        & \igual{\red{!!!}} &
            \matriz{cccc}{
            \frac{2}{5} & \frac{2}{5}       & \frac{2}{5} & \frac{2}{5} \\
            \frac{1}{5} & \frac{1}{5}       & \frac{1}{5} & \frac{1}{5} \\
            \frac{1}{5} & \frac{1}{5}       & \frac{1}{5} & \frac{1}{5} \\
            \frac{1}{5} & \frac{1}{5}       & \frac{1}{5} & \frac{1}{5}
            }                                                           \\
                        & \igual{\red{!}}   &
            \frac{1}{5}
            \matriz{cccc}{
            2           & 2                 & 2           & 2           \\
            1           & 1                 & 1           & 1           \\
            1           & 1                 & 1           & 1           \\
            1           & 1                 & 1           & 1
            }
          \end{array}
        $$
        En el \red{!!!} encuentro que $a = b = c = d = \frac{1}{5}$ porque $P^{\red{\infinito}}$ debe ser una matriz de Markov, por
        lo tanto sus columnas deben sumar 1.

        La matriz tiene como todas sus columnas al autovector de autovalor $\lambda = 1$. Multiplicar una matriz $A$ por un vector
        $\bm{v}$ es equivalente a escribir al vector como una combinación lineal de las columnas de la matriz. En este caso
        todas las columnas son el vector normalizado $(\frac{2}{5},\frac{1}{5},\frac{1}{5},\frac{1}{5})^t$,
        por lo que \underline{sin importar el estado}, es decir el vector $\bm{v}$ que uses, en \underline{una sola} multiplicación ya
        se va al equilibrio.

        \bigskip

        Puede surgir la pregunta:
        \parrafoDestacado{
          ¿Qué pasa si encuentro un estado inicial $\green{\bm{v}^{(0)}}$ cuyas coordenadas en la \textit{base de autovectores}
          de $P$ $\set{\bm{v}^{eq}, \bm{v}_2, \cdots, \bm{v}_n}$ tenga \magenta{0} para el autovector $\bm{v}^{eq}$, el autovector asociado a $\lambda = 1$?
        }
        Es decir:
        $$
          \green{\bm{v}^{(0)}} = \ua{\red{c_1}}{\magenta{0}} \bm{v}^{eq} + \red{c_2} \bm{v}_2 + \cdots + \red{c_n} \bm{v}_n
        $$
        Bueno, resulta que en una base de autovectores de una \textit{matriz de Markov} eso no va a pasar nunca. Por
        esto:
        $$
          \llave{l}{
            \vec{1}^t = (1, \cdots, 1)   \\
            P : \textit{ matriz de Markov} \\
            B = \set{\bm{v}_1, \bm{v}_2, \cdots, \bm{v}_n} \textit{ autovectores de $P$},
          }
        $$
        La siguiente cuenta hace todo:
        $$
          0 =
          \vec{1}^t P \bm{v}_i -  \vec{1}^t P \bm{v}_i =
          (\vec{1}^t P) \bm{v}_i - \vec{1}^t (P \bm{v}_i) =
          \vec{1}^t \bm{v}_i - \vec{1}^t \lambda_i \bm{v}_i =
          (1 - \lambda) \vec{1}^t \bm{v}_i
          \Sii{\red{!!!}}
          \llave{l}{
            1 - \lambda = 0 \\
            \text{ o }\\
            \sumatoria{\blue{j} = 1}{n} (\bm{v}_i)_{\blue{j}} = 0
          }
        $$
        Por lo tanto si el autovector no es $\bm{v}^{eq}$, el autovector de $P$ asociado al $\lambda = 1$, los otros autovectores
        no son \textit{vectores de probabilidad}, más aún \red{la suma de sus coordenadas es 0}. Y una combinación así:
        $$
          \vec{1}^t \cdot \green{\bm{v}^{(0)}} =
          \vec{1}^t \cdot
          \ob{
            (\ua{
              \red{c_1}
            }{
              \magenta{0}
            }
            \bm{v}^{eq} + \red{c_2} \bm{v}_2 + \cdots + \red{c_n} \bm{v}_n)
          }{
            \green{\bm{v}^{(0)}}
          } \igual{\red{!}} \red{c_2} \bm{v}_2 + \cdots + \red{c_n} \bm{v}_n = 0
        $$
        \underline{nunca podría ser un vector de probabilidad, porque sus coordenadas sumarían 0.}
\end{enumerate}

\begin{aportes}
  \item \aporte{\dirRepo}{naD GarRaz \github}
  \item \aporte{https://github.com/misProyectosPropios}{Iñaki Frutos \github}
  \item \aporte{https://github.com/koopardo/}{Marcos Zea \github}
\end{aportes}
