\begin{enunciado}{\ejercicio}
  Considerar la matrices
  $$
    A =
    \matriz{cc}{
      1 & \frac{1}{\epsilon}\\
      \epsilon & 1
    },
    \qquad
    B =
    \matriz{cc}{
      1 & \frac{1}{\epsilon}\\
      0 & 1
    },
  $$
  donde $\epsilon \ll 1$ es arbitrario. Calcular los polinomios característicos y los autovalores de $A$ y de $B$.
  Concluir que pequeñas perturbaciones en los coeficientes de un polinomio pueden conducir
  a grandes variaciones en sus raíces (el problema está mal condicionado). En particular, esto
  afecta el cómputo de autovalores como raíces del polinomio característico.
\end{enunciado}

El polinomio característico de $A$ y $B$:
$$
  \begin{array}{rcl}
    \mathcal{X}_A & = & (1 - \lambda)^2 - 1 = \lambda \cdot (\lambda - 2) = 0
    \sii
    \llave{l}{
    \lambda_1 = 0                                                             \\
      \lambda_2 = 2
    }                                                                         \\
    \mathcal{X}_B & = & (1 - \lambda)^2 = 0
    \sii
    \llave{l}{
    \lambda_1 = 1                                                             \\
      \lambda_2 = 1
    }
  \end{array}
$$
Medio que el enunciado cuenta todo. En particular se puede acotar la condición de esas matrices.
Por ejemplo para
$C =
  \matriz{cc}{
    1 & \frac{1}{\epsilon}\\
    0 & 0
  }$:
$$
  \begin{array}{c}
    \condicion_\infinito(A) \geq \frac{\norma{A}_\infinito}{\norma{A - C}_\infinito} =
    \frac{1 + \frac{1}{\epsilon}}{\epsilon + 1}
    \flecha{$\epsilon \to 0$}
    \infinito
  \end{array}
$$
Lo mismo se puede hacer para la matriz $B$. Esas matrices están mal condicionadas y como se puede ver en los autovalores,
a pesar de tener elementos similares los resultados en el cálculo de los \textit{autovalores} las resultados pueden variar mucho.

\begin{aportes}
  \item \aporte{\dirRepo}{naD GarRaz \github}
\end{aportes}
