\begin{enunciado}{\ejercicio}
  Un sujeto en evidente estado de ebriedad oscila entre su casa y el bar, separados
  por $n$ pasos. En cada instante de tiempo da un paso hacia adelante (acercándose a su casa),
  con probabilidad $p$ o hacia atrás (acercándose nuevamente al bar), con probabilidad $1 - p$.
  Si llega a alguno de los dos extremos, se queda allí y no vuelve a moverse.

  \begin{enumerate}[label=(\alph*)]
    \item Sin hacer ninguna cuenta, mostrar que el proceso admite al menos dos estados límite
          linealmente independientes entre sí. Implementar un programa que reciba como \texttt{input} la
          distancia entre la casa y el bar ($n$) y la probabilidad $p$ y devuelva la matriz de transición
          del proceso. Verificar que el resultado sea correcto corriéndolo para $n = 5$ y $p = 0.5$.

    \item Para $n = 20$, tomar $p = 0.5$ y $\bm{v}^0$ el vector que corresponde a ubicar al sujeto
          en cualquiera de los puntos intermedios del trayecto con igual probabilidad. Realizar una
          simulación del proceso hasta que se estabilice ¿Cuál es el estado límite? ¿Cómo se interpreta?

    \item Repetir la simulaciones con $p = 0.8$. ¿Qué se observa?

    \item Explicar los resultados de todas las simulaciones a partir del análisis de los autovalores y autovectores de la matriz.
  \end{enumerate}
\end{enunciado}

\begin{enumerate}[label=(\alph*)]
  \item Después de que no se me ocurriera como hacer esto, quizás porque, no hay ninguna razón para eso,
        lo que quedó es:
        \begin{itemize}
          \item Sin dibujar un grafo esto no se te debería ocurrir ni a palos:
                $$
                  \begin{tikzpicture}[
                    node distance=2cm,
                    estado/.style={circle, draw, minimum size=1cm, inner sep=0pt},
                    proba/.style={-{Latex[length=2mm]}, thin}
                    ]
                    \node[estado, Cerulean] (0) {B};
                    \node[estado, right of=0] (1) {1};
                    \node[estado, right of=1] (2) {2};
                    \node[estado, dashed, gray, right of=2, node distance=2cm] (dots) {$\cdots$};
                    \node[estado, right of=dots, node distance=2cm] (n-1) {$n-1$};
                    \node[estado, right of=n-1, node distance=2cm, Cerulean] (n) {$C$};

                    \draw[proba, bend left=20] (1) to node[above] {$p$} (2);
                    \draw[proba, bend left=20] (2) to node[above] {$p$} (dots.north west);
                    \draw[proba, bend left=20] (dots.north east) to node[above] {$p$} (n-1);
                    \draw[proba, bend left=20] (n-1) to node[above] {$p$} (n);

                    \draw[proba, bend left=20] (1) to node[below] {$1-p$} (0);
                    \draw[proba, bend left=20] (2) to node[below] {$1-p$} (1);
                    \draw[proba, bend left=20] (dots.south west) to node[below] {$1-p$} (2);
                    \draw[proba, bend left=20] (n-1) to node[below] {$1-p$} (dots.south east);

                    \draw[proba] (0) to [out=180, in=90, looseness=5]  node[left] {$1$} (0.north);
                    \draw[proba] (n) to [out=0, in=-90, looseness=5]  node[right] {$1$} (n.south);
                  \end{tikzpicture}
                $$
          \item Hay 2 estados de equilibrio. Voy a tener un $\lambda = 1$ de multiplicidad 2.
                Esos estados son cuando el loco está en \blue{C} o \blue{B}.

          \item Cada nodo es una posible ubicación del tipo, y como de un paso a otro siempre se mueve, en la matriz hay ceros
                en los elementos $a_{ij}$, excepto en los límites.

          \item Con esta data, ponele que tenés 3 pasos para ir desde el \blue{Bar} hasta tu \blue{Casa}, i.e. $n = 3$:
                $$
                  A =
                  \matriz{c|cccc}{
                    \text{\yellow{\faIcon{beer}}} & \blue{B} & 1 & 2 & \blue{C} \\ \hline
                    \blue{B}    & 1 & 1 - p & 0 & 0 \\
                    1    & 0 & 0 & 1-p & 0 \\
                    2    & 0 & p & 0 & 0 \\
                    \blue{C}    & 0 & 0 & p & 1
                  }
                $$

          \item Ahora vas y le pedís a algún LLM y le pedís un programita al que le das $n$, $p$ y te devuelva esa \textit{matriz de Markov}
                o también si te interesa hacerlo a mano, tenés mi \blue{bendición}.
        \end{itemize}
        \hacer

  \item Si $p = 0.5$ el tipo se tomó el \textit{trago de Schrödinger}, es decir que está borracho y sobrio a la vez, hasta que le hacen el test
        de alcoholemia y el resultado tiene que ser o que bien está en pedo o no, o algo así, no sé, ni que fuera físico ni bartender.
        \hacer

  \item Acá el chabón le aflojó a la sidra, se tomó una coquita y arrancó para casa que está la mujer esperando que hoy le toca cocinar, así que
        se despide de la gente arranca y bueh, llega antes, probabilísticamente antes.
        \hacer

  \item Mirando la matriz tengo unos autovectores:
        $$
          E_{\lambda = 1} =
          \ket{
            (1,0,\ldots,0), (0,\ldots, 0,1)
          }
        $$
        y los otros te los debo.
        \hacer
\end{enumerate}

\begin{aportes}
  \item \aporte{\dirRepo}{naD GarRaz \github}
  \item \aporte{https://github.com/misProyectosPropios}{Iñaki Frutos \github}
\end{aportes}
