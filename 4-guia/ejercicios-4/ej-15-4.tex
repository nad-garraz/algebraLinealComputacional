\begin{enunciado}{\ejercicio}
  Un sujeto en evidente estado de ebriedad oscila entre su casa y el bar, separados
  por $n$ pasos. En cada instante de tiempo da un paso hacia adelante (acercándose a su casa),
  con probabilidad $p$ o hacia atrás (acercándose nuevamente al bar), con probabilidad $1 - p$.
  Si llega a alguno de los dos extremos, se queda allí y no vuelve a moverse.

  \begin{enumerate}[label=(\alph*)]
    \item Sin hacer ninguna cuenta, mostrar que el proceso admite al menos dos estados límite
          linealmente independientes entre sí. Implementar un programa que reciba como \texttt{input} la
          distancia entre la casa y el bar ($n$) y la probabilidad $p$ y devuelva la matriz de transición
          del proceso. Verificar que el resultado sea correcto corriéndolo para $n = 5$ y $p = 0.5$.

    \item Para $n = 20$, tomar $p = 0.5$ y $\bm{v}^0$ el vector que corresponde a ubicar al sujeto
          en cualquiera de los puntos intermedios del trayecto con igual probabilidad. Realizar una
          simulación del proceso hasta que se estabilice ¿Cuál es el estado límite? ¿Cómo se interpreta?

    \item Repetir la simulación tomando como vector inicial $\bm{v}^0 = \bm{e}_2$ (el segundo canónico).
          Interpretar el resultado.

    \item Repetir la simulaciones con $p = 0.8$. ¿Qué se observa?

    \item Explicar los resultados de todas las simulaciones a partir del análisis de los autovalores y autovectores de la matriz.
  \end{enumerate}
\end{enunciado}

\begin{enumerate}[label=(\alph*)]
  \item Después de que no se me ocurriera como hacer esto, quizás porque, no hay ninguna razón para eso,
        lo que quedó es:
        \begin{itemize}
          \item Sin dibujar un grafo esto no se te debería ocurrir ni a palos:
                $$
                  \begin{tikzpicture}[
                    node distance=2cm,
                    estado/.style={circle, draw, minimum size=1cm, inner sep=0pt},
                    proba/.style={-{Latex[length=2mm]}, thin}
                    ]
                    \node[estado, Cerulean] (0) {B};
                    \node[estado, right of=0] (1) {1};
                    \node[estado, right of=1] (2) {2};
                    \node[estado, dashed, gray, right of=2, node distance=2cm] (dots) {$\cdots$};
                    \node[estado, right of=dots, node distance=2cm] (n-1) {$n-1$};
                    \node[estado, right of=n-1, node distance=2cm] (n) {$n$};
                    \node[estado, right of=n, node distance=2cm, Cerulean] (C) {$C$};

                    \draw[proba, bend left=20] (1) to node[above] {$p$} (2);
                    \draw[proba, bend left=20] (2) to node[above] {$p$} (dots.north west);
                    \draw[proba, bend left=20] (dots.north east) to node[above] {$p$} (n-1);
                    \draw[proba, bend left=20] (n-1) to node[above] {$p$} (n);
                    \draw[proba, bend left=20] (n) to node[above] {$p$} (C);

                    \draw[proba, bend left=20] (1) to node[below] {$1-p$} (0);
                    \draw[proba, bend left=20] (2) to node[below] {$1-p$} (1);
                    \draw[proba, bend left=20] (dots.south west) to node[below] {$1-p$} (2);
                    \draw[proba, bend left=20] (n-1) to node[below] {$1-p$} (dots.south east);
                    \draw[proba, bend left=20] (n) to node[below] {$1-p$} (n-1.south east);

                    \draw[proba] (0) to [out=180, in=90, looseness=5]  node[left] {$1$} (0.north);
                    \draw[proba] (C) to [out=0, in=-90, looseness=5]  node[right] {$1$} (C.south);
                  \end{tikzpicture}
                $$
          \item Hay 2 estados de equilibrio. Voy a tener un $\lambda = 1$ de multiplicidad 2.
                Esos estados son cuando el loco está en \blue{C} o \blue{B}. Fijate que tienen una flecha saliendo con
                valor 1. Es decir que ese el único valor distinto de 0 que va a haber en la columna, para que sea
                una \textit{matriz estocástica}.

          \item Cada nodo es una posible ubicación del tipo, y como de un paso a otro siempre se mueve, en la matriz hay ceros
                en los elementos $a_{ij}$, excepto en los límites.

          \item Con esta data, ponele que tenés 2 pasos para ir desde el \blue{Bar} hasta tu \blue{Casa}, i.e. $n = 2$:
                $$
                  A =
                  \matriz{c|cccc}{
                    \text{\yellow{\faIcon{beer}}} & \blue{B} & 1 & 2 & \blue{C} \\ \hline
                    \blue{B}    & 1 & 1 - p & 0 & 0 \\
                    1    & 0 & 0 & 1-p & 0 \\
                    2    & 0 & p & 0 & 0 \\
                    \blue{C}    & 0 & 0 & p & 1
                  }
                $$

          \item Ahora vas y le pedís a algún LLM y le pedís un programita al que le das $n$, $p$ y te devuelva esa \textit{matriz de Markov}
                o también si te interesa hacerlo a mano, tenés mi \blue{bendición}.

                A mí me saió esto:
        \end{itemize}
        \codigoPython{ej-15/codigo4-15-1.py}

  \item Si $p = 0.5$ el tipo se tomó el \textit{trago de Schrödinger}, es decir que está borracho y sobrio a la vez, hasta que le hacen el test
        de alcoholemia y el resultado tiene que ser o que bien está en pedo o no, o algo así, no sé, ni que fuera físico ni bartender.

        Lo que si sabemos es que el tipo va a estar acá:
        {\small
        $$
          \bm{v}^0 = \left(0,
          \foreach \x in {1,...,20} {
            \frac{1}{20},
          }
          0
          \right)^t
          \
          \text{con}
          \
          \norma{\bm{v}^0}_1 = 1
        $$
        }
        Es decir con igual probabilidad en alguno de los pasos entre el \blue{bar} y la \blue{casa}
        y seguro que no está ni en el \blue{bar} ni en la \blue{casa}.
        \codigoPython{ej-15/codigo4-15-bcd.py}

        El código te va a mostrar 2 gráficos, uno para 10 y otro para 1000 iteraciones. El estado evoluciona hacia el:
        {\small
        $$
          \bm{v}^\infinito = (
          \frac{1}{2},
          \foreach \x in {1,...,20} {
              0,
            }
          \frac{1}{2}
          ),
        $$
        }
        $$
          \begin{array}{cc}
            \begin{tikzpicture}[every node/.style = {font=\tiny}]
              \begin{axis}[
                  ymode = log,
                  xlabel={Pasos},
                  ylabel={Probalidad},
                  title={Primeras iteraciones de \textit{Probabilidad vs Pasos} del borracho},
                  grid=major,
                  % width=0.5\linewidth,
                  % height=8cm,
                  legend style={
                      cells={anchor=center},
                      font =\tiny,
                      at = {(1,-0.15)},
                      anchor=center,
                      at = {(0.5,-0.25)},
                      legend columns=4
                    },
                  legend entries =
                    {
                      $\text{estado } v^{(0)}$,
                      $\text{estado } v^{(1)}$,
                      $\text{estado } v^{(10)}$,
                      $\text{estado } v^{(15)}$,
                      $\text{estado } v^{(21)}$
                    },
                  cycle list name=color list,
                  every axis plot/.append style={only marks, mark size=2pt},
                ]
                \foreach \x in {0,...,3}{
                    \addplot+ table {./ejercicios-4/dataFiles/item-b-plot/\x-step-item-b.data};
                  }
              \end{axis}
            \end{tikzpicture}
             &
            \begin{tikzpicture}[every node/.style = {font=\tiny}]
              \begin{axis}[
                  ymode = log,
                  ymax = 1,
                  xlabel={Pasos},
                  ylabel={Probalidad},
                  title={Evolución \textit{Probabilidad vs Pasos} del borracho. 1000 iteraciones.},
                  legend style={
                      cells={anchor=center},
                      font =\tiny,
                      anchor=center,
                      at = {(0.5,-0.25)},
                      legend columns=4
                    },
                  legend entries =
                    {
                      $\text{estado } v^{(100)}$,
                      $\text{estado } v^{(200)}$,
                      $\text{estado } v^{(500)}$,
                      $\text{estado } v^{(1000)}$,
                    },
                  cycle list name=color list,
                  every axis plot/.append style={only marks, mark size=1.5pt},
                  grid=major
                ]
                \foreach \x in {0,...,3}{
                    \addplot table {./ejercicios-4/dataFiles/item-b-plot/\x-step-item-b-final.data};
                  }
              \end{axis}
            \end{tikzpicture}
          \end{array}
        $$
        \magenta{¡Atención a la escala logarítmica del eje \text{y}}, que parece que no se cumple que los estados suman 1, pero suman.

        Algo que podría ser esperable. Mientras tanto el tipo va a pasar más tiempo en el medio del camino que en otro lugar. Cuando se
        aleja del centro del camino la probabilidad aumenta para que vuelva al centro. Eso se ve en los gráficos, porque
        para un estado en particular, siempre el siguiente paso tiene más probabilidad de ir hacia el centro
        para.

        Después de $1000$ pasos el tipo tiene igual probabilidad, $p \approx 0.5$ de estar en el \blue{bar} que en su \blue{casa}.

  \item Ahora la simulación arranca en:
        {\small
        $$
          \bm{v}^0 = \left(0,1,
          \foreach \x in {1,...,19} {
            0,
          }
          0
          \right)^t
          \
          \text{con}
          \
          \norma{\bm{v}^0}_1 = 1
        $$
        }
        Haciendo los mismo cálculos de antes el resultado que sale de la simulación
        $$
          \begin{array}{cc}
            \begin{tikzpicture}[every node/.style = {font=\tiny}]
              \begin{axis}[
                  ymode = log,
                  ymax = 10,
                  xlabel={Pasos},
                  ylabel={Probalidad},
                  title={Primeras iteraciones de \textit{Probabilidad vs Pasos} del borracho},
                  grid=major,
                  % width=0.5\linewidth,
                  % height=8cm,
                  legend style={
                      cells={anchor=center},
                      font =\tiny,
                      anchor=center,
                      at = {(0.5,-0.25)},
                      legend columns=4
                    },
                  legend entries =
                    {
                      $\text{estado } v^{(0)}$,
                      $\text{estado } v^{(1)}$,
                      $\text{estado } v^{(10)}$,
                      $\text{estado } v^{(15)}$,
                      $\text{estado } v^{(21)}$,
                    },
                  cycle list name=color list,
                  every axis plot/.append style={only marks, mark size=2pt},
                ]
                \foreach \x in {0,...,3}{
                    \addplot table {./ejercicios-4/dataFiles/item-c-plot/\x-step-item-c.data};
                  }
              \end{axis}
            \end{tikzpicture}
             &
            \begin{tikzpicture}[every node/.style = {font=\tiny}]
              \begin{axis}[
                  ymode = log,
                  ymax = 1,
                  xlabel={Pasos},
                  title={Evolución \textit{Probabilidad vs Pasos} del borracho. 1000 iteraciones.},
                  % width=0.5\linewidth,
                  % height=10cm,
                  smooth,
                  legend style={
                      cells={anchor=center},
                      font =\tiny,
                      anchor=center,
                      at = {(0.5,-0.25)},
                      legend columns=4
                    },
                  legend entries =
                    {
                      $\text{estado } v^{(100)}$,
                      $\text{estado } v^{(200)}$,
                      $\text{estado } v^{(500)}$,
                      $\text{estado } v^{(1000)}$,
                    },
                  grid=major,
                  cycle list name=color list,
                  every axis plot/.append style={only marks, mark size=2pt},
                ]
                \foreach \x in {0,...,3}{
                    \addplot table {./ejercicios-4/dataFiles/item-c-plot/\x-step-item-c-final.data};
                  }
              \end{axis}
            \end{tikzpicture}
          \end{array}
        $$
        \magenta{¡Atención a la escala logarítmica del eje \text{y}}, que parece que no se cumple que los estados suman 1, pero suman.

        En los gráficos puede verse que en las primeras iteraciones la probabilidad de quedarse
        en el \blue{bar} crece mucho más que la probabilidad de acercarse a la \blue{casa}. De hecho,
        después de dar 1000 pasos, 500 veces más pasos de lo necesario, la probabilidad de que el tipo llegue a su \blue{casa}
        es de $\approx 0.05$, durísimo.

  \item Acá el chabón le aflojó a la sidra, se tomó una coquita y arrancó para casa que está la mujer esperando que hoy le toca cocinar, así que
        se despide de la gente arranca y bueh, llega antes, probabilísticamente antes.

        En estas circunstancias, la probabilidad de que el tipo esté caminando 200 pasos para llegar a la \blue{casa} es bajísima $p \approx 10^{-10}$
        $$
          \begin{array}{cc}
            \begin{tikzpicture}[every node/.style = {font=\tiny}]
              \begin{axis}[
                  ymode = log,
                  xlabel={Pasos},
                  ylabel={Probalidad},
                  title={Primeras iteraciones de \textit{Probabilidad vs Pasos} del borracho},
                  grid=major,
                  line width= 0,
                  % width=0.5\linewidth,
                  % height=8cm,
                  legend style={
                      cells={anchor=center},
                      font =\tiny,
                      at = {(1,-0.15)},
                      anchor=center,
                      at = {(0.5,-0.25)},
                      legend columns=4
                    },
                  legend entries =
                    {
                      $\text{estado } v^{(0)}$,
                      $\text{estado } v^{(1)}$,
                      $\text{estado } v^{(10)}$,
                      $\text{estado } v^{(15)}$,
                      $\text{estado } v^{(21)}$,
                    },
                  cycle list name=color list,
                  every axis plot/.append style={only marks, mark size=2pt},
                ]
                \foreach \x in {0,...,3}{
                    \addplot table {./ejercicios-4/dataFiles/item-d-plot/\x-step-item-d.data};
                  }
              \end{axis}
            \end{tikzpicture}
             &
            \begin{tikzpicture}[every node/.style = {font=\tiny}]
              \begin{axis}[
                  ymode = log,
                  ymax = 1,
                  xlabel={Pasos},
                  title={Evolución \textit{Probabilidad vs Pasos} del borracho. 1000 iteraciones.},
                  legend style={
                      cells={anchor=center},
                      font =\tiny,
                      anchor=center,
                      at = {(0.5,-0.25)},
                      legend columns=4
                    },
                  legend entries =
                    {
                      $\text{estado } v^{(100)}$,
                      $\text{estado } v^{(200)}$,
                      $\text{estado } v^{(500)}$,
                      $\text{estado } v^{(1000)}$,
                    },
                  cycle list name=color list,
                  every axis plot/.append style={only marks, mark size=2pt},
                  grid=major
                ]
                \foreach \x in {0,...,3}{
                    \addplot table {./ejercicios-4/dataFiles/item-d-plot/\x-step-item-d-final.data};
                  }
              \end{axis}
            \end{tikzpicture} \\
            \begin{tikzpicture}[every node/.style = {font=\tiny}]
              \begin{axis}[
                  ymode = log,
                  xlabel={Pasos},
                  ylabel={Probalidad},
                  title={Primeras iteraciones de \textit{Probabilidad vs Pasos} del borracho},
                  grid=major,
                  line width= 0,
                  % width=0.5\linewidth,
                  % height=8cm,
                  legend style={
                      cells={anchor=center},
                      font =\tiny,
                      at = {(1,-0.15)},
                      anchor=center,
                      at = {(0.5,-0.25)},
                      legend columns=4
                    },
                  legend entries =
                    {
                      $\text{estado } v^{(0)}$,
                      $\text{estado } v^{(4)}$,
                      $\text{estado } v^{(8)}$,
                      $\text{estado } v^{(12)}$,
                      $\text{estado } v^{(16)}$,
                      $\text{estado } v^{(20)}$,
                    },
                  cycle list name=color list,
                  every axis plot/.append style={only marks, mark size=2pt},
                ]
                \foreach \x in {0,...,3}{
                    \addplot table {./ejercicios-4/dataFiles/item-d-plot/\x-step-item-d2.data};
                  }
              \end{axis}
            \end{tikzpicture}
             &
            \begin{tikzpicture}[every node/.style = {font=\tiny}]
              \begin{axis}[
                  ymode = log,
                  ymax = 1,
                  xlabel={Pasos},
                  title={Evolución \textit{Probabilidad vs Pasos} del borracho. 1000 iteraciones.},
                  legend style={
                      cells={anchor=center},
                      font =\tiny,
                      anchor=center,
                      at = {(0.5,-0.25)},
                      legend columns=4
                    },
                  legend entries =
                    {
                      $\text{estado } v^{(0)}$,
                      $\text{estado } v^{(200)}$,
                      $\text{estado } v^{(400)}$,
                      $\text{estado } v^{(600)}$,
                      $\text{estado } v^{(800)}$,
                      $\text{estado } v^{(1000)}$,
                    },
                  cycle list name=color list,
                  every axis plot/.append style={only marks, mark size=2pt},
                  grid=major
                ]
                \foreach \x in {0,...,3}{
                    \addplot table {./ejercicios-4/dataFiles/item-d-plot/\x-step-item-d2-final.data};
                  }
              \end{axis}
            \end{tikzpicture}
          \end{array}
        $$

  \item Mirando la matriz tengo unos autovectores:
        $$
          E_{\lambda = 1} =
          \ket{
            (1,0,\ldots,0), (0,\ldots, 0,1)
          }
        $$
        y los otros te los debo.
        \hacer
\end{enumerate}

\begin{aportes}
  \item \aporte{\dirRepo}{naD GarRaz \github}
  \item \aporte{https://github.com/misProyectosPropios}{Iñaki Frutos \github}
\end{aportes}
