\begin{enunciado}{\ejercicio}
  Calcular el polinomio característico, los autovalores y los autovectores de la matriz $A$ en cada uno de los siguientes
  casos (analizar por separado los casos $K = \reales$ y  $K = \complejos$):
  \begin{enumerate}[label=(\alph*)]
    \begin{multicols}{3}
      \item $A =
        \matriz{cc}{
          0 & a \\
          -a & 0
        }$

      \item $A =
        \matriz{ccc}{
          0 & 2 & 1 \\
          -2 & 0 & 2 \\
          -1 & -2 & 0
        }$

      \item $A =
        \matriz{ccc}{
          3 & 1 & 0 \\
          -4 & -1 & 0 \\
          4 & -8 & -2
        }$

      \item $A =
        \matriz{ccc}{
          a & 1 & 1 \\
          1 & a & 1 \\
          1 & 1 & a
        }$

      \item $A =
        \matriz{cccc}{
          0 & 1 & 0 & 1 \\
          1 & 0 & 1 & 0 \\
          0 & 1 & 0 & 1 \\
          1 & 0 & 1 & 0
        }$

      \item $A =
        \matriz{cccc}{
          0 & 0 & 0 & 0 \\
          1 & 0 & 0 & 0 \\
          0 & 1 & 0 & 0 \\
          0 & 0 & 0 & 1
        }$
    \end{multicols}
  \end{enumerate}
\end{enunciado}

\begin{enumerate}[label=(\alph*)]
  \item \textit{Ecuación característica, a polinomio característico:}
        $$
          (A - \lambda I)v_\lambda = 0
          \sii
          \deter{cc}{
            -\lambda & a        \\
            -a       & -\lambda
          } = 0
          \sii
          (\lambda^2 + a^2) = 0
          \sii
          \llave{lcl}{
            \lambda = -ia  &\quad\text{con}\quad &  v_{\lambda = -ia} = (1,-i)\\
            \lambda = ia  &\quad\text{con}\quad & v_{\lambda = ia} = (1,i)
          }
        $$
        Quedaría algo así diagonalizada:
        $$
          A =
          \ub{
            \matriz{cc}{
              1 & 1 \\
              -i & i
            }
          }{C}
          \ub{
            \matriz{cc}{
              -ia & 0 \\
              0 & ia
            }
          }{D}
          \ub{
            \matriz{cc}{
              \frac{1}{2} & \frac{i}{2} \\
              \frac{1}{2} & -\frac{i}{2}
            }
          }{C^{-1}}
        $$

  \item  \hacer
  \item  \hacer

  \item \textit{Ecuación característica, a polinomio característico:}
        $$
          (A - \lambda I)v_\lambda = 0
          \sii
          \deter{ccc}{
            a -\lambda & 1          & 1          \\
            1          & a -\lambda & 1          \\
            1          & 1          & a -\lambda \\
          } = 0
          \sii
          (a-\lambda)^3 - 3 (a - \lambda) + 2  = 0 \llamada1
        $$
        Que lindo ejercicio \red{\angry}.

        Si hago $x = (a-\lambda)$ entonces $\llamada1$:
        $$
          x^3 - 3 x + 2 = (x - 1)^2(x + 2) = 0
          \Sii{\red{!}}
          \big((a - \lambda) - 1\big)^2\big((a-\lambda) + 2) = 0
        $$
        Por lo tanto:
        $$
          \sii
          \llave{lcl}{
            \lambda_1 = a - 1  &\quad\text{con}\quad &  E_{\lambda = a - 1} = \ket{(-1,1,0), (-1, 0, 1}\\
            \lambda_2 = a + 2  &\quad\text{con}\quad & E_{\lambda = a + 2} = \ket{(1,1,1)}
          }
        $$
        Quedaría algo así diagonalizada:
        $$
          A =
          \ub{
            \matriz{ccc}{
              -1 & -1 & 1 \\
              1 & 0 & 1 \\
              0 & 1 & 1
            }
          }{C}
          \ub{
            \matriz{ccc}{
              a - 1 & 0 & 0 \\
              0 & a - 1 & 0 \\
              0 & 0 & a + 2
            }
          }{D}
          \ub{
            \matriz{ccc}{
              -\frac{1}{3} & \frac{2}{3} & -\frac{1}{3} \\
              -\frac{1}{3} & -\frac{1}{3} & \frac{2}{3} \\
              \frac{1}{3} & \frac{1}{3} & \frac{1}{3}
            }
          }{C^{-1}}
        $$

  \item  \hacer
  \item  \hacer
\end{enumerate}
