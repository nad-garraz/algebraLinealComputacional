\begin{enunciado}{\ejercicio}
  \begin{enumerate}[label=(\alph*)]
    \item Sea $A \en \reales^{3 \times 3}$ diagonalizable con $\traza(A) = -4$. Calcular los autovalores de
          $A$ sabiendo que los autovalores de $A^2 + 2A$ son $-1,\,3$ y $8$.

    \item Sea $A \en \reales^{4 \times 4}$ tal que $\det(A) = 6;\ 1 $ y $-2$ son autovalores de $A$ y $-4$ es
          autovalor de la matriz $A - 3I$. Hallar los restantes autovalores de $A$.
  \end{enumerate}
\end{enunciado}

\begin{enumerate}[label=(\alph*)]
  \item
        Truquini de escribir la cosita y sacar factor común las cositas de los costaditos:
        $$
          A = C D C^{-1}
          \entonces
          \llave{l}{
            A^2 = C D^2 C^{-1} \\
            2A = C 2D C^{-1}
          }
          \entonces
          A^2 + 2A = C D^2 C^{-1} + C 2D C^{-1}
          \igual{\red{!}}
          C
          \ub{
            (D^2 + 2D)
          }{
            \lambda'_i = \magenta{\lambda_i}^2 + 2\magenta{\lambda_i}
          }
          C^{-1}
        $$
        Donde $\lambda'_i$ son los autovalores de $A^2 + 2A$ mientras que los $\magenta{\lambda_i}$ los autovalores de $A$. Por enunciado:
        $$
          \llave{rcl}{
            -1 & = &  \magenta{\lambda_1}^2 + 2\magenta{\lambda_1} \sii \magenta{\lambda_1} = -1\\
            3  & = &  \magenta{\lambda_2}^2 + 2\magenta{\lambda_2} \sii \magenta{\lambda_2} \en \set{-3, 1} \\
            8  & = &  \magenta{\lambda_3}^2 + 2\magenta{\lambda_3} \sii \magenta{\lambda_3} \en \set{-4, 2}
          }
        $$
        Tenemos un millón de \textit{posibles autovalores} para $A$, busquemos la combineta que haga que $\traza(A) = -4$:
        $$
          \cajaResultado{
            \llave{rcc}{
              \magenta{\lambda_1} & = & -1\\
              \magenta{\lambda_2} & = & 1\\
              \magenta{\lambda_3} & = & -4
            }
          }
        $$

  \item Sabemos que determinante de una matriz es igual al producto de sus autovalores:
        $$
          \det(A) = \productoria{i = 1}{n} \lambda_i
        $$
        En este caso:
        $$
          \det(A) = 6 =
          \ua{\lambda_1}{1} \cdot
          \ua{\lambda_2}{-2} \cdot
          \lambda_3 \cdot
          \lambda_4
          \sii
          \lambda_3 \cdot
          \lambda_4
          = -3
        $$
        Luego tenemos por la \textit{definición} de lo que es un autovector:
        $$
          (A - 3I)v = -4v \sii Av = -v
        $$
        Es decir que encontré otro autovalor:
        $$
          \lambda_3 = -1
          \entonces
          \ua{\lambda_3}{-1} \cdot
          \lambda_4
          = -3
          \sii
          \lambda_4
          = 3
        $$
        Los autovalores de $A$:
        $$
          \cajaResultado{
            \llave{rcc}{
              \lambda_1 & = & 1 \\
              \lambda_2 & = & -2 \\
              \lambda_3 & = & -1 \\
              \lambda_4 & = & 3
            }
          }
        $$
\end{enumerate}

\begin{aportes}
  \item \aporte{\dirRepo}{naD GarRaz \github}
\end{aportes}
