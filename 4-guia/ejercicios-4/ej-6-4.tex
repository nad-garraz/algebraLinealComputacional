\begin{enunciado}{\ejercicio}
  Sea $A \in \complejos^{n \times n}$ y $\lambda$ un autovalor de $A$. Probar que:
  \begin{enumerate}[label=(\alph*)]
    \item Si $A$ es triangular, sus autovalores son los elementos de la diagonal.
    \item $\lambda^k$ es autovalor de $A^k$, con el mismo autovector.
    \item $\lambda + \mu$ es autovalor de $A + \mu I$, con el mismo autovector.
    \item Si $p$ es un polinomio, $p(\lambda)$ es autovalor de $p(A)$.
  \end{enumerate}
\end{enunciado}

\begin{enumerate}[label=(\alph*)]
  \item
        Sea $A$ triangular

        \parrafoDestacado{
          \magenta{Arranca el lema}
        }
        Voy a usar y demostrar el lema:
        \parrafoDestacado{
          \textit{Si $A$ es una matriz triangular, entonces su determinante es la multiplicación de sus elementos diagonales}.
        }

        ¡¡A demostrarlo!!

        \textit{Caso base:}
        \parrafoDestacado{
          $p(\blue{2})$ : una matriz $M \en K^{\blue{2} \times \blue{2}}$ triangular, entonces su determinante es la multiplicación de sus elementos
          diagonales
        }

        Sea $M \en K^{2 \times 2}$ triangular inferior {\tiny(la $1 \times 1$ es trivial, no es divertido)}, el caso triangular superior
        es análogo:
        $
          M =
          \matriz{cc}{
            a      & 0 \\
            c_{21} & b
          },
        $
        entonces $\det(M) = a \cdot b - 0 \cdot c_{21} = a \cdot b$ cumpliendo así el caso base.

        \bigskip

        \textit{Paso inductivo}:

        Asumo que
        $$
          p(\blue{h}) : \text{M triangular inferior,} \paratodo M \en K^{\blue{h} \times \blue{h}} \text{ se tiene que }
          \ub{\det(M) = \productoria{i=1}{\blue{h}} m_{ii}}{\text{\purple{hipótesis inductiva}}},
        $$
        es verdadera para algún $\blue{h} \en \naturales$, entonces quiero probar que:
        $$
          p(\blue{h+1}) : \text{M triangular inferior,}
          \paratodo M \en K^{(\blue{h+1}) \times (\blue{h+1})} \text{ se tiene que } \ub{\det(M) = \productoria{i=1}{\blue{h+1}} m_{ii}}{\text{\purple{hipótesis inductiva}}}
        $$
        también sea verdadera.

        Nuevamente voy a hacerlo en el caso en que sea triangular inferior, el caso superior es enteramente análogo.
        $$
          A =
          \matriz{ccccc}{
            a_{11}     & 0          & \cdots & 0     & \green{0}              \\
            a_{21}     & a_{22}     & \cdots & 0     & \green{0}              \\
            \vdots     & \vdots     & \ddots & \vdots& \green{\vdots}         \\
            a_{\blue{h}1} & a_{\blue{h}2} & \cdots & a_{\blue{hh}} & \green{0} \\
            a_{(\blue{h+1})1} & a_{(\blue{h+1})2} & \cdots & a_{(\blue{h+1})(\blue{h})} & \green{a_{(\blue{h+1})(\blue{h+1})}}

          }
        $$
        Calculo el determinate. Lo voy a hacer desarrollando por la \green{última columna}:
        $$
          \det(A) = \green{0} + \green{0} + \green{\cdots} + \green{0} + \green{a_{(\blue{h+1})(\blue{h+1})}}
          \cdot
          \deter{cccc}{
            a_{11}                      & 0                           & \cdots & 0                          \\
            a_{21}                      & a_{22}                      & \cdots & 0                          \\
            \vdots                      & \vdots                      & \ddots & \vdots                     \\
            a_{\blue{h}1} & a_{\blue{h}2} & \cdots & a_{\blue{hh}}
          }
          \igual{\purple{HI}}
          a_{(\blue{h+1})(\blue{h+1})}  \cdot \productoria{i = 1}{\blue{h}} a_{ii} =
          \productoria{i=1}{\blue{h+1}} a_{ii}
        $$
        El lema queda probado. La demo de cuando es triangular superior que la haga Dios, o vos, pero no yo.
        \parrafoDestacado{
          \magenta{Terminó el lema}
        }

        Ahora volviendo con la demostración del ejercicio.
        $$
          (A - \lambda I) =
          \matriz{cccc}{
            a_{11} - \lambda & 0                & \cdots & 0                \\
            a_{21}           & a_{22} - \lambda & \cdots & 0                \\
            \vdots           & \vdots           & \ddots & \vdots           \\
            a_{n1}           & a_{n2}           & \cdots & a_{nn} - \lambda
          }
        $$
        Por lema y recordando que $A$ es triangular por lo que la resta de $A$ con una matriz diagonal seguirá siéndolo:
        $$
          \det(A - \lambda I) = \productoria{i=1}{n}(a_{ii} - \lambda)
        $$

        \textit{¡Ta rahh!}, los $a_{ii}$ son autovalores de $A$ $\paratodo i \leq n$.

  \item\label{ej-6:itemb} Supongo que $\lambda$ es autovalor de $A$.

        Demostracion por inducción:

        \textit{Caso base:}
        $$
          p(\blue{1}) : A^{\blue{1}}v = \lambda^{\blue{1}} v
        $$
        Es verdadera por simple definción de autovalor.

        \textit{Paso inductivo:}
        Asumo como verdadera la proposición:
        $$
          p(\blue{k}) : A^{\blue{k}}v = \lambda^{\blue{k}}v \text{ con el autovector de }  Av = \lambda v
        $$
        para algún $k \en \naturales$, entonces quiero probar que:
        $$
          p(\blue{k+1}) : A^{\blue{k + 1}}v = A^{\blue{k+1}}v
        $$
        también lo sea.
        $$
          A^{\blue{k + 1}}v = A \cdot A^{\blue{k}}v
          \igual{\purple{HI}}
          A \cdot \lambda^{\blue{k}} v
          \igual{\red{!}}
          \lambda^{\blue{k+1}} v
        $$

        Fin

  \item\label{ej-6:itemc}
        Sea $\lambda$ autovalor de $A$ con su autovector correspondiente $v$ . Sea $\mu$ un número.

        Tenemos que por definición:
        $$
          Av = \lambda v
        $$
        Veamos
        $$
          (A + \mu I)v = Av + \mu I v
          \igual{def}
          \lambda v + \mu Iv = \lambda v + \mu v = (\lambda + \mu) v
        $$
        Fin.

  \item
        Sea $p$ un polinomio, $\lambda$ un autovalor con $v$ autovector asociado de $A$

        Demostración por inducción en el grado del polinomio $p_n$. Quiero probar que:
        $$
          \red{p}(n) : p(\lambda) \text{ es autovalor de } p(A)
        $$

        \textit{Caso base:}
        $$
          \red{p}(\gr(p) = 1) : p_1(\lambda) \quad \text{ es autovalor de } \quad p_1(A) = a_1 A + a_0 \ua{A^0}{I_n}
        $$
        Y de lo que vio en el ítem \ref{ej-6:itemc}:
        $$
          p_1(A)\orange{v} = a_1A\orange{v} + a_0I_n \orange{v}
          \sii
          \ub{(a_1A + a_0I_n)}{p(A)} \orange{v} = \ub{(a_1\lambda + a_0)}{p(\lambda)}\orange{v}
        $$
        Por lo cual la proposición $\red{p}(\gr(p) = 1)$ resultó verdadera.

        \textit{Paso inductivo:}

        Asumo como verdadera la proposición:
        $$
          \red{p}(\gr(p) = \blue{k}) :
          \ub{
            p_{\blue{k}}(\lambda)
            \quad \text{ es autovalor de } \quad
            p_{\blue{k}}(A) = \sumatoria{i = 0}{\blue{k}} a_i A^i
          }{\text{\purple{hipótesis inductiva}}}
        $$
        para algún $\blue{k} en \naturales$. Entonces quiero probar que
        $$
          \red{p}(\gr(p) = \blue{k+1}) : p_{\blue{k+1}}(\lambda)
          \quad \text{ es autovalor de } \quad
          p_{\blue{k}}(A) = \sumatoria{i = 0}{\blue{k+1}} a_i A^i
        $$
        Veamos un polinomio de grado $k+1$:
        $$
          p_{\blue{k+1}}(X) =
          \sumatoria{i = 0}{\blue{k + 1}} a_{i} \cdot X^i =
          a_{\blue{k+1}}X^{\blue{k+1}} + \sumatoria{i = 0}{\blue{k}} a_{i} \cdot X^i
        $$
        Evalúo en $A$ y multiplico por $v$ autovector de $A$:
        $$
          p_{\blue{k+1}}(A)v = a_{\blue{k+1}} A^{\blue{k+1}}v + \sumatoria{i = 0}{\blue{k}} a_i \cdot A^i v
          \igual{\purple{HI}}[\ref{ej-6:itemb}]
          a_{\blue{k+1}} \lambda^{\blue{k+1}}v + \sumatoria{i = 0}{\blue{k}} a_i \cdot \lambda^i v
          =
          \ub{\sumatoria{i = 0}{\blue{k+1}} a_i \cdot \lambda^i}{p(\blue{k+1})(\lambda)} v
        $$
        Concluyendo así que
        $$
          p_{\blue{k+1}}(A)v \igual{\red{!!}} p_{\blue{k+1}}(\lambda)v
        $$

        Entonces, probé que es verdadera la proposición.
\end{enumerate}

\begin{aportes}
  \item \aporte{https://github.com/misProyectosPropios}{Iñaki Frutos \github}
\end{aportes}
