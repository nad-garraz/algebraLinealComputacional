\begin{enunciado}{\ejercicio}
Sea $A \in \mathbb{C}^{nxn}$ y $\lambda$ un autovalor de $A$. Probar que:
\begin{itemize}
    \item[(a)] Si $A$ es triangular, sus autovalores son los elementos de la diagonal.
    \item[(b)] $\lambda^k$ es autovalor de $A^k$, con el mismo autovector.
    \item[(c)] $\lambda + \mu$ es autovalor de $A + \mu I$, con el mismo autovector.
    \item[(d)] Si $p$ es un polinomio, $p(\lambda)$ es autovalor de $p(A)$.
\end{itemize}
\end{enunciado}

\begin{itemize}
    \item [(a)]
    Sea $A$ triangular

    Voy a usar un lema: \textit{Si A triangular, entonces su determinante es la multiplicacion de su diagonal}.

    A demostrarlo!!

    Caso base: \\
        Matriz 2x2: (la 1x1 es trivial, no es divertido)
        $
        \begin{bmatrix}
            a & c12 \\
            c21 & b  \\
            \end{bmatrix}
            $
        con c12 = 0 (o excluyente) c21 = 0

        Supongamos que c12 = 0, entonces $det(M) = a * b - 0 * c21 \Rightarrow a*b$ cumpliendo asi el caso base
        El otro caso es analogo
    
    \underline{Paso inductivo}:

        \underline{HI}: $\forall M \in \mathbb{K}^{nxn}: det(M) = \prod_{i=1}^{n}m_{ii}$

        Sup $P(n) \rightarrow P(n-1)$

        Voy a hacerlo en el caso de que sea triangular, en el otro caso queda de la misma forma

        \[
        A = \begin{bmatrix}
            a_{11} & 0 & \cdots & 0 \\
            a_{21} & a_{22} & \cdots & 0  \\
            \vdots & \vdots & \ddots & \vdots \\
            a_{(n+1)1} & a_{(n+1)2} & \cdots & a_{(n+1)(n+1)}
        \end{bmatrix}
            \]

        Calculo el determinate. Lo voy a hacer desde la ultima columna. 

        $$ det(A) = 0 + 0 + \cdots + 0 + a_{(n+1)(n+1)} * det(A[:n][:n]) $$
        Por HI, $det(A[:n][:n]) = \prod_{i=1}^{n} a_{ii}$

        $\Rightarrow det(A) = a_{(n+1)(n+1)} * \prod_{i=1}^{n}a_{ii}$ \\
        $\Rightarrow det(A) = \prod_{i=1}^{n + 1}a_{ii}$

        Que es lo que se queria probar del lema
        PD: el caso dodnde es triangular inferior es lo mismo, solo se hace usando la fila y no la columna

    Ahora volviendo con la demostracion del ejercicio.
   

    \[
        (A - \lambda I) = \begin{bmatrix}
            a_{11} - \lambda & 0 & \cdots & 0 \\
            a_{21} & a_{22} - \lambda & \cdots & 0  \\
            \vdots & \vdots & \ddots & \vdots \\
            a_{n1} & a_{n2} & \cdots & a_{nn} - \lambda
        \end{bmatrix}
            \]

    Por lema, el $det(A - \lambda I) = \prod_{i=1}^{n}a_{ii}$ 
    $\Rightarrow \prod_{i=1}^{n}(a_{ii} - \lambda) $

    Ta rahh, $a_{ii}$ es autovalor de A

    \item[(b)] Sup. $\lambda$ es autovalor de A.

    Demostracion por induccion:
        
    CB (k = 1): \\  
        $Av = \lambda v$ (es por definicion)

    PI: \\

    \underline{HI}: $\lambda^{k}$ autovalor de $A^{k}$ con v el mismo autovalor de $Av = \lambda v$ o lo mismo $A^{k}v = \lambda^{k}v$

    Veamos

    $$A^{k + 1}v = A * A^{k}v \overset{\text{HI}}{=} A \lambda^{k}v = \lambda^{k} Av = \lambda^{k} \lambda^{k} \lambda v = \lambda^{k + 1} v  $$

    Fin

    \item[(c)] Sea $\lambda$ autovalor de $A$ con su autovector correspondiente $v$ . Sea $\mu$ un numero

    Tenemos que: $\Rightarrow Av = \lambda v$ por definicion
    
    Veamos
    
    $(A + \mu I)v = Av + \mu I v \overset{\text{def}}{=} \lambda v + \mu Iv = \lambda v + \mu v = (\lambda + \mu) v $


    \item [(d)]
    Sea $p$ un polinomio, $\lambda$ un autovalor con $v$ autovector asociado de $A$

    Demostracion por induccion en el grado del polinomio:

    \underline{CB} (grado 1):
        $p = a_{1} x + a_{0} x^{0} $

    Por la demostracion anterior, sabemos que es verdadero.

    PI:

    \underline{HI}: {$\forall p / gr(p) = k \land p(A) v = p(\lambda) v$} 

    Ahora quiero ver para $p$ con grado k + 1

    Veamos

    $p = \sum_{i = 0}^{k + 1} \alpha_{i} * x^{i} \Rightarrow \alpha_{k+1} x^{k+1} + \sum_{i = 0}^{k} \alpha_{i} * x^{i} $ \\
    
    Evaluo en $p$ a $A$

    $p(A) = \alpha_{k+1} A^{k+1} + \sum_{i = 0}^{k} \alpha_{i} * A^{i}$

    Y veo si v es un autovector de p(A) utilizado $p(\lambda)$

    Por HI, la 2da parte de la suma es verdadera, veamos la otra parte

    $p(A) v= \alpha_{k+1} A^{k+1}v \overset{\text{Demo 6b) y aAv = a $\lambda$ v}}{=} \alpha_{k+1} \lambda^{k} v$, que es lo mismo que evaluarlo en p a $\lambda$

    Entonces, como ambas partes del polinomio son disjuntos y al juntarlas formo a p con grado k + 1, probé que es verdadero esta afirmación
\end{itemize}