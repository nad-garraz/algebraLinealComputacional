\begin{enunciado}{\ejercicio}
  Probar que $P$ y $Q$ son matrices estocásticas, entonces:
  \begin{enumerate}[label=(\alph*)]
    \item $PQ$ es estocástica.
    \item $P^n$ es estocástica $(n \en \naturales)$.
    \item $P^nQ^m$ es estocástica $(n,\, m \en \naturales)$.
  \end{enumerate}
\end{enunciado}

\begin{enumerate}[label=(\alph*)]
  \item  Los elementos de cada columna de una matriz estocástica suman 1. La idea es probar que al multiplicar dos matrices estocásticas
        se sigue cumpliendo esa propiedad:
        $$
          PQ
          =
          \matriz{c|c|c}{
            &&\\
            P\columna_{\magenta{1}}(Q) & \cdots & P \columna_{\magenta{n}}(Q) \\
            &&\\
          }
        $$
        En particular para el elemento $[PQ]_{\green{k}\magenta{i}}$, el elemento de la fila $\green{k}-$ésima y columna $\magenta{i}-$ésima:
        $$
          \begin{array}{rcl}
            \left[
              P\columna_{\magenta{i}}(Q)
              \right]_{\green{k}}
            = \sumatoria{\blue{j} = 1}{n} p_{\green{k}\blue{j}} q_{\blue{j}\magenta{i}}
             & \Sii{sumando en}[toda la columna $\magenta{i}$] &
            \sumatoria{\green{k}=1}{n}
            \left[
              P\columna_{\magenta{i}}(Q)
              \right]_{\green{k}}=
            \sumatoria{\green{k}=1}{n} \sumatoria{\blue{j} = 1}{n} p_{\green{k}\blue{j}} q_{\blue{j}\magenta{i}} \\
             & \Sii{\red{!}}                                   &
            \sumatoria{\green{k}=1}{n}
            \left[
              P\columna_{\magenta{i}}(Q)
              \right]_{\green{k}}=
            \sumatoria{\blue{j} = 1}{n} q_{\blue{j}\magenta{i}} \cdot
            \Big(
            \sumatoria{\green{k}=1}{n} p_{\green{k}\blue{j}}
            \Big)
            =
            \sumatoria{\blue{j} = 1}{n} q_{\blue{j}\magenta{i}} = 1
            \\
          \end{array}
        $$
        Se uso repetidas veces que las columnas de este tipo de matrices suman 1. Es así que el producto de dos
        matrices de \textit{Markov} sigue siéndo una matriz de \textit{Markov}.

  \item Quiero probar la proposición:
        $$
          p(n) : \text{Si $A$ es una matriz de \textit{Markov} } \entonces  A^n  \text{ también es una matriz de \textit{Markov}}
        $$

        \textit{Caso base}:
        $$
          p(\blue{2}) : \text{Si $A$ es una matriz de \textit{Markov} } \entonces  A^{\blue{2}}  \text{ también es una matriz de \textit{Markov}}
        $$
        Lo cual es verdadero por el ejercicio anterior, donde se vio que si $A$ y $B$ son estocásticas, entonces $A\cdot B$ también lo es.

        \textit{Paso inductivo}

        Asumo para un $\blue{k} \en \naturales_{>=2}$ que:
        $$
          p(\blue{k}) :
          \ub{
            \text{Si $A$ es una matriz de \textit{Markov} } \entonces  A^{\blue{k}}  \text{ también es una matriz de \textit{Markov}}
          }{
            \text{\purple{hipótesis inductiva}}
          }
        $$
        es verdadera, entonces ahora quiero probar que:
        $$
          p(\blue{k + 1}) : \text{Si $A$ es una matriz de \textit{Markov} } \entonces  A^{\blue{k+1}}  \text{ también es una matriz de \textit{Markov}}
        $$
        también sea verdadera.

        Es inmediato:
        $$
          A^{\blue{k+1}} = \ub{A}{\text{es de}\\\textit{Markov}} \cdot \ob{A^{\blue{k}}}{\text{es de}\\\textit{Markov}}
          \Entonces{el producto también}[es de \textit{Markov}] p(\blue{k+1}) \text{ es verdadera.}
        $$

        Por principio de inducción y dado que $p(2), \, p(\blue{k}) \ytext p(\blue{k+1})$ son verdaderas, también lo es $p(n) \ \paratodo n \en \naturales_{\geq 2}$

  \item Es lo mismo que el anterior. Quizás acá se amerita mencionar la siguiente definición:

        \parrafoDestacado{
          \textit{Vector de probabilidad $\bm{v}$}:

          $\bm{v}$ es un vector de probabilidad si cumple que:

          $v_i \geq 0$ y $\sumatoria{i = 1}{n} v_i = 1$
        }
        Con esa definción se pueden demostrar todos estos ejercicios probando que una matriz de \textit{Markov}
        \ul{mantiene las proporciones}. Las columnas de una matriz de \textit{Markov} son vectores de probabilidad.
        Esto quiere decir que para un vector $\bm{w}$ y una matriz de \textit{Markov} $A$:
        $$
          w = \matriz{c}{
            w_i\\
            \vdots\\
            w_n
          }
          \ytext
          A
          \matriz{c}{
            w_i\\
            \vdots\\
            w_n
          }
          =
          \matriz{c}{
            y_i\\
            \vdots\\
            y_n
          }
          \Entonces{se conserva la}[\textit{'normalización'}]
          \sumatoria{i = 1}{n} w_i = \sumatoria{i = 1}{n} y_i
        $$

        \begin{enumerate}[label=(\alph*)] 
         \item \hacer 
         \item \hacer 
         \item \hacer 
         \item \hacer 
         \end{enumerate}

\end{enumerate}

\begin{aportes}
  \item \aporte{\dirRepo}{naD GarRaz \github}
\end{aportes}
