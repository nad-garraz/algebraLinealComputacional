\begin{enunciado}{\ejercicio}
  Sea $A \en \reales^{n \times n}$ tal que admite una base de autovectores $\mathcal{B} = \set{v_1, \ldots, v_n}$
  (que supondremos normalizados) y, además, tiene un único autovalor de máximo módulo (digamos: $\lambda_1$).
  Es decir, sus autovalores satisfacen:
  $$
    |\lambda_1| >
    |\lambda_2| >
    |\lambda_3| >
    \dots >
    |\lambda_n|.
  $$
  Dado $v^{(0)}$ un vector cualquiera tal que sus coordenadas en base $\mathcal{B}$ son $(a_1, \ldots, a_n),$ con $a_1 \distinto 0$.

  Definimos $v^{(k+1)} = Av^{(k)} = A^k v^{(0)}$.

  \begin{enumerate}[label=\alph*)]
    \item Probar que $A v^{(k)} = a_1 \lambda_1^k v_1 + \cdots + a_n \lambda_n^k v_n$.

    \item Deducir que $Av^{(k)} = \lambda_1^k(a_1v_1 + \varepsilon_k)$, donde $\varepsilon_k \to 0$ cuando $k \to \infinito$.

    \item Sea $\varphi : \complejos^n \to \complejos$ una funcional lineal tal que $\varphi(v_1) \distinto 0$. Probar que:
          $$
            \frac{\varphi(Av^{(k)})}{\varphi(v^{(k)})} \to \lambda_1.
          $$

    \item Para evitar que $\norma{v^{(k)}}$ tianda a $0$ o a $\infinito$ es usual normalizar $v^{(k)}$ al cabo de cada iteración.
          Probar que en tal caso, si $\lambda_1$ es real positivo, se tiene que $v^{(k)} \to v_1$.
  \end{enumerate}
\end{enunciado}

\begin{enumerate}[label=\alph*)]
  \item \hacer
  \item \hacer
  \item \hacer
  \item \hacer
\end{enumerate}
