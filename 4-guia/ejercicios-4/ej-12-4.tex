\begin{enunciado}{\ejercicio}
  Una matriz $P = (p_{ij})_{1\leq i, j \leq n}$ se dice estocástica (o de Markov) si sus elementos
  son todos no negativos y sus columnas suman uno. Los elementos $p_{ij}$ representan la proporción
  de individuos que pasan del estado $j$ al estado $i$ en cada iteración (también pueden interpretarse)
  como la probabilidad de pasar $j$ a $i$).

  \begin{enumerate}[label=(\alph*)]
    \item Probar que si $\lambda$ es autovalor de $P$, entonces $|\lambda| \leq 1$.
    \item Sea $\bm{1}$ el vector con todas sus coordenadas iguales a 1. Mostrar que $\bm{1}^t P = \bm{1}$.
          De hecho: $P$ es estocástica si y solo si sus elementos son no negativos y $\bm{1}^t P = \bm{1}$
    \item Probar que toda matriz estocástica tiene a 1 por autovalor.
  \end{enumerate}
\end{enunciado}

\begin{enumerate}[label=(\alph*)]
  \item
        La matriz $P$ es estocástica, sus columnas suman 1. Si $\lambda$ es autovalor de $P$:
        $$
          \begin{array}{rcl}
            P\bm{v} = \lambda \bm{v}
             & \sii                                            &
            \fila_{\magenta{i}}(P) \cdot \bm{v} = (\lambda \bm{v})_{\magenta{i}}                        \\
             & \sii                                            &
            \sumatoria{\blue{j} = 1}{n} p_{\magenta{i}\blue{j}} v_{\blue{j}} =  \lambda v_{\magenta{i}} \\
             & \sii                                            &
            \big|
            \sumatoria{\blue{j} = 1}{n} p_{\magenta{i}\blue{j}} v_{\blue{j}}
            \big|
            =  |\lambda| |v_{\magenta{i}}|                                                              \\
             & \Sii{sumo todas las}[coordenadas $\magenta{i}$] &
            \sumatoria{\magenta{i} = 1}{n} \sumatoria{\blue{j} = 1}{n} |p_{\magenta{i}\blue{j}} v_{\blue{j}}|
            = |\lambda|  \cdot \sumatoria{\magenta{i} = 1}{n} |v_{\magenta{i}}|                         \\
             & \Sii{desigualdad}[triangular]                   &
            \sumatoria{\magenta{i} = 1}{n} \sumatoria{\blue{j} = 1}{n} |p_{\magenta{i}\blue{j}}| |v_{\blue{j}}|
            \geq
            \sumatoria{\magenta{i} = 1}{n} \sumatoria{\blue{j} = 1}{n} |p_{\magenta{i}\blue{j}} v_{\blue{j}}|
            = |\lambda|  \cdot \sumatoria{\magenta{i} = 1}{n} |v_{\magenta{i}}|                         \\
             & \sii                                            &
            \sumatoria{\magenta{i} = 1}{n} \sumatoria{\blue{j} = 1}{n} |p_{\magenta{i}\blue{j}}| |v_{\blue{j}}|
            \geq
            |\lambda|  \cdot \sumatoria{\magenta{i} = 1}{n} |v_{\magenta{i}}|                           \\
             & \Sii{\red{!}}                                   &
            \Big(\sumatoria{\blue{j} = 1}{n} |v_{\blue{j}}|\Big) \Big(\sumatoria{\magenta{i} = 1}{n} |p_{\magenta{i}\blue{j}}|\Big)
            \geq
            |\lambda|  \cdot \sumatoria{\magenta{i} = 1}{n} |v_{\magenta{i}}|                           \\
             & \Sii{\red{!!}}                                  &
            \norma{\bm{v}}_1 \geq |\lambda|  \cdot  \norma{\bm{v}}_1                                    \\
             & \sii                                            &
            \cajaResultado{
              1 \geq |\lambda|
            }
          \end{array}
        $$

  \item
        {\small
        $$
          \bm{1}^t
          P =
          \matriz{c|c|c}{
            &&\\
            \bm{1}^t \cdot \columna_{\blue{1}}(P) & \cdots & \bm{1}^t \cdot  \columna_{\blue{n}}(P) \\
            &&
          }
          =
          \matriz{ccc}{
            &&\\
            \sumatoria{\magenta{i} = 1}{n} (\columna_{\blue{1}}(P))_{\magenta{i}}
            & \cdots &
            \sumatoria{\magenta{i} = 1}{n} (\columna_{\blue{n}}(P))_{\magenta{i}} \\
            &&
          }
          =
          \matriz{ccc}{
            1 & \cdots & 1\\
          } = \bm{1}^t
        $$
        }
        Dado que $P$ es estocástica y sus columnas suman 1.

  \item
        La matriz $P$ es estocástica:
        $$
          P =
          \matriz{c|c|c}{
            &&\\
            \columna_{\blue{1}}(P) & \cdots & \columna_{\blue{n}}(P) \\
            &&
          }
          \qquad
          \text{con}
          \qquad
          \sumatoria{\magenta{i} = 1}{n} (\columna_{\blue{j}}(P))_{\magenta{i}} = 1 \quad \paratodo \blue{j} \en [1,n]
        $$
        Ahora si calculo la matriz traspuesta de $P$ el producto por un vector $v$ queda como el producto escalar de las columnas
        de $P$ por el vector:
        $$
          \begin{array}{c}
            P^t =
            \matriz{ccc}{
            \quad & (\columna_{\blue{1}}(P))^t & \quad                                          \\ \hline
            \quad & \vdots                     & \quad                                          \\ \hline
            \quad & (\columna_{\blue{n}}(P))^t & \quad
            }
            \flecha{en}[particular]                                                             \\
            P^t \cdot
            \matriz{c}{
            \lambda                                                                             \\
            \vdots                                                                              \\
              \lambda
            }
            \igual{\red{!!}}
            \matriz{c}{
            \lambda \cdot \sumatoria{\magenta{i} = 1}{n} (\columna_{\blue{1}}(P))_{\magenta{i}} \\
            \vdots                                                                              \\
              \lambda \cdot \sumatoria{\magenta{i} = 1}{n} (\columna_{\blue{n}}(P))_{\magenta{i}}
            }
            =
            \matriz{c}{
            \lambda                                                                             \\
            \vdots                                                                              \\
              \lambda
            }
            \entonces P^tv = v \sii 1 \text{ es autovalor de } P^t
          \end{array}
        $$
        Y dado que si:
        $$
          A = CDC^{-1}
          \flecha{trasponer}
          A^t = (CDC^{-1})^t
          =
          (C^{-1})^tD^tC^t
          \igual{\red{!}}
          (C^{-1})^tDC^t
          \Sii{\red{!}}[$M = (C^{-1})^t$]
          A^t = M D M^{-1}
        $$
        los autovalores son comunes a $P$ y $P^t$, se obtiene que si $P^t$ tiene autovalor 1, entonces también lo tiene $P$.
\end{enumerate}

\begin{aportes}
  \item \aporte{\dirRepo}{naD GarRaz \github}
\end{aportes}
