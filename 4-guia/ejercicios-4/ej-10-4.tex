\begin{enunciado}{\ejercicio}
  Sea $f: \reales^3 \to \reales^3$ la transformación lineal dada por:
  $$
    [f] =
    \matriz{ccc}{
      -3 & 2 & 0\\
      -6 & 4 & 0\\
      -9 & 6 & 0
    }
  $$
  Probar que $f$ es un proyector y hallar una base $B$ tal que $[f]_{BB}$ sea diagonal.
\end{enunciado}

Por inspección, sino calculalos, ese proyector tiene:
$$
  \imagen(P) = \set{(1,2,3)}
  \ ,\quad
  \nucleo(P) = \set{(2,3,0), (0,0,1)}
  \ytext
  \nucleo(P) \inter \imagen(P) = \set{0}
$$
Se ve que $\underline{Pv = v} \paratodo v \en \imagen(P)$, y ya esa ecuación que escribí te dice que:
$$
  E_{\lambda = 1} = \set{v} = \set{(1,2,3)} = \imagen(P)
$$
Similar sucede con los elementos del núcleo:
$$
  E_{\lambda = 0} = \set{(2,3,0), (0,0,1)} = \nucleo(P)
$$
En forma diagonal para \underline{una} base $B = \set{(1,2,3),(2,3,0), (0,0,1)}$:
$$
  P = C D C^{-1} =
  \matriz{ccc}{
    1 & 2 & 0                                        \\
    2 & 3 & 0                                        \\
    3 & 0 & 1
  }
  \matriz{ccc}{
    1  & 0  & 0 \\
    0  & 0  & 0 \\
    0  & 0  & 0
  }
  \matriz{ccc}{
    1 & 2 & 0                                        \\
    2 & 3 & 0                                        \\
    3 & 0 & 1
  }^{-1}
  =
  \matriz{ccc}{
    1 & 2 & 0                                        \\
    2 & 3 & 0                                        \\
    3 & 0 & 1
  }
  \matriz{ccc}{
    1  & 0  & 0 \\
    0  & 0  & 0 \\
    0  & 0  & 0
  }
  \matriz{ccc}{
    -3 & 2 & 0                                        \\
    2 & -1 & 0                                        \\
    9 & -6 & 1
  }
$$

\begin{aportes}
  \item \aporte{\dirRepo}{naD GarRaz \github}
\end{aportes}
