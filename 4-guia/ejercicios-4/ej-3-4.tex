\begin{enunciado}{\ejercicio}
  Considerar la sucesión de Fibonacci, dada por la recursión:
  $$
    \llave{l}{
      F_0 = 0,\\
      F_1 = 1,\\
      F_{n+1} = F_n + F_{n-1}
    }
  $$
  \begin{enumerate}[label=(\alph*)]
    \item Hallar una matriz $A$ tal que
          $
            \matriz{c}{
              F_n\\
              F_{n+1}
            }
            = A
            \matriz{c}{
              F_{n-1}\\
              F_n
            }$.
          Mostrar que
          $\matriz{c}{
              F_n\\
              F_{n+1}
            }
            = A^n
            \matriz{c}{
              F_0\\
              F_1
            }
          $

    \item Diagonalizar $A$.

    \item Dar una fórmula cerrada para $F_n$.
  \end{enumerate}
\end{enunciado}

\begin{enumerate}[label=(\alph*)]
  \item
        Quiero una matriz $A \en \reales^{2 \times 2}$ tal que:
        $$
          \matriz{c}{
            F_n\\
            F_{n+1}
          }
          =
          A
          \matriz{c}{
            F_{n-1}\\
            F_n
          }
          \sii
          \matriz{c}{
            F_n\\
            F_{n+1}
          }
          =
          \matriz{cc}{
            a & b\\
            c & d
          }
          \matriz{c}{
            F_{n-1}\\
            F_n
          }
          \sii
          \llave{l}{
            aF_{n+1} + b F_n = F_n \\
            cF_{n-1} + d F_n = F_{n+1} \igual{\red{!}} F_n + F_{n-1}
          }
        $$
        Resolviendo ese sistemita:
        $$
          A =
          \matriz{cc}{
            0 & 1\\
            1 & 1
          }
        $$
        Para mostrar lo que sigue, inducción. Quiero mostrar la siguiente proposición:
        $$
          p(n) : A^n
          \matriz{c}{
            F_0\\
            F_1
          } =\matriz{c}{
            F_n\\
            F_{n+1}
          }
          \quad
          \text{con}
          \quad
          A =
          \matriz{cc}{
            0 & 1\\
            1 & 1
          }
        $$
        \textit{Caso base:}
        $$
          p(\blue{1}) : A^{\blue{1}}
          \matriz{c}{
            F_0\\
            F_1
          } =
          \matriz{cc}{
            0 & 1\\
            1 & 1
          }
          \matriz{c}{
            F_0\\
            F_1
          } =
          \matriz{c}{
            0 + F_1\\
            F_0 + F_1
          }
          \igual{def}
          \matriz{c}{
            F_1\\
            F_2
          }
        $$
        Es así que la proposición $p(\blue{1})$ resultó verdadera.

        \textit{Paso inductivo:}

        Asumo que para algún $\blue{k} \en \naturales$ la proposición:
        $$
          p(\blue{k}) :
          \ub{
            A^{\blue{k}}
            \matriz{c}{
              F_0\\
              F_1
            } =
            \matriz{c}{
              F_{\blue{k}}\\
              F_{\blue{k}+1}
            }
          }{\text{\purple{hipótesis inductiva}}}
        $$
        es verdadera. Entonces quiero ver ahora que la proposición:
        $$
          p(\blue{k+1}) : A^{\blue{k+1}}
          \matriz{c}{
            F_0\\
            F_1
          } =
          \matriz{c}{
            F_{\blue{k+1}}\\
            F_{\blue{k+1}+1}
          }
          =
          \matriz{c}{
            F_{\blue{k+1}}\\
            F_{\blue{k+2}}
          }
        $$
        también lo sea.
        $$
          A^{\blue{k+1}}
          \matriz{c}{
            F_0\\
            F_1
          }
          =
          A
          \cdot
          A^{\blue{k}}
          \matriz{c}{
            F_0\\
            F_1
          }
          \igual{\purple{HI}}
          A
          \cdot
          \matriz{c}{
            F_{\blue{k}}\\
            F_{\blue{k}+1}
          }
          =
          \matriz{c}{
            F_{\blue{k+1}}\\
            F_{\blue{k}} + F_{\blue{k}+1}
          }
          \igual{def}
          \matriz{c}{
            F_{\blue{k+1}}\\
            F_{\blue{k+2}}
          }
        $$
        Tuqui, también resulta ser verdadera.

        Es así que $p(1), p(k) \ytext p(k+1)$ resultaron verdaderas y por el principio de inducción
        la proposición $p(n)$ también lo será $\paratodo n \en \naturales$.

  \item \textit{Ecuación característica a polinomio característico:}
        $$
          A =
          \matriz{cc}{
            0 & 1\\
            1 & 1
          }
          \flecha{ecuación}[característica]
          (A - \lambda I)v_\lambda = 0
          \flecha{polinomio}[característico]
          \deter{ccc}{
            -\lambda & 1           \\
            1        & 1 - \lambda
          }
          = \lambda^2 - \lambda - 1 = 0
          \sii
          \llave{l}{
            \lambda_1 = \frac{1 + \sqrt{5}}{2} = \varphi\\
            \lambda_2 = \frac{1 - \sqrt{5}}{2} = -\frac{1}{\varphi}
          }
        $$
        Esa notación se complementa con:
        $$
          \llave{l}{
            \frac{1}{\varphi} = \varphi - 1
          }
        $$
        Diagonalizar esta matriz tiene un montón de \textit{droga}:
        $$
          \matriz{cc}{
            0 & 1\\
            1 & 1
          }
          \matriz{c}{
            1 \\
            \varphi
          }
          \igual{\red{!}}
          \varphi
          \matriz{c}{
            1 \\
            \varphi
          }
          \quad
          \ytext
          \quad
          \matriz{cc}{
            0 & 1\\
            1 & 1
          }
          \matriz{c}{
            1 \\
            -\frac{1}{\varphi}
          }
          \igual{\red{!}}
          -\frac{1}{\varphi}
          \matriz{c}{
            1 \\
            -\frac{1}{\varphi}
          }
        $$
        \parrafoDestacado{
          No sé si están bien las cuentas, pero, a veces es mejor ni preguntar. \textit{Beware} \red{\atencion}.
        }

        $$
          A =
          \matriz{cc}{
            1  & 1\\
            \varphi & -\frac{1}{\varphi}
          }
          \matriz{cc}{
            \varphi & 0 \\
            0 & -\frac{1}{\varphi}
          }
          \matriz{cc}{
            \frac{1}{1+\varphi^2}  & \frac{\varphi}{1+\varphi^2}\\
            \frac{\varphi^2}{1+\varphi^2} & -\frac{\varphi}{1+\varphi^2}
          }
        $$

  \item  Voy a agarrar la primera coordenada de este \faIcon{ghost}:
        $$
          \matriz{cc}{
            1  & 1\\
            \varphi & -\frac{1}{\varphi}
          }
          \matriz{cc}{
            \varphi^{\red{n}} & 0 \\
            0 & (-\frac{1}{\varphi})^{\red{n}}
          }
          \matriz{cc}{
            \frac{1}{1+\varphi^2}  & \frac{\varphi}{1+\varphi^2}\\
            \frac{\varphi^2}{1+\varphi^2} & -\frac{\varphi}{1+\varphi^2}
          }
          \matriz{c}{
            F_0\\
            F_1
          }
        $$
        Entonces la fórmula cerrada:
        $$
          F_{\red{n}} =
          \frac{1}{1 + \varphi^2}
          \parentesis{
            (\varphi^{\red{n}} + (\frac{-1}{\varphi})^{\red{n}} \varphi^2)F_0 + (\varphi^{\red{n}+1}) - \frac{-1}{\varphi})^{\red{n}} \varphi)F_1
          },
        $$
        ponele.
\end{enumerate}

\begin{aportes}
  \item \aporte{\dirRepo}{naD GarRaz \github}
\end{aportes}
