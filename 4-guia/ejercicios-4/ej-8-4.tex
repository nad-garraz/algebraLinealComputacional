\begin{enunciado}{\ejercicio}
  Sea $A \en \reales^{n \times n}$. Probar:
  \begin{enumerate}[label=(\alph*)]
    \item Si los autovalores de $A$ son todos reales, sus autovectores pueden tomarse con coordenadas reales.

    \item Si $A$ es simétrica, entonces sus autovalores son reales.

    \item Si $A$ es simétrica y definida positiva (negativa), entonces todos sus autovalores son positivos (negativos)

    \item Si $A$ es simétrica y $\lambda_1$ y $\lambda_2$ son autovalores distintos, entonces sus correspondientes autovectores
          son ortogonales entre sí.
  \end{enumerate}
\end{enunciado}

\begin{enumerate}[label=(\alph*)]
  \item
        $$
          A v_i = \lambda_i v_i
          \ytext
          \conj{A v_i} = \conj{\lambda_i v_i}
          \Sii{\red{!}}
          A \conj{v}_i = \lambda_i \conj{v}_i
        $$
        Ahora la papa está en usar que $(\text{\poo} + \conj{\text{\poo}}) \en \reales$:
        $$
          \begin{array}{rcl}
            Av_i + A \conj{v}_i = \lambda_i v_i + \lambda_i \conj{v}_i
             & \sii &
            A(v_i + \conj{v}_i) = \lambda_i (v_i + \conj{v}_i)                                      \\
             & \sii &
            A(\ub{2\re(v_i)}{=w_i \en \reales^n}) = \lambda_i (\ub{2\re(v_i)}{= w_i \en \reales^n}) \\
             & \sii &
            Aw_i = \lambda_i w_i
          \end{array}
        $$
        Queda por lo tanto que si $A \en \reales^{n \times n}$ con un autovector $\lambda \en \reales$
        entonces su autovector asociado tendrá coordenadas reales.

  \item $A$ es simétrica:
        $$
          v^* A v = \lambda v^*v \igual{$\llamada1$} \lambda \norma{v}_2^2 \en \reales
        $$
        Ahora la idea es conjugar esa expresión y ver que da lo mismo:
        $$
          (v^* A v)^* = (v^* \lambda v)^*
          \Entonces{fuaa el loco vivía las implicaciones al 1000\%}
          [$
              \begin{array}{rcl}
                (v^* A v)^*  =  v^* (Av^*)^* = v^* A^* v & \igual{\red{!}} & v^* A v \llamada1                                  \\
                                                         & =               & (v^* \lambda v)^* = \conj{\lambda (\norma{v}_2^2)} \\
                                                         & =               & (v^* \lambda v)^* = \conj{\lambda} \norma{v}_2^2
              \end{array}
            $]
          v^* A v \igual{$\llamada2$} \conj{\lambda} \norma{v}_2^2
        $$
        De ahí sale que $\llamada1$ y $\llamada2$ tienen que ser iguales, si bien en la expresión de $\llamada2$ el autovalor está conjugado. Por lo tanto
        para que se cumpla la igualdad tengo que tener:
        $$
          \lambda = \conj{\lambda}
          \sisolosi
          \lambda \en \reales
        $$

  \item \hacer

  \item \hacer
\end{enumerate}

