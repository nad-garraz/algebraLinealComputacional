\begin{enunciado}{\ejercicio}
  Sea $A \en \reales^{n \times n}$. Probar:
  \begin{enumerate}[label=(\alph*)]
    \item Si los autovalores de $A$ son todos reales, sus autovectores pueden tomarse con coordenadas reales.

    \item Si $A$ es simétrica, entonces sus autovalores son reales.

    \item Si $A$ es simétrica y definida positiva (negativa), entonces todos sus autovalores son positivos (negativos)

    \item Si $A$ es simétrica y $\lambda_1$ y $\lambda_2$ son autovalores distintos, entonces sus correspondientes autovectores
          son ortogonales entre sí.
  \end{enumerate}
\end{enunciado}

\begin{enumerate}[label=(\alph*)]
  \item Si $A \en \reales^{n \times n}$ y sus autovalores $\lambda_i \en \reales$, eso se calculó con un polinomio característico:
        $$
          \mathcal{X}(A) = \det(A - \lambda_i)
        $$
        Los autovalores resultaron reales y ahora tengo que calcular los autovectores asociados resolviendo el sistema:
        $$
          (A - \lambda I_n)v = 0 .
        $$
        Donde no hay razón \textit{necesaria} para que la solución de ese sistema tenga elementos con parte imaginaria no nula. Los
        $v$ serán vectores con todas sus coordenadas reales.
          {\footnotesize
            \parrafoDestacado[\orange\atencion]{
              Siempre uno puede complicar la vida y hacer cosas como:
              $$
                \matriz{cc}{
                  1 & 0 \\
                  0 & 0
                }
                \matriz{cc}{
                  0\\
                  i
                }
                =
                \magenta{0}
                \matriz{cc}{
                  0\\
                  i
                }
                =
                \matriz{cc}{
                  0\\
                  0
                }
              $$
              concluyendo que el autovector
              $
                \matriz{cc}{
                  0\\
                  i
                }
              $
              es un autovector $\en \complejos$ asociado al autovalor $\magenta{0}$, pero el \textit{\ul{pueden tomarse del enunciado}} justamente dice que
              para no pegarnos un tiro en el pie, \textit{tomaríamos} el
              $
                \matriz{cc}{
                  0\\
                  1
                }
              $ y listo el \simpleicon{kfc}.
            }
          }
  \item $A$ es simétrica, supogno que $\lambda$ es un autovalor con autovector asociado $v$:
        $$
          \llave{l}{
            v^* A v = \lambda v^*v = \lambda \norma{v}_2^2\\
            v^* A v \igual{\red{!!}} (Av)^* v = \bar{\lambda} v^*v  = \bar{\lambda} \norma{v}_2^2
          }
          \entonces
          \lambda \norma{v}_2^2 = \bar{\lambda} \norma{v}_2^2 \sii \lambda = \bar{\lambda} \sii \lambda \en \reales
        $$

  \item Si a es una matriz SDP, por lo visto antes tiene autovalores reales. Si $\lambda$ es un autovalor de $A$ con autovector asociado $v$:
        $$
          v^t A v \mayor{$(<)$} 0
          \Sii{en particular}[para $Av = \lambda v$]
          \lambda \norma{v}_2^2 \mayor{$(<)$} 0
          \sii
          \lambda \mayor{$(<)$} 0
        $$

  \item Si a es una matriz simétrica, y tengo 2 autovalores $\lambda_1 \ytext \lambda_2$ con autovectores asociado $v_1 \ytext v_2$ respectivamente:
        $$
          \llave{l}{
            v_2^t A v_1 = \lambda_1 v_2^t v_1\\
            v_2^t A v_1 \igual{\red{!!}} (Av_2)^t A v_1 = \lambda_2 v_2^t v_1\\
          }
          \Sii{restando}[M.A.M.]
          0 =  \ub{(\lambda_1 - \lambda_2)}{\distinto 0} v_2^t \cdot v_1
          \entonces
          v_1 \perp v_2
        $$
\end{enumerate}

\begin{aportes}
  \item \aporte{\dirRepo}{naD GarRaz \github}
\end{aportes}
