\begin{enunciado}{\ejercicio}
  Sea $A \en \reales^{n \times n}$. Probar:
  \begin{enumerate}[label=(\alph*)]
    \item Si los autovalores de $A$ son todos reales, sus autovectores pueden tomarse con coordenadas reales.

    \item Si $A$ es simétrica, entonces sus autovalores son reales.

    \item Si $A$ es simétrica y definida positiva (negativa), entonces todos sus autovalores son positivos (negativos)

    \item Si $A$ es simétrica y $\lambda_1$ y $\lambda_2$ son autovalores distintos, entonces sus correspondientes autovectores
          son ortogonales entre sí.
  \end{enumerate}
\end{enunciado}

\begin{enumerate}[label=(\alph*)]
  \item

        $$
          v^*Av = v^* \lambda v =
          \lambda v^* v =
          \lambda\norma{v}_2^2  \en \reales
          \ytext
          \conj{v^*Av}=
          \conj{v^*}A\conj{v} = v^t \lambda \conj{v} =
          \lambda v^t\conj{v} = \lambda \norma{v}_2^2  \en \reales
        $$
        Donde se puede ver que el conjugar a los autovectores da lo mismo que no conjugarlos, es decir que puedo tomarlos como $v \en \reales^n$.

  \item $A$ es simétrica:
        $$
          v^* A v      =  \lambda v^*v \igual{$\llamada1$} \lambda \norma{v}_2^2 \en \reales
        $$
        Ahora la idea es conjugar esa expresión y ver que da lo mismo:
        $$
          (v^* A v)^* = (v^* \lambda v)^*
          \Entonces{fuaa el loco vivía las implicaciones al 1000\%}
          [$
              \begin{array}{rcl}
                (v^* A v)^*  =  v^* (Av^*)^* = v^* A^* v & \igual{\red{!}} & v^* A v \llamada1                               \\
                                                         & =               & (v^* \lambda v)^* = \lambda^* (\norma{v}_2^2)^* \\
                                                         & =               & (v^* \lambda v)^* = \lambda^* \norma{v}_2^2
              \end{array}
            $]
          v^* A v \igual{$\llamada2$} \lambda^* \norma{v}_2^2
        $$
        De ahí sale que $\llamada1$ y $\llamada2$ tienen que ser iguales, si bien en la expresión de $\llamada2$ el autovalor está conjugado. Por lo tanto
        para que se cumpla la igualdad tengo que tener:
        $$
          \lambda = \lambda^*
          \sisolosi
          \lambda \en \reales
        $$

  \item \hacer

  \item \hacer
\end{enumerate}

