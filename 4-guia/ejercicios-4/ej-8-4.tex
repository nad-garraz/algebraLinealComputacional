\begin{enunciado}{\ejercicio}
  Sea $A \en \reales^{n \times n}$. Probar:
  \begin{enumerate}[label=(\alph*)]
    \item Si los autovalores de $A$ son todos reales, sus autovectores pueden tomarse con coordenadas reales.

    \item Si $A$ es simétrica, entonces sus autovalores son reales.

    \item Si $A$ es simétrica y definida positiva (negativa), entonces todos sus autovalores son positivos (negativos)

    \item Si $A$ es simétrica y $\lambda_1$ y $\lambda_2$ son autovalores distintos, entonces sus correspondientes autovectores
          son ortogonales entre sí.
  \end{enumerate}
\end{enunciado}

\begin{enumerate}[label=(\alph*)]
  \item \hacer

  \item $A$ es simétrica:
        $$
          A v = \lambda v
          \Sii{transpongo}[M.A.M]
          v^t A^t = v^t A = \lambda v^t
          \Sii{$\ot \times v$}
          v^t A v = \lambda v^t v
          \sii
          v^t v = \lambda v^t v
          \sii
          \norma{v}_2^2 = \lambda \norma{v}_2^2
        $$

  \item \hacer

  \item \hacer
\end{enumerate}

