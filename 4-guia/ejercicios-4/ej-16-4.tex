\begin{enunciado}{\ejercicio}
  El movimiento anual entre 4 ciudades está refido por el siguiente diagrama de transición:
  $$
    \begin{tikzpicture}[
      node distance=3cm,
      estado/.style={circle, draw, fill=violet!20!white, minimum size=1cm, inner sep=0pt},
      proba/.style={-{Latex[length=2mm]}, thin},
      every node/.style={font={\Large}}
      ]
      \node[estado] (1) {$\bm{1}$};
      \node[estado, right of=1] (2) {$\bm{2}$};
      \node[estado, below of=1] (3) {$\bm{3}$};
      \node[estado, below of=2] (4) {$\bm{4}$};

      \draw[proba] (3) to node[left] {$a$} (1);
      \draw[proba, bend left=20] (2) to node[below] {$\frac{1}{2}$} (1.east);
      \draw[proba, bend left=20] (1) to node[above] {$\frac{1}{2}$} (2.west);
      \draw[proba]  (4.north) to node[left] {$d$}(2.south);
      \draw[proba]  (2.south west) to node[left, below] {$\frac{1}{2}$}(3.north east);

      \draw[proba, bend left=20] (3) to node[above] {$c$} (4.west);
      \draw[proba, bend left=20] (4) to node[below] {$e$} (3.east);

      \draw[proba] (1) to [out=180, in=90, looseness=5]  node[left] {$\frac{1}{2}$} (1.north);
      \draw[proba] (3) to [out=270, in=180, looseness=5]  node[left] {$b$} (3.west);
    \end{tikzpicture}
  $$
  Se sabe que $\bm{v} =
    \matriz{c}{
      0\\
      0\\
      \frac{1}{2}\\
      \frac{1}{2}
    }
  $
  es un estado de equilibrio.
  \begin{enumerate}[label=(\alph*)]
    \item Hallar la matriz de transición $P$.

    \item Determinar la distribución de población después de 10 años, si la distribución inicial es de $\bm{v}_0 = (\frac{1}{2}, 0, \frac{1}{2}, 0)^t$.

    \item ¿Existe un estado límite cualquiera sea el estado inicial? ¿Existe $P^{\infinito}$?

    \item ¿Existe estado límite para $\bm{v_0} = (0,0,\frac{1}{3}, \frac{2}{3})^t$?
  \end{enumerate}
\end{enunciado}

\begin{enumerate}[label=(\alph*)]
  \item Matriz de transición: Tiene en el elemento $p_{ij}$ la probabilidad
        de que se pase del estado $i$ al $j$
        $$
          P =
          \matriz{cccc}{
            \frac{1}{2} & \frac{1}{2}   & a             & 0 \\
            \frac{1}{2} & 0             & 0   & d \\
            0           & \frac{1}{2}   & b             & e \\
            0           &   0           & c             & 0
          }
        $$
        Para que eso sea una matriz de \textit{Markov} necesito que las columnas sumen cada una 1:
        $$
          \llamada1
          \llave{rcl}{
            a + b + c & = & 1 \\
            d + e & = & 1
          }
        $$
        Multiplicar por el vector de equilibrio $\bm{v}$ el cual cumple que $P \bm{v} = \bm{v}$ me va a dar más data para
        determinar los valores de las constantes:
        $$
          P\bm{v} =
          \matriz{cccc}{
            \frac{1}{2} & \frac{1}{2}   & a             & 0 \\
            \frac{1}{2} & 0             & 0   & d \\
            0           & \frac{1}{2}   & b             & e \\
            0           &   0           & c             & 0
          }
          \matriz{c}{
            0\\
            0\\
            \frac{1}{2}\\
            \frac{1}{2}
          }
          =
          \matriz{c}{
            \frac{a}{2}\\
            \frac{d}{2}\\
            \frac{b + e}{2} \\
            \frac{c}{2}
          }
          \igual{\red{!}}
          \matriz{c}{
            0\\
            0\\
            \frac{1}{2}\\
            \frac{1}{2}
          }
          \sii
          \llave{rcl}{
            a & = & 0 \\
            d & = & 0 \\
            b + e & = & 1 \\
            c & = & 1
          }
        $$
        Juntando $\llamada1$ con este último resultado:
        $$
          \cajaResultado{
            P =
            \matriz{cccc}{
              \frac{1}{2} & \frac{1}{2}   & 0             & 0 \\
              \frac{1}{2} & 0             & 0             & 0 \\
              0           & \frac{1}{2}   & 0             & 1 \\
              0           &   0           & 1             & 0
            }
          }
        $$

  \item Si $\bm{v}_0 = (\frac{1}{2}, 0, \frac{1}{2}, 0)^t$ hay que hacer $P^{10}\bm{v}_0$ o ir de a poco con:
        $$
          v^{(k+1)} = P v^{(k)}
        $$
        ¡Arranco!
        $$
          \bm{v}^{(1)} =
          P \bm{v}^{(0)} =
          \matriz{c}{
            \frac{1}{4}\\
            \frac{1}{4}\\
            0 \\
            \frac{1}{2}
          }
          ,\
          \bm{v}^{(2)} =
          P \bm{v}^{(1)} =
          \matriz{c}{
            \frac{1}{4}\\
            \frac{1}{8}\\
            \frac{5}{8}\\
            0 \\
          },\
          \bm{v}^{(3)} =
          P \bm{v}^{(2)} =
          \matriz{c}{
            \frac{3}{16}\\
            \frac{1}{8}\\
            \frac{1}{16}\\
            \frac{5}{8} \\
          },\
          \bm{v}^{(4)} =
          P \bm{v}^{(3)} =
          \matriz{c}{
            \frac{5}{32}\\
            \frac{3}{32}\\
            \frac{11}{16}\\
            \frac{1}{16}\\
          }
        $$
        No sé que onda, esto pero me rindo.

  \item Calculo autovalores para ver que el único autovalor de módulo 1 sea el 1:
        $$
          \begin{array}{rcl}
            \det(A - I_4\lambda) = 0
                                  & \sisolosi   &
            (-1)
            \deter{ccc}{
            \frac{1}{2} - \lambda & \frac{1}{2} & 0                                                                                   \\
            \frac{1}{2}           & - \lambda   & 0                                                                                   \\
            0                     & 0           & 1
            }
            +
            (-\lambda)
            \deter{ccc}{
            \frac{1}{2} - \lambda & \frac{1}{2} & 0                                                                                   \\
            \frac{1}{2}           & - \lambda   & 0                                                                                   \\
            0                     & \frac{1}{2} & - \lambda
            } = 0                                                                                                                     \\
                                  & \sisolosi   &
            (-1)[(\frac{1}{2} - \lambda) (-\lambda) - \frac{1}{4}] + \lambda^2 [(\frac{1}{2} - \lambda) (-\lambda) - \frac{1}{4}] = 0 \\
                                  & \sisolosi   &
            \frac{1}{4}[4\lambda^2 - 2\lambda - 1] (\lambda^2 - 1)  = 0                                                               \\
                                  & \sisolosi   &
            \llave{rcc}{
            \lambda_1             & =           & 1                                                                                   \\
            \lambda_2             & =           & \red{-1}                                                                            \\
            \lambda_3             & =           & \blue{\frac{1}{4}(1 + \sqrt{5})}                                                    \\
            \lambda_4             & =           & \blue{\frac{1}{4}(1 - \sqrt{5})}
            }
          \end{array}
        $$
        Con los autovalores se puede contestar que \ul{no}.
        Para que haya un estado límite, debe existir el límite:
        $$
          \limite{k}{\infty} P^k\bm{v}^{(0)} = \bm{v}^{(\infinito)}
        $$
        Para que exista el estado límite para un $\bm{v}^{(0)}$ se necesita que para una combinación lineal en la base de autovectores,
        el estado inicial tenga coordenada 0 en el autovector $\bm{v}_{-1}$ asociado al autovalor $\lambda = -1$:
        $$
          \bm{v}^{(0)} = c_1 \bm{v}^{eq} +
          \ua{
            c_2
          }{
            \text{\red{debe}}\\\text{\red{ser 0}}
          }\bm{v}_{_{\red{-1}}} +
          c_3 \bm{v}_{_{\blue{\frac{1}{4}(1 + \sqrt{5})}}} +
          c_4 \bm{v}_{_{\blue{\frac{1}{4}(1 - \sqrt{5})}}}
        $$

        La matriz $P^\infinito$ no existe, porque no existe el límite:
        $$
          \limite{k}{\infinito}
          P^k =
          \limite{k}{\infinito}
          C D^k C^{-1} =
          \limite{k}{\infinito}
          C
          \matriz{cccc}{
            1 & 0 & 0 & 0 \\
            0 & (\red{-1})^k & 0 & 0 \\
            0 & 0 & \blue{\frac{1}{4^k}(1 + \sqrt{5})}^k & 0 \\
            0 & 0 & 0 & \blue{\frac{1}{4^k}(1 - \sqrt{5})}^k
          }
          C^{-1}
        $$
        Y bueh: $\limite{k}{\infinito} (-1)^k \noexiste$

  \item Otra forma de decir esto del límite es que la sucesión de estados,
        $
          \set{
            v^{(0)},
            v^{(1)},
            \ldots,
            v^{(\infinito)},
            \ldots,
            v^{(\infinito)},
            \ldots
          }$ tiene que converger. Para el estado inicial esto no sucede
        $$
          v^{(0)} = (0,0,\frac{1}{3}, \frac{2}{3})^t,\
          v^{(1)} = Pv^{(0)} = (0,0,\frac{2}{3}, \frac{1}{3})^t
          v^{(2)} = Pv^{(1)} = (0,0,\frac{1}{3}, \frac{2}{3})^t,
        $$
        Y así los sucesivos estados van a estar oscilando entre un estado y otro. Más aún:
        $$
          \textstyle
          v^{(0)} =
          \magenta{1} \cdot
          \ub{
          (0, 0, \frac{1}{2}, \frac{1}{2})
          }{
          \bm{v}_{_{1}}
          }
          + \magenta{\frac{1}{3}}
          \ub{
          (0, 0, -\frac{1}{2}, \frac{1}{2})
          }{
          \bm{v}_{_{\red{-1}}}
          } =
          (0,0,\frac{1}{3}, \frac{2}{3})
        $$
        Claramente la coordenada del autovector asociado al $\lambda = \red{-1}$ en la combineta de autovectores no es 0, por lo tanto
        \ul{ese estado inicial no tiene un estado límite}.
\end{enumerate}
