\begin{enunciado}{\ejercicio}
  El movimiento anual entre 4 ciudades está refido por el siguiente diagrama de transición:
  $$
    \begin{tikzpicture}[
      node distance=3cm,
      estado/.style={circle, draw, fill=violet!20!white, minimum size=1cm, inner sep=0pt},
      proba/.style={-{Latex[length=2mm]}, thin},
      every node/.style={font={\Large}}
      ]
      \node[estado] (1) {$\bm{1}$};
      \node[estado, right of=1] (2) {$\bm{2}$};
      \node[estado, below of=1] (3) {$\bm{3}$};
      \node[estado, below of=2] (4) {$\bm{4}$};

      \draw[proba] (3) to node[left] {$a$} (1);
      \draw[proba, bend left=20] (2) to node[below] {$\frac{1}{2}$} (1.east);
      \draw[proba, bend left=20] (1) to node[above] {$\frac{1}{2}$} (2.west);
      \draw[proba]  (4.north) to node[left] {$d$}(2.south);
      \draw[proba]  (2.south west) to node[left, below] {$\frac{1}{2}$}(3.north east);

      \draw[proba, bend left=20] (3) to node[above] {$c$} (4.west);
      \draw[proba, bend left=20] (4) to node[below] {$e$} (3.east);

      \draw[proba] (1) to [out=180, in=90, looseness=5]  node[left] {$\frac{1}{2}$} (1.north);
      \draw[proba] (3) to [out=270, in=180, looseness=5]  node[left] {$b$} (3.west);
    \end{tikzpicture}
  $$
  Se sabe que $\bm{v} =
    \matriz{c}{
      0\\
      0\\
      \frac{1}{2}\\
      \frac{1}{2}
    }$
  es un estado de equilibrio.
  \begin{enumerate}[label=(\alph*)]
    \item Hallar la matriz de transición $P$.

    \item Determinar la distribución de población después de 10 años, si la distribución inicial es de $\bm{v}_0 = (\frac{1}{2}, 0, \frac{1}{2}, 0)^t$.

    \item ¿Existe un estado límite cualquiera sea el estado inicial? ¿Existe $P^{\infinito}$?

    \item ¿Existe estado límite para $\bm{v_0} = (0,0,\frac{1}{3}, \frac{2}{3})^t$?
  \end{enumerate}
\end{enunciado}

\begin{enumerate}[label=(\alph*)]
    \item Matriz de transición: Tiene en el elemento $p_{ij}$ la probabilidad
        de que se pase del estado $i$ al $j$
        $$
        P = 
        \matriz{cccc}{
            \frac{1}{2} & \frac{1}{2} & 0 & 0 \\  
            \frac{1}{2} & 0 & 0 & 0 \\  
            \frac{1}{2} & \frac{1}{2} & 0 & 0 \\  
            \frac{1}{2} & \frac{1}{2} & 0 & 0
        }
        $$
      
  \item \hacer
  \item \hacer
  \item \hacer
\end{enumerate}
