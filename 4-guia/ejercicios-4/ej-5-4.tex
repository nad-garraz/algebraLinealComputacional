\begin{enunciado}{\ejercicio}
  Sea $A \en K^{n \times n}$. Probar que $A$ y $A^t$ tienen los mismos autovalores. Dar un ejemplo en el
  que los autovectores sean distintos.
\end{enunciado}

Demostracion:

Por propiedades del determinante sabemos que:
$$
  \det(A) = \det(A^t)
$$
Sabemos que los autovalores $\lambda$ son los que tienen la siguiente propiedad:
$$
  \det(A - \lambda I) = 0
$$

Usando la propiedad del determinante, tenemos que:
$$
  \det(A - \lambda I) = \det((A - \lambda I)^t)
$$

Y, como sabemos que $\lambda$ es un autovalor de $A$
$$
  0 =
  \ub{
    \det(A - \lambda I)
  }{\mathcal{X}_A(\lambda)} =
  \ub{
    \det((A - \lambda I)^t)
  }{\mathcal{X}_{A^t}(\lambda)}
  \sii
  \mathcal{X}_A(\lambda) = \mathcal{X}_{A^t}(\lambda) = 0
$$

Probando así que tienen los mismos autovalores, dado que los \textit{polinomios característicos de ambas expresiones} son iguales

Si tengo la siguiente matriz:
$$
  \ub{
    A =
    \matriz{cc}{
      0 & 1 \\
      0 & 0
    }
  }{
    E_{\lambda_1 = \lambda_2 = 0} = \set{(1,0)}
  }
  \Entonces{transponiendo}
  \ub{
    A^t =
    \matriz{cc}{
      0 & 0 \\
      1 & 0
    }
  }{
    E_{\lambda_1 = \lambda_2 = 0} = \set{(0,1)}
  }
$$
Esas matrices \underline{no son diagonalizables}. Ambas tienen los mismos autovalores $\lambda_1 = \lambda_2 = 0$,
pero \ul{no generan una base de autovectores} para poder diagonalizar la matriz.

\begin{aportes}
  \item \aporte{https://github.com/misProyectosPropios}{Iñaki Frutos \github}
  \item \aporte{\dirRepo}{naD GarRaz \github}
\end{aportes}
