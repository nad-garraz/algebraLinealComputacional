\begin{enunciado}{\ejercicio}
Sea $A \in \mathbb{K}^{nxn}$. Probar que $A$ y $A^{t}$ tienen los mismos autovalores. Dar un ejemplo en el
 que los autovectores sean distintos.
\end{enunciado}

Demostracion:

Por propiedades del $det$ sabemos que: $det(A) = det(A^{t})$

Sabemos que los autovalores $\lambda$ son los que tienen la siguietne propiedad:
    $$det(A - \lambda I) = 0 $$


Usando la propiedad del determinante, tenemos que $det(A - \lambda I) = det((A - \lambda I)^{t})$

Y, como sabemos que $\lambda$ es un autovalor de A, $0 = det(A - \lambda I) = det((A - \lambda I)^{t})$

Probando así que tienen los mismos autovalores

Falta ejemplo{\hacer}