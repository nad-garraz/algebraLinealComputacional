\begin{enunciado}{\ejercicio}
  Recordando que la solución de la ecuación diferencial
  $$
    x'(t) = ax(t), \quad a \en \reales
  $$
  con condición inicial $x(0) c _0$ es $x(t)  c_0e^{at}$, resolver el siguiente sistema de ecuaciones diferenciales
  $$
    \llave{rcl}{
      x'(t) & = & 6 x(t) + 2 y(t) \\
      y'(t) & = & 2 x(t) + 3 y(t)
    }
  $$
  con condiciones iniciales $x(0) = 3$, $y(0)= -1$.

  \parbox{0.9\textwidth}{
    Sugerencia:
    Hallar una matriz $C$ tal que
    $C^{-1}
      \matriz{cc}{
        6 & 2\\
        2 & 3
      }
      C
    $
    sea diagonal y hacer el cambio de variables
    $
      \matriz{c}{
        u(t)\\
        v(t)
      }
      =
      C^{-1} \cdot
      \matriz{c}{
        x(t)\\
        y(t)
      }
    $.
  }
\end{enunciado}

Enunciado aterrador, pero es un ejercicio para desacoplar las ecuaciones, cosa que no se mezclen la $x$ con las $y$.
Lo primer es escribir la matriz de coeficientes en forma diagonal:
$$
  \llave{rcl}{
    x'(t) & = & 6 x(t) + 2 y(t) \\
    y'(t) & = & 2 x(t) + 3 y(t)
  }
  \Entonces{forma}[matricial]
  \matriz{c}{
    x'(t)\\
    y'(t)
  }
  =
  \ub{
    \matriz{cc}{
      6 & 2\\
      2 & 3
    }
  }{A}
  \matriz{c}{
    x(t)\\
    y(t)
  }
$$
Diagonalizo la matriz:
$$
  \deter{cc}{
    6-\lambda & 2         \\
    2         & 3-\lambda
  }
  = \lambda^2 - 9\lambda + 14 = 0 \sii \lambda \en \set{7,2}
  \entonces
  \matriz{cc}{
    6 & 2\\
    2 & 3
  }
  =
  \matriz{cc}{
    2 & 1  \\
    1 & -2
  }
  \matriz{cc}{
    7 & 0  \\
    0 & 2
  }
  \matriz{cc}{
    \frac{2}{5} & \frac{1}{5}  \\
    \frac{1}{5} & -\frac{2}{5}
  }
$$
El cambio de variables planteado:
$$
  \matriz{c}{
    u(t)\\
    v(t)
  }
  \igual{$\llamada1$}
  C^{-1} \cdot
  \matriz{c}{
    x(t)\\
    y(t)
  }
  \sii
  C\matriz{c}{
    u(t)\\
    v(t)
  }
  =
  \matriz{c}{
    x(t)\\
    y(t)
  }
$$
Multiplico la ecuación diferencial a izquierda por $C^{-1}$:
$$
  \ub{
    C^{-1}
    \matriz{c}{
      x'(t)\\
      y'(t)
    }
  }{
    \matriz{c}{
      u'(t)\\
      v'(t)
    }
  }
  =
  C^{-1}
  \matriz{cc}{
    6 & 2\\
    2 & 3
  }
  \ub{
    \matriz{c}{
      x(t)\\
      y(t)
    }
  }{
    C
    \matriz{c}{
      u(t)\\
      v(t)
    }
  }
  \sii
  \matriz{c}{
    u'(t)\\
    v'(t)
  }
  =
  \ub{
    C^{-1}A C
  }{
    \matriz{cc}{
      7 & 0  \\
      0 & 2
    }
  }
  \matriz{c}{
    u(t)\\
    v(t)
  }
$$
Ahora el sistema queda desacoplado, \textit{no hay mezcla} de las cosas de $u$ con las cosas de $v$ y se
puede resolver como dos ecuaciones diferenciales por separación de variables:
$$
  \matriz{c}{
    u'(t)\\
    v'(t)
  }
  =
  \matriz{cc}{
    7 & 0  \\
    0 & 2
  }
  \matriz{c}{
    u(t)\\
    v(t)
  }
  \sii
  \llave{rcl}{
    u'(t) & = & 7 u(t) \sii u(t) = c_0 e^{7t} \flecha{condiciones}[iniciales $\llamada1$] u(0) = 1 = c_0 e^{7\cdot 0} \sii c_0 = 1\\
    v'(t) & = & 2 v(t) \sii v(t) = c_1 e^{2t} \flecha{condiciones}[iniciales $\llamada1$] v(0) = 1 = c_1 e^{2\cdot 0} \sii c_1 = 1\\
  }
$$
Ahora hay que volver a las variables originales:
$$
  \matriz{c}{
    x(t)\\
    y(t)
  }
  =
  C
  \matriz{c}{
    e^{7t}\\
    e^{2t}
  }
  =
  \matriz{c}{
    2 e^{7t} +  e^{2t}\\
    e^{7t} -2  e^{2t}\\
  }
$$

$$
  \cajaResultado{
    \matriz{c}{
      x(t)\\
      y(t)
    }
    =
    \matriz{c}{
      2 e^{7t} +  e^{2t}\\
      e^{7t} -2  e^{2t}
    }
  }
$$

\begin{aportes}
  \item \aporte{\dirRepo}{naD GarRaz \github}
\end{aportes}
