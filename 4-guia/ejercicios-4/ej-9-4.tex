\begin{enunciado}{\ejercicio}
  Una transformación lineal $f: K^n \to K^n$ se llama \textit{proyector} si verifica $f(f(x)) = f(x)$ para todo $x \en K^n$.
  Probar que los únicos autovalores de un proyector son 1 y 0.
\end{enunciado}

Dejame escribir al proyector como $P$ en vez de $f$, porque me da \textit{cosita} sino.
Tenemos un proyector y por definición:
$$
  P \circ P = P
$$
Si el proyector tiene forma diagonal:
$$
  \begin{array}{c}
    P = C D C^{-1}
    \sii
    P = C
    \matriz{ccc}{
    \lambda_1               & \dots  & 0                                        \\
    0                       & \ddots & 0                                        \\
    0                       & \dots  & \lambda_n
    }
    C^{-1}                                                                      \\
    \\
    P \circ P = C D C^{-1} C D C^{-1} = C D^2 C^{-1} =
    \ub{
      C
      \matriz{ccc}{
    \lambda_1^2             & \dots  & 0                                        \\
    0                       & \ddots & 0                                        \\
    0                       & \dots  & \lambda_n^2
      }
      C^{-1}
    }{
      P \circ P
    }
    =
    \ub{
      C
      \matriz{ccc}{
    \lambda_1               & \dots  & 0                                        \\
    0                       & \ddots & 0                                        \\
    0                       & \dots  & \lambda_n
      }
      C^{-1}
    }{P}                                                                        \\
    \\
    \sii
    \matriz{ccc}{
    \lambda_1^2             & \dots  & 0                                        \\
    0                       & \ddots & 0                                        \\
    0                       & \dots  & \lambda_n^2
    }
    =
    \matriz{ccc}{
    \lambda_1               & \dots  & 0                                        \\
    0                       & \ddots & 0                                        \\
    0                       & \dots  & \lambda_n
    }
    \sii
    \llave{ccc}{
    \lambda_1^2 = \lambda_1 & \sii   & \cajaResultado{ \lambda_1 \en \set{0,1}} \\
    \vdots                  & \vdots & \vdots                                   \\
    \lambda_n^2 = \lambda_n & \sii   & \cajaResultado{ \lambda_n \en \set{0,1}}
    }
  \end{array}
$$

\begin{aportes}
  \item \aporte{\dirRepo}{naD GarRaz \github}
\end{aportes}
