\begin{enumerate}[label=\tiny\purple{\faIcon{snowman}}]
\item\hypertarget{teoria-4:markov}{\textit{Procesos de Markov:}}

Sucesión de vectores $\bm{v}_k$ con $k \en \naturales_0$
$$
  \set{\bm{v}_0, \bm{v}_1, \ldots}
  \quad
  \text{con}
  \quad
  \bm{v}_{k+1} = M \bm{v}_k.
$$
$M$ es una matriz de \textit{Markov} si es una matriz estocástica \underline{por columnas}, es decir:
\begin{itemize}
  \item Todas los elementos $m_{ij}$ de la matriz $M$ son \underline{no negativos}.

  \item Cada columna de $M$ suma 1:
        $$
          \left[\sumatoria{i = 1}{n} m_{i\blue{j}}\right]_{\blue{j}} = 1
          \quad
          \paratodo \blue{j} \en \naturales_{\leq n}
        $$

  \item $M$ tiene por lo menos un autovalor $\lambda = 1$.

  \item Los autovalores de $M$ cumplen que $|\lambda| \leq 1$.
\end{itemize}
\parrafoDestacado[\red{\atencion}]{
  La interpretación para un elemento $a_{\magenta{i}\blue{j}}$, se lee:
  \parrafoDestacado{
    $a_{\magenta{i}\blue{j}}$
    es la probabilidad de que \ul{la componente en el estado $\magenta{i}$}
    pase a estar \ul{al estado $\blue{j}$}.
  }
}

\item \textit{Vector estocástico, de estado o de probabilidad:}

Sea un $\bm{v} = (v_1, \ldots, v_n)$ cumple que sus coordenadas son \underline{no negativas} y suman 1.
Las coordenada $j-$ésima corresponde \ul{la probabilidad de estar en el estado $j-$ésimo o la proporción
  de la población que se encuentra en ese estado}.

\item
\textit{El estado límite $v^{(\infinito)}$} es el estado para el cual existe el límite:
$$
  \limite{k}{\infinito} M^kv^{(0)} =
  \limite{k}{\infinito} v^{(k+1)} =  v^{(\infinito)}
$$
\begin{enumerate}[label=\red{\tiny\faIcon{pray}$_{\arabic*}$}]
  \item La existencia del estado límite, $v^{(\infinito)}$, como se ve en la definición, depende del estado inicial $v^{(0)}$.

  \item Si los autovalores $\lambda_i$ de $A$ cumplen que si
        $|\lambda_i| = 1 \entonces \lambda_i = 1 $
        $\entonces$ podría existir \purple{existe el \textit{estado límite}}.
\end{enumerate}

\item \textit{El estado de equilibrio $v^{(eq)}$}:
$$
  M v^{(eq)} = v^{(eq)},
$$
por lo tanto $v^{(eq)}$ no es otra cosa que el autovector asociado al autovalor 1 de $M$
\begin{enumerate}[label=\blue{\tiny\faIcon{pray}$_{\arabic*}$}]
  \item Si la matriz $A \en \reales^{n \times n}$ \ul{diagonalizable} y tiene un \ul{único autovalor de valor $\lambda = 1$} y
        $\lambda_i \distinto -1$ con autovector asociado $v^{eq}$, entonces existe:
        $$
          \limite{k}{\infinito} A^k = A^{(\infinito)}
          \ytext
          A^{(\infinito)} =
          \matriz{c|c|c}{
            \quad &  & \quad \\
            v^{eq} & \cdots & v^{eq} \\
            &  &
          }.
        $$
        Con ese resultado, para cualquier estado inicial escrito como combineta de la base de autovectores:
        $$
          v^{(0)} = \red{c_1} v^{eq} + \cdots +  \red{c_n}v_n \text { con }  \red{c_1} \distinto 0
        $$.
        Eventualmente $A v^{(0)} \flecha{tiende}[a] v^{eq}$.
        \parrafoDestacado{
          \textit{Si hay estado límite $v^{(\infinito)}$ entonces es $v^{eq}$}.
        }

  \item Si la matriz $A\en \reales^{n \times n}$ \ul{diagonalizable}, tiene \ul{$\lambda = 1$ con multiplicidad $m$} y $\lambda_i \distinto -1$ con
        autoespacio $E_{\lambda = 1} = \set{v_1, \ldots, v_m}$
        entonces el estado $v^{(\infinito)}$ es una combinación lineal de los vectores de $E_{\lambda = 1}$
\end{enumerate}
