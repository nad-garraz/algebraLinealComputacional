\begin{enunciado}{\ejExtra}
  Sean $\alpha > 0$ y $A(\alpha) \en \reales^{4 \times 4}$ un conjunto de matrices que dependen de $\alpha$, dadas por:
  $$
    A(\alpha) =
    \matriz{cccc}{
      1 & 2 & 0 & 1 \\
      \alpha & \alpha^2 + 2\alpha & \alpha & 2 \alpha \\
      1 & 2 & \alpha^2 - 1 & 1 \\
      1 & 2 & 0 & 2 \\
    }
  $$

  \begin{enumerate}[label=\alph*)]
    \item Encontrar los valores de $\alpha$ para los cuales $A(\alpha)$ no es inversible.
    \item Probar que $\condicion_\infinito(A(\alpha)) \flecha{}[$\alpha \to 1$] +\infinito$.
    \item ¿Qué puede decir de $\limite{\alpha}{1} \condicion_2(A(\alpha))$
  \end{enumerate}
\end{enunciado}

\begin{enumerate}[label=\alph*)]
  \item Igualar el determinante a cero:
        $$
          A(\alpha) =
          \deter{cccc}{
            1      & 2                  & 0            & 1        \\
            \alpha & \alpha^2 + 2\alpha & \alpha       & 2 \alpha \\
            1      & 2                  & \alpha^2 - 1 & 1        \\
            1      & 2                  & 0            & 2        \\
          }
          =
          (\alpha - 2)^2 \cdot (\alpha - 1) \cdot (\alpha + 1)
          \igual{\red{!}}
          0
          \sii
          \alpha \en \set{-1, 1, 2}
        $$

  \item Uso el mega resultado.
        $$
          \condicion(A)
          \geq
          \supremo \set{\frac{\norma{A}}{\norma{A - B}} : B \text{ es singular}}.
        $$
        Elijo una matriz singular $B$:
        $$
          B =
          \matriz{cccc}{\rowcolor{red!5}
            1      & 2                  & 0            & 1        \\
            \alpha & \alpha^2 + 2\alpha & \alpha       & 2 \alpha \\\rowcolor{red!5}
            1      & 2                  & 0 & 1        \\
            1      & 2                  & 0            & 2
          }
        $$
        Por lo tanto queda:
        $$
          A - B =
          \matriz{cccc}{\rowcolor{red!5}
            0      & 0                  & 0            & 0        \\
            0      & 0                  & 0            & 0        \\
            0      & 0                  & \alpha^2 - 1            & 0        \\
            0      & 0                  & 0            & 0
          }
        $$

        $$
          \limite{\alpha}{1} \frac{\norma{A}_\infinito}{\norma{A - B}_\infinito} =
          \limite{\alpha}{1} \bigg|\frac{6\alpha + \alpha^2}{\alpha^2 - 1}\bigg| = +\infinito
        $$

  \item
        Uso el resultado:
        $$
          \frac{1}{\sqrt{n}} \norma{A}_\infinito \menor{$\llamada1$} \norma{A}_2 \leq \sqrt{n} \norma{A}_\infinito
        $$
        Dado que $A(\alpha) \en \reales^{4 \times 4}$
        $$
          \begin{array}{rcl}\vspace{3pt}
            \condicion_2(A(\alpha))
             & \igual{def}              &
            \norma{A(\alpha)}_2 \cdot \norma{A(\alpha)^{-1}}_2                                                     \\
             & \mayorIgual{$\llamada1$} &
            \frac{1}{\sqrt{4}}\norma{A(\alpha)}_\infinito \cdot \frac{1}{\sqrt{4}}\norma{A(\alpha)^{-1}}_\infinito \\
             & =                        &
            \frac{1}{4}\norma{A(\alpha)}_\infinito \cdot \norma{A(\alpha)^{-1}}_\infinito                          \\
             & \igual{def}              &
            \frac{1}{4} \condicion_\infinito(A(\alpha))                                                            \\
             & \flecha{$\alpha \to 1$}  & \infinito
          \end{array}
        $$
        Se concluye que:
        $$
          \cajaResultado{
            \condicion_2(A(\alpha))
            \flecha{$\alpha \to \infinito$}
            \infinito
          }
        $$
        La cota superior también tiende a infinito. Se prueba igual pero no es necesario.
\end{enumerate}

\begin{aportes}
  \item \aporte{\dirRepo}{naD GarRaz \github}
\end{aportes}
