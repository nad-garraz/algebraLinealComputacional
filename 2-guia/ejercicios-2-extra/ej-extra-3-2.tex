\begin{enunciado}{\ejExtra}
  \fechaEjercicio{final 20/7/23}
  Sea $\varepsilon > 0$ y
  $A =
    \matriz{cccc}{
      7 & 2 & 5 & 3 \\
      1 & -2 & 3 & 4 \\
      8 & \frac{1}{\varepsilon} & 8 & 7 \\
      0 & 0 & 3 & 1
    }$
  \begin{enumerate}[label=\alph*)]
    \item Probar que $\condicion_\infinito(A)\to \infinito$ cuando $\varepsilon \to \infinito$.
    \item Probar que $\condicion_\infinito(A)\to \infinito$ cuando $\varepsilon \to 0^+$.
    \item Probar que $\condicion_2(A)\to \infinito$ cuando $\varepsilon \to 0^+$ y cuando $\varepsilon \to \infinito$.
  \end{enumerate}
\end{enunciado}

\begin{enumerate}[label=\alph*)]
  \item Uso el resultado:
        $$
          \condicion(A)
          \mayorIgual{$\llamada1$}
          \supremo \set{\frac{\norma{A}}{\norma{A - B}} : B \text{ es singular}}.
        $$
        Notar que sumando la \textit{primera} y \textit{segunda} filas queda casi la tercerca fila, por lo tanto la $B$ singular:
        $$
          B =
          \matriz{cccc}{
            7 & 2 & 5 & 3 \\
            1 & -2 & 3 & 4 \\
            8 & 0 & 8 & 7 \\
            0 & 0 & 3 & 1
          }
        $$
        Usando $\llamada1$ para la $\normaBullet_\infinito$:
        $$
          \condicion_\infinito(A)
          \geq
          \frac{\norma{A}_\infinito}{\norma{A - B}\infinito} =
          \frac{23}{\frac{1}{\varepsilon}} = 23\varepsilon
        $$
        Calculando el límite:
        $$
          \limite{\varepsilon}{\infinito} 23\varepsilon =
          \infinito \leq \condicion_\infinito(A)
          \entonces
          \cajaResultado{
            \condicion_\infinito(A) \flecha{$\varepsilon \to \infinito$} \infinito
          }
        $$

  \item
        Parecido al anterior ahora la $B$ singular candidata:
        $$
          B =
          \matriz{cccc}{
            7 & 2 & 5 & 3 \\
            1 & -2 & 3 & 4 \\
            8 & \frac{1}{\epsilon} & 8 & 7 \\
            0 & 0 & 0 & 0
          }
        $$
        Usando nuevamente $\llamada1$ para la $\normaBullet_\infinito$:
        $$
          \condicion_\infinito(A)
          \geq
          \frac{\norma{A}_\infinito}{\norma{A - B}\infinito} =
          \frac{23 + \frac{1}{\varepsilon}}{4} =
        $$
        Calculando el límite:
        $$
          \limite{\varepsilon}{0^+} \frac{23 + \frac{1}{\varepsilon}}{4} = \infinito
          \leq \condicion_\infinito(A)
          \entonces
          \cajaResultado{
            \condicion_\infinito(A) \flecha{$\varepsilon \to 0^+$} \infinito
          }
        $$

  \item Usamos la útil y misma de siempre, \textit{sanguchinni de cotardi}:
        $$
          \frac{1}{\sqrt{n}} \norma{A}_\infinito \menor{$\llamada2$} \norma{A}_2 \leq \sqrt{n} \norma{A}_\infinito
        $$
        Si queda acotada inferiormente por infinito, listo:
        $$
          \begin{array}{rcl}\vspace{3pt}
            \condicion_2(A)
             & \igual{def}                     &
            \norma{A}_2 \cdot \norma{A^{-1}}_2                                                     \\
             & \mayorIgual{$\llamada2$}[$n=4$] &
            \frac{1}{\sqrt{4}}\norma{A}_\infinito \cdot \frac{1}{\sqrt{4}}\norma{A^{-1}}_\infinito \\
             & =                               &
            \frac{1}{4}\norma{A}_\infinito \cdot \norma{A^{-1}}_\infinito                          \\
             & \igual{def}                     &
            \frac{1}{4} \condicion_\infinito(A)  \flecha{$\varepsilon \to 0^+ ~\lor~ \varepsilon \to \infinito$}[\red{!!!}]  \infinito
          \end{array}
        $$
        Donde lo último es lo calculado en los ítems anteriores \rollingEyes.

        Finiquitando:
        $$
          \cajaResultado{
            \condicion_2(A)
            \flecha{$\varepsilon \to 0^+$}\infinito
            \ytext
            \condicion_2(A)
            \flecha{$\varepsilon \to \infinito$}\infinito
          }
        $$
        La cota superior también tiende a infinito. Se prueba igual pero no es necesario.

\end{enumerate}

\begin{aportes}
  \item \aporte{\dirRepo}{naD GarRaz \github}
\end{aportes}
