\begin{enunciado}{\ejExtra} \fechaEjercicio{final 09/07/24}
  Sea $A_n \en \reales^{n \times n}$ la matriz con coeficientes dados por,
  $$
    a_{ij} =
    \llave{cl}{
      n & \text{ si } i = 1 \text{ o } j = 1 \\
      n/i & \text{ si } i = j \\
      0 & \text{ en otro caso}
    }
  $$
  \begin{enumerate}[label=\alph*)]
    \item Probar que $\condicion_\infinito (A_n) \geq cn^2$ para alguna constante independiente de $n$.
    \item Probar que $\condicion_2(A_n) \to \infinito$ cuando $n \to \infinito$.
  \end{enumerate}
\end{enunciado}

Mirá el ejercicio \refEjercicio{ej:23} para inspiración y bueh, el ejercicio \refEjercicio{ej:25} que es \textit{exactamente igual} a este.

\begin{enumerate}[label=\alph*)]
  \item
        Hay que encontrar una $B$ (antes de verla, mirá el ejercicio \refEjercicio{ej:23} para inspirarte)
        $$
          \condicion_\infinito(A)
          \geq
          \supremo \set{\frac{\norma{A}_\infinito}{\norma{A - B}_\infinito} : B \text{ es singular}}
        $$
        El caso con $n = 2$ se puede calcular a mano:
        $$
          A =
          \matriz{cc}{
            2 & 1 \\
            1 & 1
          }
          \ytext
          A^{-1} =
          \matriz{cc}{
            1 & -1 \\
            -1 & 2
          }
          \entonces
          \condicion_\infinito(A) =
          \norma{A}_\infinito \cdot
          \norma{A^{-1}}_\infinito = 3 \cdot 3 = 9 \geq c \cdot \ua{2}{n}^2
        $$
        Para $n \geq 3$:
        $$
          b_{ij} =
          \llave{cl}{
            n & \text{ si } i = 1 ~ \lor ~  j = 1 \\
            n / i & \text{ si } i = j ~ \land ~ \magenta{i,j < n - 1}\\
            0 & \text{ en otro caso}
          }
        $$
        Las últimas 2 filas son iguales, así que \ul{$B$ es singular} con solo 2 entradas distintas de 0:
        $$
          A - B =
          \llave{cl}{
            \frac{n}{n-1} & \text{ si } i = j = n - 1\\
            1 & \text{ si } i = j = n\\
            0 & \text{ en otro caso}
          }
        $$
        Entonces queda que:
        $$
          \textstyle
          \condicion_\infinito(A) \geq \frac{\norma{A}_\infinito}{\norma{A - B}_\infinito} =
          \frac{n^2}{\frac{n}{n - 1}} =
          \frac{n-1}{n} \cdot n^2 \geq c \cdot n^2
        $$
        Esto último es verdadero, en particular, para cualquier $c < 1$.

  \item Pispeá el ejercicio \refEjercicio{ej:16}, ahí están las acotaciones falopa de la normas.

        Entonces usando que:
        $$
          \frac{1}{\sqrt{n}} \norma{A}_\infinito \menorIgual{$\llamada1$} \norma{A}_2 \leq \sqrt{n} \norma{A}_\infinito
        $$
        sale con fritas \simpleicon{kfc}.
        $$
          \begin{array}{rcl}
            \limite{n}{\infinito} \condicion_2(A)
             & =                        &
            \limite{n}{\infinito} \norma{A}_2 \cdot \norma{A^{-1}}_2                                                     \\
             & \mayorIgual{$\llamada1$} &
            \limite{n}{\infinito} \frac{1}{\sqrt{n}}\norma{A}_\infinito \cdot \frac{1}{\sqrt{n}}\norma{A^{-1}}_\infinito \\
             & =                        &
            \limite{n}{\infinito} \frac{1}{n} \norma{A}_\infinito \cdot \norma{A^{-1}}_\infinito                         \\
             & \igual{def}              &
            \limite{n}{\infinito}
            \frac{1}{n} \condicion_\infinito(A)                                                                          \\
             & \mayorIgual{\red{!}}     &
            \limite{n}{\infinito}
            \frac{1}{n} c \cdot n^2                                                                                      \\
             & =                        &
            \limite{n}{\infinito}
            c \cdot n
            = \infinito
          \end{array}
        $$
\end{enumerate}

\begin{aportes}
  \item \aporte{\dirRepo}{naD GarRaz \github}
\end{aportes}
