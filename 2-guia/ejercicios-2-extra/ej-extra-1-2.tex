\begin{enunciado}{\ejExtra}
  \begin{enumerate}[label=(\alph*)]
    \item
          Sea $M \en \reales^{n \times n}$ una matriz tal que
          existe una norma (subordinada a una norma vectorial)
          que cumple que $\norma{M} < 1$. Probar que $I_n - M$ es inversible,
          siendo $I_n$ la matriz identidad de orden $n$.

    \item
          Sean $\alpha, \gamma > 0, A \en \reales^{n \times n}$ tal que
          $\norma{A}_\infinito = \alpha$ y $f : \reales^{n \times n} \to \reales^{n \times n}$
          una transformación lineal tal que
          $\paratodo i,j \en \set{1,\ldots, n} : f(E_{ij}) = \gamma E_{ji}$ siendo $E_{ij}$ la matriz canónica de
          $\reales^{n \times n}$ que tiene un 1 en la coordenada $(i, j)$.
          \begin{enumerate}[label=\roman*)]
            \item Calcular $\norma{f(A)}_1$.
            \item Probar que si $\gamma \alpha < 1$ entonces $I_n - f(A)$ es inversible.
          \end{enumerate}
  \end{enumerate}
\end{enunciado}

\begin{enumerate}[label=(\alph*)]
  \item\label{extra-1:itema} Quiero ver si la matriz $I_n - M$ es inversible. Para eso puedo buscar a ver si hay algún vector $x_0$ tal que:
        $$
          (I_n - M)x_0 = 0,
        $$
        de ocurrir eso, la matriz trendría un $\nucleo(I_n - M) \distinto \set{0}$ o columnas linealmente independientes o como te guste decirle.
        Laburo un poco la expresión:
        $$
          (I_n - M)x_0 = 0
          \sii
          Mx_0 \igual{$\llamada1$} x_0.
        $$
        Tengo de dato que $\norma{M} < 1$ y la definición de norma subordinada dice algo como:
        $$
          \norma{M} = \maximo[x \distinto 0]\set{{\frac{\norma{Mx}}{\norma{x}}}}, \text{ para alguna norma vectorial}.
        $$
        En particular agarro de ese conjunto el vector $x_0$ que es por hipótesis distinto de 0, sino no tiene gracia. Tengo que:
        $$
          \frac{\norma{Mx_0}}{\norma{x_0}}
          \menorIgual{\red{!}}
          \norma{M} < 1
          \sisolosi
          \norma{Mx_0} < \norma{x_0}
          \Sii{$\llamada1$}
          \norma{x_0} \menor{\red{!}} \norma{x_0}.
        $$
        Lo cual terminando resultando en un absurdo. Por lo tanto no puede haber un vector $x_0$ tal que
        $(I_n - M) x = 0$.
        $$
          \cajaResultado{
            (I_n - M)  \text{ es inversible.}
          }
        $$

  \item
        \begin{enumerate}[label=\roman*)]
          \item\label{extra-1:itembi}
                Juntemos esa info:
                \begin{itemize}
                  \item $\norma{A}_\infty = \alpha$ quiere decir que la \textit{\underline{fila} cuya sumatoria sea máxima} vale $\alpha$.

                  \item La transformación $f$ transpone y multiplica por $\gamma$.
                        $$
                          [f(A)]_{ij} = \gamma \cdot [A]_{ji}
                        $$
                \end{itemize}

                Entonces como $\norma{A}_1$ sería la \textit{\underline{columna} cuya sumatoria sea máxima}:
                $$
                  \norma{f(A)}_1 =
                  \norma{\gamma \cdot A^t}_1 =
                  |\gamma| \cdot \norma{A^t}_1 \igual{\red{!}}
                  \gamma \cdot \norma{A}_\infinito \igual{!}
                  \gamma \cdot \alpha
                $$

                $$
                  \cajaResultado{
                    \norma{f(A)}_1 = \gamma \cdot \alpha
                  }
                $$

          \item ¿Hay que hacer lo mismo que en el ítem \ref{extra-1:itema}?
                $$
                  (I_n - f(A))x_0 = 0
                  \sii
                  f(A)x_0 = x_0.
                $$
                Del ítem \ref{extra-1:itembi} sé que $\norma{f(A)}_1 < 1$, entonces:
                $$
                  \frac{\norma{f(A)}_1}{\norma{A}_1}
                  \menorIgual{\red{!}}
                  \norma{f(A)}_1 =
                  \gamma \cdot \norma{A^t}_\infinito =
                  \gamma \cdot \alpha < 1
                  \Sii{\red{!}}
                  \norma{f(A)x_0}_1 < \norma{x_0}_1.
                $$
                Llegando a la misma conclusión que en el ítem \ref{extra-1:itema}:
                $$
                  \cajaResultado{
                    (I_n - f(A))  \text{ es inversible.}
                  }
                $$
                Al parecer para cualquier matriz que tenga $\norma{A} < 1 \to (I_n - A)$ será inversible.

        \end{enumerate}

\end{enumerate}

\begin{aportes}
  \item \aporte{\dirRepo}{naD GarRaz \github}
\end{aportes}
