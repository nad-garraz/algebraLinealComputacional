\begin{enunciado}{\ejercicio}
  Sea $D_n = \frac{1}{10} I_n$. Verificar que $\det(D_n) \to 0$ si $n \to \infinito$. ¿$D_n$ está mal
  condicionada? ¿Es el determinante un buen indicador de cuan cerca está una matriz de ser
  singular?
\end{enunciado}

\medskip

$D_n$ es la matriz identidad de $n \times n$ multiplicado por $\frac{1}{10}$,
por lo tanto es la matriz de $n \times n$
que en su diagonal tiene $\frac{1}{10}$.
Al ser una matriz diagonal su determinante es el producto de los elementos en su diagonal:
$$
  \det(D_n) =
  \productoria{1}{n} \frac{1}{10} =
  \left(\frac{1}{10}\right)^n=
  \frac{1}{10^n}
$$

para verificar $\det(D_n) \to 0$ si $n \to \infinito$ tomo límite:

$$
  \limite{n}{\infinito}\frac{1}{10^n} = 0
$$
La matriz está bien condicionada, se ve fácilmente que:
$$
  \norma{A}_\infinito = \frac{1}{10} \ytext \norma{A^{-1}}_\infinito = 10,
$$
ya que tiene todos 10 en la diagonal (inversa de matriz diagonal). Entonces: $\condicion_\infinito(A) = 1$ por lo que está perfectamente condicionada.

\begin{aportes}
  \item \aporte{https://github.com/juandelia03}{Juan D Elia \github}
\end{aportes}
