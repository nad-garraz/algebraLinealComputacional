\begin{enunciado}{\ejercicio}
Sea $D_n = \frac{1}{10} I_n.$  Verificar que $det(D_n) \rightarrow 0$ si $n \rightarrow \infinito$. ¿$D_n$ esta mal
condicionada? ¿Es el determinante un buen indicador de cuan cerca esta una matriz de ser
singular?
\end{enunciado}

\medskip

$D_n$ es la matriz identidad de $nxn$ multiplicado por $\frac{1}{10}$, por lo tanto es la matriz de $nxn$ 
que en su diagonal tiene $\frac{1}{10}$. Al ser una matriz diagonal su determinante es el producto de los elementos en su diagonal:

$
det(D_n) = 
\productoria{1}{n} \frac{1}{10} = 
(\frac{1}{10})^{n}=
\frac{1}{10^n}
$

para verificar $det(D_n) \rightarrow 0$ si $n \rightarrow \infinito$ tomo limite:

$\limite{n}{\infinito}\frac{1}{10^n} = 0$

La matriz esta bien condicionada, se ve facilmente que:

$||A||_\infinito = \frac{1}{10}$ y $||A^{-1}||_\infinito = 10$ ya que tiene todos 10 en la diagonal (inversa de matriz diagonal).

Entonces: $Cond_\infinito(A) = 1$ por lo que esta perfecamente condicionada.

\begin{aportes}
    \item \aporte{https://github.com/juandelia03}{Juan D Elia \github}
\end{aportes}