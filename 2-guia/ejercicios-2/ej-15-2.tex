\begin{enunciado}{\ejercicio}
    Dada una sucesión de vectores $\set{\bm{x}_n}_{n \en \naturales} \subset \reales^k$  probar
    $$
      \norma{\bm{x}_n}_1 \flecha{}[$n \to \infty$] 0
      \sisolosi
       (x_n)_i \flecha{}[$n \to \infty$] 0, 1 \leq i \leq k
    $$
    donde $(x_n)_i$ es la i-esima coordenada de $x_n$.

\end{enunciado}

\medskip

Tener en cuenta que: $||x_n||_1 \flecha{}[$n \to \infty$] 0$ es $\limite{n}{\infty} \sum_{i=1} ^{n} |x_i| = 0$

\begin{enumerate}[label=(\roman*)]
\item veo la ida:
\\
Como $||x_n||_1 \flecha{}[$n \to \infty$] 0$ y la norma se compone de sumar valores mayores o iguales a cero, se puede 
implicar que cada elemento con n tendiendo a infinito debe ser 0, es decir $(x_n)_i \flecha{}[$n \to \infty$] 0$ para todo i.
\\
Mas formal:

Por el absurdo: asumo que existe un $(x_n)_i$ distinto de cero

por la implicacion se que $\limite{n}{\infty} \sum_{i=1} ^{n} |x_i| = 0$, como solo sumo positivos es absurdo si hay algun $x_i$ tq: $x_i$ no es cero
\item veo la vuelta:
\\ 
Si $(x_n)_i \flecha{}[$n \to \infty$] 0$ para todo i, puedo implicar  que $\norma{\bm{x}_n}_1 \flecha{}[$n \to \infty$] 0$,
porque es sumar todos esos elementos. Mas formal:

$\limite{n}{\infty} \sum_{i=1} ^{n} |x_i| = 0$ con cada $x_i = 0$ con n tendiendo a infinito:

$\limite{n}{\infty} \sum_{i=1} ^{n} 0 = 0$ vale
\end{enumerate}

\begin{aportes}
    \item \aporte{https://github.com/juandelia03}{Juan D Elia \github}
\end{aportes}
  