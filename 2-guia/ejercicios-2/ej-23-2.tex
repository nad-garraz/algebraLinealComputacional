\begin{enunciado}{\ejercicio}
Para la matriz

$$
A =
\matriz{ccc}{
  1 & n & 5n \\
  1 & 3n & 3n \\
  1 & n & 2n
}$$

con $n \en \naturales$, probar que existe una constante $c > 0$ tal que $cond_\infinito(A) \geq cn$ para todo $n \en \naturales$,
y deducir que $cond_\infinito(A) \rightarrow \infinito$ cuando $n \rightarrow \infinito$
\end{enunciado}

\medskip

Primero voy a calcular el numero de condicion. Para eso tengo que ver la norma de $A$ y $A^{-1}$

Para $A$:

La norma infinito es sumar los elementos de cada fila en modulo y quedarnos con la suma mas grande.
En este caso se ve a ojo que:

$||A||_\infinito = 1 + 6n$

Para $A^{-1}$:

Aca hay que calcular la inversa de A, no voy a escribir todos los pasos (ninguno de hecho).
$$
A^{-1} =
\matriz{ccc}{
  \frac{-1}{2} & \frac{-1}{2} & 2 \\ 
  \frac{-1}{6n} & \frac{1}{2n} & \frac{-1}{3n} \\
  \frac{1}{3n} & 0 & \frac{-1}{3n}
}$$

y se ve que $|\frac{-1}{2}| + |\frac{-1}{2}| + |2|$ es la fila que mas suma 
(el resto tiene elementos con n el denominador con n>0)

$||A^{-1}||_\infinito = 3$

Por lo tanto $cond(A)_\infinito = 3(1+6n) = 3+18n$

El enunciado pide que muestre que existe una $c$ que cumpla $cond(A)_\infinito \geq cn$ para todo n.

Elijo c = 2:

$
3+18n \geq 2n \sii 3 + 16n \geq 0 \sii n \geq \frac{-3}{16}
$, Como n es natural vale para todo n

Por ultimo deducir que $cond_\infinito(A) \rightarrow \infinito$ cuando $n \rightarrow \infinito$:

$\limite{n}{\infinito}cond_\infinito(A) = \limite{n}{\infinito} 3+18n = \infinito$

\begin{aportes}
    \item \aporte{https://github.com/juandelia03}{Juan D Elia \github}
\end{aportes}