\begin{enunciado}{\ejercicio}
  Para la matriz
  $$
    A =
    \matriz{ccc}{
      1 & n & 5n \\
      1 & 3n & 3n \\
      1 & n & 2n
    }$$

  con $n \en \naturales$, probar que existe una constante $c > 0$ tal que $\condicion_\infinito(A) \geq cn$ para todo $n \en \naturales$,
  y deducir que $\condicion_\infinito(A) \to \infinito$ cuando $n \to \infinito$
\end{enunciado}

\bigskip

Como sobra la creatividad en este repo, acá van 2 formas de resolver el ejercicio:
\begin{enumerate}[label=\red{\angry$_{(\arabic*)}$}]
  \item Es fácil calcular $A_\infinito$:
        $$
          \norma{A}_\infinito = 6n + 1
        $$
        Ahora voy a buscar una matriz singular $B$, conveniente para usar:
        $$
          \condicion_\infinito(A)
          \geq
          \supremo \set{\frac{\norma{A}_\infinito}{\norma{A - B}_\infinito} : B \text{ es singular}}
        $$
        por ejemplo:
        $$
          B =
          \matriz{ccc}{
            0 & n & 5n  \\
            0 & 3n & 3n \\
            0 & n & 2n
          }
          \entonces
          A - B =
          \matriz{ccc}{
            1 & 0 & 0  \\
            1 & 0 & 0 \\
            1 & 0 & 0
          }
          \entonces
          \frac{\norma{A}_{\infinito}}{\norma{A - B}_\infinito} =
          \frac{6n + 1}{1} = 6n + 1 > \ua{6}{c}n  \quad \paratodo n \en \naturales
        $$
        Y ahí quedó esa constante $c$:
        $$
          \cajaResultado{
            c = 6
          }
        $$

        \bigskip

  \item
          Primero voy a calcular el número de condición. Para eso tengo que ver la norma de $A$ y $A^{-1}$

        \textit{Para $A$:}

        La norma infinito es sumar los elementos de cada fila en módulo y quedarnos con la suma más grande.
        En este caso se ve a ojo que:
        $$
          \norma{A}_\infinito = 1 + 6n
        $$

        \textit{Para $A^{-1}$:}

        Aca hay que calcular la inversa de $A$, $A^{-1}$, no voy a escribir todos los pasos (ninguno de hecho).
        $$
          \everymath{\displaystyle}
          A^{-1} =
          \matriz{ccc}{
            \frac{-1}{2} & \frac{-1}{2} & 2 \\ \\
            \frac{-1}{6n} & \frac{1}{2n} & \frac{-1}{3n} \\\\
            \frac{1}{3n} & 0 & \frac{-1}{3n}
          }
        $$
        y se ve que $|\frac{-1}{2}| + |\frac{-1}{2}| + |2|$ es la fila que más suma
        (el resto tiene elementos con $n$ el denominador con $n > 0$)
        $$
          \norma{A^{-1}}_\infinito = 3
        $$
        Por lo tanto:
        $$
          \condicion(A)_\infinito = 3(1 + 6n) = 3 + 18n
        $$

        El enunciado pide que muestre que existe una $c$ que cumpla $\condicion(A)_\infinito \geq cn$ para todo n.

        Elijo $c = 2$:
        $$
          3 + 18n \geq 2n \sii 3 + 16n \geq 0 \sii n \geq \frac{-3}{16}.
        $$
        Como $n \en \naturales$ vale para todo $n$.

        Por último deducir que $\condicion_\infinito(A) \to \infinito$ cuando $n \to \infinito$:

        $$
          \limite{n}{\infinito}\condicion_\infinito(A) = \limite{n}{\infinito} 3+18n = \infinito
        $$
\end{enumerate}

\begin{aportes}
  \item \aporte{https://github.com/juandelia03}{Juan D Elia \github}
  \item \aporte{\dirRepo}{naD GarRaz \github}
\end{aportes}
