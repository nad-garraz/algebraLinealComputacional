\begin{enunciado}{\ejercicio}
    Escribir la matriz de las siguientes transformaciones lineales en base canónica.
Interpretar geométricamente cada transformación. \newline
La parte de Interpretar geometricamente \hacer 
    \begin{enumerate}[label=(\alph*)]
    \item \( f(x,y) = (x,0) \)
    \item \( f(x,y) = (x,-y) \)
    \item \( f(x,y) = \left( \frac{1}{2} (x + y), \frac{1}{2} (x + y) \right) \)
    \item \( f(x,y) = (x \cos t - y \sin t, x \sin t + y \cos t) \)
    
    \end{enumerate}
  \end{enunciado}
  
  \begin{enumerate}[label=(\alph*)]
    \item Para la base canónica:
    \[
    f(1,0) = (1,0), \quad f(0,1) = (0,0)
    \]
    Entonces, la matriz asociada es:
    \[
    M =
    \begin{bmatrix}
    1 & 0 \\
    0 & 0
    \end{bmatrix}
    \]
   
    \item Para la base canónica:
    \[
    f(1,0) = (1,0), \quad f(0,1) = (0,-1)
    \]
    Entonces, la matriz asociada es:
    \[
    M =
    \begin{bmatrix}
    1 & 0 \\
    0 & -1
    \end{bmatrix}
    \]

    \item Para la base canónica:
    \[
    f(1,0) = \left( \frac{1}{2}, \frac{1}{2} \right), \quad f(0,1) = \left( \frac{1}{2}, \frac{1}{2} \right)
    \]
    Entonces, la matriz asociada es:
    \[
    M =
    \begin{bmatrix}
    \frac{1}{2} & \frac{1}{2} \\
    \frac{1}{2} & \frac{1}{2}
    \end{bmatrix}
    \]

    \item Para la base canónica:
    \[
    f(1,0) = (\cos t, \sin t), \quad f(0,1) = (-\sin t, \cos t)
    \]
    Entonces, la matriz asociada es:
    \[
    M =
    \begin{bmatrix}
    \cos t & -\sin t \\
    \sin t & \cos t
    \end{bmatrix}
    \]
  
  \end{enumerate}
  
  \begin{aportes}
    \item \aporte{https://github.com/juandelia03}{Juan D Elia \github} 
  \end{aportes}