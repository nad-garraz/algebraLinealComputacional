\begin{enunciado}{\ejercicio}
  Escribir la matriz de las siguientes transformaciones lineales en base canónica.
  Interpretar geométricamente cada transformación.
  \begin{enumerate}[label=(\alph*)]
    \item  $f(x,y) = (x,0) $
    \item  $f(x,y) = (x,-y) $
    \item  $f(x,y) = (\frac{1}{2} (x + y), \frac{1}{2} (x + y))$
    \item  $f(x,y) = (x \cos t - y \sin t, x \sin t + y \cos t) )$
  \end{enumerate}
\end{enunciado}

\begin{enumerate}[label=(\alph*)]
  \item Para la base canónica:
        $$
          f(1,0) = (1,0), \quad f(0,1) = (0,0)
        $$
        Entonces, la matriz asociada es:
        $$
          M =
          \begin{bmatrix}
            1 & 0 \\
            0 & 0
          \end{bmatrix}
        $$
        \begin{minipage}{0.5\textwidth}
          Geométricamente estamos proyectando al eje $x_0$.
        \end{minipage}
        \begin{minipage}{0.5\textwidth}
          \begin{tikzpicture}[scale=0.8, every node/.style={font={\tiny}}]
            \draw[-latex] (-1.5,0) -- (3,0) node[below] {$x_0$};
            \draw[-latex] (0,-1.5) -- (0,2) node[left] {$x_1$};

            \coordinate (P1) at (2,1.5);
            \fill[Cerulean] (P1) circle (0.08) node[above right] {$(x_0,x_1)$};
            \draw[dotted, Cerulean, -latex] (P1) -- ($(P1) -(0,1.5)$) node[midway, left] {$M$};

            \fill[Cerulean] ($(P1) -(0,1.5)$) circle (0.05);
            \node[below, Cerulean] at ($(P1) -(0,1.5)$) {$(x_0,0)$};

            \coordinate (P2) at (-0.5,-1);
            \fill[purple] (P2) circle (0.08) node[below left] {$(x_0',x_1')$};

            \draw[dotted, purple, -latex] (P2) -- ($(P2) -(0,-1)$) node[midway, left] {$M$};

            \fill[purple] ($(P2) -(0,-1)$) circle (0.05);
            \node[above, purple] at ($(P2) -(0,-1)$) {$(x_0',0)$};
          \end{tikzpicture}
        \end{minipage}

  \item Para la base canónica:
        $$
          f(1,0) = (1,0), \quad f(0,1) = (0,-1)
        $$
        Entonces, la matriz asociada es:
        $$
          M =
          \begin{bmatrix}
            1 & 0  \\
            0 & -1
          \end{bmatrix}
        $$
        \begin{minipage}{0.65\textwidth}
          Geométricamente estamos haciendo una reflexión respecto del eje $x_0$.
        \end{minipage}
        \begin{minipage}{0.35\textwidth}
          \begin{tikzpicture}[scale=0.8, every node/.style={font={\tiny}}]
            \draw[-latex] (-1.5,0) -- (3,0) node[below] {$x_0$};
            \draw[-latex] (0,-1.5) -- (0,2) node[left] {$x_1$};

            \coordinate (P1) at (2,1.5);
            \fill[Cerulean] (P1) circle (0.08) node[above right] {$(x_0,x_1)$};
            \draw[dotted, Cerulean, -latex] (P1) -- (2,-1.5) node[midway, below left] {$M$};
            \fill[Cerulean] (2,-1.5) circle (0.05) node[below, Cerulean] {$(x_0,-x_1)$};

            \coordinate (P2) at (-0.5,-1);
            \fill[purple] (P2) circle (0.08) node[below left] {$(x'_0,x'_1)$};
            \draw[dotted, purple, -latex] (P2) -- (-0.5,1) node[midway,above left] {$M$};
            \fill[purple] (-0.5, 1) circle (0.05)node[above left, purple] at (-0.5, 1) {$(x'_0,-x'_1)$};
          \end{tikzpicture}
        \end{minipage}

  \item Para la base canónica:
        $$
          f(1,0) = \left( \frac{1}{2}, \frac{1}{2} \right), \quad f(0,1) = \left( \frac{1}{2}, \frac{1}{2} \right)
        $$
        Entonces, la matriz asociada es:
        $$
          M =
          \begin{bmatrix}
            \frac{1}{2} & \frac{1}{2} \\
            \frac{1}{2} & \frac{1}{2}
          \end{bmatrix}
        $$

        Geométricamente estamos \textit{haciendo, llevando {\tiny (mejores palabras serán bienvenidas)}} todo a la dirección $(1,1)$, ponele.
        $$
          f(x_0,x_1) = \purple{\frac{1}{2}(x_0 + x_1)} \cdot (1,1) \approx \purple{\lambda} \cdot (1,1)
        $$

  \item Para la base canónica:
        $$
          f(1,0) = (\cos t, \sin t), \quad f(0,1) = (-\sin t, \cos t)
        $$
        Entonces, la matriz asociada es:
        $$
          M =
          \begin{bmatrix}
            \cos t & -\sin t \\
            \sin t & \cos t
          \end{bmatrix}
        $$
        \begin{minipage}{0.65\textwidth}
          Geométricamente estamos rotando en sentido antihorario al eje $\ua{x_2}{z}$.
        \end{minipage}
        \begin{minipage}{0.35\textwidth}
          \begin{tikzpicture}[scale=0.8, every node/.style={font={\tiny}}]
            \draw[-latex] (-1.5,0) -- (3,0) node[below] {$x_0$};
            \draw[-latex] (0,-1.5) -- (0,2) node[left] {$x_1$};

            \coordinate (P1) at (1,1);
            \fill[Cerulean] (P1) circle (0.08) node[above right] {$(x_0,x_1)$};
                  \draw[dotted, Cerulean, -latex] (P1) arc (45:225:{sqrt(2)}) node[midway, left] {$M_{(t = \pi)}$};

            \fill[Cerulean] (-1,-1) circle (0.05) node[below left, Cerulean] {$M_{(t = \pi)}(x_0,x_1)$};
          \end{tikzpicture}
        \end{minipage}

\end{enumerate}

\begin{aportes}
  \item \aporte{https://github.com/juandelia03}{Juan D Elia \github}
\end{aportes}
