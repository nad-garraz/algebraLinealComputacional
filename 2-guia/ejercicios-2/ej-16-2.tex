\begin{enunciado}{\ejercicio}
  Sea $A \en \reales ^{nxn}$, probar que las constantes de equivalencia entre las normas $\norma{.}_1$ y $\norma{.}_2$
  y entre las normas $\norma{.}_1$ y $\norma{.}_\infinito$ vienen dadas por:

  $$
    \frac{1}{\sqrt{n}} \norma{A}_\infinito \leq \norma{A}_2 \leq \sqrt{n} \norma{A}_\infinito
  $$
  $$
    \frac{1}{\sqrt{n}} \norma{A}_1 \leq \norma{A}_2 \leq \sqrt{n} \norma{A}_1
  $$
\end{enunciado}

La norma matricial \textit{es fea}, por lo cual voy a usar normas inducidas:

\begin{enumerate}[label=(\alph*)]
  \item Veo los dos lados de la desigualdad
        \begin{enumerate}[label=(\roman*)]
          \item
                Arranco por la parte izquierda de la desigualdad. Quiero probar que:
                $$
                  \frac{1}{\sqrt{n}} \norma{A}_\infinito \leq \norma{A}_2
                $$
                Norma inducida:
                $$
                  \norma{A}_2 = \maximo_{x \distinto 0} \set{\frac{\norma{Ax}_2}{\norma{x}_2} }
                $$
                Quiero  achicar el numerador y agrandar el denominador para armar una expresión más chica.
                Recordar por ejercicio \ref{ej:11} que:
                $$
                  \begin{array}{c}
                    \sqrt{n} \norma{x}_\infinito
                    \mayorIgual{$\llamada1$}
                    \norma{x}_2,
                    \entonces
                    \frac{1}{\sqrt{n} \cdot \norma{x}_\infinito}
                    \menorIgual{$\llamada2$}
                    \frac{1}{\norma{x}_2}         \\
                    \ytext
                    \norma{x}_2
                    \mayorIgual{$\llamada3$} \norma{x}_\infinito
                    \entonces
                    \frac{1}{\norma{x}_2}
                    \menorIgual{$\llamada4$}
                    \frac{1}{\norma{x}_\infinito} \\
                  \end{array}
                $$
                entonces dado que $Ax \en \reales^n$ es un vector, por \textit{norma vectorial}:
                $$
                  \norma{Ax}_2
                  \igual{def}
                  \maximo_{x \distinto 0} \set{\frac{\norma{Ax}_2}{\norma{x}_2}}
                  \mayorIgual{$\llamada3$}[$\llamada2$]
                  \maximo_{x \distinto 0} \set{\frac{\norma{Ax}_\infinito}{\sqrt{n} \norma{x}_\infinito}} =
                  \frac{1}{\sqrt{n}} \cdot \maximo_{x \distinto 0} \set{\frac{\norma{Ax}_\infinito}{\norma{x}_\infinito}}
                  \igual{def}
                  \frac{1}{\sqrt{n}}  \cdot \norma{A}_\infinito
                $$
                Donde se tiene en cuenta que si un conjunto $A$ tiene sus elementos menores al de otro conjunto $B$ en particular sus máximos también
                esa relación.

          \item
                Veo la parte derecha de la desigualdad quiero probar que:
                $$
                  \norma{A}_2 \leq \sqrt{n} \norma{A}_\infinito
                $$
                Ahora como quiero ver que $\norma{A}_2$ es mas chico que otra cosa. \textit{agrando} el numerador y \textit{achico} denominador
                para armar una expresión más grande. Nuevamente usamos el resultado del ejercicio \ref{ej:11}:
                $$
                  \norma{A}_2
                  \igual{def}
                  \maximo_{x \distinto 0}
                  \set{\frac{\norma{Ax}_2}{\norma{x}_2} }
                  \menorIgual{$\llamada1$}[$\llamada4$]
                  \maximo_{x \distinto 0} \set{\frac{\sqrt{n} \norma{Ax}_\infinito}{\norma{x}_\infinito} } =
                  \sqrt{n} \cdot \maximo_{x \distinto 0} \set{\frac{ \norma{Ax}_\infinito}{\norma{x}_\infinito} }
                  \igual{def}
                  \sqrt{n} \norma{A}_\infinito
                $$
        \end{enumerate}
        Queda demostrado.
  \item
        Veo los dos lados de la desigualdad. Parecido a lo que se hizo en el anterior:

        \begin{enumerate}[label=(\roman*)]
          \item
                Para ver que es más grande, tengo que armar una expresión mas chica. Achico el numerador y agrando el denominador.

                Recordar por ejercicio 11 que: $\norma{x}_2 \geq \frac{1}{\sqrt{n}} \norma{x}_1$ y $\norma{x}_2 \leq  \norma{x}_1$, entonces:
                $$
                  \begin{array}{rcl}
                    \norma{x}_2
                    \mayorIgual{$\llamada1$}
                    \frac{1}{\sqrt{n}} \norma{x}_1
                     & \sisolosi &
                    \frac{\sqrt{n}}{\norma{x}_1}
                    \mayorIgual{$\llamada2$}
                    \frac{1}{\norma{x}_2} \\
                     & \ytext    &        \\
                    \norma{x}_2
                    \menorIgual{$\llamada3$} \norma{x}_1
                     & \sisolosi &
                    \frac{1}{\norma{x}_1}
                    \menorIgual{$\llamada4$}
                    \frac{1}{\norma{x}_2} \\
                  \end{array}
                $$
                Usando la norma inducida:
                $$
                  \norma{A}_2
                  \igual{def}
                  \maximo_{x \distinto 0} \set{\frac{\norma{Ax}_2}{\norma{x}_2} }
                  \mayorIgual{$\llamada1$}[$\llamada4$]
                  \maximo_{x \distinto 0} \set{\frac{\frac{1}{\sqrt{n}} \norma{Ax}_1}{\norma{x}_1} } =
                  \frac{1}{\sqrt{n}} \cdot  \maximo_{x \distinto 0} \set{\frac{\norma{Ax}_1}{\norma{x}_1} }
                  \igual{def}
                  \frac{1}{\sqrt{n}} \cdot \norma{A}_1
                $$

          \item
                Para ver que es más chico, tengo que armar una expresión más grande. Agrando el numerador y achico el denominador.
                Usando nuevamente los resultados del ejercicio \ref{ej:11} para normas vectoriales:
                $$
                  \norma{A}_2
                  \igual{def}
                  \maximo_{x \distinto 0} \set{\frac{\norma{Ax}_2}{\norma{x}_2} }
                  \menorIgual{$\llamada3$}[$\llamada2$]
                  \maximo_{x \distinto 0} \set{\norma{Ax}_1 \cdot \frac{\sqrt{n}}{\norma{x}_1}} =
                  \sqrt{n} \cdot  \maximo_{x \distinto 0} \set{\frac{\norma{Ax}_1}{\norma{x}_1}}
                  \igual{def}
                  \sqrt{n} \cdot \norma{A}_1
                $$
                Queda demostrado
        \end{enumerate}

\end{enumerate}

\begin{aportes}
  \item \aporte{https://github.com/juandelia03}{Juan D Elia \github}
\end{aportes}
