\begin{enunciado}{\ejercicio}
    Sea $A \en \reales ^{nxn}$, probar que las constantes de equivalencia entre las normas $||.||_1$ y $||.||_2$
    y entre las normas $||.||_1$ y $||.||_\infinito$ vienen dadas por:
    
    $$
    \frac{1}{\sqrt{n}} ||A||_\infinito \leq ||A||_2 \leq \sqrt{n} ||A||_\infinito
    $$
    $$
    \frac{1}{\sqrt{n}} ||A||_1 \leq ||A||_2 \leq \sqrt{n} ||A||_1
    $$
    
    \end{enunciado}
    
    \begin{enumerate}[label=(\alph*)]
    \item 
        Veo los dos lados de la desigualdad
        \begin{enumerate}[label=(\roman*)]
            \item 
            veo la parte izquierda de la desigualdad:
    
            $||A||_2 = max \{\frac{||Ax||_2}{||x||_2} \}$
    
            Quiero  achicar el numerador y agrandar el denominador para armar una expesion mas chica.\\ 
            Recordar por ejercicio 11 que:$||x||_2 \geq ||x||_\infinito$ y $\sqrt{n} ||x||_\infinito \geq  ||x||_2$, entonces:
    
            $max \{\frac{||Ax||_2}{||x||_2} \} \geq max \{\frac{||Ax||_\infinito}{\sqrt{n} ||x||_\infinito} \} = \frac{1}{\sqrt{n}} ||A||_\infinito$ 
            \item 
            veo la parte derecha de la desigualdad:
            
            Ahora como quiero ver que $||A||_2$ es mas chico que otra cosa. "agrando" el numerador y "achico" denominador
            para armar una expresion mas grande.
            
            Recordar por ejercicio 11 que:$||x||_2 \leq \sqrt{n} ||x||_\infinito$ y $||x||_\infinito \leq  ||x||_2$, entonces:
            
            $||A||_2=max \{\frac{||Ax||_2}{||x||_2} \} \leq max \{\frac{\sqrt{n} ||Ax||_\infinito}{||x||_\infinito} \} = \sqrt{n} ||A||_\infinito$ 
        \end{enumerate}
        Queda demostrado.
    \item 
        Veo los dos lados de la desigualdad
        \begin{enumerate}[label=(\roman*)]
            \item
            Para ver que es mas grande, tengo que armar una expresion mas chica. Achico el numerador y agrando el denominador.
            
            Recordar por ejercicio 11 que $||x||_2 \geq \frac{1}{\sqrt{n}} ||x||_1$ y $||x||_2 \leq  ||x||_1$, entonces:
    
            $||A||_2 = max \{\frac{||Ax||_2}{||x||_2} \} \geq  max \{\frac{\frac{1}{\sqrt{n}} ||Ax||_1}{||x||_1} \} =
            \frac{1}{\sqrt{n}} ||A||_1$
            \item
            Para ver que es mas chico, tengo que armar una expresion mas grande. Agrando el numerador y achico el denominador.
    
            Recordar por ejercicio 11 que $||x||_ \geq \frac{1}{\sqrt{n}} ||x||_1$ y $||x||_2 \leq  ||x||_1$, entonces:
    
            $||A||_2 = max \{\frac{||Ax||_2}{||x||_2} \} \leq  max \{\frac{||Ax||_1}{\frac{1}{\sqrt{n}} ||x||_1} \} =
            \sqrt{n} ||A||_1$
    
            Queda demostrado
        \end{enumerate}
    
    \end{enumerate}
    
\begin{aportes}
    \item \aporte{https://github.com/juandelia03}{Juan D Elia \github}
\end{aportes}
      