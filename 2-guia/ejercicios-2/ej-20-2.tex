\begin{enunciado}{\ejercicio}
  Sea
  $$
    A =
    \matriz{ccc}{
      3 & 0 & 0 \\
      0 & \frac{5}{4} & \frac{3}{4} \\
      0 & \frac{3}{4} & \frac{5}{4}
    }$$
  \begin{enumerate}[label=(\alph*)]
    \item Calcular $\condicion_\infinito(A)$
    \item
          Cuan chico debe ser el error relativo en los datos $\frac{\norma{b-\tilde{b}}}{\norma{b}}$, si se desea que el error relativo en
          la aproximacion de la solucion $\frac{\norma{x-\tilde{x}}}{\norma{x}}$ sea menor que $10^{-4}$ en ($\norma{.}_\infinito$)

    \item Realizar experimentos numéricos para verificar las estimaciones del ítem anterior. Considerar $\bm{b} = (3,2,2)^t$, que se corresponde
          con la solución exacta $\bm{x} = (1,1,1)^t$. Generar vectores de error aleatorios y perturbar $\bm{b}$, obteniendo $\tilde{\bm{b}}$.
          Finalmente, resolver $\bm{A}\bm{\tilde{x}} = \tilde{\bm{b}}$ y verificar que $\norma{\tilde{\bm{x}} - \bm{x}} < 10^{-4}$
  \end{enumerate}

\end{enunciado}

\medskip
\begin{enumerate}[label=(\alph*)]
  \item Para calcular $\condicion(A)$ calculo la norma de $A$ y $A^{-1}$:

        $\norma{A}_\infinito = \maximo{}\set{3,2,2} = 3$

        Calculamos $A^{-1}$:
        $$
          \matriz{ccc}{
            \frac{1}{3} & 0 & 0 \\
            0 & \frac{5}{4} & \frac{-3}{4} \\
            0 & \frac{-3}{4} & \frac{5}{4}
          }
        $$

        Se ve a ojo que $\norma{A^{-1}}_\infinito = 2 $

        Por lo tanto: $\condicion_\infinito(A) = 3.2 = 6$

  \item
        Quiero: $\frac{\norma{x - \tilde{x}}}{\norma{x}} < 10^{-4}$

        Por el ejercicio \ref{ej:19} sabemos que $\frac{\norma{x-\tilde{x}}}{\norma{x}} \leq \condicion(A) \frac{\norma{b-\tilde{b}}}{\norma{b}}$

        Entonces quiero que $6 \cdot \frac{\norma{b-\tilde{b}}}{\norma{b}} < 10^4
          \sii
          \frac{\norma{b-\tilde{b}}}{\norma{b}} < \frac{10^{-4}}{6}$

  \item \hacer
\end{enumerate}

\begin{aportes}
  \item \aporte{https://github.com/juandelia03}{Juan D Elia \github}
\end{aportes}
