\begin{enunciado}{\ejercicio}
  Para las siguientes matrices
  $$
    A =
    \matriz{ccc}{
      1 & 2 & 1 \\
      2 & 3 - \varepsilon & 2 + \varepsilon \\
      0 & 1 + \varepsilon & \varepsilon
    }
    \qquad
    b =
    \matriz{c}{
      0  \\
      0.1 \\
      0.1
    }
  $$
  \begin{enumerate}[label=(\alph*)]
    \item Tomando $\varepsilon = 0.001$, resolver el sistema $Ax = b$ mediante eliminación gaussiana sin
          intercambio de filas usando aritmética de punto flotante en base 10 con 3 dígitos de mantisa
          y sistema de redondeo.

    \item Para $\varepsilon = 0.001$, hallar la solución exacta $x$ del sistema y comparar con la solución del ítem anterior
          ¿Cómo explica la diferencia?
  \end{enumerate}
\end{enunciado}

Voy a usar \blue{3 dígitos de mantisa}, es decir \blue{3} números significativos:

$$
  \begin{array}{l}
    3.01 = 0.\blue{301} \cdot 10^1                                                       \\
    3.001 = 0.\blue{300}\red{1} \cdot 10^1 \flecha{trunca}[redondea] 0.3 \cdot 10^1  = 3 \\
    3.005 = 0.\blue{300}\red{5} \cdot 10^1 \flecha{trunca}[redondea] 0.301 \cdot 10^1 = 3.01
  \end{array}
$$

$$
  \varepsilon = 0.001 = 0.1 \cdot 10^{-2}
$$

\begin{enumerate}[label=(\alph*)]
  \item$$
          \begin{array}{clll}
            (A|b)
            =
            \matriz{ccc|c}{
            1 & 2                               & 1                               & 0   \\
            2 & 3 - \varepsilon                 & 2 + \varepsilon                 & 0.1 \\
            0 & 1 + \varepsilon                 & \varepsilon                     & 0.1
            }
              & \igual{\red{!}}                 &
            \matriz{ccc|c}{
            1 & 2                               & 1                               & 0   \\
            2 & 0.3 \taa{\llamada1}{}\cdot 10^1 & 0.2 \taa{\llamada2}{}\cdot 10^1 & 0.1 \\
            0 & 0.1 \taa{\llamada3}{}\cdot 10^1 & 0.1 \cdot 10^{-2}               & 0.1
            }                                                                           \\
            \triangulacion{
              F_2 - 2 F_1 \to F_2
            }
              &                                 &
            \matriz{ccc|c}{
            1 & 2                               & 1                               & 0   \\
            0 & -1                              & 0                               & 0.1 \\
            0 & 1                               & 0.001                           & 0.1
            }                                                                           \\
            \triangulacion{
              F_3 +  F_2 \to F_3
            }
              & \igual{\red{!}}                 &
            \matriz{ccc|c}{
            1 & 2                               & 1                               & 0   \\
            0 & -1                              & 0                               & 0.1 \\
            0 & 0                               & 0.001                           & 0.2
            }                                                                           \\
          \end{array}
        $$

        Esas cuentas falopas con \textit{punto flotante}:

        La solución sería:
        $$
          \llave{l}{
            x =  0.2 - 200 = -\blue{199}.8 \igual{$\llamada4$} -200\\
            y = -0.1\\
            z = 0.2 \div (0.1 \cdot 10^{-2}) = 200
          }
        $$

        $
          \llave{l}{
            \llamada1 2.999 \flecha{trunca y}[redondea] 3\\
            \llamada2 2.001 \flecha{trunca y}[redondea] 2\\
            \llamada3 1.001 \flecha{trunca y}[redondea] 1\\
            \llamada4 -199.8 \flecha{trunca y}[redondea] -200
          }.
        $

  \item $$
          \begin{array}{cllll}
            (A|b)
            =
            \matriz{ccc|c}{
            1 & 2               & 1               & 0   \\
            2 & 3 - \varepsilon & 2 + \varepsilon & 0.1 \\
            0 & 1 + \varepsilon & \varepsilon     & 0.1
            }
              & \igual{\red{!}} &
            \matriz{ccc|c}{
            1 & 2               & 1               & 0   \\
            2 & 2.999           & 2.001           & 0.1 \\
            0 & 1.001           & 0.001           & 0.1
            }                                           \\
            \triangulacion{
              F_2 - 2 F_1 \to F_2
            }
              &                 &
            \matriz{ccc|c}{
            1 & 2               & 1               & 0   \\
            0 & -1.001          & 0.001           & 0.1 \\
            0 & 1.001           & 0.001           & 0.1
            }                                           \\
            \triangulacion{
              F_3 +  F_2 \to F_3
            }
              &                 &
            \matriz{ccc|c}{
            1 & 2               & 1               & 0   \\
            0 & -1.001          & 0.001           & 0.1 \\
            0 & 0               & 0.002           & 0.2
            }                                           \\
          \end{array}
        $$

        La solución sería:
        $$
          \llave{l}{
            x = -100\\
            y = 0\\
            z = 100
          }
        $$
\end{enumerate}

La solución exacta difiere mucho de la original. A continuación un para de  cuentas hechas en \python.

\copyPaste
\codigoPython{ej-9/codigo2-9-1.py}

\begin{aportes}
  \item \aporte{\dirRepo}{naD GarRaz \github}
\end{aportes}
