\begin{enunciado}{\ejercicio}
  Sean $ f: \reales^3 \to \reales^4 $ definido por:
  $$
    f(x_1, x_2, x_3) = (x_1 + x_2, x_1 + x_3, 0, 0)
  $$
  y $ g: \reales^4 \to \reales^2 $ definido por:
  $$
    g(x_1, x_2, x_3, x_4) = (x_1 - x_2, 2x_1 - x_2).
  $$
  Calcular el núcleo y la imagen de $f$, de $g$ y de $g \circ f$.

  Decidir si son monomorfismos, epimorfismos o isomorfismos.
\end{enunciado}

\medskip

\textbf{Cálculo de la imagen de $f$}

Aplicamos $f$ a los vectores canónicos de $\reales^3$:

$$ f(1,0,0) = (1,1,0,0) $$
$$ f(0,1,0) = (1,0,0,0) $$
$$ f(0,0,1) = (0,1,0,0) $$

Por lo tanto, el generador de la imagen de $ f $ es:

$$ \imagen(f) = \ket{(1,1,0,0), (1,0,0,0), (0,1,0,0)} $$
Como (1,1,0,0) LD:
$$ \imagen(f) = \ket{(1,0,0,0), (0,1,0,0)} $$

La dimensión de la imagen es 2.

\medskip

\textbf{Cálculo del núcleo de $ f $}

Buscamos los coeficientes $ \alpha, \beta, \gamma $ tales que:

$$ \alpha(1,1,0,0) + \beta(1,0,0,0) + \gamma(0,1,0,0) = (0,0,0,0) $$

Esto da el sistema:

$$ \alpha + \beta = 0 $$
$$ \alpha + \gamma = 0 $$

Resolviendo:

$$ \alpha = -\beta $$
$$ \alpha = -\gamma $$

Reemplazo en los vectores de salida:
$$ \alpha(1,0,0) - \alpha(0,1,0) - \alpha(0,0,1) = \alpha(1,-1,-1) $$

Por lo tanto, el núcleo de $ f $ es:

$$ \nucleo(f) = ket{(1,-1,-1)} $$

Entonces podemos concluir:
\newline
i) Como $ \imagen(f) \neq \reales^3 $, no es epimorfismo.
\newline
ii)Como $ \nucleo(f) \neq \set{0} $, no es monomorfismo.

\subsection*{Cálculo de la imagen de $ g $}

Aplicamos $g$ a los vectores canónicos de $\reales^4$:

$$ g(1,0,0,0) = (1,2) $$
$$ g(0,1,0,0) = (-1,-1) $$
$$ g(0,0,1,0) = (0,0) $$
$$ g(0,0,0,1) = (0,0) $$

Por lo tanto, el generador de la imagen de $ g $ es:

$$ \imagen(g) = \ket{(1,2), (-1,-1)} $$

Los vectores son LI, asi que la dimension es 2. Implica que Im g es $ \reales^2 $.

\subsection*{Cálculo del núcleo de $ g $}

Buscamos los coeficientes $ \alpha, \beta, \gamma, \delta $ tales que:

$$ \alpha(1,2) + \beta(-1,-1) + \gamma(0,0) + \delta(0,0) = (0,0) $$

Esto nos lleva al sistema de ecuaciones:

$$ \alpha - \beta = 0 $$
$$ 2\alpha - \beta = 0 $$

Resolviendo:

$$ \alpha = \beta $$
$$ 2\alpha - \alpha = 0 \Rightarrow \alpha = 0, \beta = 0 $$

Por lo tanto, el núcleo de $ g $ es:
$$ 0(1,0,0,0) + 0(0,1,0,0) + \gamma(0,0,1,0) + \delta(0,0,0,1) = \gamma(0,0,1,0) + \delta(0,0,0,1) $$
$$ \nucleo(g) = \ket{(0,0,1,0), (0,0,0,1)} $$

En conclusion

\begin{itemize}
  \item Como $ \imagen(g) = \reales^2 $, es epimorfismo.
  \item Como $ \nucleo(g) \neq \set{0} $, no es monomorfismo.
  \item No es isomorfismo.
\end{itemize}

\subsection*{Calculo $ g\circ f $}
$$ g(f(x_1,x_2,x_3)) = g(x_1+x_2,x_1+x_3,0,0) =$$
$$ g(x_1+x_2-x_1-x_3,2x_1+2x_2-x_1-x_3) =   $$
$$ (x_2-x_3,x_1+2x_2-x_3) = g\circ f (x_1,x_2,x_3) $$

\subsection*{Cálculo de la imagen de $ g\circ f $}

Usando los canonicos como vectores de salida:

$$ g(1,0,0) = (0,1) $$
$$ g(0,1,0) = (1,2) $$
$$ g(0,0,1) = (-1,-1) $$

Por lo tanto, el generador de la imagen de $ g $ es:

$$ \imagen(g) = \ket{(0,1), (1,2), (-1,-1)} $$
Es LD (-1,-1):
$$ \imagen(g) = \ket{(0,1), (1,2)} $$

La dimension es 2. Implica que Im es $ \reales^2 $.

\textbf{Cálculo del núcleo de $ g \circ f $}

Buscamos los coeficientes $ \alpha, \beta, \gamma, \delta $ tales que:

$$ \alpha(0,1) + \beta(1,2) + \gamma(-1,-1) = 0 $$

Resolviendo obtenemos:
$$ \beta = \gamma$$
$$ \alpha = -\beta $$

Por lo tanto, el núcleo de $ g $ es:
$$ -\beta(1,0,0) + \beta(0,1,0) + \beta(0,0,1) = \beta(-1,1,1) $$
$$ \nucleo(g) = \ket{ (-1,1,1,) } $$

En conclusion

\begin{itemize}
  \item Como $ \imagen(g) = \reales^2 $, es epimorfismo.
  \item Como $ \nucleo(g) \neq \set{0} $, no es monomorfismo.
  \item No es isomorfismo.
\end{itemize}

\begin{aportes}
  \item \aporte{https://github.com/juandelia03}{Juan D Elia \github}
\end{aportes}
