\begin{enunciado}{\ejercicio}
    Sean \( f: \mathbb{R}^3 \to \mathbb{R}^4 \) definido por:
    \[
    f(x_1, x_2, x_3) = (x_1 + x_2, x_1 + x_3, 0, 0)
    \]
    y \( g: \mathbb{R}^4 \to \mathbb{R}^2 \) definido por:
    \[
    g(x_1, x_2, x_3, x_4) = (x_1 - x_2, 2x_1 - x_2).
    \]
    Calcular el núcleo y la imagen de \( f \), de \( g \) y de \( g \circ f \).  
    
    Decidir si son monomorfismos, epimorfismos o isomorfismos.
\end{enunciado}

\subsection*{Cálculo de la imagen de \( f \)}

Aplicamos \( f \) a los vectores canónicos de \( \mathbb{R}^3 \):

\[ f(1,0,0) = (1,1,0,0) \]
\[ f(0,1,0) = (1,0,0,0) \]
\[ f(0,0,1) = (0,1,0,0) \]

Por lo tanto, el generador de la imagen de \( f \) es:

\[ \text{Im}(f) = \langle (1,1,0,0), (1,0,0,0), (0,1,0,0) \rangle \]
Como (1,1,0,0) es LI:
\[ \text{Im}(f) = \langle  (1,0,0,0), (0,1,0,0) \rangle \]


La dimensión de la imagen es 2.

\subsection*{Cálculo del núcleo de \( f \)}

Buscamos los coeficientes \( \alpha, \beta, \gamma \) tales que:

\[ \alpha(1,1,0,0) + \beta(1,0,0,0) + \gamma(0,1,0,0) = (0,0,0,0) \]

Esto da el sistema:

\[ \alpha + \beta = 0 \]
\[ \alpha + \gamma = 0 \]

Resolviendo:

\[ \alpha = -\beta \]
\[ \alpha = -\gamma \]

Reemplazo en los vectores de salida:
\[ \alpha(1,0,0) - \alpha(0,1,0) - \alpha(0,0,1) = \alpha(1,-1,-1) \]


Por lo tanto, el núcleo de \( f \) es:

\[ \text{Nu}(f) = \langle (1,-1,-1) \rangle \]


Entonces podemos concluir:
\newline
i) Como \( \text{Im}(f) \neq \mathbb{R}^3 \), no es epimorfismo.
\newline
ii)Como \( \text{Nu}(f) \neq \{0\} \), no es monomorfismo.


\begin{aportes}
    \item \aporte{https://github.com/juandelia03}{Juan D Elia \github}
\end{aportes}