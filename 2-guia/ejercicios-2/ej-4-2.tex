\begin{enunciado}{\ejercicio}
    Hallar todos los \( a \in \mathbb{R} \) para los cuales exista una transformación lineal 

    \[
    f: \mathbb{R}^3 \to \mathbb{R}^3
    \]
    
    que satisfaga:
    
    \[
    f(1,-1,1) = (2, a, -1),
    \]
    
    \[
    f(1,-1,2) = (a^2, -1, 1),
    \]
    
    \[
    f(1,-1,-2) = (5, -1, -7).
    \]
\end{enunciado}

Si los vectores de la salida son linealmente independientes, la transformación lineal existe para cualquier \( a \). Si alguno de ellos es linealmente dependiente, hay que buscar \( a \) para que no indetermine el sistema.

\[
\begin{bmatrix}
1 & -1 & 1 \\
1 & -1 & 2 \\
1 & -1 & -2
\end{bmatrix}
\longrightarrow
\begin{bmatrix}
1 & -1 & 1   \\
0 & 0 & 1   \\
0 & 0 & -3 
\end{bmatrix}
\]

Como el tercer vector es LD se puede escribir:

\[
\alpha (1,-1,1) + \beta (1,-1,2) = (1,-1,-2).
\]

Hallamos \( \alpha \) y \( \beta \) resolviendo:

\[
\begin{bmatrix}
1 & 1 \\
-1 & -1 \\
1 & 2 
\end{bmatrix}
=
\begin{bmatrix}
1 \\
-1 \\
-2
\end{bmatrix}
\]

Resolviendo tenemos \( \alpha = 4 \), \( \beta = -3 \).

Entonces:

\[
    f(1,-1,-2) = f(4(1,-1,1) - 3(1,-1,2)) = \\ 
    = 4(2,a,-1) - 3(a^2,-1,1) = (8-3a^2,4a+3,-7)
\]

Solo es T.L si ese vector es igual al (5,-1,-7) Esto da el sistema:

\[
8 - 3a^2 = 5,
\]

\[
4a + 3 = -1.
\]

Resolviendo:

\[
4a = -4 \Rightarrow a = -1.
\]

\[
8 - 3(-1)^2 = 5 \Rightarrow 8 - 3 = 5, \quad \text{(se cumple).}
\]

Por lo tanto, la transformación lineal existe si y solo si \( a = -1 \).

\begin{aportes}
    \item \aporte{https://github.com/juandelia03}{Juan D Elia \github}
\end{aportes}
