\begin{enunciado}{\ejercicio}
  Para cada una de las siguientes sucesiones de vectores $\set{\bm{x}_n}_{n\en\naturales}$ en $\reales^2$,
  determinar si existe $\lim_{n \to \infty} \bm{x}_n$, y en caso afirmativo hallarlo.
  \begin{enumerate}[label=\alph*)]
    \begin{multicols}{2}
      \item $\bm{x}_n = (1 + \frac{1}{n}, 3)$,

      \item $\bm{x}_n = ((-1)^n, e^{-n})$,

      \item $\bm{x}_n =
        \llave{rl}{
          (\frac{1}{n}, 0) & \text{si $n$ es par}\\
          (0,- \frac{1}{n}) & \text{si $n$ es impar}
        },
      $

      \item $
        \bm{x}_n = (\frac{1}{2^n}, 4, \sin(\pi n)).
      $
    \end{multicols}
  \end{enumerate}
\end{enunciado}

{\huge\red{\atencion CONSULTAR, demasiado simple?}}

\begin{enumerate}[label=\alph*)]
  \item Calculo de una:
        $$
          \cajaResultado{
            \bm{x}_n = (1 + \frac{1}{n}, 3) \flecha{$n \to \infty$} (1,3)
          }
        $$

  \item $\bm{x}_n = ((-1)^n, e^{-n})$, \orange{\underline{\negro{no existe}}} ver ejercicio \ref{ej:12} \ref{ej12-item-c}

  \item $\bm{x}_n =
          \llave{rl}{
            (\frac{1}{n}, 0) & \text{si $n$ es par}\\
            (0,- \frac{1}{n}) & \text{si $n$ es impar}
          },
        $
        $$
          \cajaResultado{
            \bm{x}_n \flecha{}[$n \to \infty$] (0,0)
          }
        $$

  \item $
          \bm{x}_n = (\frac{1}{2^n}, 4, \sin(\pi n)).
        $

        Dado que $\sin(\pi \cdot n) = 0 \quad \paratodo n \en \naturales$
        $$
          \cajaResultado{
            \bm{x}_n \flecha{}[$n \to \infty$] (0,4,0)
          }
        $$
\end{enumerate}

\begin{aportes}
  \item \aporte{\dirRepo}{naD GarRaz \github}
\end{aportes}
