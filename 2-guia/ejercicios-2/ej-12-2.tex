\begin{enunciado}{\ejercicio}
  Para cada una de ls siguientes sucesiones $\set{x_n}_{n \en \naturales}$, determinar si existe
  $\lim_{n\to \infty}$, y en caso afirmativo hallarlo.
  \begin{enumerate}[label=\alph*)]
    \begin{multicols}{2}
      \item $x_n = \frac{1}{n}$,
      \item $x_n = \frac{n^2 + 1}{n^2 - 1}$,
      \item $x_n = (-1)^n$,
      \item $x_n = (-1)^n e^{-n}$.
    \end{multicols}
  \end{enumerate}
\end{enunciado}

Hay que calcular los límites sin ninguna cosa rara, son ejercicios \textit{premonitorios} de los límites sin muchas complicaciones
que hay para \textit{acotar} las condiciones de matrices.

\begin{enumerate}[label=\alph*)]
  \item $x_n = \frac{1}{n}$
        $$
          \limite{n}{\infty} \frac{1}{n} = 0
        $$

  \item $x_n = \frac{n^2 + 1}{n^2 - 1}$
        $$
          \limite{n}{\infty} \frac{n^2 + 1}{n^2 - 1} = 1
        $$

  \item\label{ej12-item-c} $x_n = (-1)^n$

        Uso subsucesiones, para mostrar que no existe. La idea es que de existir el límite, sin importar como me acerque a $\infty$ todo
        camino debería llegar al mismo resultado:
        $$
          \llave{rcccc}{
            \blue{a_{2n}} & = & (-1)^{2n} & \flecha{$n \to \infty$} & 1 \\
            \green{a_{2n-1}} & = & (-1)^{2n-1} & \flecha{$n \to \infty$} & -1
          }
        $$
        Calculo los límites
        $$
          \limite{n}{\infty} x_{\blue{a_{2n}}} = 1
          \ytext
          \limite{n}{\infty} x_{\green{a_{2n-1}}} = -1
        $$
        Dado que los límites no coiciden el límite no existe.

  \item $x_n = (-1)^n e^{-n}$
        $$
          \limite{n}{\infty} (-1)^n \cdot \ua{\frac{1}{e^n}}{\to 0} = \text{acotado} \cdot 0 = 0.
        $$
\end{enumerate}

\begin{aportes}
  \item \aporte{\dirRepo}{naD GarRaz \github}
\end{aportes}

