\begin{enunciado}{\ejercicio}
  Determinar cuáles de las siguientes aplicaciones son lineales.
  \begin{enumerate}[label=(\alph*)]
  \item $f: \mathbb{R}^3 \to \mathbb{R}^2$ definida como:   \[
    f(x_1, x_2, x_3) = (x_2 - 3x_1 + \sqrt{2}x_3, x_1 - \frac{1}{2} x_2)
    \]
  \item \[f(x_1, x_2) = (x_1 + x_2, |x_1|)\]

  \item \[f \begin{pmatrix} a_{11} & a_{12} \\ a_{21} & a_{22} \end{pmatrix} =
  a_{11} a_{22} - a_{12} a_{21}
  \]
  \item \[f \begin{pmatrix} a_{11} & a_{12} \\ a_{21} & a_{22} \end{pmatrix} =
  \begin{pmatrix} a_{22} & 0 & a_{12} + a_{21} \\ 0 & a_{11} & a_{22} - a_{11} \end{pmatrix}
  \]

  \end{enumerate}
\end{enunciado}

\begin{enumerate}[label=(\alph*)]
  \item
  Primero veamos que la suma es lineal. Tomemos dos vectores cualesquiera: 
  
  \[
  v = (x_1, y_1, z_1), \quad w = (x_2, y_2, z_2)
  \]
  
  Entonces,
  
  \[
  f(v + w) = f(x_1 + x_2, y_1 + y_2, z_1 + z_2)
  \]
  
  \[
  = (y_1 + y_2 - 3(x_1 + x_2) + \sqrt{2} (z_1 + z_2), x_1 + x_2 - \frac{1}{2} (y_1 + y_2))
  \]
  
  Ahora veo:
  
  \[
  f(v) + f(w) = (y_1 - 3x_1 + \sqrt{2} z_1, x_1 - \frac{1}{2} y_1) + (y_2 - 3x_2 + \sqrt{2} z_2, x_2 - \frac{1}{2} y_2)
  \]
  
  \[
  = (y_1 + y_2 - 3(x_1 + x_2) + \sqrt{2} (z_1 + z_2), x_1 + x_2 - \frac{1}{2} (y_1 + y_2))
  \]
  
  Son iguales, la suma es lineal
  
  Veamos que el producto es lineal. Tomemos un escalar $\alpha \in \mathbb{R}$ y un vector $v = (x, y, z)$. Entonces,
  
  \[
  f(\alpha v) = f(\alpha x, \alpha y, \alpha z)
  \]
  
  \[
  = (\alpha y - 3\alpha x + \sqrt{2} \alpha z, \alpha x - \frac{1}{2} \alpha y)
  \]
  
  \[
  = \alpha (y - 3x + \sqrt{2}z, x - \frac{1}{2} y) = \alpha f(x, y, z)
  \]
  
  El producto es lineal
  \[
  f \text{ es una transformación lineal.}
  \]
  %incciso b
  \newpage
  \item 
  Tomemos dos vectores cualesquiera y veamos la suma: 

\[
v = (x_1, y_1), \quad w = (x_2, y_2)
\]

Entonces,

\[
f(v + w) = f(x_1 + x_2, y_1 + y_2)
\]

\[
= (x_1 + x_2 + y_1 + y_2, |x_1 + x_2|)
\]

Ahora veamos:

\[
f(v) + f(w) = (x_1 + y_1, |x_1|) + (x_2 + y_2, |x_2|)
\]

\[
= (x_1 + x_2 + y_1 + y_2, |x_1| + |x_2|)
\]

$|x_1 + x_2| \neq |x_1| + |x_2|$, la suma no es lineal.

\[
\Rightarrow f \text{ no es una transformación lineal.}
\]
  %inciso c
  \item Veamos que vale la suma, tomo dos matrices cualesquiera $A$ y $B$:

  \[
  f(A + B) = f \left( \begin{pmatrix} a_{11} & a_{12} \\ a_{21} & a_{22} \end{pmatrix} +
  \begin{pmatrix} b_{11} & b_{12} \\ b_{21} & b_{22} \end{pmatrix} \right)
  \]
  
  \[
  = f \begin{pmatrix} a_{11} + b_{11} & a_{12} + b_{12} \\ a_{21} + b_{21} & a_{22} + b_{22} \end{pmatrix}
  \]
  
  \[
  = (a_{11} + b_{11})(a_{22} + b_{22}) - (a_{12} + b_{12})(a_{21} + b_{21})
  \]
  
  Ahora vemos:
  
  \[
  f(A) + f(B) = (a_{11} a_{22} - a_{12} a_{21}) + (b_{11} b_{22} - b_{12} b_{21})
  \]
  
  \[
  = a_{11} a_{22} - a_{12} a_{21} + b_{11} b_{22} - b_{12} b_{21}
  \]
  
  Se ve que:
  
  \[
  (a_{11} + b_{11})(a_{22} + b_{22}) - (a_{12} + b_{12})(a_{21} + b_{21}) \neq a_{11} a_{22} - a_{12} a_{21} + b_{11} b_{22} - b_{12} b_{21}
  \]
  
  La suma no es lineal.\[
  \Rightarrow f \text{ no es una transformación lineal.}
  \]
  
  %inciso d
  \item Veo que valga la suma:

  Sea $A, B$ matrices cualesquiera:
  
  \[
  A = \begin{pmatrix} a_{11} & a_{12} \\ a_{21} & a_{22} \end{pmatrix}, \quad
  B = \begin{pmatrix} b_{11} & b_{12} \\ b_{21} & b_{22} \end{pmatrix}
  \]
  
  \[
  f(A + B) = 
  \begin{pmatrix} (a_{22} + b_{22}) & 0 & (a_{12} + b_{12}) + (a_{21} + b_{21}) \\ 0 & (a_{11} + b_{11}) & (a_{22} + b_{22}) - (a_{11} + b_{11}) \end{pmatrix}
  \]
  
  Ahora miro,
  
  \[
  f(A) + f(B) =
  \begin{pmatrix} a_{22} & 0 & a_{12} + a_{21} \\ 0 & a_{11} & a_{22} - a_{11} \end{pmatrix} +
  \begin{pmatrix} b_{22} & 0 & b_{12} + b_{21} \\ 0 & b_{11} & b_{22} - b_{11} \end{pmatrix}
  \]
  
  \[
  =
  \begin{pmatrix} a_{22} + b_{22} & 0 & (a_{12} + a_{21}) + (b_{12} + b_{21}) \\ 0 & a_{11} + b_{11} & (a_{22} - a_{11}) + (b_{22} - b_{11}) \end{pmatrix}
  \]
  
  La suma es lineal.
  
  Ahora veo el producto:
  
  \[
  f(\alpha A) =
  f \begin{pmatrix} \alpha a_{11} & \alpha a_{12} \\ \alpha a_{21} & \alpha a_{22} \end{pmatrix}
  =
  \begin{pmatrix} \alpha a_{22} & 0 & \alpha (a_{12} + a_{21}) \\ 0 & \alpha a_{11} & \alpha (a_{22} - a_{11}) \end{pmatrix}
  = \alpha f(A)
  \]
  
  El producto y la suma son lineales, f es transformacion lineal

\end{enumerate}

\begin{aportes}
  \item \aporte{https://github.com/juandelia03}{Juan D Elia \github} 
\end{aportes}