\begin{enunciado}{\ejercicio}
  Determinar cuáles de las siguientes aplicaciones son lineales.
  \begin{enumerate}[label=(\alph*)]
  \item $f: \mathbb{R}^3 \to \mathbb{R}^2$ definida como:   \[
    f(x_1, x_2, x_3) = (x_2 - 3x_1 + \sqrt{2}x_3, x_1 - \frac{1}{2} x_2)
    \]
  \item hacer
  \item hacer
  \item hacer

  \end{enumerate}
\end{enunciado}

\begin{enumerate}[label=(\alph*)]
  \item
  Primero veamos que la suma es lineal. Tomemos dos vectores cualesquiera: 
  
  \[
  v = (x_1, y_1, z_1), \quad w = (x_2, y_2, z_2)
  \]
  
  Entonces,
  
  \[
  f(v + w) = f(x_1 + x_2, y_1 + y_2, z_1 + z_2)
  \]
  
  \[
  = (y_1 + y_2 - 3(x_1 + x_2) + \sqrt{2} (z_1 + z_2), x_1 + x_2 - \frac{1}{2} (y_1 + y_2))
  \]
  
  Ahora veo:
  
  \[
  f(v) + f(w) = (y_1 - 3x_1 + \sqrt{2} z_1, x_1 - \frac{1}{2} y_1) + (y_2 - 3x_2 + \sqrt{2} z_2, x_2 - \frac{1}{2} y_2)
  \]
  
  \[
  = (y_1 + y_2 - 3(x_1 + x_2) + \sqrt{2} (z_1 + z_2), x_1 + x_2 - \frac{1}{2} (y_1 + y_2))
  \]
  
  Son iguales, la suma es lineal
  
  Veamos que el producto es lineal. Tomemos un escalar $\alpha \in \mathbb{R}$ y un vector $v = (x, y, z)$. Entonces,
  
  \[
  f(\alpha v) = f(\alpha x, \alpha y, \alpha z)
  \]
  
  \[
  = (\alpha y - 3\alpha x + \sqrt{2} \alpha z, \alpha x - \frac{1}{2} \alpha y)
  \]
  
  \[
  = \alpha (y - 3x + \sqrt{2}z, x - \frac{1}{2} y) = \alpha f(x, y, z)
  \]
  
  El producto es lineal
  \[
  f \text{ es una transformación lineal.}
  \]
  \item hacer
  \item hacer
  \item hacer

\end{enumerate}

\begin{aportes}
  \item \aporte{https://github.com/juandelia03}{Juan D Elia \github} 
\end{aportes}