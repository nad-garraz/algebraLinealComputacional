\begin{enunciado}{\ejercicio}
  Determinar cuáles de las siguientes aplicaciones son lineales.
  \begin{enumerate}[label=(\alph*)]
    \item $f: \reales^3 \to \reales^2,\quad f(x_1, x_2, x_3) = (x_2 - 3x_1 + \sqrt{2}x_3, x_1 - \frac{1}{2} x_2)$

    \item $f: \reales^2 \to \reales^2,\quad f(x_1, x_2) = (x_1 + x_2, |x_1|)$

    \item $f: \reales^{2 \times 2} \to \reales,\quad
            f \matriz{cc}{
              a_{11} & a_{12} \\
              a_{21} & a_{22}
            }
            =
            a_{11} \cdot a_{22} - a_{12} \cdot a_{21}
          $

    \item $f: \reales^{2 \times 2} \to \reales^{2 \times 3},\quad
            f
            \matriz{cc}{
              a_{11} & a_{12} \\
              a_{21} & a_{22}
            }
            =
            \matriz{ccc}{
              a_{22} & 0 & a_{12} + a_{21} \\
              0 & a_{11} & a_{22} - a_{11}
            }
          $
  \end{enumerate}
\end{enunciado}

\begin{enumerate}[label=(\alph*)]
  \item
        Primero veamos que la suma es lineal. Tomemos dos vectores cualesquiera:
        $$
          v = (x_1, y_1, z_1), \quad w = (x_2, y_2, z_2)
        $$
        Entonces,
        $$
          f(v + w) = f(x_1 + x_2, y_1 + y_2, z_1 + z_2) =
          (y_1 + y_2 - 3(x_1 + x_2) + \sqrt{2} (z_1 + z_2), x_1 + x_2 - \frac{1}{2} (y_1 + y_2))
        $$
        Ahora veo que:
        $$
          \begin{array}{rcl}
            f(v) + f(w) & = & (y_1 - 3x_1 + \sqrt{2} z_1, x_1 - \frac{1}{2} y_1) + (y_2 - 3x_2 + \sqrt{2} z_2, x_2 - \frac{1}{2} y_2) \\
                        & = & (y_1 + y_2 - 3(x_1 + x_2) + \sqrt{2} (z_1 + z_2), x_1 + x_2 - \frac{1}{2} (y_1 + y_2))
          \end{array}
        $$
        Son iguales, la suma es lineal
        Veamos que el producto es lineal. Tomemos un escalar $\alpha \in \mathbb{R}$ y un vector $v = (x, y, z)$. Entonces,
        $$
          f(\alpha v) = f(\alpha x, \alpha y, \alpha z) =
          (\alpha y - 3\alpha x + \sqrt{2} \alpha z, \alpha x - \frac{1}{2} \alpha y) =
          \alpha (y - 3x + \sqrt{2}z, x - \frac{1}{2} y) = \alpha f(x, y, z)
        $$
        El producto es lineal
        $$
          \cajaResultado{
            f \text{ es una transformación lineal.}
          }
        $$

  \item
        Tomemos dos vectores cualesquiera y veamos la suma:
        $$
          v = (x_1, y_1), \quad w = (x_2, y_2)
        $$
        Entonces,
        $$
          f(v + w) = f(x_1 + x_2, y_1 + y_2) =
          (x_1 + x_2 + y_1 + y_2, |x_1 + x_2|)
        $$
        Ahora veamos que:
        $$
          \begin{array}{rcl}
            f(v) + f(w) & = & (x_1 + y_1, |x_1|) + (x_2 + y_2, |x_2|) \\
                        & = & (x_1 + x_2 + y_1 + y_2, |x_1| + |x_2|)
          \end{array}
        $$
        dado que $|x_1 + x_2| \neq |x_1| + |x_2|$, la suma no es lineal.
        $$
          \cajaResultado{
            \entonces f \text{ no es una transformación lineal.}
          }
        $$

  \item Veamos que vale la suma, tomo dos matrices cualesquiera $A$ y $B$:
        $$
          \begin{array}{rcl}
            f(A + B) = f \parentesis{
              \matriz{cc}{
            a_{11}          & a_{12}                                                                                    \\
            a_{21}          & a_{22}
              } +
              \matriz{cc}{
            b_{11}          & b_{12}                                                                                    \\
            b_{21}          & b_{22}
              }
            }
                            & =               &
            f \matriz{cc}{
            a_{11} + b_{11} & a_{12} + b_{12}                                                                           \\
            a_{21} + b_{21} & a_{22} + b_{22}
            }                                                                                                           \\
                            & =               & (a_{11} + b_{11})(a_{22} + b_{22}) - (a_{12} + b_{12})(a_{21} + b_{21})
          \end{array}
        $$
        Ahora vemos:
        $$
          f(A) + f(B) = (a_{11} a_{22} - a_{12} a_{21}) + (b_{11} b_{22} - b_{12} b_{21}) = a_{11} a_{22} - a_{12} a_{21} + b_{11} b_{22} - b_{12} b_{21}
        $$
        Se ve que:
        $$
          (a_{11} + b_{11})(a_{22} + b_{22}) - (a_{12} + b_{12})(a_{21} + b_{21}) \neq a_{11} a_{22} - a_{12} a_{21} + b_{11} b_{22} - b_{12} b_{21}
        $$
        La suma no es lineal.
        $$
          \cajaResultado{
            \entonces f \text{ no es una transformación lineal.}
          }
        $$

  \item Veo que valga la suma:
        Sea $A, B$ matrices cualesquiera:
        {\small
        $$
          A = \matriz{cc}{
            a_{11} & a_{12} \\
            a_{21} & a_{22}
          },
          \quad
          B = \matriz{cc}{
            b_{11} & b_{12} \\
            b_{21} & b_{22}
          }
          \entonces
          f(A + B) =
          \matriz{ccc}{
            (a_{22} + b_{22}) & 0 & (a_{12} + b_{12}) + (a_{21} + b_{21}) \\
            0 & (a_{11} + b_{11}) & (a_{22} + b_{22}) - (a_{11} + b_{11})
          }
        $$
        }
        Ahora miro,
        {\small
            $$
              f(A) + f(B) =
              \matriz{ccc}{
                a_{22} & 0 & a_{12} + a_{21} \\
                0 & a_{11} & a_{22} - a_{11}
              } +
              \matriz{ccc}{
                b_{22} & 0 & b_{12} + b_{21} \\
                0 & b_{11} & b_{22} - b_{11}
              }
              =
              \matriz{ccc}{
                a_{22} + b_{22} & 0 & (a_{12} + a_{21}) + (b_{12} + b_{21}) \\
                0 & a_{11} + b_{11} & (a_{22} - a_{11}) + (b_{22} - b_{11})
              }
            $$
          }
        La suma es lineal. Ahora veo el producto:
        $$
          f(\alpha A) =
          f \matriz{cc}{
            \alpha a_{11} & \alpha a_{12} \\
            \alpha a_{21} & \alpha a_{22}
          }
          =
          \matriz{ccc}{
            \alpha a_{22} & 0             & \alpha (a_{12} + a_{21}) \\
            0             & \alpha a_{11} & \alpha (a_{22} - a_{11})
          }
          = \alpha f(A)
        $$
        El producto y la suma son lineales,
        $$
          \cajaResultado{
            f \text{ es transformacion lineal}
          }
        $$
\end{enumerate}

\begin{aportes}
  \item \aporte{https://github.com/juandelia03}{Juan D Elia \github}
\end{aportes}
