\begin{enunciado}{\ejercicio}
  Si $x \en \reales^n$, probar que las constantes de equialencia
  entre las normas $\normaBullet_1$ y $\normaBullet_2$
  y entre las normas $\normaBullet_2$ y $\normaBullet_\infty$ vienen dadas por:
  $$
    \begin{array}{c}
      \norma{x}_\infty \leq \norma{x}_2 \leq \sqrt{n} \norma{x}_\infty \\
      \frac{1}{\sqrt{n}}\norma{x}_1 \leq \norma{x}_2 \leq \norma{x}_1
    \end{array}
  $$
\end{enunciado}

\hyperlink{teoria-2:normas}{Acá están las definiciones de las normas que se usan en el ejercicio {\tiny($\ot$ click {\tiny\faIcon{mouse}})}}

Si $x = (0,\ldots, 0)$ la desigualdad es el caso de la igualdad. Entonces si tengo un $x \en \reales^n$ y $x \distinto 0$:
$$
  x = (x_1, \cdots, x_n)
  \flecha{calculo}[$\normaBullet_2$]
  \norma{x}_2 = \sqrt{x_1^2 + \cdots + x_n^2} =
  |\ua{x_i}{\maximo_{1\leq i \leq n} |x_i|| \cdot
  \ub{\sqrt{\parentesis{\frac{x_1}{x_i}}^2 + \cdots + \ua{1}{i-\text{ésimo lugar}} + \cdots + \parentesis{\frac{x_n}{x_i}}^2}}{\geq 1}
  \mayorIgual{\red{!}}
  |x_i|=
  \norma{x}_\infty
$$
Ahí queda mostrado que:
$$
  \cajaResultado{
    \norma{x}_\infty \leq \norma{x}_2
  }
$$
Parecido:
$$
  \norma{x}_2 = \sqrt{|x_1|^2 + \cdots + |x_n|^2}
  \ua{\menorIgual{\red{!}}}{|x_i| = \maximo\set{|x_1|,\ldots,|x_n|}}
  \sqrt{|x_i|^2 + \cdots + |x_i|^2} =
  \sqrt{n \cdot |x_i|^2} =
  \sqrt{n} \cdot |x_i| =
  \sqrt{n} \cdot \norma{x}_\infty
$$
Ahí queda mostrado que:
$$
  \cajaResultado{
    \norma{x}_2 \leq \sqrt{n}\cdot \norma{x}_\infty
  }
$$

Ahora para la relación entre $\normaBullet_1$ y $\normaBullet_2$:

Recuerdo Desigualdad de \textit{Cauchy Schwartz:}

$$
  |x^T y| \menorIgual{$\llamada1$} \norma{x}_2 \cdot \norma{y}_2
$$
Con $y = \ub{(1,\ldots,1)}{\text{\magic}} \entonces \norma{y}_2 = \sqrt{n}$ y tomo el módulo de las coordenadas de $x$:
$$
  (|x_1|, \ldots, |x_n|) \cdot
  \matriz{c}{
    1\\
    \vdots\\
    1
  }
  \menorIgual{$\llamada1$}
  \sqrt{n} \cdot \sqrt{|x_1|^2 + \cdots + |x_n|^2}
  \sii
  \ub{|x_1| + \cdots + |x_n|}{\norma{x}_1} \leq \sqrt{n} \cdot \ub{\sqrt{|x_1|^2 + \cdots + |x_n|^2}}{\norma{x}_2}
$$
De donde pasando para acá y para allá queda que:
$$
  \cajaResultado{
    \frac{1}{\sqrt{n}}\norma{x}_1 \leq \norma{x}_2
  }
$$
La última que queda también usando al desigualdad de \textit{Cauchy Schwartz}:
$$
  |x^t \cdot y| \menorIgual{$\llamada2$} \norma{x}_1 \cdot \norma{y}_1
$$
Ahora uso $y = x$
$$
  |x^t \cdot x| \menorIgual{$\llamada2$} \norma{x}_1 \cdot \norma{x}_1
  \sii
  (\norma{x}_2)^2 \leq (\norma{x}_1)^2
  \sii
  \cajaResultado{
    \norma{x}_2 \leq \norma{x}_1
  }
$$

\begin{aportes}
  \item \aporte{\dirRepo}{naD GarRaz \github}
\end{aportes}
