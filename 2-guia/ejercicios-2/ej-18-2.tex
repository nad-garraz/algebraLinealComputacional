\begin{enunciado}{\ejercicio}
  Se quiere estimar la norma 2 de una matriz $A \en \reales^{n \times n}$ como el máximo del valor
  $ \norma{Ax}_2 / \norma{x}_2 $ entre varios vectores $x \en \reales^3$ no nulos generadosal azar.
  Hacer un programa quereciba una matriz $A$ y luego
  \begin{itemize}
    \item Genere los primeros 100 términos de la siguiente sucesión:
          $$
            s_1 = 0, \quad s_{k+1} = \maximo{}\set{s_k, \frac{\norma{Ax_k}_2}{\norma{x}_2}}
          $$
          donde los $x_k \en \reales^3$ son vectores no nulos generados al azar en la bola unitaria:
          $B = \set{x : \norma{x}_2 \leq 1}$.

    \item Grafique la sucesión calculada, junto con el valor exacto de la norma de la matriz.
  \end{itemize}

  Recordar que tanto la norma 2 puede calcularse con el comando \texttt{np.linalg.norm}. Tener en cuenta
  que los vectores generados al azar (comando \texttt{np.random.rand}) tienen coordenadas en el intervalor $[0,1]$.
\end{enunciado}


