\begin{enunciado}{\ejercicio}
  \begin{enumerate}[label=(\alph*)]
    \item Estimar la $\condicion_\infinito(A)$ de las siguientes matrices en función de $\varepsilon$ (cuando $\varepsilon \to 0$).
          \begin{enumerate}[label=(\roman*)]
            \begin{multicols}{2}
              \item $
                \matriz{ccc}{
                  1 & 1 & 1 \\
                  1 & \varepsilon &\varepsilon^2 \\
                  1 & 0 & 0 \\
                },
              $
              \item
              $
                \matriz{ccc}{
                  1 & 0 & 1 + \varepsilon \\
                  2 & 3 & 4 \\
                  1 - \varepsilon & 0 & 1 \\
                }
              $
            \end{multicols}
          \end{enumerate}

    \item Concluir que la condición de las matrices $A$ y $B$ del ítem anterior tienden a infinito, cualquiera
          sea la norma considerada.
  \end{enumerate}
\end{enunciado}

\begin{enumerate}[label=(\alph*)]
  \item
        \begin{enumerate}[label=(\roman*)]
          \item En el ejercicio \ref{ej:21} están las expresiones que sirven para dar una estimación de las normas. El truco
                está en encontrar una \textit{matriz singular} con la cual podamos hacer que algo \textit{explote} \faIcon{bomb}:
                $$
                  \condicion_\infinito(A)
                  \mayorIgual{$\llamada1$}
                  \supremo \set{\frac{\norma{A}_\infinito}{\norma{A - B}_\infinito} : B \text{ es singular}}.
                $$
                La matriz $B$:
                $$
                  B =
                  \matriz{ccc}{
                    1 & 1 & 1 \\
                    1 & 0 & 0 \\
                    1 & 0 & 0
                  }
                $$
                es singular. La expresión para la cota para norma infinito queda:
                $$
                  \frac{\norma{A}_\infinito}{\norma{A - B}_\infinito}
                  \igual{\red{!}}
                  \frac{3}{\varepsilon + \varepsilon^2}
                  \flecha{$\varepsilon \to 0 $} \infinito
                $$
                En \red{!} el valor de la $\norma{A}_\infinito = \maximo\set{3,1+ \varepsilon + \varepsilon^2}$ asumo que $\varepsilon < 1$.

                Por lo tanto cuando $\varepsilon \to 0$ el conjunto para la $B$ elegida, no tiene cota superior. Entonces
                en $\llamada1$ tenemos que:
                $$
                  \condicion_\infty(A)
                  \geq
                  \infinito
                  \entonces
                  \limite{\varepsilon}{0} \condicion_\infinito(A) = \infinito
                $$

          \item Misma estrategia ahora para estimar la $\condicion_\infinito(B)$ con la matriz singular:
                $$
                  C =
                  \matriz{ccc}{
                    1 & 0 & 1 \\
                    2 & 3 & 4 \\
                    1 & 0 & 1
                  }
                  \entonces
                  B - C =
                  \matriz{ccc}{
                    0 & 0 & \varepsilon \\
                    0 & 0 & 0 \\
                    - \varepsilon & 0 & 0
                  }
                $$
                $$
                  \condicion_\infinito(B)
                  \geq
                  \supremo \set{\frac{\norma{B}_\infinito}{\norma{B - C}_\infinito} : C \text{ es singular}} \geq
                  \frac{9}{\varepsilon}
                  \flecha{$\varepsilon \to 0$} \infinito
                  \entonces
                  \limite{\varepsilon}{0} \condicion_\infinito(B) = \infinito.
                $$

        \end{enumerate}

  \item \hacer
\end{enumerate}

\begin{aportes}
  \item \aporte{\dirRepo}{naD GarRaz \github}
\end{aportes}
