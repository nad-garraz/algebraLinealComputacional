\begin{enunciado}{\ejercicio}
    Se tiene el sistema $ Ax = b $.

    \begin{itemize}
        \item[(a)] Sea \( x \) la solución exacta y \( \tilde{x} \) la solución obtenida numéricamente. \
        Se llama \textbf{residuo} al vector: $r := b - A\tilde{x}$.
        Si notamos $e = x - \tilde{x}$, mostrar que:
        $$
        \frac{1}{cond(A)}  \frac{\|r\|}{\|b\|} \leq \frac{\|e\|}{\|x\|} \leq  cond(A) \frac{\|r\|}{\|b\|}.
        $$

        \item[(b)] En lugar del dato exacto \( b \), se conoce una aproximación \( \tilde{b} \). 
        Se tiene que \( \tilde{x} \) satisface: $A\tilde{x} = \tilde{b}.$
        Probar que:
        \[
        \frac{1}{cond(A)}  \frac{||b - \tilde{b}||}{\|b\|} \leq \frac{||x - \tilde{x}||}{\|x\|} \leq cond(A)  \frac{||b - \tilde{b}||}{||b||}.
        \]
    
        ¿Cómo se puede interpretar este resultado?
    \end{itemize}
\end{enunciado}

\medskip

Algunas consideraciones antes de empezar:
\begin{enumerate}[label=(\roman*)]
\item $b = Ax$
\item $r = Ax - A\tilde{x} = A(x - \tilde{x})$
\item $e = A^{-1} r$
\item cuando uso la propiedad de consistencia pongo una c arriba de la desigualdad,ojo: varias veces achico el denominador para armar algo mas grande
\end{enumerate}

\begin{enumerate}[label=(\alph*)]
    \item 
    Veo primero 
    $\frac{1}{cond(A)} \frac{\|r\|}{\|b\|} \leq \frac{\|e\|}{\|x\|}$:

    $\frac{1}{cond(A)} \frac{\|r\|}{\|b\|} = 
    \frac{1}{||A^{-1}|| ||A||} \frac{||A e||}{||A x||} \menorIgual{c}
    \frac{1}{||A^{-1}|| ||A||} \frac{||A|| ||e||}{||Ax||} =
    \frac{1}{||A^{-1}|| } \frac{||e||}{||Ax||} \menorIgual{c}
    \frac{||e||}{||A^{-1} Ax|| } = 
    \frac{||e||}{||x||}
    $ 

    ahora veo 
    $  \frac{\|e\|}{\|x\|} \leq  cond(A)  \frac{\|r\|}{\|b\|}$ :

    $\frac{||e||}{||x||} = 
     \frac{||A^{-1} r||}{||x||} \menorIgual{c}
     \frac{||A^{-1}|| ||r||}{||x||} =
     \frac{||A||}{||A||} \frac{||A^{-1}|| ||r||}{||x||} = 
     \frac{cond(A) ||r||}{||A|| ||x||} \menorIgual{c}
     \frac{cond(A) ||r||}{||A x||} =
     \frac{cond(A) ||r||}{||b||}
    $

    Queda demostrado
    \item Es igual que el item anterior 
    
    Veo primero: 
    $\frac{1}{cond(A)}  \frac{||b - \tilde{b}||}{\|b\|} \leq \frac{||x - \tilde{x}||}{\|x\|}$:

    $\frac{1}{cond(A)}  \frac{||b - \tilde{b}||}{\|b\|} =
     \frac{1}{||A|| ||A^{-1}||} \frac{||Ax - A\tilde{x}||}{||b||} =
     \frac{1}{||A|| ||A^{-1}||} \frac{||A(x - \tilde{x})||}{||b||} \menorIgual{c}
     \frac{1}{||A|| ||A^{-1}||} \frac{||A|| ||(x - \tilde{x})||}{||b||} =
    $

    $
    \frac{||(x - \tilde{x})||}{||A^{-1}|| ||b||} \menorIgual{c}
    \frac{||(x - \tilde{x})||}{||A^{-1} b||} = 
    \frac{||(x - \tilde{x})||}{||x||} 
    $

    Queda ver $\frac{||x - \tilde{x}||}{||x||} \leq cond(A)  \frac{||b - \tilde{b}||}{||b||}$

    $
    \frac{||x - \tilde{x}||}{||x||} =
    \frac{||A^{-1}(b - \tilde{b})||}{||x||} \menorIgual{c}
    \frac{||A^{-1}|| ||(b - \tilde{b})||}{||x||} = 
    \frac{||A|| ||A^{-1}|| ||(b - \tilde{b})||}{||A|| ||x||} =
    \frac{cond(A) ||(b - \tilde{b})||}{||A|| ||x||} \menorIgual{c}
    \frac{cond(A) ||(b - \tilde{b})||}{||Ax||} =
    cond(A)  \frac{||b - \tilde{b}||}{||b||}
    $
    
    queda demostrado
\end{enumerate}


\begin{aportes}
    \item \aporte{https://github.com/juandelia03}{Juan D Elia \github}
\end{aportes}