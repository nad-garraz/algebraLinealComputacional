\begin{enunciado}{\ejercicio}
  Se tiene el sistema $Ax = b$.

  \begin{enumerate}[label=(\alph*)]
    \item Sea  $x$ la solución exacta y \( \tilde{x} \) la solución obtenida numéricamente. \
          Se llama \textbf{residuo} al vector: $r := b - A\tilde{x}$.
          Si notamos $e = x - \tilde{x}$, mostrar que:
          $$
            \frac{1}{\condicion(A)}  \frac{\norma{r}}{\norma{b}} \leq \frac{\norma{e}}{\norma{x}} \leq  \condicion(A) \frac{\norma{r}}{\norma{b}}.
          $$

    \item En lugar del dato exacto \( b \), se conoce una aproximación \( \tilde{b} \).
          Se tiene que \( \tilde{x} \) satisface: $A\tilde{x} = \tilde{b}.$
          Probar que:
          $$
            \frac{1}{\condicion(A)}  \frac{\norma{b - \tilde{b}}}{\norma{b}}
            \leq
            \frac{\norma{x - \tilde{x}}}{\norma{x}}
            \leq
            \condicion(A)  \frac{\norma{b - \tilde{b}}}{\norma{b}}.
          $$

          ¿Cómo se puede interpretar este resultado?
  \end{enumerate}
\end{enunciado}

\medskip

Algunas consideraciones antes de empezar:
\begin{enumerate}[label=(\roman*)]
  \item $b = Ax$
  \item $r = Ax - A\tilde{x} = A(x - \tilde{x})$
  \item $e = A^{-1} r$
  \item Cuando uso la propiedad de consistencia pongo \blue{cons} arriba de la desigualdad.
        \underline{Ojo}: Varias veces achico el denominador para armar algo más grande!
\end{enumerate}

\begin{enumerate}[label=(\alph*)]
  \item
        Veo primero
        $\frac{1}{\condicion(A)} \frac{\norma{r}}{\norma{b}} \leq \frac{\norma{e}}{\norma{x}}$:

        $\frac{1}{\condicion(A)} \frac{\norma{r}}{\norma{b}} =
          \frac{1}{\norma{A^{-1}} \norma{A}} \frac{\norma{A e}}{\norma{A x}}
          \menorIgual{\blue{cons}}
          \frac{1}{\norma{A^{-1}} \norma{A}} \frac{\norma{A} \norma{e}}{\norma{Ax}} =
          \frac{1}{\norma{A^{-1}} } \frac{\norma{e}}{\norma{Ax}}
          \menorIgual{\blue{cons}}
          \frac{\norma{e}}{\norma{A^{-1} Ax} } =
          \frac{\norma{e}}{\norma{x}}
        $

        ahora veo
        $  \frac{\norma{e}}{\norma{x}} \leq  \condicion(A)  \frac{\norma{r}}{\norma{b}}$ :

        $\frac{\norma{e}}{\norma{x}} =
          \frac{\norma{A^{-1} r}}{\norma{x}} \menorIgual{\blue{cons}}
          \frac{\norma{A^{-1}} \norma{r}}{\norma{x}} =
          \frac{\norma{A}}{\norma{A}} \frac{\norma{A^{-1}} \norma{r}}{\norma{x}} =
          \frac{\condicion(A) \norma{r}}{\norma{A} \norma{x}} \menorIgual{\blue{cons}}
          \frac{\condicion(A) \norma{r}}{\norma{A x}} =
          \frac{\condicion(A) \norma{r}}{\norma{b}}
        $

        Queda demostrado
  \item Es igual que el item anterior

        Veo primero:
        $\frac{1}{\condicion(A)}  \frac{\norma{b - \tilde{b}}}{\norma{b}} \leq \frac{\norma{x - \tilde{x}}}{\norma{x}}$:

        $\frac{1}{\condicion(A)}  \frac{\norma{b - \tilde{b}}}{\norma{b}} =
          \frac{1}{\norma{A} \norma{A^{-1}}} \frac{\norma{Ax - A\tilde{x}}}{\norma{b}} =
          \frac{1}{\norma{A} \norma{A^{-1}}} \frac{\norma{A(x - \tilde{x})}}{\norma{b}} \menorIgual{\blue{cons}}
          \frac{1}{\norma{A} \norma{A^{-1}}} \frac{\norma{A} \norma{(x - \tilde{x})}}{\norma{b}} =
        $

        $
          \frac{\norma{(x - \tilde{x})}}{\norma{A^{-1}} \norma{b}} \menorIgual{\blue{cons}}
          \frac{\norma{(x - \tilde{x})}}{\norma{A^{-1} b}} =
          \frac{\norma{(x - \tilde{x})}}{\norma{x}}
        $

        Queda ver $\frac{\norma{x - \tilde{x}}}{\norma{x}} \leq \condicion(A)  \frac{\norma{b - \tilde{b}}}{\norma{b}}$

        $
          \frac{\norma{x - \tilde{x}}}{\norma{x}} =
          \frac{\norma{A^{-1}(b - \tilde{b})}}{\norma{x}}
          \menorIgual{\blue{cons}}
          \frac{\norma{A^{-1}} \norma{(b - \tilde{b})}}{\norma{x}} =
          \frac{\norma{A} \norma{A^{-1}} \norma{(b - \tilde{b})}}{\norma{A} \norma{x}} =
          \frac{\condicion(A) \norma{(b - \tilde{b})}}{\norma{A} \norma{x}}
          \menorIgual{\blue{cons}}
          \frac{\condicion(A) \norma{(b - \tilde{b})}}{\norma{Ax}} =
          \condicion(A)  \frac{\norma{b - \tilde{b}}}{\norma{b}}
        $

        queda demostrado
\end{enumerate}

\begin{aportes}
  \item \aporte{https://github.com/juandelia03}{Juan D Elia \github}
\end{aportes}
