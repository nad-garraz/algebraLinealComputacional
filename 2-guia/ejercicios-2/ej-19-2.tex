\begin{enunciado}{\ejercicio}
  Se tiene el sistema $\bm{A}\bm{x} = \bm{b}$.
  \begin{enumerate}[label=(\alph*)]
    \item Sea  $\bm{x}$ la solución exacta y $\tilde{\bm{x}}$ la solución obtenida numéricamente.
          Se llama \textit{residuo} al vector: $\bm{r} \coloneq \bm{b} - \bm{A}\tilde{\bm{x}}$.
          Si notamos $\bm{e} = \bm{x} - \tilde{\bm{x}}$, mostrar que:
          $$
            \frac{1}{\condicion(\bm{A})}  \frac{\norma{\bm{r}}}{\norma{\bm{b}}}
            \leq
            \frac{\norma{\bm{e}}}{\norma{\bm{x}}}
            \leq
            \condicion(\bm{A}) \frac{\norma{\bm{r}}}{\norma{\bm{b}}}.
          $$

    \item En lugar del dato exacto $b$, se conoce una aproximación $\tilde{b}$.
          Se tiene que $\tilde{x}$ satisface: $A\tilde{x} = \tilde{b}.$
          Probar que:
          $$
            \frac{1}{\condicion(\bm{A})}  \frac{\norma{\bm{b} - \tilde{\bm{b}}}}{\norma{\bm{b}}}
            \leq
            \frac{\norma{\bm{x} - \tilde{\bm{x}}}}{\norma{\bm{x}}}
            \leq
            \condicion(\bm{A})  \frac{\norma{\bm{b} - \tilde{\bm{b}}}}{\norma{\bm{b}}}.
          $$
          ¿Cómo se puede interpretar este resultado?
  \end{enumerate}
\end{enunciado}

\medskip

\parrafoDestacado[\red{\atencion}]{
  Esto lo quiero mencionar acá, porque se va a usar todo el tiempo en el ejercicio:
  $$
    \norma{Ax} \leq \norma{A} \norma{x}
  $$
  y mucha gente le dice \textit{Cauchy-Schwartz},
  pero no es esa desigualdad, esto sale de la definición de \textit{norma matricial}:
  $$
    \norma{A} = \maximo[x\distinto 0] \frac{\norma{Ax}}{\norma{x}}
    \Entonces{\tiny en particular para}[\tiny un $\blue{y}$ cualquiera]
    \norma{A} =  \maximo[x\distinto 0] \frac{\norma{Ax}}{\norma{x}}
    \geq
    \frac{\norma{A\blue{y}}}{\norma{\blue{y}}}
    \sii
    \boxed{
      \norma{A\blue{y}} \leq \norma{A} \norma{\blue{y}}
    }
  $$
}

\begin{enumerate}[label=\alph*)]
  \item
        No sé si a vos te gusta tener esas letras por todos lados, pero a mí no, reescribo las desigualdades
        del enunciado usando las definiciones del enunciado \rollingEyes como:
        $$
          \frac{1}{\condicion(\bm{A})} \frac{\norma{\bm{b} - \bm{A\tilde{x}}}}{\norma{\bm{b}}}
          \leq
          \frac{\norma{\bm{x - \tilde{x}}}}{\norma{\bm{x}}}
          \leq
          \condicion(\bm{A}) \frac{\norma{\bm{b} - \bm{A\tilde{x}}}}{\norma{\bm{b}}}.
        $$
        Listo, menos letras \textit{feas} menos ruido. Primero ataco la desigualdad de la izquierda:
        $$
          \frac{1}{\condicion(\bm{A})} \frac{\norma{\bm{b} - \bm{A\tilde{x}}}}{\norma{\bm{b}}}
          \leq
          \frac{\norma{\bm{x - \tilde{x}}}}{\norma{\bm{x}}}
        $$

        Sale usando que:
        $$
          \bm{Ax} =  \bm{b}
          \sii
          \bm{x} =  \bm{A^{-1}b}
          \sii
          \norma{\bm{x}} =  \norma{\bm{A^{-1}b}} \leq \norma{\bm{A^{-1}}}  \norma{\bm{b}}
          \sii
          \norma{\bm{x}} \menorIgual{$\llamada1$} \norma{\bm{A^{-1}}}  \norma{\bm{b}}
        $$

        Si compraste eso que quedo en $\llamada1$, listo solo hay que acomodar y reemplazar en la desigualdad y se prueba:
        {\small
        $$
          \frac{1}{\condicion(\bm{A})} \frac{\norma{\bm{b} - \bm{A\tilde{x}}}}{\norma{\bm{b}}}
          \igual{\red{!}}
          \frac{1}{\norma{\bm{A}} \norma{\bm{A^{-1}}}} \frac{\norma{\bm{Ax} - \bm{A\tilde{x}}}}{\norma{\bm{b}}} =
          \frac{1}{\norma{\bm{A}} } \frac{\norma{\bm{A}(\bm{x} - \bm{\tilde{x})}}}{\norma{\bm{A^{-1}}}\norma{\bm{b}}}
          \menorIgual{\red{!}}
          \frac{1}{\norma{\bm{A}} } \frac{\norma{\bm{A}}\norma{\bm{x} - \bm{\tilde{x}}}}{\norma{\bm{A^{-1}}}\norma{\bm{b}}}=
          \frac{\norma{\bm{x} - \bm{\tilde{x}}}}{\norma{\bm{A^{-1}}}\norma{\bm{b}}}
          \menorIgual{$\llamada1$}
          \frac{\norma{\bm{x} - \bm{\tilde{x}}}}{\norma{\bm{x}}}
        $$
        }
        Listo el pollo, quedó que:
        $$
          \cajaResultado{
            \frac{1}{\condicion(\bm{A})} \frac{\norma{\bm{b} - \bm{A\tilde{x}}}}{\norma{\bm{b}}}
            \leq
            \frac{\norma{\bm{x} - \bm{\tilde{x}}}}{\norma{\bm{x}}}
          }
        $$
        Probar la otra parte de la desigualdad es casi lo mismo, pero mirá con atención, porque
        marea el hecho de ir acotar en sentido contrario, pero es lo meeeeeeeeeesmo:
        $$
          \begin{array}{rcl}
            \condicion(\bm{A}) \frac{\norma{\bm{b} - \bm{A\tilde{x}}}}{\norma{\bm{b}}}
            \igual{\red{!}}
            \norma{\bm{A}} \norma{\bm{A^{-1}}} \frac{\norma{\bm{A}(\bm{x} - \bm{\tilde{x}})}}{\norma{\bm{Ax}}}
            \geq
            \norma{\bm{A}} \norma{\bm{A^{-1}}} \frac{\norma{\bm{A}(\bm{x} - \bm{\tilde{x}})}}{\norma{\bm{A}} \norma{\bm{x}}}
             & =                    &
            \frac{\norma{\bm{A^{-1}}} \norma{\bm{A}(\bm{x} - \bm{\tilde{x}})}}{ \norma{\bm{x}}} \\
             & \mayorIgual{\red{!}} &
            \frac{\norma{\bm{A^{-1}}\bm{A}(\bm{x} - \bm{\tilde{x}})}}{ \norma{\bm{x}}} =
            \frac{\norma{\bm{x - \tilde{x}}}}{\norma{\bm{x}}}
          \end{array}
        $$
        Así llegando a que:
        $$
          \cajaResultado{
            \frac{\norma{\bm{x - \tilde{x}}}}{\norma{\bm{x}}}
            \leq
            \condicion(\bm{A}) \frac{\norma{\bm{b} - \bm{A\tilde{x}}}}{\norma{\bm{b}}}.
          }
        $$

        \bigskip

        Con respecto a la interpretación:

        Lo que estamos haciendo es calcular \textit{cotas} para el \textit{error relativo} que puede haber
        al resolver numéricamente $\bm{A}\bm{x} = \bm{b}$, con solución númerica del sistema $\tilde{\bm{x}}$
        y solución exacta $\bm{x}$,
        $\frac{\norma{\bm{x} - \tilde{\bm{x}}}}{\norma{\bm{x}}}$.

        Una condición grande, será una matriz, \textit{verga}, porque nos da un intervalo para el error relativo grande.
        
        Una condición cercana a 1, será una matriz, \textit{linda}, con un error relativo sin mucho espacio para delirarla.

  \item Es igual que el item anterior

        \begin{enumerate}[label=(\roman*)]
          \item $\bm{A}\bm{x} = \bm{b} \sii \bm{x} = \bm{A}^{-1}\bm{b}$
          \item $\bm{A}\tilde{\bm{x}} = \tilde{\bm{b}} \sii \tilde{\bm{x}} = \bm{A}^{-1}\tilde{\bm{b}}$
        \end{enumerate}

        Veo primero:
        $$
          \frac{1}{\condicion(\bm{A})}  \frac{\norma{\bm{b} - \tilde{\bm{b}}}}{\norma{\bm{b}}} \leq \frac{\norma{\bm{x} - \tilde{\bm{x}}}}{\norma{\bm{x}}}
        $$

        $$
          \begin{array}{rcl}
            \frac{1}{\condicion(\bm{A})}  \frac{\norma{\bm{b} - \tilde{\bm{b}}}}{\norma{\bm{b}}} =
            \frac{1}{\norma{\bm{A}} \norma{\bm{A}^{-1}}} \frac{\norma{\bm{A}(\bm{x} - \tilde{\bm{x}})}}{\norma{\bm{b}}} \menorIgual{\red{!}}
            \frac{1}{\norma{\bm{A}} \norma{\bm{A}^{-1}}} \frac{\norma{\bm{A}} \norma{(\bm{x} - \tilde{\bm{x}})}}{\norma{\bm{b}}} =
            \frac{\norma{(\bm{x} - \tilde{\bm{x}})}}{\norma{\bm{A}^{-1}} \norma{\bm{b}}} \menorIgual{\red{!}}
            \frac{\norma{(\bm{x} - \tilde{\bm{x}})}}{\norma{\bm{A}^{-1} \bm{b}}} =
            \frac{\norma{(\bm{x} - \tilde{\bm{x}})}}{\norma{\bm{x}}}
          \end{array}
        $$
        Queda así:
        $$
          \cajaResultado{
            \frac{1}{\condicion(\bm{A})}  \frac{\norma{\bm{b} - \tilde{\bm{b}}}}{\norma{\bm{b}}} =
            \leq
            \frac{\norma{(\bm{x} - \tilde{\bm{x}})}}{\norma{\bm{x}}}
          }
        $$

        Por último queda ver que:
        $$
          \frac{\norma{\bm{x} - \tilde{\bm{x}}}}{\norma{\bm{x}}} \leq \condicion(\bm{A})  \frac{\norma{\bm{b} - \tilde{\bm{b}}}}{\norma{\bm{b}}}
        $$

        $$
          \begin{array}{c}
            \frac{\norma{\bm{x} - \tilde{\bm{x}}}}{\norma{\bm{x}}} =
            \frac{\norma{\bm{A}^{-1}(\bm{b} - \tilde{\bm{b}})}}{\norma{\bm{x}}}
            \menorIgual{\red{i}}
            \frac{\norma{\bm{A}^{-1}} \norma{(\bm{b} - \tilde{\bm{b}})}}{\norma{\bm{x}}} =
            \frac{\norma{\bm{A}} \norma{\bm{A}^{-1}} \norma{(\bm{b} - \tilde{\bm{b}})}}{\norma{\bm{A}} \norma{\bm{x}}} =
            \frac{\condicion(\bm{A}) \norma{(\bm{b} - \tilde{\bm{b}})}}{\norma{\bm{A}} \norma{\bm{x}}}
            \menorIgual{\red{i}}
            \frac{\condicion(\bm{A}) \norma{(\bm{b} - \tilde{\bm{b}})}}{\norma{\bm{A}\bm{x}}} =
            \condicion(\bm{A})  \frac{\norma{\bm{b} - \tilde{\bm{b}}}}{\norma{\bm{b}}}
          \end{array}
        $$
        Finalmente obteniendo:
        $$
          \cajaResultado{
            \frac{\norma{\bm{x} - \tilde{\bm{x}}}}{\norma{\bm{x}}} =
            \leq
            \condicion(\bm{A})  \frac{\norma{\bm{b} - \tilde{\bm{b}}}}{\norma{\bm{b}}}
          }
        $$
        queda demostrado el ejercicio.
\end{enumerate}

\begin{aportes}
  \item \aporte{https://github.com/juandelia03}{Juan D Elia \github}
  \item \aporte{\dirRepo}{naD GarRaz \github}
\end{aportes}
