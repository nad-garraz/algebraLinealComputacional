\begin{enunciado}{\ejercicio}
  Dada una sucesión de vectores $\set{\bm{x}_n}_{n \en \naturales} \subset \reales^k$ y dos normas $\normaBullet_a$ y $\normaBullet_b$ de
  $\reales^k$, usando la equivalencia de normas, probar
  $$
    \norma{\bm{x}_n}_a \flecha{}[$n \to \infty$] 0
    \sisolosi
    \norma{\bm{x}_n}_b \flecha{}[$n \to \infty$] 0.
  $$
\end{enunciado}

\medskip


En un espacio vectorial de dimension finita todas las normas son equivalentes, entonces existen c1,c2 tq:
\begin{equation*}
  c_1||x_n||_b \leq ||x_n||_a \leq c2||x_n||_b
\end{equation*}



\begin{enumerate}[label=(\roman*)]
  \item Veo la ida
  Veo la ida, reemplazo en la desigualdad tomada por limite:

  $\limite{n}{\infty} c_1||x_n||_b \leq 0 \leq \limite{n}{\infty}  c2||x_n||_b $
  
  sacando las constantes para afuera:
  
  $c_1 \limite{n}{\infty} ||x_n||_b \leq 0 \leq  c2 \limite{n}{\infty} ||x_n||_b $
  
  como  $c_1, c_2 > 0$ (renombrando $\limite{n}{\infty}$ como x):
  
  $c1x \leq 0 \rightarrow x \leq 0 $
  
  $c2x \geq 0 \rightarrow x \geq  0 $
  
  Por lo tanto, $\limite{n}{\infty} ||x_n||_b = 0$ ,como queriamos ver, vale la ida

  \item Veo la vuelta.
  Vuelvo a reemplazar tomando limite, pero en ese caso $\limite{n}{\infty} ||x_n||_b = 0$

  $0 \leq \limite{n}{\infty} ||x_n||_a \leq 0$

  Se ve entonces que: $\limite{n}{\infty} ||x_n||_a=0$

\end{enumerate}

\begin{aportes}
  \item \aporte{https://github.com/juandelia03}{Juan D Elia \github}
\end{aportes}

 

