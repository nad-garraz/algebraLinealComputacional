\begin{enunciado}{\ejercicio}
  Probar que si $A \en \reales^{n \times n}$ es una matriz inversible y $\normaBullet$ es una norma matricial,
  la condición de $A$ verifica la desigualdad:
  $$
    \frac{1}{\condicion(A)}
    \leq
    \infimo\set{\frac{\norma{A - B}}{\norma{A}} : B \text{ es singular}}.
  $$
  Deducir que
  $$
    \condicion(A)
    \geq
    \supremo \set{\frac{\norma{A}}{\norma{A - B}} : B \text{ es singular}}.
  $$
  Nota: En ambos casos, vale la igualdad, pero la otra desigualdad es un poco más complicada de probar. De
  la igualdad se puede concluir que $\condicion(A)$ mide la distancia relativa de $A$ a la matriz
  singular más próxima.
\end{enunciado}

Si $B$ es singular, significa que \textit{existe} un $x \en \reales^n$ tal que $Bx = 0$ a esto tirale un poco de \magic y sale que:
$$
  \begin{array}{rcl}
    x
    \igual{\red{!}}
    \blue{A^{-1}A}x - \blue{A^{-1}}Bx
    =
    A^{-1}(A - B)x
     & \Sii{tomo}[$\normaBullet$]       &
    \norma{x}
    =
    \norma{A^{-1}(A - B)x}                              \\
     & \Sii{\red{!!}}                   &
    \norma{x}
    \leq
    \norma{A^{-1}} \norma{A - B} \norma{x}              \\
     & \Sii{$\div \magenta{\norma{A}}$} &
    \frac{\norma{x}}{\magenta{\norma{A}} \norma{A^{-1}}}
    \leq
    \frac{\norma{A - B} \norma{x}}{\magenta{\norma{A}}} \\
     & \sii                             &
    \frac{1}{\condicion(A)}
    \leq
    \frac{\norma{A - B}}{\norma{A}}
  \end{array}
$$
Ahí quedó que los elementos del conjunto
$
  \set{\frac{\norma{A - B}}{\norma{A}} : B \text{ es singular}},
$
son mayores o iguales al número $\frac{1}{\condicion(A)}$, pero faltaría ver
la igualdad así aparece ahí el \textit{ínfimo}.

La igualdad vale debería valer para alguna $\green{B}$ singular, es decir:
$$
  \frac{1}{\condicion(A)}
  =
  \frac{\norma{A - \green{B}}}{\norma{A}}.
$$

La igualdad se asume válida porque la demostración es \textit{falopa}.

Ahora habría que mostrar que:
$$
  \condicion(A)
  \geq
  \supremo \set{\frac{\norma{A}}{\norma{A - B}} : B \text{ es singular}}.
$$

\hacer

\begin{aportes}
  \item \aporte{\dirRepo}{naD GarRaz \github}
\end{aportes}
