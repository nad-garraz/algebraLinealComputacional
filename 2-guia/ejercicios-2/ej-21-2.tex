\begin{enunciado}{\ejercicio}
  Probar que si $A \en \reales^{n \times n}$ es una matriz inversible y $\normaBullet$ es una norma matricial,
  la condición de $A$ verifica la desigualdad:
  $$
    \frac{1}{\condicion(A)}
    \leq
    \infimo \set{\frac{\norma{A - B}}{\norma{A}} : B \text{ es singular}}.
  $$
  Deducir que
  $$
    \condicion(A)
    \geq
    \supremo \set{\frac{\norma{A}}{\norma{A - B}} : B \text{ es singular}}.
  $$
  Nota: En ambos casos, vale la igualdad, pero la otra desigualdad es un poco más complicada de probar. De
  la igualdad se puede concluir que $\condicion(A)$ mide la distancia relativa de $A$ a la matriz
  singular más próxima.
\end{enunciado}
 \hacer
