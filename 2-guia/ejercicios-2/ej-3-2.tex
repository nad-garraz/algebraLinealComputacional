\begin{enunciado}{\ejercicio}
  \begin{enumerate}[label=(\alph*)]
    \item Probar que existe una única transformación lineal $f : \reales^2 \to \reales^2$ tal que
          $f(1,1) = (-5, 3)$ y $f(-1,1) = (5,2)$. Para dicha $f$, determinar $f(5,3)$ y $f(-1,2)$.

    \item ¿Existirá una transformación lineal $f : \reales^2 \to \reales^2$ tal que $f(1,1) = (2,6), f(-1,1) = (2,1)$ y
          $f(2,7) = (5,3)$?

    \item Sean $f, g: \reales^3 \to \reales^3$ transformaciones lineales tales que
          $$
            \begin{array}{l}
              f(1,0,1) = (1,2,1), f(2,1,0) = (2,1,0), f(-1,0,0) = (1,2,1) \\
              g(1,1,1) = (1,1,0), g(3,2,0) = (0,0,1), g(2,2,-1) = (3,-1,2)
            \end{array}
          $$
  \end{enumerate}
\end{enunciado}

De la teoría se tiene que:
\begin{center}
  Sea $V$ un $K-$espacio vectorial y $B=\set{v_1,\ldots, v_n}$ base de $V$.
  Podemos definir en forma única una t.l. de $V$ en $W$ definiendo cada $f(v_i) \en W$
  con $i=1,\ldots n$.
\end{center}

\begin{enumerate}[label=(\alph*)]
  \item Sale casi solo usando propiedades de \textit{transformación lineal:}
        $$
          \begin{array}{rlc}
            \llave{rcl}{
            f(1,1)  & = & (-5,3)           \\
            f(-1,1) & = & (5,2)
            }
                    &
            \triangulacion{
            F_2 + F_1 \to F_2              \\
            }
                    &
            \llave{rcl}{
            f(1,1)  & = & (-5,3)           \\
            f(0,2)  & = & (0,5)
            }                              \\
                    &
            \triangulacion{
              \frac{1}{2} F_2 \to F_2
            }
                    &
            \llave{rcl}{
            f(1,1)  & = & (-5,3)           \\
            f(0,1)  & = & (0,\frac{5}{2})
            }                              \\
                    &
            \triangulacion{
              F_1 - F_2 \to F_1
            }
                    &
            \llave{rcl}{
            f(1,0)  & = & (-5,\frac{1}{2}) \\
            f(0,1)  & = & (0,\frac{5}{2})
            }
          \end{array}
        $$
        Si bien no es necesario, puedo escribir a la \textit{transformación lineal} como:
        $$
          f
          \matriz{c}{
            x\\
            y
          }
          =
          \matriz{cc}{
            -5 & 0\\
            \frac{1}{2} & \frac{5}{2}
          }
          \cdot
          \matriz{c}{
            x\\
            y
          }
          =
          \matriz{c}{
            -5x\\
            \frac{1}{2}x + \frac{5}{2}y
          }
        $$
        Y ahora calculo lo más pancho:
        $$
          \cajaResultado{
            f(5,3) =
            \matriz{c}{
              -25\\
              10
            }
            \ytext
            f(-1,2) =
            \matriz{c}{
              5\\
              \frac{9}{2}
            }
          }
        $$

  \item  Se llega a un absurdo con algunas operaciones.
        $$
          \llave{rcl}{
            f(1,1) &=& (2,6)\\
            f(-1,1) &=& (2,1)\\
            f(2,7) &=& (5,3)
          }
          \triangulacion{
            F_2 - F_1 \to F_2\\
            F_3 - 2F_1 \to F_3
          }
          \llave{rcl}{
            f(1,1) &=& (2,6)\\
            f(0,2) &=& (4,7)\\
            f(0,5) &=& (1,-9)
          }
          \triangulacion{
            \frac{1}{2}\cdot F_2 \to F_2\\
            \frac{1}{5}\cdot F_3 \to F_3
          }
          \llave{rcl}{
            f(1,1) &=& (2,6)\\ \rowcolor{red!20}
            f(0,1) &=& (2,\frac{7}{2})\\\rowcolor{red!20}
            f(0,1) &=& (\frac{1}{5},\frac{-9}{5})
          }
        $$

        Las operaciones de triangulación aplicadas en la triangulación son lineales y se usó todo el tiempo la definición de linealidad.

  \item Ataco igual que al anterior, la idea es poder compararlos con la misma \textit{base del espacio de partida $V$}:
        $$
          \llave{rcl}{
            f(1,0,1) &=& (1,2,1)\\
            f(2,1,0) &=& (2,1,0)\\
            f(-1,0,0) &=& (1,2,1)
          }
          \flecha{\magic}
          \llave{rcl}{
            f(1,0,0) &=& (1,2,1)\\
            f(0,1,0) &=& (0,-3,-2)\\\rowcolor{Cerulean!10}
            f(0,0,1) &=& (2,4,2)
          }
        $$
        Ahora con $g$:
        $$
          \llave{rcl}{
            g(1,0,1) &=& (1,2,1)\\
            g(2,1,0) &=& (2,1,0)\\
            g(-1,0,0) &=& (1,2,1)
          }
          \triangulacion{
            F_2 - 3 F_1 \to F_1 \\
            F_3 - 2 F_1 \to F_3
          }
          \llave{rcl}{
            g(1,1,1) &=& (1,1,0)\\
            g(0,-1,-2) &=& (-3,-3,1)\\ \rowcolor{Cerulean!10}
            g(0,0,-3) &=& (1,-3,2)
          }
        $$

        Podría seguir triangulando y llegar hasta que me queden ambas expresiones en la canónica de $\reales^3$, pero pajilla.
        Resalté en azul dos filas que me \textit{gritan} que si:
        $$
          (0,0,1) \flecha{$f$} (2,4,2)
          \entonces
          (0,0,-3) \flecha{$f$}[\red{!}] (-6,-12,-6)
        $$
        No obstante:
        $$
          (0,0,-3) \flecha{$g$} (1,-3,2) \distinto (0,0,0)
        $$
        Así se concluye que :
        $$
          \cajaResultado{
            f \distinto g
          }
        $$

\end{enumerate}

\begin{aportes}
  \item \aporte{\dirRepo}{naD GarRaz \github}
\end{aportes}
