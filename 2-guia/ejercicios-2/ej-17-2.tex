\begin{enunciado}{\ejercicio}
  Probar que para toda matriz $A \en \reales^{n \times n}$
  \begin{enumerate}[label=(\alph*)]
    \begin{multicols}{2}
      \item $\norma{A}_\infty = \maximo[1 \leq i \leq n]\set{ \sumatoria{j=1}{n}|a_{ij}|}$
      \item $\norma{A}_1 = \maximo[1 \leq j \leq n]\set{ \sumatoria{i=1}{n}|a_{ij}|}$
    \end{multicols}
  \end{enumerate}
\end{enunciado}

\begin{enumerate}[label=(\alph*)]
  \item
        Definición de norma infinito:
        $$
          \norma{A}_\infty =
          \maximo[\norma{x}_\infty = 1] \set{\norma{Ax}_\infty}
        $$
        Veamos que:
        $$
          \norma{Ax}_\infty =
          \maximo[1 \leq i \leq n]
          \sumatoria{j=1}{n}  |a_{ij} \cdot x_j|
          \menorIgual{$\oa{}{\text{Des. triangular}}$}
          \maximo[1 \leq i \leq n] \sumatoria{j=1}{n}  |a_{ij}| \cdot |x_j|
          \menorIgual{$\oa{}{\text{Por ser le máx }|x_j|}$}
          \maximo[1 \leq i \leq n] \sumatoria{j=1}{n}  |a_{ij}| \cdot \norma{x}_\infty =
          \maximo[1 \leq i \leq n] \norma{x}_\infty \cdot \sumatoria{j=1}{n}  |a_{ij}|
        $$
        Luego, nos queda que
        $$
          \norma{Ax}_\infty \leq \norma{x}_\infty \maximo[1 \leq i \leq n] \sumatoria{j=1}{n}  |a_{ij}|
        $$
        Volvamos a $\norma{A}_\infty$.
        Reemplazamos $\norma{Ax}_\infty$ por la desigualdad que obtuvimos
        $$
          \norma{A}_\infty \leq \maximo[\norma{x}_\infty = 1] \norma{x}_\infty \maximo[1 \leq i \leq n] \sumatoria{j=1}{n}  |a_{ij}|
        $$
        Como $\norma{x}_\infty = 1$, nos queda que:
        $$
          \norma{A}_\infty \leq \maximo[1 \leq i \leq n] \sumatoria{j=1}{n}  |a_{ij}|
        $$
        Llamaremos a esta expresión de la derecha $M$.
        $$
          \norma{A}_\infty \leq M
        $$
        Ahora, busquemos demostrar la igualdad encontrando un $x$ particular.
        Sea $\tilde{x} = e_{\hat{i}}$ tal que $\norma{Ax}_\infty = M$.
        Es decir, vemos que se cumple que
        $$
          \hat{i} = \max \set{i \in [1, n] : \sumatoria{j=1}{n}  |a_{ij}|}
        $$
        Es decir, la fila cuya suma de módulos es la mayor.
        Además, vemos que, por ser $\tilde{x}$ un vector canónico, se cumple que $\norma{x}_\infty = 1$.
        Luego como sabemos que:
        \begin{itemize}
          \item $\norma{A}_\infty \leq M$ para todo $x$ con $\norma{x}_\infty = 1$
          \item Existe un $\tilde{x}$ tal que $\norma{A\tilde{x}}_\infty = M$
        \end{itemize}
        $$
          \norma{A}_\infty = \maximo[\norma{x}_\infty = 1]{ \norma{Ax}_\infty} = M
        $$
        Puesto que encontramos un $x$ que cumple la igualdad, y se que todo el resto son menores o iguales.

  \item Quiero probar la fórmula cerrada:
        $$
          \norma{A}_1 =
          \maximo[1 \leq j \leq n] \sumatoria{i=1}{n} |a_{ij}| =
          \ub{\maximo \set{\sumatoria{i = 1}{n} |a_{i1}|, \ldots , \sumatoria{i = 1}{n} |a_{in}|}}{\llamada1}
        $$
        Usando \textit{normas inducidas o subordinadas} voy a ponerle una cota a $\norma{A}_1$ para cualquier $x \en K^n$ con $\norma{x}_1 = 1$:
        $$
          \begin{array}{rcl}
            \norma{Ax}_1 =
            {\Bigg\|}
            \textstyle
            \matriz{c}{
            \sumatoria{j = 1}{n} a_{1j}x_j                                            \\
            \vdots                                                                    \\
            \sumatoria{j = 1}{n} a_{nj}x_j                                            \\
            }
            {\Bigg\|}_1
             & \igual{def}                           &
            | \sumatoria{j = 1}{n} a_{1j}x_j |
            + \cdots +
            |\sumatoria{j = 1}{n} a_{nj}x_j|                                          \\
             & \menorIgual{$\oa{}{\text{desigualdad}                                  \\ \text{triangular}}$} &
               \sumatoria{j = 1}{n} |a_{1j}| \cdot |x_j |
               + \cdots +
            \sumatoria{j = 1}{n} |a_{nj}| \cdot |x_j|                                 \\
             & \igual{\red{!}}                       &
            \sumatoria{i = 1}{n}
            \sumatoria{j = 1}{n}
            |a_{ij}| \cdot |x_j |                                                     \\
             & \igual{\red{!!}}                      &
            \sumatoria{j = 1}{n} \sumatoria{i = 1}{n} |a_{ij}| \cdot |x_j|
            =
            \sumatoria{j = 1}{n} |x_j| \cdot \sumatoria{i = 1}{n} |a_{ij}|            \\
             & =                                     &
            |x_1| \cdot \ub{\sumatoria{i = 1}{n} |a_{i1}|}{\norma{ \columna(A_1)}_1} +
            \cdots +
            |x_n| \cdot \ub{\sumatoria{i = 1}{n} |a_{in}|}{\norma{ \columna(A_n)}_1}  \\
             & \menorIgual{\red{!}}                  &
            \ub{\sumatoria{j = 1}{n} |x_j|}{\norma{x}_1} \cdot \maximo[1 \leq i \leq n] \set{\norma{ \columna(A_i)}_1} =
            \norma{x}_1 \cdot \maximo[1 \leq i \leq n] \set{\norma{ \columna(A_i)}_1} \\
             & \igual{HIP}[$\ua{}{\norma{x}_1 = 1}$] &
            \maximo[1 \leq i \leq n]\set{\norma{ \columna(A_i)}_1} = \llamada1
          \end{array}
        $$
        Después de ese parto se obtiene que:
        $$
          \norma{Ax}_1
          \menorIgual{$\llamada1$}
          \maximo \set{\sumatoria{i = 1}{n} |a_{i1}|, \ldots , \sumatoria{i = 1}{n} |a_{in}|}
        $$
        Ahora el razonamiento que sigue es algo así: Dado que la expresión que quedó
        $$
          \maximo \set{\sumatoria{i = 1}{n} |a_{i1}|, \ldots , \sumatoria{i = 1}{n} |a_{in}|}
        $$
        \underline{no depende} de $x$, solo son sumas de los elementos por columna de $A$ y como ya sé, la $j-$ésima columna de $A$ la puedo escribir como:
        $$
          A \cdot \hat{e}_j =
          A \cdot
          \matriz{c}{
            0\\
            \vdots\\
            1\\
            \vdots\\
            0
          }
          =
          \matriz{c}{
            a_{1j} \\
            \vdots\\
            a_{nj}
          }
          = \columna(A)_j,
        $$
        Es así que como ese hermoso $\hat{e}_j$ cumple eso va a haber \underline{algún otro vector genérico} que cumpla que justo nos dé
        \underline{el máximo del conjunto $\llamada1$}. Para creerme esto último, me gusta pensar que $Ax$ vive en el subespacio
        $\columna(A) = \ub{\imagen(A)}{\text{pensando a $A$ como una T.L.}}$.

        Entonces si en la columna $\magenta{j}$, tengo un vector \textit{cualquiera} al que
        llamo $\magenta{\tilde{x}}$ con $\norma{\magenta{\tilde{x}}} = 1$ y ¡Oh sorpresa \surprise!
        $A \cdot \magenta{\tilde{x}} \en \llamada1$ y justo es el elemento máximo {\LARGE \surprise} de $\llamada1$ por lo tanto el más poronga entre todos los
        $\norma{y} = 1$:
        $$
          \maximo[\norma{y} = 1] \frac{\norma{Ay}}{\norma{y}}
          \igual{$\oa{}{\text{el}\\ \text{más}\\ \text{groso}\\\magenta{\tilde{x}}}$}
          \norma{A \magenta{\tilde{x}}}_1 =
          \maximo \set{\sumatoria{i = 1}{n} |a_{i1}|, \ldots , \sumatoria{i = 1}{n} |a_{i\magenta{j}}|, \ldots, \sumatoria{i = 1}{n} |a_{in}|} =
          \sumatoria{i=1}{n} |a_{i\magenta{j}}|
        $$
        Y por definición (la otra con $x\distinto 0$)
        de norma inducida con norma 1:
        $\norma{A}_1 = \maximo[x \distinto 0] \frac{\norma{Ax}_1}{\norma{x}_1}$, así pudiendo expresar:
        $$
          \cajaResultado{
          \norma{A}_1 =
          \maximo \set{\sumatoria{i = 1}{n} |a_{i1}|, \ldots, \sumatoria{i = 1}{n} |a_{in}|}
          }
        $$
\end{enumerate}

\begin{aportes}
  \item \aporte{https://github.com/pedrofuentes79}{Pedro F. \github}
  \item \aporte{\dirRepo}{naD GarRaz \github}
\end{aportes}
