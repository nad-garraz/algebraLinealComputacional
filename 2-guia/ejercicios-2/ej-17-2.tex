\begin{enunciado}{\ejercicio}
  Probar que para toda matriz $A \en \reales^{n \times n}$
  \begin{enumerate}[label=(\alph*)]
    \begin{multicols}{2}
      \item $\norma{A}_\infty = \maximo{1 \leq i \leq n} \sumatoria{j=1}{n}|a_{ij}|$
      \item $\norma{A}_1 = \maximo{1 \leq j \leq n} \sumatoria{i=1}{n}|a_{ij}|$
    \end{multicols}
  \end{enumerate}
\end{enunciado}

\begin{enumerate}[label=(\alph*)]
  \item
        Definición de norma infinito:
        $$
          \norma{A}_\infty =
          \maximo{\norma{x}_\infty = 1} \norma{Ax}_\infty
        $$
        Veamos que:
        $$
          \norma{Ax}_\infty =
          \maximo{1 \leq i \leq n}
          \sumatoria{j=1}{n}  |a_{ij} \cdot x_j|
          \menorIgual{$\oa{}{\text{Des. triangular}}$}
          \maximo{1 \leq i \leq n} \sumatoria{j=1}{n}  |a_{ij}| \cdot |x_j|
          \menorIgual{$\oa{}{\text{Por ser le máx }|x_j|}$}
          \maximo{1 \leq i \leq n} \sumatoria{j=1}{n}  |a_{ij}| \cdot \norma{x}_\infty =
          \maximo{1 \leq i \leq n} \norma{x}_\infty \cdot \sumatoria{j=1}{n}  |a_{ij}|
        $$
        Luego, nos queda que
        $$
          \norma{Ax}_\infty \leq \norma{x}_\infty \maximo{1 \leq i \leq n} \sumatoria{j=1}{n}  |a_{ij}|
        $$
        Volvamos a $\norma{A}_\infty$.
        Reemplazamos $\norma{Ax}_\infty$ por la desigualdad que obtuvimos
        $$
          \norma{A}_\infty \leq \maximo{\norma{x}_\infty = 1} \norma{x}_\infty \maximo{1 \leq i \leq n} \sumatoria{j=1}{n}  |a_{ij}|
        $$
        Como $\norma{x}_\infty = 1$, nos queda que:
        $$
          \norma{A}_\infty \leq \maximo{1 \leq i \leq n} \sumatoria{j=1}{n}  |a_{ij}|
        $$
        Llamaremos a esta expresión de la derecha $M$.
        $$
          \norma{A}_\infty \leq M
        $$
        Ahora, busquemos demostrar la igualdad encontrando un $x$ particular.
        Sea $\tilde{x} = e_{\hat{i}}$ tal que $\norma{Ax}_\infty = M$.
        Es decir, vemos que se cumple que
        $$
          \hat{i} = \max \set{i \in [1, n] : \sumatoria{j=1}{n}  |a_{ij}|}
        $$
        Es decir, la fila cuya suma de modulos es la mayor.
        Además, vemos que, por ser $\tilde{x}$ un vector canónico, se cumple que $\norma{x}_\infty = 1$.
        Luego como sabemos que:
        \begin{itemize}
          \item $\norma{A}_\infty \leq M$ para todo $x$ con $\norma{x}_\infty = 1$
          \item Existe un $\tilde{x}$ tal que $\norma{A\tilde{x}}_\infty = M$
        \end{itemize}
        $$
          \norma{A}_\infty = \maximo{\norma{x}_\infty = 1} \norma{Ax}_\infty = M
        $$
        Puesto que encontramos un $x$ que cumple la igualdad, y se que todo el resto son menores o iguales.

  \item \hacer
\end{enumerate}
