\begin{enunciado}{\ejercicio}
  Sea $A_n \en \reales^n$ la matriz dada por $A_n = (a_{ij})$,
  $$
    a_{ij} =
    \llave{cl}{
      1 & \text{ si } i = 1 \text{ o } j = 1 \\
      1/i & \text{ si } i = j \\
      0 & \text{ en otro caso}
    }
  $$
  \begin{enumerate}[label=\alph*)]
    \item Probar que $\condicion_\infinito (A_n) \geq f(n)$ para alguna función $f(n) \en O(n^2)$.
    \item Probar que $\condicion_2(A_n) \to \infinito$ cuando $n \to \infinito$.
  \end{enumerate}
\end{enunciado}

\begin{enumerate}[label=\alph*)]
  \item
        Hay que encontrar una $B$ (antes de verla, mirá el ejercicio \ref{ej:23} para inspirarte)
        $$
          \condicion_\infinito(A)
          \geq
          \supremo \set{\frac{\norma{A}_\infinito}{\norma{A - B}_\infinito} : B \text{ es singular}}
        $$
        El caso con $n = 2$ se puede calcular a mano:
        $$
          A =
          \matriz{cc}{
            1 & 1 \\
            1 & \frac{1}{2}
          }
          \ytext
          A^{-1} =
          \matriz{cc}{
            -1 & 2 \\
            2 & -2
          }
          \entonces
          \condicion_\infinito(A) =
          \norma{A}_\infinito \cdot
          \norma{A^{-1}}_\infinito = 2 \cdot 4 = 8
        $$
        Para $n \geq 3$:
        $$
          b_{ij} =
          \llave{cl}{
            1 & \text{ si } i = 1 \text{ o } j = 1 \\
            1/i & \text{ si } i = j, ~ \magenta{i,j < n - 1}\\
            0 & \text{ en otro caso}
          }
        $$
        Las últimas 2 filas son iguales, así que \ul{$B$ es singular}:
        $$
          A - B =
          \llave{cl}{
            \frac{1}{n-1} & \text{ si } i = j = n - 1\\
            \frac{1}{n} & \text{ si } i = j = n\\
            0 & \text{ en otro caso}
          }
        $$
        Entonces queda que:
        $$
          \condicion_\infinito(A) \geq \frac{\norma{A}_\infinito}{\norma{A - B}_\infinito} = \frac{n}{\frac{1}{n - 1}} = n^2 - n \en O(n^2)
        $$

  \item Pispeá el ejercicio \ref{ej:16}, ahí están las acotaciones falopa de la normas.

        Entonces usando que:
        $$
          \frac{1}{\sqrt{n}} \norma{A}_\infinito \menorIgual{$\llamada1$} \norma{A}_2 \leq \sqrt{n} \norma{A}_\infinito
        $$
        sale con fritas \simpleicon{kfc}.
        $$
          \begin{array}{rcl}
            \limite{n}{\infinito} \condicion_2(A)
             & =                        &
            \limite{n}{\infinito} \norma{A}_2 \cdot \norma{A^{-1}}_2                                                     \\
             & \mayorIgual{$\llamada1$} &
            \limite{n}{\infinito} \frac{1}{\sqrt{n}}\norma{A}_\infinito \cdot \frac{1}{\sqrt{n}}\norma{A^{-1}}_\infinito \\
             & =                        &
            \limite{n}{\infinito} \frac{1}{n} \norma{A}_\infinito \cdot \norma{A^{-1}}_\infinito                         \\
             & \igual{def}              &
            \limite{n}{\infinito}
            \ub{
              \frac{1}{n} \ob{\condicion_\infinito(A)}{\en O(n^2)}
            }{
              \en O(n)
            } \flecha{$n \to \infinito$} \infinito
          \end{array}
        $$
\end{enumerate}

\begin{aportes}
  \item \aporte{\dirRepo}{naD GarRaz \github}
\end{aportes}
