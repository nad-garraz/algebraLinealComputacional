\begin{enunciado}{\ejExtra}
  Sea $\set{q_1, q_2, q_3, q_4, q_5}$ una base ortonormal de $\reales^5$, $A$ una matriz de $5 \times 3$
  con columnas $q_1,\, q_2,\, q_3$ y el vector $b = q_1 + 2q_2 + 3q_3 + 4q_4 + 5q_5$.

  \begin{enumerate}[label=(\alph*)]
    \item Mostrar que el sistema $Ax = b$ no tiene solución. Plantear las ecuaciones normales y hallar la solución $\hat{x}$ de cuadrados
          mínimos para dicho sistema.

    \item Calcular el error cometido en la aproximación.

    \item Mostrar que $A^\dagger = A^t$, siendo $A^\dagger$ la pseudoinversa de $A$.
  \end{enumerate}
\end{enunciado}

\begin{enumerate}[label=(\alph*)]
  \item En la cuentilla uso notación
        $(q_i)_j$ con $j \en [1,5]$ como la coordenada $j$-ésima del vector $q_i$:
        $$
          \begin{array}{rcl}
            A x = b
                  & \sii                    &
            \matriz{c|c|c}{
            \quad & \quad                   & \quad                    \\
            \quad & \quad                   & \quad                    \\
            q_1   & q_2                     & q_3                      \\
            \quad & \quad                   & \quad                    \\
            \quad & \quad                   & \quad
            }
            \cdot
            \matriz{c}{
            x_1                                                        \\
            x_2                                                        \\
              x_3
            }
            =
            q_1 + 2q_2 + 3q_3 + 4q_4 + 5q_5                            \\
                  & \sii                    &
            \matriz{c}{
            x_1 (q_1)_1 + x_2 (q_2)_1 + x_3 (q_3)_1                    \\
            x_1 (q_1)_2 + x_2 (q_2)_2 + x_3 (q_3)_2                    \\
            x_1 (q_1)_3 + x_2 (q_2)_3 + x_3 (q_3)_3                    \\
            x_1 (q_1)_4 + x_2 (q_2)_4 + x_3 (q_3)_4                    \\
            x_1 (q_1)_5 + x_2 (q_2)_5 + x_3 (q_3)_5                    \\
            }
            =
            q_1 + 2q_2 + 3q_3 + 4q_4 + 5q_5                            \\
                  & \sii                    &
            x_1q_1 + x_2q_2 + x_3q_3 = q_1 + 2q_2 + 3q_3 + 4q_4 + 5q_5 \\
                  & \Sii{\red{!}}           &
            q_4^t \cdot (x_1q_1 + x_2q_2 + x_3q_3) =
            q_4^t \cdot (q_1 + 2q_2 + 3q_3 + 4q_4 + 5q_5)              \\
                  & \Entonces{\red{!}}[BON] &
            0 = 4 \quad \text{\red{\faIcon{skull}}}
          \end{array}
        $$
        Por lo tanto ese sistema no tiene una solución.

        Planteo las ecuaciones normales para encontrar la solución $\hat{x}$ que mejor
        aproxima por cuadrados mínimos:
        {
        \small
        $$
          \begin{array}{rcl}
            A^tA \hat{x} = A^tb
                      & \sii          &
            \matriz{ccc}{
            \quad     & q_1^t         & \quad       \\\hline
            \quad     & q_2^t         & \quad       \\\hline
            \quad     & q_3^t         & \quad       \\
            }
            \matriz{c|c|c}{
            \quad     & \quad         & \quad       \\
            \quad     & \quad         & \quad       \\
            q_1       & q_2           & q_2         \\
            \quad     & \quad         & \quad       \\
            \quad     & \quad         & \quad
            }
            \cdot
            \matriz{c}{
            \hat{x}_1                               \\
            \hat{x}_2                               \\
              \hat{x}_3
            }
            =
            \matriz{ccc}{
            \quad     & q_1^t         & \quad       \\\hline
            \quad     & q_2^t         & \quad       \\\hline
            \quad     & q_3^t         & \quad       \\
            }
            (q_1 + 2q_2 + 3q_3 + 4q_4 + 5q_5)       \\
                      & \Sii{\red{!}} &
            I_3
            \cdot
            \matriz{c}{
            \hat{x}_1                               \\
            \hat{x}_2                               \\
              \hat{x}_3
            }
            =
            \matriz{c}{
            1                                       \\
            2                                       \\
              3
            }                                       \\
                      & \sii          &
            \llave{rcl}{
            \hat{x}_1 & =             & \magenta{1} \\
            \hat{x}_2 & =             & \magenta{2} \\
            \hat{x}_3 & =             & \magenta{3}
            }
          \end{array}
        $$
        }

  \item Si venís haciendo los ejercicio donde te dan los datos para aproximar $(\blue{x_i}, \green{y_i})$ este punto puede ser \textit{confuso}, pero
        tenés que pensar que los $\blue{x_i}$ están en la matriz $A$ y los $\green{y_i}$ son los elementos de $b$.

        El error viene dado por:
        $$
          \sumatoria{i = 1}{5} (\green{y_i} - \blue{x_i}\tilde{x})^2 =
          \norma{\green{b} - \blue{A}\tilde{x}}_2^2
          \igual{\red{!}}
          \norma{
            \green{q_1 + 2q_2 + 3q_3 + 4q_4 + 5q_5} - (\magenta{1}q_1 + \magenta{2}q_2 + \magenta{3}q_3)
          }_2^2
          =
          \norma{
            \green{4q_4 + 5q_5}
          }_2^2
        $$

  \item Recordando que la pseudoinversa es como la transpuesta, pero invirtiendo los valores singulares:
        $$
          A = U \Sigma V^*
          \Entonces{pseudo}[inversa]
          A^\dagger = V \Sigma^\dagger U^*,
        $$
        con la $\Sigma^\dagger$ que sería como $\Sigma^t$ invirtiendo los elementos diagonales $[\Sigma^\dagger]_{ii} = \frac{1}{\sigma_{ii}}$.

        En los calculos de los ítems anteriores se vio que:
        $$
          A^t \cdot A = I_3
          \Entonces{\ \red{!}\ }
          \sigma_i = 1
          \Sii{$\llamada1$}
          \frac{1}{\sigma_i} = 1
        $$
        por lo tanto:
        $$
          A = U
          \ua{
            \Sigma
          }{
            \matriz{c}{
              I_3 \\ \hline
              0
            }
          }
          V^t
          \sii
          A^t = V
          \ua{
            \Sigma^t
          }{
            \matriz{c|c}{
              I_3 & 0\\
            }
          }
          U^t
          \ytext
          A^\dagger = V
          \ua{
            \Sigma^\dagger
          }{
            \matriz{c|c}{
              I_3^{-1} & 0
            }
          }
          U^t \igual{$\llamada1$} V \Sigma^t U^t = A^t
        $$
\end{enumerate}

\begin{aportes}
  \item \aporte{\dirRepo}{naD GarRaz \github}
\end{aportes}
