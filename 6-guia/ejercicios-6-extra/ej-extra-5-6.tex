\begin{enunciado}{\ejExtra}\fechaEjercicio{final 24/02/25}
  Dada la función:
  $$
    z = ay^b e^{cx + 2}
  $$
  \begin{enumerate}[label=\arabic*.]
    \item Plantear la ecuaciones de mínimos cuadrados para estimar los parámetros $a,\, b \ytext c$.
    \item Proponer puntos de datos para que la solución sea única.
    \item Determinar la mínima cantidad de puntos necesarios para que la solución sea única.
  \end{enumerate}
\end{enunciado}

\begin{enumerate}[label=\arabic*.]
  \item Necesito que los parámetros a encontrar sean lineales en la ecuación. Para eso \textit{ logaritmo natural to the rescue} con sus ricas propiedades:
        $$
          \begin{array}{c}
            \textstyle
            \cajaResultado{\log_{\red{a}}(\green{b})+\log_{\red{a}}(\blue{c})=\log_{\red{a}}(\green{b}\cdot \blue{c})}               \\
            \textstyle
            \cajaResultado{\log_{\red{a}}(\green{b})-\log_{\red{a}}(\blue{c})=\log_{\red{a}}\left(\frac{\green{b}}{\blue{c}}\right)} \\
            \cajaResultado{\log_{\red{a}}(\green{b}^{\blue{c}})=\blue{c}\cdot\log_{\red{a}}(\green{b})}
          \end{array}
        $$
        $$
          \begin{array}{rcl}
            z = ay^b e^{cx + 2}
             & \sii          &
            \ln(z) = \ln(ay^b e^{cx + 2})                 \\
             & \Sii{\red{!}} &
            \ln(z) = \ln(a) + \ln(y^b) +  \ln(e^{cx + 2}) \\
             & \Sii{\red{!}} &
            \ln(z) = \ln(a) + b\ln(y) +  cx + 2           \\
             & \Sii{\red{!}} &
            \ub{\ln(z) - 2}{\tilde{z}} = b \ln(y) + cx + \ub{\ln(a)}{\tilde{a}}
          \end{array}
        $$
        Busco minimizar la función que ahora es lineal en sus parámetros:
        $$
          \tilde{z} = b \ln(y) + cx + \tilde{a}
        $$
        Las ecuaciones normales:
        {
        \small
        $$
          \begin{array}{rcl}
            A \alpha = \beta
                     & \sii                                  &
            \matriz{ccc}{
            \ln(y_1) & x_1                                   & 1      \\
            \vdots   & \vdots                                & \vdots \\
            \ln(y_n) & x_n                                   & 1
            }
            \cdot
            \matriz{c}{
            b                                                         \\
            c                                                         \\
              \tilde{a}
            }
            =
            \matriz{c}{
            \tilde{z}_1                                               \\
            \vdots                                                    \\
            \tilde{z}_n                                               \\
            }                                                         \\
                     & \Sii{$\times A^t \to $}[Ec. normales] &
            \matriz{ccc}{
            \ln(y_1) & x_1                                   & 1      \\
            \vdots   & \vdots                                & \vdots \\
            \ln(y_n) & x_n                                   & 1
            }^t
            \matriz{ccc}{
            \ln(y_1) & x_1                                   & 1      \\
            \vdots   & \vdots                                & \vdots \\
            \ln(y_n) & x_n                                   & 1
            }
            \cdot
            \matriz{c}{
            b                                                         \\
            c                                                         \\
              \tilde{a}
            }
            =
            \matriz{ccc}{
            \ln(y_1) & x_1                                   & 1      \\
            \vdots   & \vdots                                & \vdots \\
            \ln(y_n) & x_n                                   & 1
            }^t
            \matriz{c}{
            \tilde{z}_1                                               \\
            \vdots                                                    \\
            \tilde{z}_n                                               \\
            }                                                         \\\\
                     & \sii                                  &
            \cajaResultado{A^t A \alpha = A^t \beta}
          \end{array}
        $$
        }

  \item
        \ul{En general} un sistema de la pinta:
        $$
          \oa{M}{\en \reales^{m \times n}} \cdot \ua{\magenta{x}}{\en \reales^n} = \oa{b}{\en \reales^m}
        $$
        tiene solución si $b \en \columna(M)$, lo que equivale a decir que $b \en \imagen(M)$ si pensás a $M$ como una transformación lineal.
        Si el sistema tiene solución, para que esta sea única deberían pasar estas cosas que son todas lo mismo dicho de diferente forma:

        \begin{itemize}
          \item Si al triangular la matriz $M$ quedan tantas filas \textit{linealmente independientes} como cantidad de \textit{\magenta{incognitas}}.
          \item Que el rango de $M$, $\rango(M)$ sea completo, es decir que sus columnas sean \textit{linealmente independientes}.
          \item Si la matriz es cuadrada su determinante tiene que ser \underline{distinto} de cero.
        \end{itemize}

        La idea es resolver el sistema:
        $$
          A^t A \alpha = A^t \beta
        $$
        Entonces quiero una $A$ que tenga columnas \textit{LI} y además que $\beta \en \columna(A)$. Si tengo eso, sé que la solución
        va a ser única. Tené en cuenta que si le pido eso a $A$ para que $ A \alpha = \beta $ tenga solución única, entonces el
        sistema $A^tA x = A^t b$ también va a cumplir eso, no sé si es obvio, pero se prueba sin mucho sufrimiento.

        Propogno:
        $$
          A =
          \matriz{ccc}{
            \ln(1) & 0 & 1 \\
            \ln(1) & 1 & 1 \\
            \ln(e) & 1 & 1
          }
        $$
        Los puntos que elegí:
        $$
          \begin{array}{|c|c|c|c|}
            \hline
            x & 0   & 1   & 1   \\ \hline
            y & e   & 1   & e   \\ \hline
            z & e^2 & e^2 & e^2 \\ \hline
          \end{array}
        $$
        Para armar los puntos \textit{"no pensé"} en los $z$, porque si las columnas de $A$ generan todo $\reales^3$, $z$ seguro va a pertenecer
        al $\columna(A)$, pero dado que $z$ e $y$ están como argumentos de $\ln(\cdot)$ por lo menos tienen que ser positivos para que no explote
        todo por los aires.

  \item Por lo hecho antes necesito un sistema compatible determinado, así que necesito por lo menos 3 puntos para encontrar los 3 parámetros
        $a,\,b \ytext c$.
\end{enumerate}

\begin{aportes}
  \item \aporte{\dirRepo}{naD GarRaz \github}
\end{aportes}
