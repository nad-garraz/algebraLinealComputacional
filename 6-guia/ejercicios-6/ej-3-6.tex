\begin{enunciado}{\ejercicio}
  Para cada uno de los conjuntos de datos, plantear las ecuaciones normales y calcular los polinomios de grado 1, 2 y 3 que
  mejor aproximan la tabla en el sentido de cuadrados mínimos. Graficar los datos juntos con los tres polinomios.
  ¿Qué se observa? ¿Qué se puede decir del polinomio de grado 3?
  $$
    \begin{array}{|c||c|c|c|c|}
      \hline
      x & -1 & 0 & 2  & 3  \\ \hline
      y & -1 & 3 & 11 & 27 \\ \hline
    \end{array}
    \quad
    \begin{array}{|c||c|c|c|c|}
      \hline
      x & -1 & 0 & 1 & 2 \\ \hline
      y & -3 & 1 & 1 & 3 \\ \hline
    \end{array}
  $$
\end{enunciado}

Quiero hacer cuadrados mínimos en los conjuntos dados para los polinomios:
$$
  \llave{rcl}{
    y & = & ax + b  \\
    y & = & ax^2 + bx + c  \\
    y & = & ax^3 + bx^2 + cx + d
  }
$$
$$
  A x = y
  \sii
  \matriz{cc}{
    -1 & 1 \\
    0 & 1 \\
    2 & 1 \\
    3 & 1
  }
  \matriz{cc}{
    a\\
    b
  }
  =
  \matriz{c}{
    -1 \\
    3  \\
    11  \\
    27
  }
  \sii
  A^tAx = A^ty
  \sii

$$
