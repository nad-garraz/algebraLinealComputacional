\begin{enunciado}{\ejercicio}
  En cada uno de los siguientes casos construir un proyector $f:\reales^3 \to \reales^3$ que cumpla:
  \begin{enumerate}[label=(\roman*)]
    \item $\imagen(f) = \set{(x_1,x_2,x_3) / x_1 + x_2 + x_3 = 0 }$
    \item $\nucleo(f) = \set{(x_1,x_2,x_3) / x_1 + x_2 + x_3 = 0 }$
    \item $\nucleo(f) = \set{(x_1,x_2,x_3) / 3x_1 - x_3 = 0 } \text{ e } \imagen(f) = \ket{(1,1,1)}$
  \end{enumerate}
\end{enunciado}

\hyperlink{teoria-3:proyector}{Acá un poco de cosas de proyectores, click, click {\tiny\faIcon{mouse}}}

\begin{enumerate}[label=(\roman*)]
  \item Encuentro un \textit{sistema de generadores} de $\imagen(f) = \ket{(-1,1,0), (-1,0,1)}$, y dado que
        el subespacio $\nucleo(f) \sumaDirecta \imagen(f)$, por ejemplo $\nucleo(f) = \ket{(1,1,1)}$.

        Con esa data defino el proyector:
        $$
          \llave{rcl}{
            f(-1,1,0) &=& (-1,1,0) \\
            f(-1,0,1) &=& (-1,0,1) \\
            f(1,1,1) &=& (0,0,0)
          }
        $$
        Si tomo como base $B = \set{(-1,1,0),(-1,0,1), (1,1,1)}$:
        $$
          [f]_{BE} =
          \matriz{rcl}{
            -1 & -1 & 0\\
            1 & 0 & 0\\
            0 & 1 & 0
          }
        $$
        Busco $f$, para lo cual multiplico por:
        $$
          [C]_{BE} =
          \matriz{ccc}{
            -1 & -1 & 1 \\
            1 & 0 & 1 \\
            0 & 1 & 1
          }
          \quad\flecha{calculo la}[inversa]\quad
          [C]_{EB} =
          \matriz{rrr}{
            -\frac{1}{3} & \frac{2}{3} & -\frac{1}{3}  \\
            -\frac{1}{3} & -\frac{1}{3} & \frac{2}{3}  \\
            \frac{1}{3} & \frac{1}{3} & \frac{1}{3}
          }
        $$
        Por lo tanto:
        $$
          [f]_{EE} = [f]_{BE} \cdot [C]_{EB} =
          \matriz{rrr}{
            \frac{2}{3} & -\frac{1}{3} & -\frac{1}{3} \\
            -\frac{1}{3} & \frac{2}{3} & -\frac{1}{3} \\
            -\frac{1}{3} & -\frac{1}{3} & \frac{2}{3}
          }
        $$

        $$
          \cajaResultado{
            f(x,y,z) = \frac{1}{3}(2x - y - z, -x + 2y - z,-x - y + 2z)
          }
        $$

  \item Usando lo mismo de antes pero resuelvo de otra forma:
        Ahora tengo que una base del $\nucleo(f)$:
        $$
          B_{\nucleo(f)}  = \set{(-1,1,0), (-1,0,1)}
        $$
        Completo esa base teniendo en cuenta que el vector que agrego va a ser
        un elemento de la $\imagen(f)$:
        $$
          B_V  = \{\ub{(-1,1,0), (-1,0,1)}{\en \nucleo(f)},\ub{(0,0,1)}{\en \imagen(f)}\} = \reales^3
        $$
        \parrafoDestacado[\atencion]{
          Importante que $\nucleo(f) \inter \imagen(f) = 0$, en $\reales^3$ parece obvio, pero ya en $\reales^4$
          es más engañoso.
        }
        $$
          \llamada1
          \llave{rcl}{
            f(-1,1,0) &=& (0,0,0) \\
            f(-1,0,1) &=& (0,0,0) \\
            f(0,0,1) &=& (0,0,1)
          }
        $$
        Voy a encontrar la expresión funcional de este proyector. La idea es escribir a esa base, $B_V $,
        en función de $(x_1, x_2, x_3)$ para luego transformarla usando la propiedades de las viejas y
        queridas \textit{transformaciones lineales}:
        $$
          (x_1, x_2, x_3) \igual{$\llamada2$} \blue{a} \cdot (-1,1,0) + \blue{b} (-1,0,1) + \blue{c} (0,0,1)
          \flecha{sistema en}[forma matricial]
          \matriz{ccc|c}{
            -1 & -1 & 0 & x_1 \\
            1 & 0 & 0 & x_2\\
            0 & 1 & 1 & x_3
          }
        $$
        Resolviendo ese sistema obtengo:
        $$
          \llave{l}{
            \blue{a} = x_2 \\
            \blue{b} = -x_1 - x_2 \\
            \blue{c} = x_1 + x_2 + x_3
          },
        $$
        es decir que $\llamada2$ es:
        $$
          (x_1, x_2, x_3) =
          x_2 \cdot (-1,1,0) +
          (-x_1 - x_2) \cdot (-1,0,1) +
          (x_1 + x_2 + x_3)\cdot (0,0,1)
        $$
        Aplicando $f$:
        $$
          \begin{array}{rcl}
            f(x_1, x_2, x_3) & =                   & f\big(x_2 \cdot (-1,1,0) + (-x_1 - x_2) \cdot (-1,0,1) + (x_1 + x_2 + x_3)\cdot (0,0,1)\big) \\
            f(x_1, x_2, x_3) & =                   & x_2 \cdot f(-1,1,0) + (-x_1 - x_2) \cdot f(-1,0,1) + (x_1 + x_2 + x_3) \cdot f(0,0,1)        \\
            f(x_1, x_2, x_3) & \igual{$\llamada1$} & x_2 \cdot (0,0,0) + (-x_1 - x_2) \cdot (0,0,0) + (x_1 + x_2 + x_3) \cdot (0,0,1)             \\
            f(x_1, x_2, x_3) & =                   & (0,0,x_1 + x_2 + x_3)
          \end{array}
        $$
        Si por alguna razón quiero esto en forma matricial, transformo los canónicos y pongo los transformados como columnas:
        $$
          [f]_{EE} =
          \matriz{ccc}{
            0 & 0 & 0 \\
            0 & 0 & 0 \\
            1 & 1 & 1
          }
        $$
        $$
          \llave{rcl}{
            f(1,0,0) & = & (\frac{1}{3},\frac{1}{3},\frac{1}{3}) \\
            f(0,1,0) & = & (\frac{1}{3},\frac{1}{3},\frac{1}{3}) \\
            f(0,0,1) & = & (\frac{1}{3},\frac{1}{3},\frac{1}{3})
          }
        $$
        $$
          \cajaResultado{
            f(x,y,z) = \frac{1}{3}(x+y+z, x+y+z,x+y+z)
          }
        $$

  \item Ahora $\nucleo(f) = \ket{(3,0,1), (0,1,0)}$
        $$
          \llave{rcl}{
            f(3,0,1) &=& (0,0,0) \\
            f(0,1,0) &=& (0,0,0) \\
            f(1,1,1) &=& (1,1,1)
          }
        $$
        Aprovechando que hay muchos ceros en la matriz, puedo encontrar los transformados de los
        \textit{vectores canónicos} con un par de cuentas usando nuevamente propiedades de \textit{transformaciones lineales}:
        {\tiny
        $$
          \begin{array}{rcl}
            \llave{rcl}{
            f(3,0,1) & =                                                           & (0,0,0)                                  \\
            f(0,1,0) & =                                                           & (0,0,0)                                  \\
            f(1,1,1) & =                                                           & (1,1,1)
            }
            \flecha{$3F_3 - F_1\to F_3$}
            \llave{rcl}{
            f(3,0,1) & =                                                           & (0,0,0)                                  \\
            f(0,1,0) & =                                                           & (0,0,0)                                  \\
            f(0,3,2) & =                                                           & (3,3,3)
            }                                                                                                                 
                     & \flecha{$F_3 - 3F_2\to F_3$}                                &
            \llave{rcl}{
            f(3,0,1) & =                                                           & (0,0,0)                                  \\
            f(0,1,0) & =                                                           & (0,0,0)                                  \\
            f(0,0,2) & =                                                           & (3,3,3)
            }                                                                                                                 \\\\
                     & \flecha{$2F_1 - 3F_3\to F_1$}                               &
            \llave{rcl}{
            f(6,0,0) & =                                                           & (-3,-3,-3)                               \\
            f(0,1,0) & =                                                           & (0,0,0)                                  \\
            f(0,0,2) & =                                                           & (3,3,3)
            }                                                                                                                 \\\\
                     & \flecha{$\frac{1}{6}F_1 \to F_1$}[$\frac{1}{2}F_3 \to F_3$] &
            \llave{rcl}{
            f(1,0,0) & =                                                           & (-\frac{1}{2},-\frac{1}{2},-\frac{1}{2}) \\
            f(0,1,0) & =                                                           & (0,0,0)                                  \\
            f(0,0,1) & =                                                           & (\frac{3}{2},\frac{3}{2},\frac{3}{2})
            }
          \end{array}
        $$
        }
        En forma matricial quedaría:
        $$
          [f]_{EE} =
          \matriz{ccc}{
            -\frac{1}{2} & 0 & \frac{3}{2} \\
            -\frac{1}{2} & 0 & \frac{3}{2} \\
            -\frac{1}{2} & 0 & \frac{3}{2}
          }
        $$
        Multiplicando a $[f]_{EE}$ por $(x_1, x_2, x_3)$ se obtiene la forma funcional:
        $$
          [f]_{EE}\cdot
          \matriz{c}{
            x_1\\
            x_2\\
            x_3
          }
          =
          \matriz{c}{
            -\frac{1}{2}x_1 + \frac{3}{2}x_3 \\
            -\frac{1}{2}x_1 + \frac{3}{2}x_3 \\
            -\frac{1}{2}x_1 + \frac{3}{2}x_3
          }
        $$
        A mí me gusta escrito así:
        $$
          \cajaResultado{
            f(x_1, x_2, x_3) = \frac{1}{2} \cdot (-x_1 + 3x_3,-x_1 + 3x_3,-x_1 + 3x_3)
          }
        $$

\end{enumerate}

\begin{aportes}
  \item \aporte{\dirRepo}{naD GarRaz \github}
\end{aportes}
