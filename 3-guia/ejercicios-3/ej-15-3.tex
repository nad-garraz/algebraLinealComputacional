\begin{enunciado}{\ejercicio}
  En cada uno de los siguientes casos construir un proyector $f:\reales^3 \to \reales^3$ que cumpla:
  \begin{enumerate}[label=(\roman*)]
    \item $\imagen(f) = \set{(x_1,x_2,x_3) / x_1 + x_2 + x_3 = 0 }$
    \item $\nucleo(f) = \set{(x_1,x_2,x_3) / x_1 + x_2 + x_3 = 0 }$
    \item $\nucleo(f) = \set{(x_1,x_2,x_3) / 3x_1 - x_3 = 0 } \text{ e } \imagen(f) = \ket{(1,1,1)}$
  \end{enumerate}
\end{enunciado}

\hyperlink{teoria-3:proyector}{Acá un poco de cosas de proyectores, click, click {\tiny\faIcon{mouse}}}

\begin{enumerate}[label=(\roman*)]
  \item Encuentro un \textit{sistema de generadores} de $\imagen(f) = \ket{(-1,1,0), (-1,0,1)}$, y dado que
        el subespacio $\nucleo(f) \sumaDirecta \imagen(f)$, por ejemplo $\nucleo(f) = \ket{(1,1,1)}$.

        Con esa data defino el proyector:
        $$
          \llave{rcl}{
            f(-1,1,0) &=& (-1,1,0) \\
            f(-1,0,1) &=& (-1,0,1) \\
            f(1,1,1) &=& (0,0,0)
          }
        $$
        Si tomo como base $B = \set{(-1,1,0),(-1,0,1), (1,1,1)}$:
        $$
          [f]_{BE} =
          \matriz{rcl}{
            -1 & -1 & 0\\
            1 & 0 & 0\\
            0 & 1 & 0
          }
        $$
        Busco $f$, para lo cual multiplico por:
        $$
          [C]_{BE} =
          \matriz{ccc}{
            -1 & -1 & 1 \\
            1 & 0 & 1 \\
            0 & 1 & 1
          }
          \quad\flecha{calculo la}[inversa]\quad
          [C]_{EB} =
          \matriz{rrr}{
            -\frac{1}{3} & \frac{2}{3} & -\frac{1}{3}  \\
            -\frac{1}{3} & -\frac{1}{3} & \frac{2}{3}  \\
            \frac{1}{3} & \frac{1}{3} & \frac{1}{3}
          }
        $$
        Por lo tanto:
        $$
          [f]_{EE} = [f]_{BE} \cdot [C]_{EB} =
          \matriz{rrr}{
            \frac{2}{3} & -\frac{1}{3} & -\frac{1}{3} \\
            -\frac{1}{3} & \frac{2}{3} & -\frac{1}{3} \\
            -\frac{1}{3} & -\frac{1}{3} & \frac{2}{3}
          }
        $$

        $$
          \cajaResultado{
            f(x,y,z) = \frac{1}{3}(2x - y - z, -x + 2y - z,-x - y + 2z)
          }
        $$

  \item Usando lo mismo de antes:
        $$
          \llave{rcl}{
            f(-1,1,0) &=& (0,0,0) \\
            f(-1,0,1) &=& (0,0,0) \\
            f(1,1,1) &=& (1,1,1)
          }
          \Sii{cambio de }[base]
          \llave{rcl}{
            f(1,0,0) &=& (\frac{1}{3},\frac{1}{3},\frac{1}{3}) \\
            f(0,1,0) &=& (\frac{1}{3},\frac{1}{3},\frac{1}{3}) \\
            f(0,0,1) &=& (\frac{1}{3},\frac{1}{3},\frac{1}{3})
          }
        $$
        $$
          \cajaResultado{
            f(x,y,z) = \frac{1}{3}(x+y+z, x+y+z,x+y+z)
          }
        $$

  \item Ahora $\nucleo(f) = \ket{(3,0,1), (0,1,0)}$
        $$
          \llave{rcl}{
            f(3,0,1) &=& (0,0,0) \\
            f(0,1,0) &=& (0,0,0) \\
            f(1,1,1) &=& (1,1,1)
          }
        $$
\end{enumerate}

\begin{aportes}
  \item \aporte{\dirRepo}{naD GarRaz \github}
\end{aportes}
