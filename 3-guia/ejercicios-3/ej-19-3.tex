\begin{enunciado}{\ejercicio}
  Sea $Q \en \reales^{n \times n}$. Probar que son equivalentes:
  \begin{enumerate}[label=(\alph*)]
    \item $Q^{-1} = Q^t$.

    \item\label{ej-19:itemb} Las columnas de $Q$ forman un conjunto ortonormal.

    \item Las filas de $Q$ forman un conjunto ortonormal.

    \item\label{ej-19:itemd} $\norma{Qx}_2 = \norma{x}_2$ para todo $x \en \reales^n$.
  \end{enumerate}
  Interpretar \ref{ej-19:itemd} geométricamente.

  \textit{Sugerencia:} Para demostrar la implicación
  (\ref{ej-19:itemd} $\entonces$ \ref{ej-19:itemb}) usar que $x^ty = \frac{1}{4}(\norma{x + y}_2^2 - \norma{x - y}_2^2)$.
\end{enunciado}

\begin{enumerate}[label=\alph*)]
  \item  Quiero probar que:
        $$
          \text{Si}\quad \ub{Q^{-1} = Q^t}{\text{hipótesis}}
          \entonces
          \ub{\text{las columnas de $Q$ forman un conjunto ortonormal}}{\text{tesis}}
        $$
        $$
          Q^{-1} \cdot Q = I
          \Sii{HIP}
          Q^t \cdot Q = I
          \Sii{zoom}
          \ub{
            \matriz{c}{
              \quad \bm{q}_1^t \quad \\ \hline
              \vdots \\ \hline
              \bm{q}_n^t \\
            }
            \matriz{c|c|c}{
              & & \\
              \bm{q}_1  & \dots  & \bm{q}_n   \\
              & &
            }
          }{
            \matriz{cccc}{
              \bm{q}_1^t\bm{q}_1 & \bm{q}_1^t \bm{q}_2 & \dots & \bm{q}_1^t \bm{q}_n\\
              \vdots & \vdots  & \ddots & \vdots\\
              \bm{q}_n^t\bm{q}_1 & \bm{q}_n^t \bm{q}_2 & \dots & \bm{q}_n^t \bm{q}_n
            }
          }
          =
          \matriz{ccc}{
            1 & \cdots & 0 \\
            \vdots & \ddots & \vdots \\
            0 & \cdots & 1
          }
        $$
        Estaría quedando que:
        $$
          Q^t Q =
          \llave{rcl}{
            1 & \text{si} & i = j\\
            0 & \text{si} & i \distinto j
          }
          \entonces
          \columna(Q) = \set{\bm{q}_1, \bm{q}_2, \cdots, \bm{q}_n} \text{ es un conjunto ortonormal}
        $$

  \item Quiero probar que:
        $$
          \text{Si}\quad \ub{\columna(Q) \text{ es un conjunto ortonormal}}{\text{hipótesis}}
          \entonces
          \ub{\fila(Q) \text{ es un conjunto ortonormal}}{\text{tesis}}
        $$
        Antes vi que si la columnas de $Q$ forman un conjunto ortonormal, entonces $Q^tQ = I$.
        $$
          I = Q^t \cdot Q
          \Sii{\red{!}}
          I = Q \cdot Q^t=
          \matriz{c|c|c}{
            & & \\
            \bm{q}_1  & \dots  & \bm{q}_n   \\
            & &
          }
          \matriz{c}{
            \quad \bm{q}_1^t \quad \\ \hline
            \vdots \\ \hline
            \bm{q}_n^t
          }
          =
          \ub{
            \matriz{c}{
              \fila_1(Q) \cdot Q^t\\ \hline
              \vdots  \\ \hline
              \fila_n(Q) \cdot Q^t
            }
            =
            \matriz{ccc}{
              1 & \cdots & 0 \\ \hline
              \vdots & \ddots & \vdots \\\hline
              0 & \cdots & 1
            }
          }{\llamada1}
        $$
        En $\llamada1$ está de un lado el producto de las filas de $Q$ con la matriz \underline{$Q^t$ que tiene como columnas a las filas de $Q$}. Por lo tanto,
        para la primera fila, \textit{el producto con todas las filas de $Q$} da $\bm{e}_1^t$, en general voy a tener:
        $$
          \fila_i(Q) \cdot \fila_j(Q)^t
          =
          \llave{rcl}{
            1 & \text{si} & i = j\\
            0 & \text{si} & i \distinto j
          }
        $$

  \item Quiero probar que:
        $$
          \ub{\text{Si} \fila(Q) \text{ es un conjunto ortonormal}}{\text{hipótesis}}
          \entonces
          \ub{
            \norma{Qx}_2 = \norma{x}_2 \paratodo x \en \reales^n.
          }{\text{tesis}}
        $$
        \hacer

  \item \hacer

\end{enumerate}
