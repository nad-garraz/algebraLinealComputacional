\begin{enunciado}{\ejercicio}
  Considerar la matriz
  $$
    \matriz{ccc}{
      4 & 2 & -2 \\
      2 & 5 & 5 \\
      -2 & 5 & 11
    }
  $$
  Mostrar que es definida positiva y calcular su descomposición de Cholesky.
\end{enunciado}

\hyperlink{teoria-3:definida-positiva}{\textit{según la definición de matriz definida positiva:}}
$$
  \begin{array}{rcl}
    \bm{x}^t
    \matriz{ccc}{
    4  & 2                & -2                             \\
    2  & 5                & 5                              \\
    -2 & 5                & 11
    }
    \bm{x}
    = 4 x^2 - 4 x z + 5 y^2 + 10 y z  + 11 z^2
       & \igual{\red{!!}} &
    (4 x^2 - 4 x z + z^2) + 5( y^2 + 2 y z  +  z^2) + 5z^2 \\
       & =                & 5z^2 + 5(y+z)^2 + (2x-z)^2 > 0
  \end{array}
$$

La matriz cumple la defición de
\hyperlink{teoria-3:definida-positiva}{\textit{matriz definida positiva}} $\paratodo \bm{x}\distinto \bm{0} \en \reales^3$.
Sí, oka, hacer eso es una locura, más fácil es hacer lo que sigue y mirar los elementos de la matriz $D$:

Hay un teorema que dice algo así:
\parrafoDestacado{
  Sea $A \en \reales^{n\times n}$ una matriz simétrica y definida positiva  si y solo sí existe $L\en \reales^{n\times n}$
  triangular inferior con diagonal positiva tal que $A = LL^t$.
}

\textit{Arranco como buscando la descomposición $LU$:}

$$
  \matriz{ccc}{
    4 & 2 & -2 \\
    2 & 5 & 5 \\
    -2 & 5 & 11
  }
  \triangulacion{
    \flecha{$F_2 \magenta{- \frac{1}{2}}F_1$}\\
    \flecha{$F_3 \magenta{+ \frac{1}{2}}F_1$}
  }
  \matriz{ccc}{
    4 & 2 & -2 \\
    0 & 4 & 6 \\
    0 & 6 & 10
  }
  \triangulacion{
    \flecha{$F_3 \magenta{- \frac{3}{2}}F_2$}\\
  }
  \matriz{ccc}{
    4 & 2 & -2 \\
    0 & 4 & 6 \\
    0 & 0 & 1
  }
  = U
$$
\parrafoDestacado[\red{\angry}]{¡Sí! Ver los valores diagonales de $U$ alcanza para ver que la matriz era efectivamente definida positiva.
  ¿Pero quién puede quitarnos el placer de haberlo comprobado de ambas formas?}
Y ahora me formo la $\tilde{L}$ a partir de la \textit{eliminación gaussiana:}
$$
  \tilde{L} =
  \matriz{ccc}{
    1 & 0 & 0 \\
    \magenta{\frac{1}{2}} & 1 & 0 \\
    \magenta{-\frac{1}{2}} & \magenta{\frac{3}{2}} & 1
  }
$$
Con esto ya casi estamos:
$$
  A =
  \tilde{L}U =
  \tilde{L}D\tilde{L}^t =
  \tilde{L}\sqrt{D}\sqrt{D}\tilde{L}^t =
  \tilde{L}\sqrt{D}(\tilde{L}\sqrt{D})^t = LL^t
  \text{ con }
  L =
  \matriz{ccc}{
    2 & 0 & 0 \\
    1 & 2 & 0 \\
    -1 & 3 & 1
  }
$$

Pequeña verificación:
\begin{multicols}{3}
  {
    \tiny
    \codigoPython{ej-9/codigo3-9-1.py}
    \codigoPython{ej-9/codigo3-9-2.py}
    \codigoPython{ej-9/codigo3-9-3.py}
  }
\end{multicols}

\begin{aportes}
  \item \aporte{\dirRepo}{naD GarRaz \github}
\end{aportes}
