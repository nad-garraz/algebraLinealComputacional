\begin{enunciado}{\ejercicio}
  Sea $A \en \reales^{n \times n}$ tal que $\norma{A}_2 < 1$, siendo $\normaBullet_2$ la norma matricial inducida
  por la norma 2 vectorial.
  \begin{enumerate}[label=(\alph*)]
    \item Probar que $I - A^t A$ es simétrica definida positiva.

    \item Probar que la matriz
          $
            \matriz{cc}{
              I & A\\
              A^t &I
            }
          $
          es simétrica definida positiva.
  \end{enumerate}
\end{enunciado}

\begin{enumerate}[label=(\alph*)]
  \item\label{ej12:itema} \textit{Para la simetría:}

        Transpongo y cruzo los dedos para que quede igual:
        $$
          (I - A^t A)^{\red{t}}
          \ua{=}{\text{linealidad en la trasposición}}
          I^t - (A^tA)^t =
          I - A^tA
        $$
        \parrafoDestacado{
        {
        \small
        Sobre la linealidad de la transposición:
        $$
          [A + B]_{ij} = a_{ij} + b_{ij}
          \Entonces{transpongo}
          \left[A + B\right]_{ij}^t =
          [A + B]_{ji} = a_{ji} + b_{ji}
        $$
        }
        }
        Es simétrica.

        \bigskip

        \textit{Para ver si es definida positiva:}

        Intentamos con la definición de \textit{matriz definida positiva} y vemos que sale:
        $$
          \begin{array}{rcl}
            I - A^t A
            \Sii{$\times$}[$\bm{x} \distinto \bm{0}$]
            \blue{\bm{x}^t (I - A^t A) \bm{x}} =
            \norma{\bm{x}}_2^2 - \bm{x}^tA^t A \bm{x} =
            \norma{\bm{x}}_2^2 - \bm{x}^tA^t A \bm{x}
             & =                &
            \norma{\bm{x}}_2^2 - \norma{A\bm{x}}_2^2                                                \\
             & \igual{\red{!!}} &
            \norma{\bm{x}}_2^2( 1 - \frac{\norma{A\bm{x}}_2^2}{\norma{\bm{x}}_2^2}) \quad \llamada1 \\
          \end{array}
        $$
        Y por la definición de la norma inducida:
        $$
          \norma{A}_2 = \maximo_{\bm{x} \distinto \bm{0}} \frac{\norma{A\bm{x}}_2}{\norma{\bm{x}}_2} \ua{<}{\text{enunciado}} 1 \paratodo \bm{x} \en \reales^n
        $$
        queda entonces $\llamada1
          \ob{\norma{\bm{x}}_2^2}{> 0} \ub{
            ( 1 - \ob{
              \frac{\norma{A\bm{x}}_2^2}{\norma{\bm{x}}_2^2}
            }{<1}
            )}{ > 0} > 0
        $:
        $$
          \cajaResultado{
            \blue{\bm{x}^t (I - A^t A) \bm{x}} >0
          }
        $$

  \item
        \textit{Para la simetría:} Creo que esto se ve a simple vista:
        $$
          \begin{array}{l}
            \matriz{cc}{
            I   & A \\
            A^t & I
            }
            =
            \matriz{cc}{
            I   & A \\
            A^t & I
            }
          \end{array}
        $$

        \textit{Para ver si es definida positiva:}
        Acá debería usar \textit{Cholesky}
        \hyperlink{teoría-3:cholesky}{resumencito acá click click {\tiny \faIcon{mouse}}}

        \parrafoDestacado[\atencion]{
          $A = \tilde{L} D \tilde{L}^t$, es definida positiva si y solo si
          $D$ lo es. Como $D$ es diagonal, solo es cuestión de ver que $[D]_{ii} > 0$.
        }
        Acá surge naturalemente la pregunta de ¿Cómo \poo hago la descomposición?.
        $$
          \matriz{cc}{
            I   & A            \\
            A^t & I
          }
          =
          \ob{
            \matriz{cc}{
              I   & 0            \\
              M   & B
            }
          }{\red{L}}
          \ob{
            \matriz{cc}{
              I   & M^t          \\
              0   & B^t
            }
          }{\red{L^t}}
          =
          \matriz{cc}{
            I   & M^t          \\
            M   & MM^t + B B^t
          }
        $$
        $$
          \Sii{\red{!}}
          \llave{rcl}{
            A   & =            & M^t          \\
            A^t & =            & M            \\
            I   & =            & MM^t + B B^t \sii I - A^tA =  BB^t  \\
          }
        $$
        Y en un giro totalmente inesperado, al menos por mí, quedó que esa expresión del ítem \ref{ej12:itema}.
        $BB^t$ es una matriz simétrica y definida positiva, por lo tanto tiene factorización de Cholesky,
        $$
          C = BB^t
          = \tilde{B} \sqrt{D} \sqrt{D}\tilde{B}^t
        $$
        sé que existe esa matriz diagonal $D$ que tiene sus elementos positivos.

        Por lo tanto esto demuestra que efectivamente las matrices $\red{L}$ y $\red{L^t}$ son las matrices de la descomposición
        de Cholesky de la matriz del enunciado:
        $$
          C =
          \matriz{cc}{
            I   & A            \\
            A^t & I
          }
          = \red{L} \red{L^t}
        $$
        Como la matriz $C$ admite descomposición de \textit{Chole}, es simétrica definida positiva.

\end{enumerate}

\begin{aportes}
  \item \aporte{\dirRepo}{naD GarRaz \github}
\end{aportes}
