\begin{enunciado}{\ejercicio}
  Considerar la matriz:
  $
    A =
    \matriz{ccc}{
      0 & 1 & 2 \\
      1 & 1 & 0 \\
      1 & 0 & 3
    }.
  $
  \begin{enumerate}[label=(\alph*)]
    \item Probar que $A$ no admite descomposición $LU$.
    \item Hallar la descomposición $LU$ de $PA$ para alguna matriz de permutación P adecuada.
  \end{enumerate}
\end{enunciado}

\begin{enumerate}[label=(\alph*)]
  \item Si la matriz tiene descomposición $LU$, entonces debería poder escribir:
        {\small
        $$
          A =
          \matriz{ccc}{
            0 & 1 & 2 \\
            \magenta{1} & 1 & 0 \\
            1 & 0 & 3
          }
          =
          \matriz{ccc}{
            1 & 0 & 0 \\
            l_{21} & 1 & 0 \\
            l_{31} & l_{32} & 1
          }
          \cdot
          \matriz{ccc}{
            u_{11} & u_{12} & u_{13} \\
            0 & u_{22} & u_{23} \\
            0 & 0 & u_{33}
          }
          =
          \matriz{ccc}{
          u_{11} & u_{12} & u_{13} \\
          \magenta{l_{21}u_{11}} & u_{22} & u_{23} \\
          l_{31}u_{11} & l_{31}u_{12} + l_{32}u_{22} & l_{31}u_{13} + l_{32}u_{23} + u_{33}
          }
        $$
        }
        Tremenda matriz para que falle enseguida:
        $$
          u_{11} = 0
          \entonces
          l_{21}u_{11} = 0 \distinto \magenta{1}
          \entonces \text{ no existe }  A = LU
        $$

  \item Quiero hacer $F_1 \leftrightarrow F_3$
        $$
          PA =
          \ub{
            \matriz{ccc}{
              0 & 0 & 1 \\
              0 & 1 & 0 \\
              1 & 0 & 0
            }
          }{P}
          \ub{
            \matriz{ccc}{
              0 & 1 & 2 \\
              1 & 1 & 0 \\
              1 & 0 & 3
            }
          }{A}
          =
          \matriz{ccc}{
            1 & 0 & 3 \\
            1 & 1 & 0 \\
            0 & 1 & 2
          }
        $$
        Para hacer la factorización $PA = \tilde{A} = LU$ calculo $LU$ como siempre para una matriz $\tilde{A}$:
        $$
          \begin{array}{c}
            M_1 =
            \matriz{ccc}{
            1  & 0  & 0 \\
            -1 & 1  & 0 \\
            0  & 0  & 1
            }
            \ytext
            M_2 =
            \matriz{ccc}{
            1  & 0  & 0 \\
            0  & 1  & 0 \\
            0  & -1 & 1
            }           \\
          \end{array}
        $$
        Calculo $L$ con las inversas de las matrices de triangulación:
        $$
          \begin{array}{c}
            M_1^{-1} \cdot M_2^{-1} =
            L =
            \matriz{ccc}{
            1 & 0 & 0  \\
            1 & 1 & 0  \\
            0 & 1 & 1
            }
            \ytext
            U =
            \matriz{ccc}{
            1 & 0 & 3  \\
            0 & 1 & -3 \\
            0 & 0 & 5
            }
          \end{array}
        $$
        Por lo tanto para comprobar:
        $$
          \tilde{A} = PA =
          \ub{
            \matriz{ccc}{
              1 & 0 & 0 \\
              1 & 1 & 0 \\
              0 & 1 & 1
            }
          }{L}
          \ub{
            \matriz{ccc}{
              1 & 0 & 3 \\
              0 & 1 & -3 \\
              0 & 0 & 5
            }
          }{U}
          =
          \matriz{ccc}{
            1 & 0 & 3 \\
            1 & 1 & 0 \\
            0 & 1 & 2
          }
        $$
\end{enumerate}

\begin{aportes}
  \item \aporte{\dirRepo}{naD GarRaz \github}
\end{aportes}
