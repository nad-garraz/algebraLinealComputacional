\begin{enunciado}{\ejercicio}
  Sean las matrices $A, B \en \reales^{n \times n}$. Demostrar que $A$ es simétrica definida positiva y B
  es no singular si y solo si $BAB^t$ es simétrica definida positiva.
\end{enunciado}
Muestro una doble implicación:
\begin{itemize}
  \item[$(\red{\Rightarrow})$]
        $A$ es definida positiva:
        $\paratodo x \en \reales^n$ y  $x \distinto 0$, entonces $x^t A x > 0 $. Y $B$ es no singular, entonces $\det(B) \distinto 0$.

        $$
          x^t A x > 0
          \sii
          x^tB^{-1}B A B^t(B^t)^{-1} x > 0
          \Sii{\red{!}}
          ((B^t)^{-1}x)^tB A B^t ((B^t)^{-1} x) > 0
          \sii
          \cajaResultado{
            \blue{y}^tB A B^t \blue{y} > 0
          }
          \text{ con } \blue{y} \distinto 0
        $$
        Dado que $B$ es inversible sé que $(B^t)^{-1} \distinto 0$.

        No sé si es necesario mostrar esto o no, pero:
        $$
          (BAB^t)^t =
          (AB^t)^tB^t
          =
          BA^tB^t =
          \igual{\red{!}}
          BAB^t =
        $$

  \item[$(\red{\Leftarrow})$]
        Una propiedad de las matrices simétricas definidas positivas es que son inversibles, su definición implica que $\nucleo = \set{0}$, así que su determinante es distinto de 0.
        En un producto matricial:
        $$
          0 \distinto \det(B A B^t) =
          \det(B) \cdot \det(A) \cdot \det(B^t) =
          (\det(B))^2 \cdot \det(A)
          \entonces
          \det(A) \distinto 0  \ytext \ub{\det(B) \distinto 0}{B \text{ no singular}}
        $$
        Para demostrar que $A$ es definida positiva se puede recorrer el camino en reversa que se hizo en $(\red{\Rightarrow})$ ahora
        que se sabe que $\det(B) \distinto 0$. Para $x \text{ e } \blue{y} \distinto 0$, se tiene que $(B^t)^{-1} x = \blue{y}\quad \llamada1$ entonces por hipótesis:
        $$
          \blue{y}^tB A B^t \blue{y} \ua{>}{\text{HIP}} 0
          \Sii{$\llamada1$}
          ((B^t)^{-1}x)^tB A B^t ((B^t)^{-1} x) > 0
          \sii
          x^t B^{-1}B A B^t(B^t)^{-1} x > 0
          \sii
          \cajaResultado{
            x^t A x > 0
          } \text{ con } x \distinto 0
        $$
\end{itemize}

\begin{aportes}
  \item \aporte{\dirRepo}{naD GarRaz \github}
\end{aportes}
