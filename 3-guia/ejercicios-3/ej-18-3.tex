\begin{enunciado}{\ejercicio}
  Sea $A =
    \matriz{cc}{
      1 & 0 \\
      0 & 1 \\
      1 & 0
    }
  $. Calcular la matriz de la proyección ortogonal sobre $\imagen(A)$.
\end{enunciado}

Una proyección ortogonal $P$ debe cumplir con que $\imagen(P) \perp \nucleo(P)$:
$$
  \imagen(P) = \set{(1, 0, 1), (0, 1, 0)}
  \ytext
  \nucleo(P) = \set{(-1, 0, 1)}
$$
Por lo tanto mi candidato a \textit{proyector ortogonal}:
$$
  \llave{l}{
    P(1,0,1) = (1,0,1)\\
    P(0,1,0) = (0,1,0)\\
    P(-1,0,1) = (0,0,0)
  }
$$
Voy a buscar la expresión funcional del proyector:
$$
  (x_1, x_2, x_3)
  \igual{$\llamada1$}
  \blue{a} \cdot (1,0,1) +
  \blue{b} \cdot (0,1,0) +
  \blue{c} \cdot (-1,0,1)
  \flecha{resolver en}[forma matricial]
  \matriz{ccc|c}{
    1 & 0 & -1 & x_1 \\
    0 & 1 & 0 & x_2 \\
    1 & 0 & 1 & x_3
  }
$$
Eso queda:
$$
  \llave{l}{
    \blue{a} = \frac{1}{2}x_1 + \frac{1}{2}x_3 \\
    \blue{b} =  x_2 \\
    \blue{c} = \frac{1}{2}x_1 - \frac{1}{2}x_3
  }
  \flecha{reemplazando en $\llamada1$}[y transformando]
  P(x_1, x_2, x_3) =
  \left(\frac{1}{2} x_1 + \frac{1}{2}x_3, x_2,\frac{1}{2} x_1 + \frac{1}{2}x_3\right)
$$
Transformo los canónicos para hallar $P$ en forma matricial:
$$
  [P]_{EE} =
  \matriz{ccc}{
    \frac{1}{2} & 0 &\frac{1}{2} \\
    0 & 1 & 0 \\
    \frac{1}{2} & 0 &\frac{1}{2}
  }
$$
Quedó hermosamente simétrico, porque es un \textit{proyector ortogonal} expresado en una \textit{base ortonormal}.

\begin{aportes}
  \item \aporte{\dirRepo}{naD GarRaz \github}
\end{aportes}
