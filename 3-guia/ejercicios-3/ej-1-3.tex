\begin{enunciado}{\ejercicio}
  Sean $A$ y $B \en K^{n \times n}$. Probar que:
  \begin{enumerate}[label=(\alph*)]
    \item Si $A \ytext B$ son triangulares superiores, $AB$ es triangular superior.
    \item Si $A \ytext B$ son diagonales, $AB$ es diagonal.
    \item Si $A$ es estrictamente triangular superior (es decir, $a_{ij} = 0$ si $i\geq j$), $A^n = 0$.
  \end{enumerate}
\end{enunciado}

\def\triangularSuperior{
  \begin{tikzpicture}[baseline=-6, scale = 0.5]
    \draw[blue, thin, fill=blue!20,yshift = -4, xshift = -8] (-1,1) -- (1.7,1) -- (1.7,-0.7) -- cycle;
    \node at (0,0) {
      $
        \matriz{cc}{
          \quad &  a_{\green{i}\blue{j}} \\
          0 &
        }
      $
    };
  \end{tikzpicture}
}

\def\diagonal{
  \begin{tikzpicture}[baseline=-6, scale = 0.5]
    \draw[blue, thin, fill=blue!20,yshift = 0, xshift = 0, rotate = -45] (-1.1,0.3) rectangle (1.1,-.3);
    \node at (0,0) {
      $
        \matriz{cc}{
          &  0 \\
          0 &
        }
      $
    };
  \end{tikzpicture}
}

\def\triangularSuperiorEstricta{
  \begin{tikzpicture}[baseline=-6, scale = 0.5]
    \draw[blue, thin, fill=blue!20,yshift = 0, xshift = 0] (-0.6,0.8) -- ++(0,-0.5) -- (0.2,-0.1) -- ++(0.8,0) -- (1,0.8) -- cycle;
    \node at (0,0) {
      $
        \matriz{cc}{
          0 &  a_{\green{i}\blue{j}} \\
          0 & 0
        }
      $
    };
  \end{tikzpicture}
}

\begin{enumerate}[label=(\alph*)]
\item  Una matriz $A$ va a ser triangular superior si todos los número debajo de la diagonal son cero:
$$
  A_{\green{i}\blue{j}}
  \igual{$\llamada1$}
  \llave{ccl}{
    0 & \text{ si } & \green{i} > \blue{j} \\
    a_{\green{i}\blue{j}} & \text{ si } & \green{i} \leq \blue{j}
  }
  \qquad
  \triangularSuperior
$$
Los $a_{\green{i}\blue{j}}$ no tienen que  ser necesariamente distinto a cero. Ahora multiplico dos matrices triangulares superiores:
$$
  [A \cdot B]_{\green{i}\blue{j}} = \sumatoria{\purple{k} = 1}{n} a_{\green{i}\purple{k}} \cdot b_{\purple{k}\blue{j}}
  =
  a_{\green{i}\purple{1}} \cdot b_{\purple{1}\blue{j}} + a_{\green{i}\purple{2}} \cdot b_{\purple{2}\blue{j}} + \cdots + a_{\green{i}\purple{n}} \cdot b_{\purple{n}\blue{j}}
  \igual{$\llamada1$}
  \llave{cc}{
    \sumatoria{\green{i} \leq \purple{k}\leq  \blue{j}}{} a_{\green{i}\purple{k}} \cdot b_{\purple{k}\blue{j}} & \llamada2 \\
    0 & \text{ en otro caso}
  }
$$
Se cumple $\llamada2$ son los que tiene las
\textit{\green{filas} menores o iguales \purple{columnas}} y \textit{\purple{filas} menores o iguales \blue{columnas}},
si no son cero. Básicamente la definición de matriz triangular superior.

\item Esta es un poco más fácil. Una matriz es diagonal si:
$$
  A_{ij}
  \igual{$\llamada1$}
  \llave{ccl}{
    0 & \text{ si } & i \distinto j\\
    a_{ij} &\text{ si } & i = j
  }
  \qquad \diagonal
$$
Nuevamente, los elementos diagnonales no tienen que ser necesariamente distintos de cero. Ahora multiplico dos matrices diagonales:
$$
  [A \cdot B]_{\green{i}\blue{j}} = \sumatoria{\purple{k} = 1}{n} a_{\green{i}\purple{k}} \cdot b_{\purple{k}\blue{j}}
  =
  a_{\green{i}\purple{1}} \cdot b_{\purple{1}\blue{j}} + a_{\green{i}\purple{2}} \cdot b_{\purple{2}\blue{j}} + \cdots + a_{\green{i}\purple{n}} \cdot b_{\purple{n}\blue{j}}
  \igual{$\llamada1$}
  \llave{cc}{
    a_{\green{i}\purple{i}}\cdot b_{\purple{i}\blue{i}} & \llamada2\\
    0 & \text{ en otro caso }
  }
$$
En la sumatoria las \purple{columnas} de los elementos de $A$ coinciden con las filas de los elementos de $B$, pero
solo cuando estemos multiplicando la \green{fila} $i$ con la \blue{columna} $i$ es que ambos elementos podrían ser no nulos.

\item
Una matriz $A$ va a ser triangular superior estricta si todos los número debajo y de la diagonal son cero:
$$
  A_{\green{i}\blue{j}}
  \igual{$\llamada1$}
  \llave{ccl}{
    0 & \text{ si } & \green{i} \geq \blue{j} \\
    a_{\green{i}\blue{j}} & \text{ si } & \green{i} < \blue{j}
  }
  \qquad \triangularSuperiorEstricta
$$
Meto inducción porque es un viaje.
Quiero probar que:
\begin{center}
  $p(n) : A \ytext B \en K^{n \times n}$ matrices triangulares superior estrictas, MTSE, entonces $A \cdot B$ también lo es,
  y además tiene una submatriz también MTSE de un \textit{orden matricial} menos en la esquina superior derecha.
\end{center}

\bigskip

\textit{Caso base:}
\begin{center}
  $p(2) : A \ytext B \en K^{2 \times 2}$ MTSE, entonces $A \cdot B$ también lo es,
  y además tiene una submatriz también MTSE de un \textit{orden matricial} menos en la esquina superior derecha.
\end{center}
Cálculo directo
$$
  A \cdot B =
  \matriz{cc}{
    0 & a_{12} \\
    0 & 0
  }
  \cdot
  \matriz{cc}{
    0 & b_{12} \\
    0 & 0
  }
  =
  \matriz{cc}{
    0 & 0 \\
    0 & 0
  }
$$
por lo tanto $p(\blue{2})$ es verdadera.

\medskip

\textit{Paso inductivo:}
Voy a asumir que para algún $\blue{k} \en \enteros$
\begin{center}
  $p(k) : A \ytext B \en K^{k \times k}$ estrictamente triangular superior, entonces $A \cdot B$ también lo es
  y además tiene una submatriz también MTSE de un \textit{orden matricial} menos en la esquina superior derecha.
\end{center}
es \ul{verdadera}. Por lo tanto ahora quiero probar que:
\begin{center}
  $p(k+1) : A \ytext B \en K^{(k+1) \times (k+1)}$ estrictamente triangular superior, entonces $A \cdot B$ también lo es
  y además tiene una submatriz también MTSE de un \textit{orden matricial} menos en la esquina superior derecha.
\end{center}
Primero voy a ver que onda esto de multiplicar dos MTSE:
$$
  A \cdot B =
  \matriz{ccccc}{
    0 & \blue{a_{12}} & \blue{\cdots} & \blue{\cdots} & \blue{a_{1(k+1)}} \\
    0 & 0 & \blue{\ddots} & \blue{\ddots} & \blue{\vdots} \\
    0 & 0 & \ddots & \blue{\ddots} & \blue{\vdots} \\
    0 & 0 & \cdots & 0 & \blue{a_{k(k+1)}} \\
    0 & 0 & \cdots & 0 & 0
  }
  \cdot
  \matriz{ccccc}{
    0 & \blue{b_{12}} & \blue{\cdots} & \blue{\cdots} & \blue{b_{1(k+1)}} \\
    0 & 0 & \blue{\ddots} & \blue{\ddots} & \blue{\vdots} \\
    0 & 0 & \ddots & \blue{\ddots} & \blue{\vdots} \\
    0 & 0 & \cdots & 0 & \blue{b_{k(k+1)}} \\
    0 & 0 & \cdots & 0 & 0
  }
$$
Notar que las columnas de $A$ y $B$ tienen la forma:
{
\small
$$
  \llave{rcl}{
    \columna_1(A) = A  e_1 & = & 0 \\
    \columna_2(A) = A  e_2 & = & a_{21} e_1  \\
    \columna_3(A) = A  e_3 & = & a_{31} e_1 + a_{32} e_2\\
    \columna_4(A) = A  e_4 & = & a_{41} e_1 + a_{42} e_2 + a_{43} e_3\\
    & \vdots & \\
    \columna_k(A) = A e_k & = & \sumatoria{i = \magenta{1}}{k-1} a_{k \magenta{i}} e_i \\
    \columna_{(k+1)}(A) = A  e_{k+1} & = & \sumatoria{i = \magenta{1}}{k} a_{(k+1) \magenta{i}} e_i \\
  }
  \ytext
  \llave{rcl}{
    \columna_1(B) = B  e_1 & = & 0 \\
    \columna_2(B) = B  e_2 & = & b_{21} e_1  \\
    \columna_3(B) = B  e_3 & = & b_{31} e_1 + b_{32} e_2\\
    \columna_4(B) = B  e_4 & = & b_{41} e_1 + b_{42} e_2 + b_{43} e_3\\
    & \vdots & \\
    \columna_k(B) = B e_k & = & \sumatoria{i = \magenta{1}}{k-1} b_{k \magenta{i}} e_i \\
    \columna_{(k+1)}(B) = B  e_{k+1} & = & \sumatoria{i = \magenta{1}}{k} b_{(k+1) \magenta{i}} e_i \\
  }
$$
}
Entonces al multiplicar $A \cdot B$
$$
  \llave{rcl}{
    \columna_1(A \cdot B) = A \cdot B e_1 & = & A \cdot 0 = 0 \\
    \columna_2(A \cdot B) = A \cdot B e_2 & = & b_{21} A e_1 \igual{\red{!}} 0  \\
    \columna_3(A \cdot B) = A \cdot B e_3 & = & b_{31} A e_1 + b_{32} A e_2 \igual{\red{!}} b_{32} a_{21} e_1\\
    \columna_4(A \cdot B) = A \cdot B e_4 & = & b_{41} A e_1 + b_{42} A e_2 + b_{43} A e_3 \igual{\red{!}}  b_{42}a_{21}e_1 + b_{43}a_{31}e_1 +  b_{43}a_{32}e_2  \\
  & \vdots & \\
  \columna_{(k+1)}(A \cdot B) = A \cdot B e_{k+1}
  & = &
  \sumatoria{i = \magenta{1}}{k} b_{(k+1) \magenta{i}} A \cdot e_i =
  \sumatoria{i = \magenta{1}}{k} \sumatoria{j = \yellow{1}}{\red{k-1}}   b_{(k+1) \magenta{i}} a_{(k+1) \yellow{j}} e_{\yellow{j}}
  }
$$
A esta altura estarás preguntándote ¿que mierda es todo esto?, yo también. Lo que importa es que todas las columnas de $AB$
se construyen con vectores que \ul{tienen un cero más que antes},
entendiendo por \ul{tienen un cero más que antes}:
$$
  (a_1,a_2,a_3,0) \flecha{luego de}[multiplicar] (a'_1, a'_2, 0, 0).
$$

La matriz $AB$ tiene $\columna(AB)_1 = \columna(AB)_2 = 0$ y como todos las columnas \ul{tienen un cero más abajo} es
decir que hay una fila de ceros nueva, quedando así una MTSE de un \textit{orden matricial} menor.

\begin{tikzpicture}[
    rect/.style={draw, thick, minimum width=0.8cm, minimum height=0.8cm},
    submatriz/.style={draw, thick, color=#1, rounded corners=2pt, inner sep=0pt},
    matrix_style/.style={
        matrix of nodes,
        nodes={rect, draw=white, anchor=center, font=\small},
        column sep=0\pgflinewidth,
        row sep=0\pgflinewidth,
        left delimiter=(,
        right delimiter=)
      }
  ]
  \matrix (A) [matrix_style] {
    $0$      & $\blue{a_{12}}$ & $\blue{\cdots}$ & $\blue{\cdots}$ & $\blue{a_{1(k+1)}}$  \\
    $0$      & $0$                                    & $\blue{\ddots}$ & $\blue{\ddots}$ & $\blue{\vdots}$                         \\
    $\vdots$ & $\vdots$                               & $\ddots$        & $\blue{\ddots}$ & $\blue{\vdots}$                         \\
    $0$      & $0$                                    & $\cdots$        & $0$             & $\blue{a_{k(k+1)}}$ \\
    $0$      & $0$                                    & $\cdots$        & $0$             & $0$                                     \\
  };
  \node[submatriz=Cerulean, fit=(A-1-2.north west)(A-4-5.south east)] {};

  \matrix (B) [matrix_style, right=0.6cm of A] {
    $0$      & $\blue{b_{12}}$ & $\blue{\cdots}$ & $\blue{\cdots}$ & $\blue{b_{1(k+1)}}$ \\
    $0$      & $0$                                     & $\blue{\ddots}$ & $\blue{\ddots}$ & $\blue{\vdots}$                         \\
    $\vdots$ & $\vdots$                                & $\ddots$        & $\blue{\ddots}$ & $\blue{\vdots}$                         \\
    $0$      & $0$                                     & $\cdots$        & $0$             & $\blue{b_{k(k+1)}}$ \\
    $0$      & $0$                                     & $\cdots$        & $0$             & $0$                                     \\
  };
  \node[submatriz=Cerulean, fit=(B-1-2.north west)(B-4-5.south east)] {};

  \matrix (C) [matrix_style, right=1.2cm of B] {
    $0$ & $0$ & $\orange{*}$ & $\orange{\cdots}$ & $\orange{*}$      \\
    $0$ & $0$ & $\ddots$     & $\orange{\ddots}$ & $\orange{\vdots}$ \\
    $0$ & $0$ & $\cdots$     & $\ddots$          & $\orange{*}$      \\
    $0$ & $0$ & $\cdots$     & $0$               & $0$               \\
    $0$ & $0$ & $\cdots$     & $0$               & $0$               \\
  };

  \node[left=0.3cm of C.west] {$=$};
  \node[submatriz=orange, fit=(C-1-2.north west)(C-4-5.south east)] {};

  % Labels
  \node[below=0.3cm of A] {$A \en K^{(k+1) \times (k+1)}$};
  \node[below=0.3cm of B] {$B \en K^{(n \times n)}$};
  \node[above=0.3cm of C, align=center] {
    $A \cdot B \en K^{(k+1) \times (k+1)}$ \\
    \orange{MTSE $\en K^{k \times k}$}
  };
\end{tikzpicture}

Por lo tanto probé (creo) por inducción que al multiplicar una matriz MTSE por otra MTSE,
se obtiene una nueva MTSE con una submatriz TSE también de un \textit{orden matricial} menor.

Así es cuestión de multiplicar a $A \en K^{n \times n}$ por sí misma $n$ veces para obtener una matriz nula:
$$
  A^n = 0
$$

\begin{aportes}
  \item \aporte{\dirRepo}{naD GarRaz \github}
\end{aportes}
