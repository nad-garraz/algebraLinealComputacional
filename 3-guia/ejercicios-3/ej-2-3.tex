\begin{enunciado}{\ejercicio}
  Sea $A =
    \matriz{cccc}{
      1 & -1 & 0 & 1 \\
      0 & 1 & 4 & 0 \\
      2 & -1 & 0 & -2 \\
      -3 & 3 & 0 & -1
    } \en \reales^{4 \times 4}$

  \begin{enumerate}[label=(\alph*)]
    \item Escalonar la matriz $A$ multiplicándola a izquierda por matrices
          elementales $T^{ij}(a), a \en \reales, 1\leq i, j \leq 4$, con $i \distinto j$.

          \textit{Recordar que $T^{ij}(a) \en K^{n \times n}$ se define como:}
          $$
            T^{ij}(a) = I_n + a E^{ij}, \quad 1\leq i,j \leq n,\quad i\distinto j, a \en K,
          $$
          siendo $E^{ij}$ las matrices canónicas de $K^{n \times n}$

    \item Hallar la descomposición $LU$ de $A$.

    \item\label{ej:2-item-c} Usando la descomposición del ítem anterior resolver el sistema $Ax = b$,
          para
          $b =
            \matriz{r}{
              1\\
              -7\\
              -5\\
              1
            }$.
  \end{enumerate}
\end{enunciado}

\begin{enumerate}[label=(\alph*)]
  \item Hacer una operación entre \textit{filas}
        es multiplicar por esas matrices $T^{ij}$, pero dado que \textit{me da tremenda pajómetro}
        explota, escribo las $T^{ij}$ para la primera \textit{columna} de ceros no más.
        $$
          % \matriz{cccc}{
          % 1            & 0            & 0  & 0  \\
          % 0            & 1            & 0  & 0  \\
          % -2           & 0            & 1  & 0  \\
          % 3            & 0            & 0  & 1
          % }
          %
          \begin{array}{rcl}
            A =
            \matriz{cccc}{
            1            & -1           & 0 & 1  \\
            \magenta{0}  & 1            & 4 & 0  \\
            \magenta{2}  & -1           & 0 & -2 \\
            \magenta{-3} & 3            & 0 & -1
            }
                         & \equivalente &
            \ub{
              \matriz{cccc}{
            1            & 0            & 0 & 0  \\
            0            & 1            & 0 & 0  \\
            -2           & 0            & 1 & 0  \\
            0            & 0            & 0 & 1
              }
            }{T^{31}(-\frac{2}{1}) = I_4 + (-\frac{2}{1}) E^{31}}
            \cdot
            \matriz{cccc}{
            1            & -1           & 0 & 1  \\
            0            & 1            & 4 & 0  \\
            \magenta{2}  & -1           & 0 & -2 \\
            -3           & 3            & 0 & -1
            }
            =
            \matriz{cccc}{
            1            & -1           & 0 & 1  \\
            0            & 1            & 4 & 0  \\
            \magenta{0}  & 1            & 0 & -4 \\
            -3           & 3            & 0 & -1
            }
            \\
            \\
                         & \equivalente &
            \ub{
              \matriz{cccc}{
            1            & 0            & 0 & 0  \\
            0            & 1            & 0 & 0  \\
            0            & 0            & 1 & 0  \\
            3            & 0            & 0 & 1
              }
            }{T^{41}(\frac{3}{1}) = I_4 + (\frac{3}{1}) E^{41}}
            \cdot
            \matriz{cccc}{
            1            & -1           & 0 & 1  \\
            0            & 1            & 4 & 0  \\
            \magenta{0}  & 1            & 0 & -2 \\
            -3           & 3            & 0 & -1
            }
            =
            \matriz{cccc}{
            1            & -1           & 0 & 1  \\
            \magenta{0}  & 1            & 4 & 0  \\
            \magenta{0}  & 1            & 0 & -4 \\
            \magenta{0}  & 0            & 0 & 2
            } \llamada1
            \\
            \\
          \end{array}
        $$
        \parrafoDestacado[\magic]{
          Ahí entonces están las $T^{ij}$ para hacer ceros en la primera \textit{columna}.
          Y como la \textit{matemagia} en esta materia parece no tener parangón, cuando
          multiplicás esas matrices $T^{ij}$ da lo mismo que sumar los elementos fuera
          de la diagonal componente a componente:
        }
        $$
          T^{41} \cdot T^{31}
          =
          \matriz{cccc}{
            1            & 0            & 0 & 0  \\
            0            & 1            & 0 & 0  \\
            0            & 0            & 1 & 0  \\
            3           & 0            & 0 & 1
          }
          \cdot
          \matriz{cccc}{
            1            & 0            & 0 & 0  \\
            0            & 1            & 0 & 0  \\
            -2           & 0            & 1 & 0  \\
            0            & 0            & 0 & 1
          }
          =
          \matriz{cccc}{
            1            & 0            & 0  & 0  \\
            0            & 1            & 0  & 0  \\
            -2           & 0            & 1  & 0  \\
            3            & 0            & 0  & 1
          }
        $$
        Es gracias a ese resultado que en el próximo paso podría armar solo una matriz
        con la info para triangular toda la \textit{segunda columna}.
        \underline{solo un producto matricial}. Continúo la triangulación de $\llamada1$:
        $$
          \begin{array}{rcl}
            \matriz{cccc}{
            1           & -1           & 0           & 1  \\
            \magenta{0} & 1            & 4           & 0  \\
            \magenta{0} & 1            & 0           & -4 \\
            \magenta{0} & 0            & 0           & 2
            }
                        & \equivalente &
            \ub{
              \matriz{cccc}{
            1           & 0            & 0           & 0  \\
            \magenta{0} & 1            & 0           & 0  \\
            \magenta{0} & -1           & 1           & 0  \\
            \magenta{0} & 0            & 0           & 1
              }
            }{T^{32}(-1) = I_4 + (-\frac{1}{1}) E^{32}}
            \cdot
            \matriz{cccc}{
            1           & -1           & 0           & 1  \\
            \magenta{0} & 1            & 4           & 0  \\
            \magenta{0} & \magenta{1}  & 0           & -2 \\
            \magenta{0} & \magenta{0}  & 0           & 2
            }
            =
            \matriz{cccc}{
            1           & -1           & 0           & 1  \\
            \magenta{0} & 1            & 4           & 0  \\
            \magenta{0} & \magenta{0}  & -4          & -4 \\
            \magenta{0} & \magenta{0}  & \magenta{0} & 2
            }                                             \\
          \end{array}
        $$
        Por lo tanto para que la matriz $A$ quede triangulada superiormente:
        $$
          \ub{
            \matriz{cccc}{
              1            & 0            & 0  & 0  \\
              0            & 1            & 0  & 0  \\
              0            & -1           & 1  & 0  \\
              0            & 0            & 0  & 1
            }
            \cdot
            \matriz{cccc}{
              1            & 0            & 0 & 0  \\
              0            & 1            & 0 & 0  \\
              0            & 0            & 1 & 0  \\
              3           & 0            & 0 & 1
            }
            \cdot
            \matriz{cccc}{
              1            & 0            & 0 & 0  \\
              0            & 1            & 0 & 0  \\
              -2           & 0            & 1 & 0  \\
              0            & 0            & 0 & 1
            }
          }{   T^{32} \cdot T^{41} \cdot T^{31} =
            \matriz{cccc}{
              1            & 0            & 0  & 0  \\
              0            & 1            & 0  & 0  \\
              -2           & -1            & 1  & 0  \\
              3            & 0            & 0  & 1
            }
          }
          \cdot
          \ub{
            \matriz{cccc}{
              1            & -1           & 0  & 1  \\
              0  & 1            & 4  & 0  \\
              2  & -1           & 0  & -2 \\
              -3 & 3            & 0  & -1
            }
          }{A}
          =
          \ub{
            \matriz{cccc}{
              1            & -1           & 0  & 1  \\
              \magenta{0}            & 1            & 4  & 0  \\
              \magenta{0}            & \magenta{0}  & -4 & -4 \\
              \magenta{0}            & \magenta{0}  & \magenta{0} & 2
            }
          }{U}
        $$
        $$
          T^{32} \cdot T^{41} \cdot T^{31} \cdot A = U
        $$

  \item La $U$ está una vez triangulada la matriz $A$. Encontrar la $L$ sale con las matrices que multiplicamos para obtener la matriz triangulada:
        $$
          L^{-1} \cdot A = U
          \Entonces{$\times$ izquierda}[$L$]
          L \cdot L^{-1} \cdot A = L \cdot U
          \sii
          A = L \cdot U
        $$
        El producto de las matrices elementales me forma la inversa de $L: L^{-1}$. Por suerte encontrar la inversa de $(L^{-1})^{-1}$ es sencillo:
        $$
          L^{-1} =
          \matriz{cccc}{
            1            & 0            & 0  & 0  \\
            0            & 1            & 0  & 0  \\
            -2           & -1            & 1  & 0  \\
            3            & 0            & 0  & 1
          }
          \entonces
          L =
          \matriz{cccc}{
            1            & 0            & 0  & 0  \\
            0            & 1            & 0  & 0  \\
            \red{+}2           & \red{+}1            & 1  & 0  \\
            \red{-}3            & 0            & 0  & 1
          }
        $$
        Solo hay que cambiarle los signos a los elementos que estás por debajo de la diagonal.
        $$
          A = LU
          \sisolosi
          \matriz{cccc}{
            1            & -1           & 0  & 1  \\
            0  & 1            & 4  & 0  \\
            2  & -1           & 0  & -2 \\
            -3 & 3            & 0  & -1
          }
          =
          \matriz{cccc}{
            1            & 0            & 0  & 0  \\
            0            & 1            & 0  & 0  \\
            2           & 1            & 1  & 0  \\
            -3            & 0            & 0  & 1
          }
          \cdot
          \matriz{cccc}{
            1            & -1           & 0  & 1  \\
            0            & 1            & 4  & 0  \\
            0            & 0  & -4 & -4 \\
            0            & 0  & 0 & 2
          }
        $$

  \item
        $$
          A \cdot x = b
          \Sii{$A = LU$}
          LU\cdot x = b
          \sii
          L\ub{(U \cdot x)}{y} = b
          \sii
          \llave{ccl}{
            L \cdot y = b & \flecha{$\llamada1$} & \text{Arranco por acá.}\\
            U \cdot x = y& \flecha{$\llamada2$} & \text{Sigo por acá una vez encontrado $y$.}
          }
        $$
        Entonces resuelvo primero $\llamada1$:
        $$
          Ly = b
          \flecha{armo sistema}
          \matriz{cccc|c}{
            1            & 0            & 0  & 0 & 1 \\
            0            & 1            & 0  & 0 & -7 \\
            2           & 1            & 1  & 0  & -5\\
            -3            & 0            & 0  & 1& 1
          }
          \entonces
          y
          \igual{$\llamada3$}
          \matriz{c}{
            1 \\
            -7 \\
            0\\
            4
          }
        $$
        Con la $\llamada3$ resuelvo $\llamada2$:
        $$
          Ux = y
          \flecha{armo sistema}[con $\llamada3$]
          \matriz{cccc|c}{
            1            & -1           & 0  & 1 &1 \\
            0            & 1            & 4  & 0 & -7 \\
            0            & 0            & -4 & -4 & 0             \\
            0            & 0            & 0  &  2 & 4
          }
          \entonces
          \cajaResultado{
            x =
            \matriz{c}{
              0 \\
              1 \\
              -2\\
              2
            }
          }
        $$
\end{enumerate}

Y porque soy un tipazo {\color{blue!10!white}\tiny(y le pifié 1000 veces a las cuentas)} acá tenés el código para corroborar:
\copyPaste
\codigoPython{ej-2/codigo3-2-1.py}

\begin{aportes}
  \item \aporte{\dirRepo}{naD GarRaz \github}
  \item \aporte{\neverGonnaGiveYouUp}{Ale S. \youtube}
\end{aportes}
