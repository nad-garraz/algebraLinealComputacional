\begin{enunciado}{\ejercicio}
  Hallar la factorización $QR$ de las siguientes matrices:
  \begin{center}
    \parbox{0.6\textwidth}{
      \begin{enumerate}[label=\alph*)]
        \begin{multicols}{2}
          \item $
            A =
            \matriz{cc}{
              0 & -4 \\
              0 & 0 \\
              -5 & -2
            },
          $
          \item $
            A =
            \matriz{cc}{
              3 & 2 \\
              4 & 5
            }
          $
        \end{multicols}
      \end{enumerate}
    }
  \end{center}
\end{enunciado}

\hyperlink{teoria-3:qr}{Acá un poco de lore $A = QR$  \click}

\begin{enumerate}[label=\alph*)]
  \item La matriz no es cuadrada \surprise. A la descomposición $QR$ le chupa un huevo. ¡Gram-Schimdteo a $\columna(A)$!
        $$
          \begin{array}{rcl}
            \columna(A) =
            \set{
              \matriz{c}{
            0                                       \\
            0                                       \\
                -5
              }
              ,
              \matriz{c}{
            -4                                      \\
            0                                       \\
                -2
              }
            }
             & \quad\flecha{\magic}\quad          &
            \columna(A)_{\text{BOG}} =
            \set{
              \matriz{c}{
            0                                       \\
            0                                       \\
                -5
              }
              ,
              \matriz{c}{
            -4                                      \\
            0                                       \\
                0
              }
            }                                       \\
             & \quad\flecha{normalizo}\quad       &
            \columna(A)_{\text{BO\red{N}}} =
            \set{
              \matriz{c}{
            0                                       \\
            0                                       \\
                \frac{-5}{\norma{(0,0,-5)}}
              }
              ,
              \matriz{c}{
            \frac{-4}{\norma{(0,0,-4)}}             \\
            0                                       \\
                0
              }
            }                                       \\
             & \quad\flecha{cocinado}[queda]\quad &
            \columna(A)_{\text{BO\red{N}}} =
            \set{
              \matriz{c}{
            0                                       \\
            0                                       \\
                -1
              }
              ,
              \matriz{c}{
            -1                                      \\
            0                                       \\
                0
              }
            }                                       \\
          \end{array}
        $$
        Listo, ahora expreso a $A$ como:
        $$
          A =
          \ub{
            \matriz{cc}{
              0 & -1  \\
              0 & 0  \\
              -1 & 0
            }
          }{Q}
          \cdot
          \ub{
            \matriz{cc}{
              5 & -2  \\
              0 & 4    \\
            }
          }{R}
        $$
        Fijate que:
        $$
          A = QR
          \flecha{$\times Q^*$}[M.A.M.]
          Q^*A = \ub{Q^*Q}{I \en \reales^{2 \times 2}} R =
          \oa{R}{\text{triangular}\\\text{superior}}
          \sii
          Q^* A = R
        $$
        La $Q^*$ te triangula la $A$.

  \item Cuando te ponen una matriz de $2 \times 2$, sabés que las cuentas van a ser un cosa horrible:
        $$
          \begin{array}{rcl}
            \columna(A) =
            \set{
              \matriz{c}{
            3                                 \\
                4
              }
              ,
              \matriz{c}{
            2                                 \\
                5
              }
            }
             & \quad\flecha{\magic}\quad    &
            \columna(A)_{\text{BOG}} =
            \set{
              \matriz{c}{
            3                                 \\
                4
              }
              ,
              \matriz{c}{
            -\frac{28}{25} \vspace{1pt}       \\
                \frac{21}{25}
              }
            }                                 \\
             & \quad\flecha{normalizo}\quad &
            \columna(A)_{\text{BO\red{N}}} =
            \set{
              \matriz{c}{
            \frac{3}{5} \vspace{1pt}          \\
                \frac{4}{5}
              }
              ,
              \matriz{c}{
            -\frac{4}{5} \vspace{1pt}         \\
                \frac{3}{5}
              }
            }\text{\rosa{¡\surprise quedó hermosa!}}
          \end{array}
        $$
        Listo expreso a $A$ como:
        $$
          A =
          \ub{
            \matriz{cc}{
              \frac{3}{5} & -\frac{4}{5} \vspace{1pt}          \\
              \frac{4}{5} & \frac{3}{5}
            }
          }{Q}
          \ub{
            \matriz{cc}{
              5 & \frac{26}{5} \vspace{1pt}          \\
              0 & \frac{7}{5}
            }
          }{R}
        $$
\end{enumerate}

\begin{aportes}
  \item \aporte{\dirRepo}{naD GarRaz \github}
\end{aportes}
