\begin{enumerate}[label=\tiny\purple{\faIcon{snowman}}]
  \item \hypertarget{teoria-3:definida-positiva}{\textit{Matriz definida positiva:}}

        Sea  $A \en \reales^{n \times n}$ definida positiva:
        $$
          \paratodo x \distinto 0 \quad x^t A x > 0 \text{ con } x \en \reales^n
        $$
        Algunas propiedades de las matrices definidas positivas:
        \begin{itemize}
          \item  $A\en \reales^{n \times n} \entonces \existe A^{-1}$.
          \item  Los elementos diagonales son positivos.
          \item  Las submatrices principales también son matrices definidas positivas
        \end{itemize}

  \item \hypertarget{teoría-3:cholesky}{\textit{Descomposición de Cholesky:}}

        La descomposición de Cholesky para una matriz $A$ simétrica y definida positiva:
        $$
          A = L L^t
        $$
        con $L$ triangular inferior. A partir de la descomposición:
        $$
          A = \ob{\tilde{L} U}{\text{la de siempre}}
          \flecha{\quad\magic\quad}
          A = \tilde{L}
          \ob{
              \ua{D}{\text{matriz diagonal}\\ \text{con los elementos}\\ \text{diagonales de } U} \tilde{L}^t}
          {}.
        $$
        \parrafoDestacado[\atencion]{
          $A = \tilde{L} D \tilde{L}^t$, es definida positiva si y solo si $D$ lo es. Como $D$ es diagonal, solo es cuestión de ver que $[D]_{ii} > 0$.
        }
        Finalmente:
        $$
          A = \tilde{L} D \tilde{L}
          \sii
          A = \tilde{L} \sqrt{D} \sqrt{D} \tilde{L}^t 
          \sii
          A = LL^t
        $$
\end{enumerate}
