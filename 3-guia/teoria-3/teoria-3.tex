\begin{enumerate}[label=\tiny\purple{\faIcon{snowman}}]
  \item \textit{Descomposición $LU$:} \quad $A = LU$

        \hacer

  \item \textit{Descomposición $LU$ con swap de filas:} \quad $PA = LU$

        \hacer

  \item \hypertarget{teoria-3:definida-positiva}{\textit{Matriz definida positiva:}}

        Sea  $A \en \reales^{n \times n}$ definida positiva:
        $$
          \paratodo x \distinto 0 \quad x^t A x > 0 \text{ con } x \en \reales^n
        $$
        Algunas propiedades de las matrices definidas positivas:
        \begin{itemize}
          \item  $A\en \reales^{n \times n} \entonces \existe A^{-1}$.
          \item  Los elementos diagonales son positivos.
          \item  Las submatrices principales también son matrices definidas positivas
        \end{itemize}

  \item \hypertarget{teoría-3:cholesky}{\textit{Descomposición de Cholesky:}}

        La descomposición de Cholesky para una matriz $A$ simétrica y definida positiva:
        $$
          A = L L^t
        $$
        con $L$ triangular inferior. A partir de la descomposición:
        $$
          A = \ob{\tilde{L} U}{\text{la misma LU de siempre}}
          \flecha{\quad\magic\quad}
          A = \tilde{L}
          \ua{D}{\text{matriz diagonal}\\ \text{con los elementos}\\ \text{diagonales de } U} \tilde{L}^t
          .
        $$
        \parrafoDestacado[\atencion]{
          $A = \tilde{L} D \tilde{L}^t$, es definida positiva si y solo si
          $D$ lo es. Como $D$ es diagonal, solo es cuestión de ver que $[D]_{ii} > 0$.
        }
        Finalmente:
        $$
          A = \tilde{L} D \tilde{L}
          \sii
          A = \tilde{L} \sqrt{D} \sqrt{D} \tilde{L}^t
          \sii
          A = LL^t
        $$

  \item \hypertarget{teoria-3:proyector}{\textit{Proyectores:}}

        Se llama  \textit{Proyector} a una transformación lineal $P$ que cumple que:
        \begin{itemize}
          \item $P(v) = v$
          \item $P \circ P = P$
        \end{itemize}
        Si $P: V \to V$ es proyector, están las siguientes propiedades:
        \begin{enumerate}[label=\poo]
          \item $v - P(v) \en \nucleo(P) \paratodo v \en V$
          \item  $ \nucleo(P) \sumaDirecta \imagen(P) = V$
                lo mismo que decir que
                $\nucleo(P) \inter \imagen(P) = \set{\bm{0}}$
          \item $P$ un \textit{\blue{proyector ortogonal}}
                $\sii \nucleo(P) \perp \imagen(P)$
          \item $P$ es un \textit{\blue{proyector ortogonal}} expresado en una base \textit{\magenta{ortonormal}}
                $\entonces P = P^t \quad (P = P^* \en \complejos)$
        \end{enumerate}

  \item \textit{Gram-Schmidt:}
        \begin{enumerate}[label={\tiny\faIcon{pray}$_{\arabic*)}$}]
          \item Proceso para construir una base que generá el mismo espacio, pero con elementos perpediculares entre sí.
                $$
                  \ub{\set{v_1, \ldots, v_r}}{\text{Base inicial}}
                  \flecha{Gram-Schmidt}
                  \ub{\set{\purple{u_1}, \ldots, \purple{u_r}}}{\text{Base final}}
                  \quad \text{con } u_i \cdot u_j = 0 \paratodo i \distinto j
                $$

          \item La mecánica es cuentosa pero razonable:
                \begin{enumerate}[label=\arabic*)]
                  \item Agarro el \underline{primer vector de la base inicial} $v_1$ como primer vector de la base final:
                        $$
                          \{\ua{\purple{u_1}}{= v_1}\}
                        $$

                  \item Agarro el \underline{segundo vector de la base inicial} $v_2$ y calculo el segundo vector de la base final:
                        $$
                          \purple{u_2} = v_2 - \frac{\purple{u_1^*} \cdot v_2}{\norma{\purple{u_1}}^2}\purple{u_1}
                          \flecha{actualizo la}[base final]
                          \set{\purple{u_1}, \purple{u_2}}
                        $$

                  \item Y voy así hasta usar todos los vectores el \underline{segundo vector de la base inicial} $v_2$ y calculo el segundo vector de la base final:
                        $$
                          \purple{u_3} = v_3
                          - \frac{\purple{u_1^*} \cdot v_3}{\norma{\purple{u_1}}^2}\purple{u_1}
                          - \frac{\purple{u_2^*} \cdot v_3}{\norma{\purple{u_2}}^2}\purple{u_2}
                          \flecha{actualizo la}[base final]
                          \set{\purple{u_1}, \purple{u_2}, \purple{u_3}}
                        $$

                  \item En general cuando agarre el \underline{$i-$ésimo vector de la base inicial} y calcule el $i-$ésimo vector de la base final:
                        $$
                          \purple{u_i} = v_i
                          - \sumatoria{k = 1}{i-1} \frac{\purple{u_k^*} \cdot v_i}{\norma{\purple{u_k}}^2}\purple{u_k}
                          \flecha{actualizo la}[base final]
                          \set{\purple{u_1}, \ldots, \purple{u_i}}
                        $$

                  \item Así hasta haber usado todos los vectores de la base inicial, para obtener la base con los vectores ortogonalizados:
                        $$
                          \set{\purple{u_1}, \ldots, \purple{u_r}}
                        $$
                        es una base ortogonal, para los amigos una BOG, si quiero que sea una \textit{base ortonormal}, BON:
                        $$
                          \set{\purple{\frac{u_1}{\norma{u_1}}}, \ldots, \purple{\frac{u_r}{\norma{u_r}}}}
                        $$
                \end{enumerate}
        \end{enumerate}

  \item \textit{Matriz Ortogonal:}
        \parrafoDestacado[{{ \huge \angry} }]{
          Una matriz \red{ortogonal} tiene sus columnas \underline{\red{\Large ortonormales}}.
        }
        \begin{itemize}
          \item Si $A$ es una \textit{matriz ortogonal} entonces $A^{-1} = A^*$.
        \end{itemize}

  \item\hypertarget{teoria-3:qr}{ \textit{Factorización $QR$:}\quad  $A = Q R$}

        La factorización $QR$ no tiene un pedo que ver las $LU$, Chole y amigos. $QR$ es un fantasma
        \blue{\Large\faIcon{ghost}} que sale de \textit{Gram-Schmidt}.
        $$
          \ub{\set{v_1, \ldots, v_r}}{\text{Columnas de }A}
          \flecha{Gram-Schmidt}[más normalizar]
          \ub{\set{\purple{\frac{u_1}{\norma{u_1}}}, \ldots, \purple{\frac{u_r}{\norma{u_r}}}}}{\text{Columnas de }Q}
          \quad \text{con } u_i \cdot u_j = 0 \paratodo i \distinto j
        $$
        \begin{enumerate}[label=$\bm{\perp}\arabic*)$]
          \item $Q$ es una \textit{matriz ortogonal} que va a tener en sus columnas el resultado de \red{ortonormalizar} las columnas de $A$.

          \item $R$ es una \textit{matriz triangular superior} con las normas de las columnas de $Q$ en la diagonal

          \item Mini derivación:
                $$
                  \ob{
                    \purple{u_i} = v_i
                    - \sumatoria{k = 1}{i-1} \frac{\purple{u_k^*} \cdot v_i}{\norma{\purple{u_k}}_2^2}\purple{u_k}
                  }{\text{Gram-Schmidt, duro y puro}}
                  \flecha{despejo el $v_i$}[y acomodo]
                  \ob{
                    \ua{v_i}{
                      \text{col}$-i$\\\text{de} A
                    } =
                    \purple{u_i} + \sumatoria{k = 1}{i-1} \frac{\purple{u_k^*} \cdot v_i}{\norma{\purple{u_k}}} \frac{\purple{u_k}}{\norma{\purple{u_k}}}
                  }{\llamada1}
                $$
                Eso ahora lo uso para armar $A = QR$:
                $$
                  \ub{
                    \matriz{c|c|c}{
                      &  & \\
                      v_1 & v_2 & v_3\\
                      &  & \\
                    }
                  }{A}
                  =
                  \ub{
                    \matriz{c|c|c}{
                      &  & \\
                      \frac{\purple{u_1}}{\norma{\purple{u_1}}} &\frac{\purple{u_2}}{\norma{\purple{u_2}}} &\frac{\purple{u_3}}{\norma{\purple{u_3}}} \\
                      &  & \\
                    }
                  }{Q}
                  \cdot
                  \ub{
                    \matriz{c|c|c}{
                      \norma{\purple{u_1}} & \frac{\purple{u_1^*} \cdot v_2}{\norma{\purple{u_2}}} & \frac{\purple{u_1^*} \cdot v_3}{\norma{\purple{u_1}}} \\
                      0 & \norma{\purple{u_2}} & \frac{\purple{u_2^*} \cdot v_3}{\norma{\purple{u_2}}}\\
                      0 & 0 &\norma{\purple{u_3}}
                    }
                  }{R}
                $$
                Ya sé que hice el ejemplo matricial en $\reales^3$, pero con el poder de tu imaginación fijate que la columna $j-$ésima para una matriz
                imaginaria de $n\times n$ sería algo así:
                $$
                  \columna(A)_j =
                  v_j
                  \igual{$\llamada1$}
                  \sumatoria{k = 1}{j-1} \frac{\purple{u_k^*} \cdot v_j}{\norma{\purple{u_k}}} \frac{\purple{u_k}}{\norma{\purple{u_k}}}
                  +  \purple{u_j}
                $$

          \item No sé si ayuda a la notación o no, pero quiero cambiar la notación así:
                $$
                  \set{\purple{\frac{u_1}{\norma{u_1}}}, \ldots, \purple{\frac{u_r}{\norma{u_r}}}}
                  \flecha{le pongo la gorra}
                  \set{\green{\hat{u}_1}, \ldots, \green{\hat{u}_r}}
                  \text{ donde los }
                  \green{\hat{u}_i} \cdot \green{\hat{u}_j} \ua{=}{i\distinto j} 0 \ytext \norma{\green{\hat{u}}_i} = 1 \paratodo i
                $$
                Dejando así la expresión de la descomposión:
                $$
                  \ub{
                    \matriz{c|c|c}{
                      &  & \\
                      v_1 & v_2 & v_3\\
                      &  & \\
                    }
                  }{A}
                  =
                  \ub{
                    \matriz{c|c|c}{
                      &  & \\
                      \green{\hat{u}_1} & \green{\hat{u}_2} &\green{\hat{u}_3} \\
                      &  & \\
                    }
                  }{Q \text{ con cols \red{ortonormales}} }
                  \cdot
                  \ub{
                    \matriz{c|c|c}{
                      \norma{\purple{u_1}} & \green{\hat{u}_1^*} \cdot v_2 & \green{\hat{u}_1^*} \cdot v_3 \\
                      0 & \norma{\purple{u_2}} & \green{\hat{u}_2^*} \cdot v_3 \\
                      0 & 0 &\norma{\purple{u_3}}
                    }
                  }{R \text{ triangular superior}}
                $$
        \end{enumerate}

  \item \textit{Matriz de HouseHolder:}
        \hacer
\end{enumerate}
