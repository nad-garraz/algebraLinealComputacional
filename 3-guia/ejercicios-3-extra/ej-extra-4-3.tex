\begin{enunciado}{\ejExtra}
  Para cada $a \en \reales$ sea la matriz
  $A(a) =
    \matriz{cccc}{
      1 & -1 & 0 & 1 \\
      0 & 2 & -1 & 3 \\
      -1 & 1 & 2 & -4 \\
      1 & 1 & 1 & a
    }.
  $

  \begin{enumerate}[label=\alph*)]
    \item Hallar la descomposición $LU$ de $A(a)$ para cada $a \en \reales$.
    \item Calcular $\condicion_\infinito(A(a))$ cuando $a \to 1$.
    \item Explicar las razones por las cuales
          $M(a) =
            \matriz{cccc}{
              0 & 2 & -1 & 3 \\
              -1 & 1 & 2 & -4 \\
              1 & -1 & 0 & 1 \\
              1 & 1 & 1 & a
            }
          $
  \end{enumerate}
\end{enunciado}

\begin{enumerate}[label=\alph*)]
  \item\label{ej-extra-4:itema} Triángulo y después le cambio los signos para encontrar la $L$.
        $$
          A =
          \ub{
            \matriz{cccc}{
              1 & 0 & 0 & 0 \\
              0 & 1 & 0 & 0 \\
              -1 & 0 & 1 & 0 \\
              1 & 1 & 1 & 1
            }
          }{
            L
          }
          \ub{
            \matriz{cccc}{
              1 & -1 & 0 & 1 \\
              0 & 2 & -1 & 3 \\
              0 & 0 & 2 & -3 \\
              0 & 0 & 0 & a-1
            }
          }{
            U
          }
        $$

  \item Cómo calcular la condición cuando $a \to 1$. En esta clase de ejercicio hay que usar la cota para la condición:
        $$
          \condicion(A)
          \geq
          \supremo \set{\frac{\norma{A}}{\norma{A - B}} : B \text{ es singular}}.
        $$
        Una $B$ singular:
        $$
          B =
          \matriz{cccc}{
            1 & -1 & 0 & 1 \\
            0 & 2 & -1 & 3 \\
            -1 & 1 & 2 & -4 \\
            1 & 1 & 1 & \red{1}
          }
        $$
        La elección de esa $B$ sale con ayuda de la $U$ del ítem \ref{ej-extra-4:itema}, donde si $a = 1$ me queda una fila de \red{ceros}.

        $$
          \limite{a}{1} \condicion_\infinito(A(a))
          \geq
          \limite{a}{1} \frac{\norma{A(a)}_\infinito}{\norma{A(a) - B}_\infinito} =
          \limite{a}{1} \frac{8}{|a - \red{1}|} = +\infinito
        $$
        Por lo tanto:
        $$
          \cajaResultado{
            \limite{a}{1} \condicion_\infinito(A(a)) = +\infinito
          }
        $$

  \item No voy a poder encontrar una descomposición que cumpla con las condiciones que tienen que cumplir
        tanto la matriz $L$ como la $M$:
        $$
          A =  L \cdot U
          \entonces
          [A]_{11} = 0 = [L]_{11} [U]_{11} \sii [U]_{11} = 0
        $$
        dado que $L$ tienen unos en su diagonal. Luego:
        $$
          [A]_{21} = -1 = [L]_{21} \ub{[U]_{11}}{ = \red{0}} \faIcon{skull} = 0 \quad \text{absurdo}.
        $$

        Puedo permutar filas para hacer que $M(a) \to A(a)$:
        $$
          P_1 =
          \matriz{cccc}{
            0 & 1 & 0 & 0 \\
            1 & 0 & 0 & 0 \\
            0 & 0 & 1 & 0 \\
            0 & 0 & 0 & 1
          }
          \ytext
          P_2 =
          \matriz{cccc}{
            1 & 0 & 0 & 0 \\
            0 & 0 & 1 & 0 \\
            0 & 1 & 0 & 0 \\
            0 & 0 & 0 & 1
          }
        $$
        Por lo tanto:
        $$
          P M(a) = P_2 P_1 M(a) = A(a)
          \quad
          \text{ con }
          \quad
          P =
          \matriz{cccc}{
            0 & 0 & 1 & 0 \\
            1 & 0 & 0 & 0 \\
            0 & 1 & 0 & 0 \\
            0 & 0 & 0 & 1
          }
        $$
\end{enumerate}

\begin{aportes}
  \item \aporte{\dirRepo}{naD GarRaz \github}
\end{aportes}
