\begin{enunciado}{\ejExtra}\fechaEjercicio{final 5/3/24}
  Dada la base $B = \set{1,2,0}; (0,1,1); (0,0,1)$ de $\reales^3$ y la transformación lineal
  $f: \reales^3 \to \reales^3$ tal que
  $$
    |f|_{BB} =
    \matriz{ccc}{
      1 & 0 & 0 \\
      1 & 0 & 1 \\
      1 & 0 & 1
    }
  $$
  \begin{enumerate}[label=\alph*)]
    \item Dar una base de $\nucleo(f)$ y de $\imagen(f)$.
    \item Decidir si $\reales^3 = \nucleo(f) \sumaDirecta \imagen(f)$.
    \item Definir $P: \reales^3 \to \reales^3$ proyector ortogonal tal que $\imagen(P) = \imagen(f)$.
  \end{enumerate}
\end{enunciado}

\begin{enumerate}[label=\alph*)]
  \item El subespacio imagen de una transformación está generado por los vectores columnas. Se ven a ojo \ul{dos} vectores \textit{linealmente independientes}
        si bien \magenta{son las coordenadas en base $B$ de los vectores} con los que se puede armar la base.

        Si $\dim(Im(f)) = 2$ entonces $\dim(\nucleo(f)) = 1$, por el teorema de la dimensión para transformaciones
        lineales. Las bases serían:
        $$
          B_{\imagen(f)} = \set{(1,3,2)^t, (0,1,2)^t}
          \ytext
          B_{\nucleo(f)} = \set{(0,1,1)^t}
        $$

  \item Es cuestión de ver si los vectores son \textit{linealmente independientes}. La \textit{suma directa} se da cuando los subespacios son disjuntos:
        $$
          \matriz{ccc}{
            1 & 3 & 2  \\
            0 & 1 & 2  \\
            0 & 1 & 1
          }
          \flecha{$F_3 - F_2 \to F_3$}
          \matriz{ccc}{
            1 & 3 & 2  \\
            0 & 1 & 2  \\
            0 & 0 & -1
          }
        $$
        Sí, el núcleo y la imagen de la transformación $f$ están en suma directa.

  \item Un proyector ortogonal cumple la defición de proyector (¡Dah!) y también que su $\nucleo(P) \perp \imagen(P)$.

        Para encontrar un vector perpendicular a la imagen de $f$ y así usarlo como núcleo de $P$:
        $$
          \llave{rcl}{
            (x_1, x_2, x_3) \cdot (1, 3, 2) = 0\\
            (x_1, x_2, x_3) \cdot (0, 1, 2) = 0
          }
          \sii
          (x_1, x_2, x_3) = (4, -2, 1) \sii \boxed{B_{\nucleo(P)} = \set{(4,-2,1)}}
        $$
        $$
          \cajaResultado{
            \llave{rcl}{
              P(1,3,2) & = & (1,3,2) \\
              P(0,1,2) & = & (0,1,2) \\
              P(4,-2,1) & = & (0,0,0)
            }
          }
        $$
        Queda así definido el \textit{proyector ortogonal} pedido. Se cumple que:
        $$
          Pv = v \paratodo v \en \imagen(P),\, \nucleo(P) \perp \imagen(P) \ytext \imagen(P) = \imagen(f).
        $$
\end{enumerate}

\begin{aportes}
  \item \aporte{\dirRepo}{naD GarRaz \github}
\end{aportes}
