% ======================
% =========CODIGO=======
% ======================
%

\definecolor{pythonbg}{RGB}{171, 178, 191}        % Dark background
\definecolor{pythonfg}{RGB}{40, 44, 52}     % Light gray text

\newcommand{\codigoPython}[1]{
  \begin{tcolorbox}[
      breakable,
      colback=pythonbg,
      colframe=pythonbg,
      rounded corners,
      boxrule=0pt,
      left=2pt,
      right=2pt,
      top=1pt,
      bottom=1pt,
      width=0.85\textwidth
    ]
    \lstinputlisting[
      language=python,
      basicstyle=\color{pythonfg}\ttfamily\footnotesize,
      backgroundcolor=\color{pythonbg},
      frame=jigsaw,
      numbers=none,
      numberstyle=\tiny\color{gray},
      breaklines=true,
      showstringspaces=false
    ]{./codigos-\guia/#1}
  \end{tcolorbox}
}

% \newcommand{\codigoPython}[1]{
%   \begin{snugshade} % el color está definido en shadecolor  
%     \lstinputlisting[language=python]{./codigos-\guia/#1}
%   \end{snugshade}
% }

% pgfplots
\pgfplotsset{every axis legend/.style={
      cells = {anchor=center},
      rounded corners = 3pt,
      draw=black,
      fill=white,
      shape=rectangle,
    },
  every axis/.style={
      rounded corners = 2pt
    }
}

% Definiciones y nuevos comandos:def
% =============
% Conjuntos
\DeclareMathOperator{\partes}{\mathcal P}
\DeclareMathOperator{\relacion}{\,\mathcal{R}\,}
\DeclareMathOperator{\norelacion}{\,\cancel{\relacion}\,}
\DeclareMathOperator{\universo}{\mathcal U}
\DeclareMathOperator{\reales}{\mathbb R}
\DeclareMathOperator{\naturales}{\mathbb N}
\DeclareMathOperator{\enteros}{\mathbb Z}
\DeclareMathOperator{\racionales}{\mathbb Q}
\DeclareMathOperator{\irracionales}{\mathbb I}
\DeclareMathOperator{\complejos}{\mathbb C}
\DeclareMathOperator{\accion}{\xspace\red{\cdot}\xspace}
\DeclareMathOperator{\columna}{\text{Col}}
\DeclareMathOperator{\fila}{\text{Fila}}
\DeclareMathOperator{\K}{\mathbb K} % cuerpo K
\DeclareMathOperator{\vacio}{\varnothing}
\DeclareMathOperator{\union}{\cup}
\DeclareMathOperator{\inter}{\,\cap\,}
\DeclareMathOperator{\sumaDirecta}{\oplus}
\DeclareMathOperator{\diferencia}{\ \setminus \ }
\DeclareMathOperator{\y}{\land}
\DeclareMathOperator{\nucleo}{\text{Nu}}
\DeclareMathOperator{\imagen}{\text{Im}}
\DeclareMathOperator{\dimension}{\text{dim}}
\DeclareMathOperator{\rango}{\text{rg}}
\DeclareMathOperator{\traza}{\text{tr}}
\DeclareMathOperator{\normaBullet}{\Vert\cdot\Vert}
\newcommand{\norma}[1]{\left\Vert #1 \right\Vert}
\DeclareMathOperator{\condicion}{\text{cond}}
\DeclareMathOperator{\multiGeo}{\text{mg}}
\DeclareMathOperator{\multiAri}{\text{ma}}

\def\o{\lor}
\def\neg{\sim}

\DeclareMathOperator{\equivalente}{\leftrightsquigarrow}
\def\entonces{\implies}
\def\noEntonces{\centernot\Rightarrow}

\def\sisolosi{\iff} % largo
\def\sii{\Leftrightarrow} % corto

\def\clase{\overline}
\def\ord{\text{ord}}

\def\existe{\,\exists\,}
\def\noexiste{\,\nexists\,} \def\paratodo{\ \, \forall}
\def\distinto{\neq}
\def\en{\in}
\def\talque{\;/\;}

% =====
\def\qvq{\text{ quiero ver que }}

%funciones
\DeclareMathOperator{\dom}{Dom}
\DeclareMathOperator{\cod}{Cod}
\def\F{\mathcal F}
\def\comp{\circ}
\def\inv{^{-1}}
\def\infinito{\infty}

% Llaves, paréntesis, contenedores
\newcommand{\llave}[2]{ \left\{ \begin{array}{#1} #2 \end{array}\right. }
\newcommand{\llaveInv}[2]{ \left\} \begin{array}{#1} #2 \end{array}\right. }
\newcommand{\llaves}[2]{ \left\{ \begin{array}{#1} #2 \end{array} \right\} }

\newcommand{\matriz}[2]{\left( \begin{array}{#1} #2 \end{array} \right)}
\newcommand{\bloque}[2]{\left[ \begin{array}{#1} #2 \end{array} \right]}
\newcommand{\triangulacion}[1]{\ensuremath{\begin{array}{c} #1 \end{array}}}
\newcommand{\deter}[2]{\left| \begin{array}{#1} #2 \end{array} \right|}
\newcommand{\lista}[2][(1)]{\begin{enumerate}[\bf #1]\setlength\itemsep{-0.6ex} #2 \end{enumerate}}
\newcommand{\listal}[2][-0.6ex]{\begin{enumerate}[\bf(a)]\setlength\itemsep{#1} #2 \end{enumerate}}

% naturales
\newcommand{\limite}[2]{\lim\limits_{#1 \to #2}}
\newcommand{\sumatoria}[2]{\sum\limits_{#1}^{#2}}
\newcommand{\productoria}[2]{\prod\limits_{#1}^{#2}}
\DeclareMathOperator*{\maximo}{\text{máx}}
\DeclareMathOperator*{\minimo}{\text{mín}}
\DeclareMathOperator*{\infimo}{\text{ínf}}
\DeclareMathOperator*{\supremo}{\text{sup}}
\newcommand{\kmasuno}[1]{\underbrace{#1}_{k+1\text{-ésimo}}}
\newcommand{\HI}[1]{\underbrace{#1}_{\text{HI}}}

% % enteros
\def\divideA{\, \big| \,}
\def\noDivide{\centernot\divideA}
\def\congruente{\, \equiv \,}
\def\noCongruente{\, \not\equiv \,}
\newcommand{\congruencia}[3]{#1 \equiv #2 \;(#3)}
\newcommand{\noCongruencia}[3]{#1 \not\equiv #2 \;(#3)}
\newcommand{\conga}[1]{\stackrel{\mathclap{(#1)}}{\congruente}}
\newcommand{\divset}[2]{\mathcal{D}(#1) = \set{#2}}
\newcommand{\divsetP}[2]{\mathcal{D_+}(#1) = \set{#2}}
\newcommand{\ub}[2]{ \underbrace{\textstyle #1}_{\mathclap{\substack{#2}}} }
\newcommand{\ob}[2]{ \overbrace{\textstyle #1}^{\mathclap{\substack{#2}}} }
\newcommand{\ua}[2]{\underset{\everymath{\textstyle}\mathclap{\substack{\downarrow \\  #2}}}{#1}}
\newcommand{\oa}[2]{\overset{\everymath{\textstyle}\mathclap{\substack{#2 \\ \uparrow }}}{#1}}
\def\cop{\, \perp \, }
\def\nocop{\, \not\perp \, }

% complejos
\DeclareMathOperator{\re}{Re}
\DeclareMathOperator{\im}{Im}
\DeclareMathOperator{\argumento}{arg}
\newcommand{\conj}[1]{\overline{#1}}

% Polinomios
\DeclareMathOperator{\cp}{cp}
\DeclareMathOperator{\gr}{gr}
\DeclareMathOperator{\mult}{mult}
\newcommand{\divPol}[2]{\polylongdiv[style=D]{#1}{#2}}
\newcommand{\mcd}[2]{\polylonggcd{#1}{#2}}

% =====
% Miscelanea
% =====
\def\ot{\leftarrow}
\newcommand{\estabien}{{\color{blue} Consultado, está bien. \checkmark}}
\newcommand{\hacer}{
  {\color{red!80!black}{\tiny\faIcon{flushed}... hay que hacerlo! \faIcon{sad-cry}}}\par
  {\color{black!70!white}
    \tiny Si querés mandá la solución $\to$ \href{\dirTelegram}{\blue{al grupo de Telegram}  \small\telegram},
    o  mejor aún si querés subirlo en \LaTeX $\to$ \href{\dirRepo}{una \textit{pull request} al \small \github}.
  }\par
}

\newcommand{\Hacer}{{\color{black!30!red}\Large Hacer!}}
\def\Tilde{\quad\checkmark}
\def\ytext{\text{\ \, y\ \, }}
\def\otext{\quad\text{o}\quad}
\newcommand{\cajaResultado}[1]{\fcolorbox{orange}{white}{\ensuremath{\displaystyle#1}}}

% Estrellita para hacer llamadas de atención, viene en divertidos colores
% para coleccionar.
\newcommand{\llamada}[1]{
  \begingroup% Scope para la variable colortemp
  \ifcase \numexpr#1 mod 6\relax
    \def\colortemp{cyan}
  \or
    \def\colortemp{magenta}
  \or
    \def\colortemp{OliveGreen}
  \or
    \def\colortemp{YellowOrange}
  \or
    \def\colortemp{Cerulean}
  \or
    \def\colortemp{Violet}
  \or
    \def\colortemp{Purple}
  \fi
  \textcolor{\colortemp}{\text{{\tiny \faIcon{star}}}^{\scriptscriptstyle#1}}
  \endgroup
}

% separadores
\newcommand{\separador}{
  \par\noindent\rule{\linewidth}{0.4pt}\par
}
\newcommand{\separadorCorto}{
  \par\noindent\rule{0.5\linewidth}{0.4pt}\par
}

% Colores
\newcommand{\red}[1]{\textcolor{red}{#1}}
\newcommand{\green}[1]{\textcolor{OliveGreen}{#1}}
\newcommand{\blue}[1]{\textcolor{Cerulean}{#1}}
\newcommand{\cyan}[1]{\textcolor{cyan}{#1}}
\newcommand{\yellow}[1]{\textcolor{YellowOrange}{#1}}
\newcommand{\orange}[1]{\textcolor{Orange}{#1}}
\newcommand{\magenta}[1]{\textcolor{magenta}{#1}}
\newcommand{\marron}[1]{\textcolor{brown}{#1}}
\newcommand{\purple}[1]{\textcolor{purple}{#1}}
\newcommand{\violet}[1]{\textcolor{violet}{#1}}
\newcommand{\rosa}[1]{\textcolor{pink}{#1}}
\newcommand{\negro}[1]{\textcolor{black}{#1}}
\newcommand{\transparente}[1]{\color[rgb]{1,1,1,0.5}{#1}}

\definecolor{shadecolor}{RGB}{240,255,255}
\newenvironment{sombrita}{\begin{snugshade}}{\end{snugshade}}

% Conjuntos entre llaves y paréntesis
% te ahorrás escribir los \left y \right, así dejando el código más legible.
\newcommand{\set}[1] {\left\{ #1 \right\}}
\newcommand{\ket}[1] {\left\langle #1 \right\rangle}
\newcommand{\parentesis}[1]{ \left( #1 \right) }

% Stackrel text. Es para ahorrarse ecribir el \text
\newcommand{\stacktext}[2]{ \stackrel{\text{#1}}{#2} }

% Dado que muchas veces ponemos cosas sobre un signo '='
%  acá está el comando para escribir \igual{arriba}[abajo] con texto!
\NewDocumentCommand{\igual}{m o}{
  \IfNoValueTF{#2}{
    \overset{\mathclap{\text{#1}}}=
  }{
    \overset{\mathclap{\text{#1}}}{\underset{\mathclap{\text{#2}}}=}
  }
}
% Dado que muchas veces ponemos cosas sobre un signo '='
%  acá está el comando para escribir \igual{arriba}[abajo] con texto!
\NewDocumentCommand{\mayorIgual}{m o}{
  \IfNoValueTF{#2}{
    \overset{\mathclap{\text{#1}}}\geq
  }{
    \overset{\mathclap{\text{#1}}}{\underset{\mathclap{\text{#2}}}\geq}
  }
}
% Dado que muchas veces ponemos cosas sobre un signo '='
%  acá está el comando para escribir \igual{arriba}[abajo] con texto!
\NewDocumentCommand{\menorIgual}{m o}{
  \IfNoValueTF{#2}{
    \overset{\mathclap{\text{#1}}}\leq
  }{
    \overset{\mathclap{\text{#1}}}{\underset{\mathclap{\text{#2}}}\leq}
  }
}

\NewDocumentCommand{\menor}{m o}{
  \IfNoValueTF{#2}{
    \overset{\mathclap{\text{#1}}}<
  }{
    \overset{\mathclap{\text{#1}}}{\underset{\mathclap{\text{#2}}}<}
  }
}

%=======================================================
% Comandos con flechas extensibles.
%=======================================================
% *Flechita* extensible con texto {arriba} y [abajo] 
\NewDocumentCommand{\flecha}{m o}{%
  \IfNoValueTF{#2}{%
    \xrightarrow[]{\text{#1}}
  }{
    \xrightarrow[\text{#2}]{\text{#1}}
  }
}
% *Si solo si* extensible con texto {arriba} y [abajo] 
\NewDocumentCommand{\Sii}{m o}{%
  \IfNoValueTF{#2}{%
    \xLeftrightarrow[]{\text{#1}}
  }{
    \xLeftrightarrow[\text{#2}]{\text{#1}}
  }
}

% *Si solo si* extensible con texto {arriba} y [abajo] 
\NewDocumentCommand{\Entonces}{m o}{%
  \IfNoValueTF{#2}{%
    \xRightarrow[]{\text{#1}}
  }{
    \xRightarrow[\text{#2}]{\text{#1}}
  }
}

%=======================================================
% fin comandos con flechas extensibles.

% como el stackrel pero también se puede poner algo debajo
\newcommand{\taa}[3]{ % [t]exto [a]rriba y [a]bajo
  \overset{\mathclap{#1}}{\underset{\mathclap{#2}}{#3}}
}

%Update time
\def\update{\tiny
  {\today\ @ \currenttime}
}
\def\updateDos{
  {\today\ @ \currenttime}
}
\newcommand{\parrafoAcotado}[2][0.6]{
  \begin{center}
    \parbox{#1\textwidth}{
      \centering
      #2
    }
  \end{center}
}

% Párrafo destacado:
\newcommand{\parrafoDestacado}[2][]{
  \begin{center}
    #1\qquad
    \parbox{0.7\textwidth}{
      \centering
      #2
    }
    \qquad#1
  \end{center}
}

% Contributors
\newcommand{\aporte}[2]{
  \href{#1}{#2}
}

\newenvironment{aportes}{
  \vspace{10pt}
  \par
  \begin{minipage}{0.9\linewidth}
    \tt\footnotesize
    Dale las gracias y un poco de amor \rosa{\faIcon{heart}} a los que contribuyeron!
    Gracias por tu aporte:
    \vspace{-10pt}
    \begin{multicols}{3}
      \begin{itemize}[label={\tiny\yellow{\faIcon{medal}}}]
        }{
      \end{itemize}
    \end{multicols}
  \end{minipage}
  \medskip
}

%=======================================================
% sección ejercicio con su respectivo formato y contador
%=======================================================
\newcounter{ejercicio}[section] % contador que se resetea en cada sección
\renewcommand{\theejercicio}{\arabic{ejercicio}} % el contador es un número arabic
\newcommand{\ejercicio}{%
  \stepcounter{ejercicio}% incremento en uno
  \titleformat{\section}[runin]{\bfseries}{\theejercicio}{}{}%
  \section*{Ejercicio \theejercicio.}\labelEjercicio{ej:\theejercicio}
}

% Label y refencia para ejercicio hay alguna forma más elegante de hacer esto?
\newcommand{\labelEjercicio}[1]{
  \addtocounter{ejercicio}{-1} % counter - 1
  \refstepcounter{ejercicio} % referencia al anterior y luego + 1
  \label{#1}}
\newcommand{\refEjercicio}[1]{{\bf\ref{#1}.}}

\def\fueguito{{\color{orange}{\faIcon{fire}}}}
\newcounter{ejExtra}[section] % contador que se resetea en cada sección
\renewcommand{\theejExtra}{\arabic{ejExtra}} % el contador es un número arabic
\newcommand{\ejExtra}{%
  \stepcounter{ejExtra}% incremento en uno
  \titleformat{\section}[runin]{\bfseries}{\theejExtra}{1em}{}%
  % Es como una sección. Le pongo un ícono, luego el número del ejercicio con la etiqueta para poder
  % linkearlo en el índice u otro lugar.
  % con \ref{ejExtra:{numero del ejercicio}} es que salto al ejercicio.
  \section*{\fueguito\theejExtra.}\labelEjExtra{ejExtra:\theejExtra}
}

% Label y refencia para ejercicio hay alguna forma más elegante de hacer esto?
\newcommand{\labelEjExtra}[1]{
  \addtocounter{ejExtra}{-1} % counter - 1
  \refstepcounter{ejExtra} % referencia al anterior y luego + 1
  \label{#1} % etiqueta para cada ejercicio extra
}
% Con esto llamos al ejercicio extra
\newcommand{\refEjExtra}[1]{
  {\fueguito\bf\ref{#1}.}
}

%=======================================================
% fin sección ejercicio con su respectivo formato y contador
%=======================================================

\newenvironment{enunciado}[1]{% Toma un parametro obligatorio: \ejExtra o \ejercicio 
  \par
  \noindent
  \begin{minipage}{\linewidth}
    \separador % linea sobre el enunciado
    #1
    }% contenido
    {
    \separadorCorto % linea debajo del enunciado
    \par
  \end{minipage}\par
}

%%% Emoticones
\def\poo{\marron{\faIcon{poo}}\xspace}
\def\angry{\faIcon[regular]{angry}\xspace}
\def\atencion{\faIcon{exclamation-triangle}\xspace}
\def\grimace{\faIcon[regular]{grimace}\xspace}
\def\meh{\faIcon[regular]{meh}\xspace}
\def\mehBlank{\faIcon[regular]{meh-blank}\xspace}
\def\rollingEyes{\faIcon[regular]{meh-rolling-eyes}\xspace}
\def\surprise{\faIcon[regular]{surprise}\xspace}
\def\magic{\href{https://www.youtube.com/watch?v=4u20AEDYhcw}{\faIcon{magic}}\xspace}
\def\python{\texttt{Python} \faIcon{python}\xspace}
\def\click{{\tiny click click \faIcon{mouse}}}
\def\copyPaste{
  \begin{center}
    \fcolorbox{green}{white}{
      \faIcon{linux}
      Si hacés un \texttt{copy paste} de este código debería funcionar lo más bien
      \faIcon{linux}
    }
  \end{center}
}

%%% (☞⌐▀͡ ͜ʖ͡▀ )☞ Yo mama
\def\superIdol{https://www.youtube.com/watch?v=DKpaKHUlyBY}
\def\videosTeresa{https://www.youtube.com/@AlgebraIC-gu7oc/videos}
\def\videosPracticas{https://www.youtube.com/playlist?list=PLEtdiZTXB5c5e9BOcqbnROTHevanNfz6U}
\def\justDoIt{https://www.youtube.com/watch?v=ZXsQAXx_ao0}
\def\dontWorryAboutAThing{https://www.youtube.com/watch?v=4k2PJFPu57Y}
\def\zanguango{https://www.youtube.com/watch?v=Uzcl2gNL3zg&t=10s}
\def\chinito{https://www.youtube.com/watch?v=ebz4xuPf-is}
\def\neverGonnaGiveYouUp{https://www.youtube.com/watch?v=dQw4w9WgXcQ}
\def\mindExplosion{https://www.youtube.com/watch?v=9CS7j5I6aOc}

%%% Iconos más usados
\def\github{{\color{violet!80!black}{\faIcon{github}}}}
\def\instagram{\faIcon{instagram}}
\def\tiktok{\faIcon{tiktok}}
\def\linkedin{\blue{\faIcon{linkedin}}}
\def\youtube{\color{red!70}{\faIcon{youtube}}}
\def\telegram{\blue{\faIcon{telegram}}}
