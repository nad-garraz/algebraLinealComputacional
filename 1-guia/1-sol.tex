\documentclass[12pt, a4paper, spanish, twoside]{article}
% Sacar draft para que aparezcan las imagenes.
% Opciones: 12pt, 10pt, 11pt, landscape, twocolumn, fleqn, leqno...
% Opciones de clase: article, report, letter, beamer...

% Paquetes:
% =========
\usepackage[headheight=110pt, top = 2cm, bottom = 2cm, left=1cm, right=1cm]{geometry} %modifico márgenes
\usepackage[T1]{fontenc} % tildes
\usepackage[utf8]{inputenc} % Para poder escribir con tildes en el editor.
\usepackage[english]{babel} % Para cortar las palabras en silabas, creo.
\usepackage[ddmmyy]{datetime}
\usepackage{amsmath} % Soporte de mathmatics
\usepackage{mathtools} % Más herramientas para matemáctica
\usepackage{amssymb} % fuentes de mathmatics
\usepackage{array} % Para tablas y eso
\usepackage[dvipsnames,table]{xcolor} % Para colorear el texto: black, blue, brown, cyan, darkgray, gray, green, lightgray, lime, magenta, olive, orange, pink, purple, red, teal, violet, white, yellow.
\usepackage{color} % Para colorear el texto: black, blue, brown, cyan, darkgray, gray, green, lightgray, lime, magenta, olive, orange, pink, purple, red, teal, violet, white, yellow.
\usepackage{enumitem} % Cambiar labels y más flexibilidad para el enumerate
\usepackage{multicol} 
\usepackage{tikz} % para graficar
\usepackage{cancel} % cancelar fórmulas
\usepackage{titlesec} % para editar titulos y hacer secciones con formato a medida
\usepackage{ulem}
\usepackage{centernot} % tacha cosas
\usepackage{bbding} % símbolos de donde uso FiveStar
\usepackage{skull} % símbolos de donde uso Skull
\usepackage{soul} % Para tachar texto en text y math mode
\usepackage{polynom} % para división de polinomios y mcd
\usepackage{fontawesome5} % fuentes "extras"
\usepackage{simpleicons} % fuentes "extras"
\usepackage{venndiagram} % Para los diagramas de Venn
\usepackage{qrcode} % genera código qr
\usepackage{xspace}% para control de espacios en macros
\usepackage{aligned-overset}% para alinear simbolos con anotaciones sobre/abajo de ellos

\usepackage{listings} % Escribir código
\usepackage{framed} % para hacer recuadros y sombrear el codigo
\renewcommand{\ttdefault}{pcr}
\lstset{
emph={row_echelon},
emphstyle={\bfseries},
breaklines=true,
basicstyle=\ttfamily,
columns=fullflexible,
keepspaces=true,
showspaces=false,
showstringspaces=false,
inputencoding=utf8,
literate=%
         {á}{{\'a}}1
         {é}{{\'e}}1
         {í}{{\'i}}1
         {ó}{{\'o}}1
         {ú}{{\'u}}1
         {ñ}{{\~n}}1
}


\usepackage{fancyhdr} % Encabezados y pie de páginas



% para hacer los graficos tipo grafos
\usetikzlibrary{shapes,arrows.meta, chains, matrix, calc, trees, positioning, fit}
\usetikzlibrary{external,decorations.pathreplacing,angles,quotes}

% En general quiero que este paquete sea el último en importarse
\usepackage{hyperref} % para que haya links navegables en el PDF
\hypersetup{
    pdftitle={Apunté Único de Álgebra I},
    pdfauthor={Por los alumnos y exalumnos de Álgebra I},
    pdfkeywords={algebra 1, resueltos},
    colorlinks=true,
    pdfborder={0,0,0}, % sin border
    citecolor=cyan,
    refcolor=magenta,
    linkcolor=blue!80!red,
    filecolor=green,
    urlcolor=red!10!purple,
    }
\urlstyle{same}

\setlength{\parindent}{0pt} % Para que no haya indentación en las nuevas líneas.

%% Info SOCIAL
\def\dirRepo{https://github.com/nad-garraz/algebraUno}
\def\dirTelegram{https://t.me/+1znt2GV1i8cwMTNh}
\newcommand{\dirGuia}[1]{\dirRepo/blob/main/#1-guia/#1-sol.pdf}


% Definiciones y macros para que se me haga más ameno el codeo.
% ======================
% =========CODIGO=======
% ======================

\newcommand{\codigoPython}[1]{
  \begin{snugshade} % el color está definido en shadecolor  
    \lstinputlisting[language=python]{./codigos-\guia/#1}
  \end{snugshade}
}

% Definiciones y nuevos comandos:def
% =============
% Conjuntos
\DeclareMathOperator{\partes}{\mathcal P}
\DeclareMathOperator{\relacion}{\,\mathcal{R}\,}
\DeclareMathOperator{\norelacion}{\,\cancel{\relacion}\,}
\DeclareMathOperator{\universo}{\mathcal U}
\DeclareMathOperator{\reales}{\mathbb R}
\DeclareMathOperator{\naturales}{\mathbb N}
\DeclareMathOperator{\enteros}{\mathbb Z}
\DeclareMathOperator{\racionales}{\mathbb Q}
\DeclareMathOperator{\irracionales}{\mathbb I}
\DeclareMathOperator{\complejos}{\mathbb C}
\DeclareMathOperator{\accion}{\xspace\red{\cdot}\xspace}
\DeclareMathOperator{\columna}{\text{Col}}
\DeclareMathOperator{\K}{\mathbb K} % cuerpo K
\DeclareMathOperator{\vacio}{\varnothing}
\DeclareMathOperator{\union}{\cup}
\DeclareMathOperator{\inter}{\,\cap\,}
\DeclareMathOperator{\sumaDirecta}{\oplus}
\DeclareMathOperator{\diferencia}{\ \setminus \ }
\DeclareMathOperator{\y}{\land}
\DeclareMathOperator{\nucleo}{\text{Nu}}
\DeclareMathOperator{\imagen}{\text{Im}}
\DeclareMathOperator{\dimension}{\text{dim}}
\DeclareMathOperator{\rango}{\text{rg}}
\DeclareMathOperator{\normaBullet}{\Vert\cdot\Vert}
\newcommand{\norma}[1]{\Vert #1 \Vert}
\DeclareMathOperator{\condicion}{\text{cond}}

\def\o{\lor}
\def\neg{\sim}

\DeclareMathOperator{\equivalente}{\leftrightsquigarrow}
\def\entonces{\implies}
\def\noEntonces{\centernot\Rightarrow}

\def\sisolosi{\iff} % largo
\def\sii{\Leftrightarrow} % corto

\def\clase{\overline}
\def\ord{\text{ord}}

\def\existe{\,\exists\,}
\def\noexiste{\,\nexists\,} \def\paratodo{\ \, \forall}
\def\distinto{\neq}
\def\en{\in}
\def\talque{\;/\;}

% =====
\def\qvq{\text{ quiero ver que }}

%funciones
\DeclareMathOperator{\dom}{Dom}
\DeclareMathOperator{\cod}{Cod}
\def\F{\mathcal F}
\def\comp{\circ}
\def\inv{^{-1}}
\def\infinito{\infty}

% Llaves, paréntesis, contenedores
\newcommand{\llave}[2]{ \left\{ \begin{array}{#1} #2 \end{array}\right. }
\newcommand{\llaveInv}[2]{ \left\} \begin{array}{#1} #2 \end{array}\right. }
\newcommand{\llaves}[2]{ \left\{ \begin{array}{#1} #2 \end{array} \right\} }

\newcommand{\matriz}[2]{\left( \begin{array}{#1} #2 \end{array} \right)}
\newcommand{\bloque}[2]{\left[ \begin{array}{#1} #2 \end{array} \right]}
\newcommand{\triangulacion}[1]{\ensuremath{\begin{array}{c} #1 \end{array}}}
\newcommand{\deter}[2]{\left| \begin{array}{#1} #2 \end{array} \right|}
\newcommand{\lista}[2][(1)]{\begin{enumerate}[\bf #1]\setlength\itemsep{-0.6ex} #2 \end{enumerate}}
\newcommand{\listal}[2][-0.6ex]{\begin{enumerate}[\bf(a)]\setlength\itemsep{#1} #2 \end{enumerate}}

% naturales
\newcommand{\limite}[2]{\lim\limits_{#1 \to #2}}
\newcommand{\sumatoria}[2]{\sum\limits_{#1}^{#2}}
\newcommand{\productoria}[2]{\prod\limits_{#1}^{#2}}
\newcommand{\maximo}[1][]{\max\limits_{#1}}
\newcommand{\infimo}[1][]{\inf\limits_{#1}}
\newcommand{\supremo}[1][]{\sup\limits_{#1}}
\newcommand{\kmasuno}[1]{\underbrace{#1}_{k+1\text{-ésimo}}}
\newcommand{\HI}[1]{\underbrace{#1}_{\text{HI}}}

% % enteros
\def\divideA{\, \big| \,}
\def\noDivide{\centernot\divideA}
\def\congruente{\, \equiv \,}
\def\noCongruente{\, \not\equiv \,}
\newcommand{\congruencia}[3]{#1 \equiv #2 \;(#3)}
\newcommand{\noCongruencia}[3]{#1 \not\equiv #2 \;(#3)}
\newcommand{\conga}[1]{\stackrel{\mathclap{(#1)}}{\congruente}}
\newcommand{\divset}[2]{\mathcal{D}(#1) = \set{#2}}
\newcommand{\divsetP}[2]{\mathcal{D_+}(#1) = \set{#2}}
\newcommand{\ub}[2]{ \underbrace{\textstyle #1}_{\mathclap{#2}} }
\newcommand{\ob}[2]{ \overbrace{\textstyle #1}^{\mathclap{#2}} }
\newcommand{\ua}[2]{\underset{\everymath{\textstyle}\mathclap{\substack{\downarrow \\  #2}}}{#1}}
\newcommand{\oa}[2]{\overset{\everymath{\textstyle}\mathclap{\substack{#2 \\ \uparrow }}}{#1}}
\def\cop{\, \perp \, }
\def\nocop{\, \not\perp \, }

% complejos
\DeclareMathOperator{\re}{Re}
\DeclareMathOperator{\im}{Im}
\DeclareMathOperator{\argumento}{arg}
\newcommand{\conj}[1]{\overline{#1}}

% Polinomios
\DeclareMathOperator{\cp}{cp}
\DeclareMathOperator{\gr}{gr}
\DeclareMathOperator{\mult}{mult}
\newcommand{\divPol}[2]{\polylongdiv[style=D]{#1}{#2}}
\newcommand{\mcd}[2]{\polylonggcd{#1}{#2}}

% =====
% Miscelanea
% =====
\def\ot{\leftarrow}
\newcommand{\estabien}{{\color{blue} Consultado, está bien. \checkmark}}
\newcommand{\hacer}{
  {\color{red!80!black}{\tiny\faIcon{flushed}... hay que hacerlo! \faIcon{sad-cry}}}\par
  {\color{black!70!white}
    \tiny Si querés mandá la solución $\to$ \href{\dirTelegram}{\blue{al grupo de Telegram}  \small\telegram},
    o  mejor aún si querés subirlo en \LaTeX $\to$ \href{\dirRepo}{una \textit{pull request} al \small \github}.
  }\par
}

\newcommand{\Hacer}{{\color{black!30!red}\Large Hacer!}}
\def\Tilde{\quad\checkmark}
\def\ytext{\text{\ \, y\ \, }}
\def\otext{\quad\text{o}\quad}
\newcommand{\cajaResultado}[1]{\fcolorbox{orange}{white}{\ensuremath{\displaystyle#1}}}

% Estrellita para hacer llamadas de atención, viene en divertidos colores
% para coleccionar.
\newcommand{\llamada}[1]{
  \begingroup% Scope para la variable colortemp
  \ifcase \numexpr#1 mod 6\relax
    \def\colortemp{cyan}
  \or
    \def\colortemp{magenta}
  \or
    \def\colortemp{OliveGreen}
  \or
    \def\colortemp{YellowOrange}
  \or
    \def\colortemp{Cerulean}
  \or
    \def\colortemp{Violet}
  \or
    \def\colortemp{Purple}
  \fi
  \textcolor{\colortemp}{\text{{\tiny \faIcon{star}}}^{\scriptscriptstyle#1}}
  \endgroup
}

% separadores
\newcommand{\separador}{
  \par\noindent\rule{\linewidth}{0.4pt}\par
}
\newcommand{\separadorCorto}{
  \par\noindent\rule{0.5\linewidth}{0.4pt}\par
}

% Colores
\newcommand{\red}[1]{\textcolor{red}{#1}}
\newcommand{\green}[1]{\textcolor{OliveGreen}{#1}}
\newcommand{\blue}[1]{\textcolor{Cerulean}{#1}}
\newcommand{\cyan}[1]{\textcolor{cyan}{#1}}
\newcommand{\yellow}[1]{\textcolor{YellowOrange}{#1}}
\newcommand{\orange}[1]{\textcolor{Orange}{#1}}
\newcommand{\magenta}[1]{\textcolor{magenta}{#1}}
\newcommand{\marron}[1]{\textcolor{brown}{#1}}
\newcommand{\purple}[1]{\textcolor{purple}{#1}}
\newcommand{\violet}[1]{\textcolor{violet}{#1}}
\newcommand{\rosa}[1]{\textcolor{pink}{#1}}
\newcommand{\negro}[1]{\textcolor{black}{#1}}
\newcommand{\transparente}[1]{\color[rgb]{1,1,1,0.5}{#1}}

\definecolor{shadecolor}{RGB}{240,255,255}
\newenvironment{sombrita}{\begin{snugshade}}{\end{snugshade}}

% Conjuntos entre llaves y paréntesis
% te ahorrás escribir los \left y \right, así dejando el código más legible.
\newcommand{\set}[1] {\left\{ #1 \right\}}
\newcommand{\ket}[1] {\left\langle #1 \right\rangle}
\newcommand{\parentesis}[1]{ \left( #1 \right) }

% Stackrel text. Es para ahorrarse ecribir el \text
\newcommand{\stacktext}[2]{ \stackrel{\text{#1}}{#2} }

% Dado que muchas veces ponemos cosas sobre un signo '='
%  acá está el comando para escribir \igual{arriba}[abajo] con texto!
\NewDocumentCommand{\igual}{m o}{
  \IfNoValueTF{#2}{
    \overset{\mathclap{\text{#1}}}=
  }{
    \overset{\mathclap{\text{#1}}}{\underset{\mathclap{\text{#2}}}=}
  }
}
% Dado que muchas veces ponemos cosas sobre un signo '='
%  acá está el comando para escribir \igual{arriba}[abajo] con texto!
\NewDocumentCommand{\mayorIgual}{m o}{
  \IfNoValueTF{#2}{
    \overset{\mathclap{\text{#1}}}\geq
  }{
    \overset{\mathclap{\text{#1}}}{\underset{\mathclap{\text{#2}}}\geq}
  }
}
% Dado que muchas veces ponemos cosas sobre un signo '='
%  acá está el comando para escribir \igual{arriba}[abajo] con texto!
\NewDocumentCommand{\menorIgual}{m o}{
  \IfNoValueTF{#2}{
    \overset{\mathclap{\text{#1}}}\leq
  }{
    \overset{\mathclap{\text{#1}}}{\underset{\mathclap{\text{#2}}}\leq}
  }
}

%=======================================================
% Comandos con flechas extensibles.
%=======================================================
% *Flechita* extensible con texto {arriba} y [abajo] 
\NewDocumentCommand{\flecha}{m o}{%
  \IfNoValueTF{#2}{%
    \xrightarrow[]{\text{#1}}
  }{
    \xrightarrow[\text{#2}]{\text{#1}}
  }
}
% *Si solo si* extensible con texto {arriba} y [abajo] 
\NewDocumentCommand{\Sii}{m o}{%
  \IfNoValueTF{#2}{%
    \xLeftrightarrow[]{\text{#1}}
  }{
    \xLeftrightarrow[\text{#2}]{\text{#1}}
  }
}

% *Si solo si* extensible con texto {arriba} y [abajo] 
\NewDocumentCommand{\Entonces}{m o}{%
  \IfNoValueTF{#2}{%
    \xRightarrow[]{\text{#1}}
  }{
    \xRightarrow[\text{#2}]{\text{#1}}
  }
}

%=======================================================
% fin comandos con flechas extensibles.

% como el stackrel pero también se puede poner algo debajo
\newcommand{\taa}[3]{ % [t]exto [a]rriba y [a]bajo
  \overset{\mathclap{#1}}{\underset{\mathclap{#2}}{#3}}
}

%Update time
\def\update{\tiny
  {\today\ @ \currenttime}
}
\def\updateDos{
  {\today\ @ \currenttime}
}
\newcommand{\parrafoAcotado}[2][0.6]{
  \begin{center}
    \parbox{#1\textwidth}{
      \centering
      #2
    }
  \end{center}
}

% Párrafo destacado:
\newcommand{\parrafoDestacado}[2][]{
  \begin{center}
    #1\qquad
    \parbox{0.7\textwidth}{
      \centering
      #2
    }
    \qquad#1
  \end{center}
}

% Contributors
\newcommand{\aporte}[2]{
  \href{#1}{#2}
}

\newenvironment{aportes}{
  \vspace{10pt}
  \par
  \begin{minipage}{0.9\linewidth}
    \tt\footnotesize
    Dale las gracias y un poco de amor \rosa{\faIcon{heart}} a los que contribuyeron!
    Gracias por tu aporte:
    \vspace{-10pt}
    \begin{multicols}{3}
      \begin{itemize}[label={\tiny\yellow{\faIcon{medal}}}]
        }{
      \end{itemize}
    \end{multicols}
  \end{minipage}
  \medskip
}

%=======================================================
% sección ejercicio con su respectivo formato y contador
%=======================================================
\newcounter{ejercicio}[section] % contador que se resetea en cada sección
\renewcommand{\theejercicio}{\arabic{ejercicio}} % el contador es un número arabic
\newcommand{\ejercicio}{%
  \stepcounter{ejercicio}% incremento en uno
  \titleformat{\section}[runin]{\bfseries}{\theejercicio}{}{}%
  \section*{Ejercicio \theejercicio.}\labelEjercicio{ej:\theejercicio}
}

% Label y refencia para ejercicio hay alguna forma más elegante de hacer esto?
\newcommand{\labelEjercicio}[1]{
  \addtocounter{ejercicio}{-1} % counter - 1
  \refstepcounter{ejercicio} % referencia al anterior y luego + 1
  \label{#1}}
\newcommand{\refEjercicio}[1]{{\bf\ref{#1}.}}

\def\fueguito{{\color{orange}{\faIcon{fire}}}}
\newcounter{ejExtra}[section] % contador que se resetea en cada sección
\renewcommand{\theejExtra}{\arabic{ejExtra}} % el contador es un número arabic
\newcommand{\ejExtra}{%
  \stepcounter{ejExtra}% incremento en uno
  \titleformat{\section}[runin]{\bfseries}{\theejExtra}{1em}{}%
  % Es como una sección. Le pongo un ícono, luego el número del ejercicio con la etiqueta para poder
  % linkearlo en el índice u otro lugar.
  % con \ref{ejExtra:{numero del ejercicio}} es que salto al ejercicio.
  \section*{\fueguito\theejExtra.}\labelEjExtra{ejExtra:\theejExtra}
}

% Label y refencia para ejercicio hay alguna forma más elegante de hacer esto?
\newcommand{\labelEjExtra}[1]{
  \addtocounter{ejExtra}{-1} % counter - 1
  \refstepcounter{ejExtra} % referencia al anterior y luego + 1
  \label{#1} % etiqueta para cada ejercicio extra
}
% Con esto llamos al ejercicio extra
\newcommand{\refEjExtra}[1]{
  {\fueguito\bf\ref{#1}.}
}

%=======================================================
% fin sección ejercicio con su respectivo formato y contador
%=======================================================

\newenvironment{enunciado}[1]{% Toma un parametro obligatorio: \ejExtra o \ejercicio 
  \par
  \noindent
  \begin{minipage}{\linewidth}
    \separador % linea sobre el enunciado
    #1
    }% contenido
    {
    \separadorCorto % linea debajo del enunciado
    \par
  \end{minipage}\par
}

%%% Emoticones
\def\poo{\marron{\faIcon{poo}}\xspace}
\def\angry{\faIcon[regular]{angry}\xspace}
\def\atencion{\faIcon{exclamation-triangle}\xspace}
\def\meh{\faIcon[regular]{meh}\xspace}
\def\mehBlank{\faIcon[regular]{meh-blank}\xspace}
\def\rollingEyes{\faIcon[regular]{meh-rolling-eyes}\xspace}
\def\surprise{\faIcon[regular]{surprise}}
\def\magic{\href{https://www.youtube.com/watch?v=4u20AEDYhcw}{\faIcon{magic}}\xspace}
\def\python{\texttt{Python} \faIcon{python}\xspace}
\def\copyPaste{
  \begin{center}
    \fcolorbox{green}{white}{
      \faIcon{linux}
      Si hacés un \texttt{copy paste} de este código debería funcionar lo más bien
      \faIcon{linux}
    }
  \end{center}
}

%%% (☞⌐▀͡ ͜ʖ͡▀ )☞ Yo mama
\def\superIdol{https://www.youtube.com/watch?v=DKpaKHUlyBY}
\def\videosTeresa{https://www.youtube.com/@AlgebraIC-gu7oc/videos}
\def\videosPracticas{https://www.youtube.com/playlist?list=PLEtdiZTXB5c5e9BOcqbnROTHevanNfz6U}
\def\justDoIt{https://www.youtube.com/watch?v=ZXsQAXx_ao0}
\def\dontWorryAboutAThing{https://www.youtube.com/watch?v=4k2PJFPu57Y}
\def\zanguango{https://www.youtube.com/watch?v=Uzcl2gNL3zg&t=10s}
\def\chinito{https://www.youtube.com/watch?v=ebz4xuPf-is}
\def\neverGonnaGiveYouUp{https://www.youtube.com/watch?v=dQw4w9WgXcQ}
\def\mindExplosion{https://www.youtube.com/watch?v=9CS7j5I6aOc}

%%% Iconos más usados
\def\github{{\color{violet!80!black}{\faIcon{github}}}}
\def\instagram{\faIcon{instagram}}
\def\tiktok{\faIcon{tiktok}}
\def\linkedin{\blue{\faIcon{linkedin}}}
\def\youtube{\color{red!70}{\faIcon{youtube}}}
\def\telegram{\blue{\faIcon{telegram}}}


\begin{document}

% IMPORTANTE: Estos valores son lo único referente a la guía en particular.
\def\guia{1} % <-- El número de la guía
\def\cantidadEjerciciosGuia{21}
\def\cantidadEjerciciosExtras{1}   % <-- Modificar si se agrega un ejercicio EXTRA

% Info para armar título.

\title{Apunte único: Álgebra Lineal Computacional - Práctica \guia} % título
\author{Por alumnos de ALC\\
  Facultad de Ciencias Exactas y Naturales \\
  UBA} % autores y lugar
\date{} % Así no aparece la fecha 


\thispagestyle{empty} % borro el número de la primera página

\pagestyle{fancy} % activo los headers y footers
\fancyhead{} % borro lo que haya en los headers
\fancyfoot{} % borro lo que haya en los headers
\fancyhead[L]{ALC} % encabezado izquierdo
\fancyhead[C]{Práctica \guia} % encabezado central
\fancyhead[R]{Página \thepage} %% encabezado derecho

\fancyfoot[EL]{
  \small\github
  ¡Aportá con correcciones, mandando ejercicios, \yellow{\faIcon{star}} \href{\dirRepo}{al repo}, críticas, todo sirve.\\
  La idea es que la guía esté actualizada y con el mínimo de errores.
  
}    % pie izquierdo pares
\fancyfoot[OL]{
  \small \telegram ¿Errores? \href{\dirTelegram}{Avisá acá} así se corrige y ganamos todos.\\
  \small Compilado: \update. Chequeá si hay una \href{\dirGuia{\guia}}{versión nueva $\to$ acá.}
} % pie izquierdo impares

% \fancyfoot[EC]{\small \update}                                                                        % pie izquierdo pares
\fancyfoot[OR]{
  \small\hyperlink{indice-\guia}{
    Ir a índice $\uparrow$
  }
}             % pie derecho pares

\fancyfoot[ER]{
  \small \hyperlink{indice-\guia}{Ir al índice $\uparrow$}
}                            % pie derecho impares

        % <-- Inyecto código de encabezado y pie
%=========================
% Un poco de Bling-Bling, carátula e índice
%=========================

%------IMPORTANTE
%el comando "\guia" está definido en padres de este archivo
%------


\vfill

\begin{center}
  \hypertarget{indice-\guia}{\Large\textit{Choose your destiny: }}\par
  {\tiny\textit{(dobleclick en los ejercicio para saltar) }}

  \begin{itemize}
    \item[\tiny\faIcon{meh-blank}] \hyperlink{teoria-\guia}{Notas teóricas}

    \item[\tiny\faIcon{meh}]
          Ejercicios de la guía:
          \begin{multicols}{8}
            \foreach \ejer in {1,...,\cantidadEjerciciosGuia}{
                \refEjercicio{ej:\ejer}\\
              }
          \end{multicols}

    \item[\tiny\faIcon{angry}] Ejercicios de Parciales
          \begin{multicols}{8}
            \foreach \extras in {1,...,\cantidadEjerciciosExtras}{
                \refEjExtra{ejExtra:\extras}\\
              }
          \end{multicols}

  \end{itemize}
\end{center}

\vfill

\newpage % nueva página
                % <-- Inyecto código del índice y carátula
%=========================
%  Disclaimer y QR
%=========================


%\thispagestyle{empty}
\underline{Disclaimer:}\par
Dirigido para aquél que esté listo para leerlo, o no tanto. Va con onda.\par \bigskip

\vspace*{\fill}
\noindent\makebox[\textwidth]{
  \begin{minipage}{0.7\textwidth}
    \centering
    {\Large \red{¡Recomendación para sacarle jugo al apunte!}}\par\bigskip

    Estudiar con resueltos puede ser un arma de doble filo.
    Si estás trabado, antes de saltar a la solución que hizo otra persona:

          \begin{enumerate}[label=\purple{\faIcon{toilet-paper}$_{\arabic*}$}]

      \item Mirar la solución ni bien te trabás, te \textit{condicionas pavlovianamente} a \textbf{no} pensar.
            Necesitás darle tiempo al cerebro para llegar a la solución.

      \item Intentá un ejercicio similar, pero \textbf{más fácil}.

      \item ¿No sale el fácil? Intentá uno \textbf{\large aún más fácil}.

      \item Fijate si tenés un ejercicio similar hecho en clase. Y mirá ese, así no quemás el ejercicio de la guía.

      \item Tomate 2 minutos para formular una pregunta que realmente
            sea lo que \textbf{no} entendés. Decir \textit{`no me sale'} $\noexiste +$. Escribí
            esa pregunta, vas a dormir mejor.
    \end{enumerate}\par \medskip

    Ahora sí mirá la solución.\par \medskip

    \textit{Si no te salen los ejercicios fáciles sin ayuda}, no te van a salir los ejercicios
    más difíciles: \cyan{Sentido común}.\par\medskip

    ¡Los más fáciles van a salir! Son el alimento de nuestra confiaza.\par\medskip

    Si mirás miles de soluciones a parciales en el afán de tener un ejemplo hecho de todas las
    variantes, estás apelando demasiado a la suerte de que te toque uno igual, \textit{pero no estás aprendiendo nada}.
    Hacer un parcial bien lleva entre 3 y 4 horas. Así que si vos en 4 horas "hiciste" 3 o 4 parciales, \textit{algo raro debe haber}.
    A los parciales se va a \textbf{pensar} y eso hay que practicarlo desde el primer día.\par\bigskip

    Mirá los videos de las teóricas:

          \href{\videosTeresa}{de Teresa que son buenísimos \faIcon{youtube}}.

          \medskip

    Videos de prácticas de pandemia, complemento extra:

          \href{\videosPracticas}{Prácticas Pandemia \faIcon{youtube}}.
          
          \bigskip

    Los ejercicios que se dan en clase suelen ser similares a los parciales,
    a veces más difíciles, repasalos siempre \href{\justDoIt}{Just Do IT \faIcon{jedi}\faIcon{hand-sparkles}\faIcon{jedi}!}
  \end{minipage}
}
\vspace*{\fill}

Eh, loco, fatalista, distópico, \href{\dontWorryAboutAThing}{relajá un toque te vas a\
  quedar (más) pelado... \faIcon{dove}\faIcon{dove}\faIcon{dove} \textit{va a salir todo bien!}}


\newpage

%%% MACRO LOCAL
\newcommand{\qrWeb}[1]{
  \qrcode{#1}\par
  \texttt{\tiny#1}
}
%%% FIN MACRO LOCAL
\thispagestyle{empty}

\vspace*{\fill}

\noindent\makebox[\textwidth]{
  \begin{minipage}{0.7\textwidth}
    \centering

    {\LARGE
      Esta Guía \guia\ que tenés se actualizó por última vez:

      \red{\updateDos}

      Escaneá el QR para bajarte (quizás) una versión
      más nueva:
      \par

      \bigskip

      Guía \guia \par\medskip
    }
    \qrcode[height=4cm]{\dirGuia{\guia}}
    \vspace{2cm}

    { \Large
      El resto de las guías repo en \href{\dirRepo}{github\github} para
      descargar las guías con los últimos updates.\par\medskip
    }
    \qrcode[height=4cm]{\dirRepo}
    \vspace{2cm}

    { \Large
      Si querés mandar un ejercicio o avisar de algún error,
      lo más fácil es por \href{\dirTelegram}{Telegram \telegram}.\par\medskip
    }
    \qrcode[height=4cm]{\dirTelegram}
  \end{minipage}
}

\vspace*{\fill}


\newpage


%=========================
% Un poco de teoría
%=========================
\subsubsection*{\hypertarget{teoria-\guia}{Notas teóricas:}}
\input{./teoria-\guia/teoria-\guia.tex}

\newpage % página nueva

%=========================
% Ejercicios guia
%=========================
\subsubsection*{Ejercicios de la guía:}

\foreach \x in {1,...,\cantidadEjerciciosGuia} { % cantidad de ejercicios de la guía
		\IfFileExists{./ejercicios-\guia/ej-\x-\guia.tex}{
			\input{./ejercicios-\guia/ej-\x-\guia.tex}
		}{
			\typeout{¡Atención! El archivo ./ejercicios-\guia/ej-\x-\guia.tex no está.
            Revisar variable: 'cantidadEjerciciosGuia` en \guia-sol.tex}
		}
	}

%-------------------------------------------------------
%--------------------------------------------------------
%---------------------------------------------------------
%-----SECCION PARA PONER PARCIALES-------------------------
%-----------------------------------------------------------
%------------------------------------------------------------
%-----------------------------------------------------------
%-----SECCION PARA PONER PARCIALES-------------------------
%---------------------------------------------------------
%--------------------------------------------------------
%-------------------------------------------------------


\newpage % página nueva

%=========================
% Ejercicios extras, parciales, etc.
%=========================

\subsubsection*{\hypertarget{extras-\guia}{{\Large\color{orange}{\faIcon{fire}}} Ejercicios de parciales:}}

\foreach \x in {1,...,\cantidadEjerciciosExtras} { % cantidad de ejercicios extras
		\IfFileExists{./ejercicios-\guia-extra/ej-extra-\x-\guia.tex}{
			\input{./ejercicios-\guia-extra/ej-extra-\x-\guia.tex}
		}{
			\typeout{¡Atención! El archivo ./ejercicios-\guia-extra/ej-extra-\x-\guia.tex no está.
             Revisar variable: 'cantidadEjerciciosExtras' en \guia-sol.tex}
		}
	}

%-------------------------------------------------------
%--------------------------------------------------------
%---------------------------------------------------------
%-----SECCION PARA PONER PARCIALES-------------------------
%-----------------------------------------------------------
%------------------------------------------------------------
%-----------------------------------------------------------
%-----SECCION PARA PONER PARCIALES-------------------------
%---------------------------------------------------------
%--------------------------------------------------------
%-------------------------------------------------------
 % <-- Inyecto código teoría, ejercicios y ejercicio extra

\end{document}
