\begin{enunciado}{\ejExtra}
  Sean $f: \reales^4 \to \reales^4$ definida por:
  $$
    f(x_1, x_2, x_3, x_4) = (2x_1 + x_2, x_1 + \alpha x_2 + \alpha x_3 - x_4, -\alpha x_3 + x_4, x_3 - x_4)
  $$
  y los subespacios:
  $$
    S = \set{x \en \reales^4 : x_1 - x_2 = 0, 2x_1  + x_3 + x_4 = 0}\\
  $$
  $$
    T = \ket{(1,1,1,-3), (1,-1,0,0), (1, -3, -1, -3)}
  $$
  \begin{enumerate}[label=\alph*)]
    \item (1 pt.) Hallar todos los valores de $\alpha \en \reales$ tales que $\dim(\imagen(f)) = 3$.

    \item (1 pt.) Para $\alpha = 1$, decidir si existe una transformación lineal $g: \reales^4 \to \reales^4$
          tal que $g(S+T) = \imagen(f)$ y que $g(\nucleo(f)) = (0, 0, 0, 0)$. En caso afirmativo, exhibir un ejemplo.
          En caso negativo, explicar por qué.

    \item (1 pt.) Para $\alpha = 1$ y considerando $h : \reales^3 \to \reales^4$ dada por:
          $$
            h(x_1, x_2, x_3) = (x_1 - x_2, x_3, x_2 - 2x_3, x_1)
          $$
          decidir si $f \circ h$ es monomorfismo. En caso contrario, hallar una base de $\nucleo(f \circ h)$.
  \end{enumerate}
\end{enunciado}

\begin{enumerate}[label=\alph*)]
  \item $\alpha = 1$ y $\alpha = \frac{1}{2}$

  \item $g = f$

  \item $$
          [h] =
          \matriz{ccc}{
            1 & -1 & 0 \\
            0 & 0  & 1 \\
            0 & 1 & -2 \\
            1 & 0 & 0
          }
          , \ytext
          \matriz{ccc}{
            1 & -1 & 0 \\
            0 & 0  & 1 \\
            0 & 1 & -2 \\
            1 & 0 & 0
          }
          \matriz{c}{
            1  \\
            1  \\
            0
          }
          =
          \matriz{c}{
            0  \\
            0  \\
            1 \\
            1
          }
          \en
          \nucleo(f)
        $$
        La composición $f \circ h$ no es mono
\end{enumerate}
