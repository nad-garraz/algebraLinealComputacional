\begin{enunciado}{\ejExtra}
  Sean $a$, $b \en \reales$,
  $$
    S = \set{(x_1,x_2,x_3,x_4) \en \reales^4 : x_1 + ax_2 + bx_3 + ax_4 = 0}
  $$
  y
  $$
    T = \set{(x_1,x_2,x_3,x_4) \en \reales^4 :
      -2x_1 + 3x_3 + x_4 = 0,\ 
      x_2 - x_3 + 4x_4 = 0}
  $$
  subespacios de $\reales^4$.
  \begin{enumerate}[label=\alph*)]
    \item Hallar todos los valores de $a, b \en \reales$ para los que $\dimension(S \inter T) = 1$.
    \item Para el caso en que $a = -1$ y $b = 1$. Hallar $f: \reales^4 \to \reales^4$ una transformación lineal
          tal que $\nucleo(f) = S \inter T$ e $\imagen(f) = S$.
  \end{enumerate}
\end{enunciado}

\begin{enumerate}[label=\alph*)]
  \item Para encontrar la intersección entre dos subespacios dados con ecuaciones, puedo resover todas las ecuaciones en simultáneo:
        $$
          \llamada1
          \llave{rcl}{
            x_1 + ax_2 + bx_3 + ax_4 & = & 0\\
            -2x_1 + 3x_3 + x_4 & = & 0\\
            x_2 - x_3 + 4x_4 & = & 0
          }
        $$
        Ahora el sistema en forma matricial y triangulo.
        La idea es que me queden \red{3 ecuaciones} \textit{linealmente independientes}, de esa
        manera quedará \underline{solo una variables libre}, por lo tanto la solución al sistema tendrá dimensión 1:
        $$
          \llamada1
          \equivalente
          \begin{array}{rcl}
            \matriz{cccc|c}{
            1  & a                                              & b           & a      & 0 \\
            -2 & 0                                              & 3           & 1      & 0 \\
            0  & 1                                              & -1          & 4      & 0
            }
               & \flecha{$F_2 + 2F_1 \to F_2$}                  &
            \matriz{cccc|c}{
            1  & a                                              & b           & a      & 0 \\
            0  & 2a                                             & 3 + 2b      & 1 + 2a & 0 \\
            0  & 1                                              & -1          & 4      & 0
            }                                                                              \\
               & \flecha{$2aF_3 - F_2 \to F_3$}[$a\distinto 0$] &
            \matriz{cccc|c}{
            1  & a                                              & b           & a      & 0 \\
            0  & 2a                                             & 3 + 2b      & 1 + 2a & 0 \\
            0  & 0                                              & -2a - 3 -2b & 6a - 1 & 0
            }
          \end{array}
        $$
        Por un lado se puede ver a ojo que cuando $a = 0$ quedan 3 ecuaciones \textit{linealmente independientes} así que no jode.
        Ahora quiero ver para cuales valores de $a \ytext b$ se borra la última fila:
        $$
          \llave{rcl}{
            -2a - 3 - 2b & = & 0 \Sii{$a = \frac{1}{6}$} b = -\frac{5}{3}\\
            6a - 1 & = & 0 \sii a = \frac{1}{6}
          }
        $$
        Por lo tanto la intersección va a tener dimensión 1, $\dimension(S \inter T) = 1$ cuando:
        $$
          \cajaResultado{
            \llave{l}{
              a = \frac{1}{6} \\
              b = - \frac{5}{3}
            }
          }
        $$

\end{enumerate}
\begin{aportes}
  \item \aporte{\dirRepo}{naD GarRaz \github}
\end{aportes}
