\begin{enunciado}{\ejExtra}
  Sean los subespacios de $\reales^4$
  $$
    \begin{array}{rcl}
      S & = & \set{(x_1, x_2, x_3, x_4) \en \reales^4 : x_2 + x_3 + x_4 = 0, x_1 + 3x_2 - 2x_3 + x_4 = 0} \\
      T & = & \ket{(4, -2, 1, 3), (0, -1, 0, 1), (0,0,1,1)}.
    \end{array}
  $$
  \begin{enumerate}[label=\alph*)]
    \item Definir una transformación lineal no nula $f: \reales^4 \to \reales^4$ tal que $f(S) \subseteq T$ y
          $f(T) \subseteq S$.

          Justificar la buena definición.

    \item Determinar $p_S : \reales^4 \to \reales^4$ la proyección ortogonal sobre $S$ y calcular $p_S(S \inter T)$.
  \end{enumerate}
\end{enunciado}

\begin{enumerate}[label=\alph*)]
  \item Calculo intersección entre $S$ y $T$ y sistema de generadores de $S$. Para la intersección meto un genérico de $T$ en las ecuacione de $S$ y así obtengo:
        $$
          S \inter T = \ket{(2,-1,0,1)}
        $$
        El sistema de generadores es resolver las ecuaciones que deben cumplir los elementos del subespacio:
        $$
          S =\ket{(5. -1, 1, 0), (2,-1,0,1)}
        $$
        \underline{Que haya aparecido la intersección es casualidad}.

        Ahora quiero formar una transformación lineal que cumpla lo del enunciado, donde voy a usar de \textit{comodín}
        a esa intersección:
        $$
          \llave{rcl}{
            f(s_1) & = & (t_1)  \\
            f(s \inter t) & = & (s \inter t) \\
            f(t_1) & = & (s_1) \\
            f(t_2) & = & (0)
          }
        $$
        Tengo que asegurarme de que $\set{s_1, s \inter t, t_1, t_2}$ sean una base de $\reales^4$, es decir,
        tienen que ser \textit{linealmente independientes}.

        Una posible transformación que satisface todo eso sería:
        $$
          f(x_1, x_2, x_3, x_4) = (2, -1, 0, 1),
        $$
        la cual manda todo lo que le des a algo de $S$ y a algo de $T$ a la vez.

  \item Para armar un \textit{proyector ortogonal}, $p_S$, hay que asegurarse que:
        $$
          \nucleo(p_S) \inter \imagen(p_S) = 0
          \quad \text{con} \quad
          S = \ket{(2,-1,0,1), (5,-1,1,0)}
        $$

        $$
          \llamada1
          \llave{lcl}{
            p_S(2,-1,0,1) & = & (2,-1,0,1) \\
            p_S(5,-1,1,0) & = & (5,-1,1,0) \\
            p_S(0,1,1,1) & = & (0,0,0,0) \\
            p_S(1,3,-2,1) & = & (0,0,0,0)
          }
        $$
        Para encontrar la expresión funcional del proyector se puede hacer la combinación lineal
        de los elementos de la base de partida igualada a un genérico de $\reales^4$ es decir:
        $$
          (x_1, x_2, x_3, x_4) \igual{$\llamada2$}
          \blue{a} \cdot (2,-1,0,1) +
          \blue{b} \cdot (5,-1,1,0) +
          \blue{c} \cdot (0,1,1,1) +
          \blue{d} \cdot (1,3,-2,1)
        $$
        Resolver eso es resolver el sistema:
        $$
          \matriz{cccc|c}{
            2 & 5 & 0 & 1 & x_1 \\
            -1 & -1 & 1 & 3 & x_2 \\
            0 & 1 & 1 & -2 & x_3 \\
            1 & 0 & 1 & 1 & x_4
          }
          \flecha{\magic}
          \matriz{cccc|l}{
            2 & 5 & 0 & 1 & x_1 \\
            0 & \frac{3}{2} & 1 & \frac{7}{2} & \frac{1}{2}x_1 + x_2 \\
            0 & 0 & \frac{1}{3} & -\frac{13}{3} & x_3 - \frac{1}{3}x_1 - \frac{2}{3}x_2 \\
            0 & 0 & 0 & \frac{88}{3} & x_4 - 8x_3 + 3x_1 + 7x_2
          }
        $$
        y luego transformando $\llamada2$.

        No sé si la idea es escribir la forma funcional o sencillamente definir el proyector $p_s$ como en
        $\llamada1$. Las cuentas son largas cuando se triangula ese sistema y no tengo ganas de hacerlo \grimace. Sea como sea.
        Se puede calcular fácil que:
        $$
          p_S\ub{(2,-1,0,1)}{S\inter T} = \ub{(2,-1,0,1)}{S\inter T}
        $$
        porque está en la definición y además es lo que tiene que hacer el proyector que proyecta a $S$.
        El elemento $S \inter T \en S$ así que el $p_S$ lo manda a sí mismo.
\end{enumerate}

\begin{aportes}
  \item \aporte{\dirRepo}{naD GarRaz \github}
\end{aportes}
