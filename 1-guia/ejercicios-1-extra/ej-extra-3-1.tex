\begin{enunciado}{\ejExtra}
  Sean los subespacios de $\reales^4$
  $$
    \begin{array}{rcl}
      S & = & \set{(x_1, x_2, x_3, x_4) \en \reales^4 : x_2 + x_3 + x_4 = 0, x_1 + 3x_2 - 2x_3 + x_4 = 0} \\
      T & = & \ket{(4, -2, 1, 3), (0, -1, 0, 1), (0,0,1,1)}.
    \end{array}
  $$
  \begin{enumerate}[label=\alph*)]
    \item Definir una transformación lineal no nula $f: \reales^4 \to \reales^4$ tal que $f(S) \subseteq T$ y
          $f(T) \subseteq S$.

          Justificar la buena definición.

    \item Determinar $p_S = : \reales^4 \to \reales^4$ la proyección ortogonal sobre $S$ y calcular $p_S(S \inter T)$.
  \end{enumerate}
\end{enunciado}

\begin{enumerate}[label=\alph*)]
  \item Calculo intersección entre $S$ y $T$. Meto un genérico de $T$ en las ecuacione de $S$ y así obtengo:
        $$
          S \inter T = \ket{(2,-1,0,1)}
        $$

        Ahora quiero formar una transformación lineal que cumpla lo del enunciado, donde voy a usar de \textit{comodín}
        a esa intersección:
        $$
          \llave{rcl}{
            f(s_1) & = & (t_1)  \\
            f(s \inter t) & = & (s \inter t) \\
            f(t_1) & = & (s_1) \\
            f(t_2) & = & (0)
          }
        $$
        Tengo que asegurarme de que $\set{s_1, s \inter t, t_1, t_2}$ sean una base de $\reales^4$, es decir,
        tienen que ser \textit{linealmente independientes}.

        Una posible transformación que satisface todo eso sería:
        $$
          f(x_1, x_2, x_3, x_4) = (2, -1, 0, 1),
        $$
        la cual manda todo lo que le des a algo de $S$ y a algo de $T$ a la vez.

  \item
        $$
          \llave{lcl}{
            p(2,-1,0,1) & = & (2,-1,0,1) \\
            p(-1,-1,0,1) & = & (-1,-1,0,1) \\
            p(0,1,1,1) & = & (0,0,0,0) \\
            p(1,3,-2,1) & = & (0,0,0,0)
          }
        $$
        
        $$
          p(S\inter T) = S\inter T
        $$
\end{enumerate}
