\begin{enunciado}{\ejExtra}
  Sea $f : \reales^4 \to \reales^4$
  la proyección ortogonal sobre
  $$
    S = \ket{(1,1,1,1), (1,2,1,0)}
  $$
  y sean
  $$
    T =\set{x \en \reales^4 : x_1 - x_2 + x_3 = 0, -x_2 + x_4 = 0}
    \ytext W = \ket{(2,0,-2,0), (-2,1,0,1), (2,1,-4,1)}.
  $$
\end{enunciado}

\begin{enumerate}[label=\alph*)]
  \item Decidir si existe alguna transformación lineal $g$ que cumpla simultáneamente:
        $$
          g(W) = \ket{(2,0,1,1), (0,-1,2,2)}
          \quad
          g(v) = f(v) \quad \paratodo v \en T
        $$
        En caso afirmativo, exhibir un ejemplo. En caso negativo, explicar por qué.

  \item
        Sea $h: \reales^4 \to \reales^3$ dada por:
        $$
          h(x_1, x_2, x_3, x_4) = (2x_1 + 4x_2 + 2x_3, -2x_1 + 4x_2 + 2x_3, 4x_4)
        $$
        halle una base de $\imagen(h \circ f)$ y decidir si $h \circ f$ es epimorfismo ¿Puede ser monomorfismo?
\end{enumerate}

\begin{enumerate}[label=\alph*)]
  \item La papa está en el subespacio al que está yendo a para $g(W)$. Hay una contradicción entre esa
        condición y que $g(v) = f(v)$.

        No existe una \textit{transformación lineal que cumpla lo pedido}

  \item
        No te dan las dimensiones. $\dimension(\nucleo(f)) = 2\llamada1$ dado
        que es un \textit{proyector ortogonal}.
        Por lo tanto $h$ va a recibir como mucho a $S$, con $\dimension(S) = 2$.
        \parrafoDestacado{
          No hay forma de que $(h \circ f)(x)$ genere más de 2 vectores
          \textit{linealmente independientes} de $\reales^3$.
        }
        $$
          \cajaResultado{
            \text{La función \underline{no} es un epimorfismo. Tampoco será mono, por $\llamada1$.}
          }
        $$
\end{enumerate}

\begin{aportes}
  \item \aporte{\dirRepo}{naD GarRaz \github}
\end{aportes}
