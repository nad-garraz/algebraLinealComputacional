\begin{enunciado}{\ejExtra}
  \begin{enumerate}[label=\alph*)]
    \item Considerar el subespacio $S$ de $\reales^4$ dado por
          $$
            \llave{rcl}{
              4x_1 - x_2 + 2x_3  & = & 0\\
              -6x_1 + x_2 - 2x_3 & = & 0
            }
          $$
          Hallar un sistema de generadores para $S$.

    \item Probar que existe una única transformación lineal $T: \reales^4 \to \reales^4$ tal que
          $$
            \llave{rcl}{
              T(1,-1,1,-1) & = & (1, -7, 1, 9)\\
              T(4,0,-2,4)  & = & (0, 8, -1, -6)\\
              T(1,1,-4,4)  & = & (1, 0, -1, 0)\\
              T(-4,2,9,-4) & = & (-2, 1, 3, 3)
            }
          $$
          Calcular $T(1,1, -4,5)$ y brindar una base de $\imagen(T)$.

    \item Determinar $S \inter \imagen(T)$.
  \end{enumerate}
\end{enunciado}

\begin{enumerate}[label=\alph*)]
  \item Busco cosas de la pinta:
        $$
          (x_1, x_2, x_3, x_4)
          \flecha{resolviendo el}[sistema]
          \llave{l}{
            x_1 = 0\\
            x_2 = 2x_3\\
            x_3 = x_3\\
            x_4 = x_4
          }
        $$
        Un posible sistema de generadores del subespacio $S$:
        $$
          S = \ket{(0,2,1,0), (0,0,0,1)}
        $$

  \item Para probar esa unicidad tengo que ver que los elementos que estoy transformando
        sean \textit{linealmente independientes}:
        $$
          (0,0,0,0) = \blue{a} \cdot (1,-1,1,-1) + \blue{b} \cdot  (4,0,-2,4) + \blue{c} \cdot   (1,1,-4,4) + \blue{d} \cdot (-4,2,9,-4) \llamada1
        $$
        Los coeficientes $\blue{a}, \blue{b}, \blue{c} \ytext \blue{d}$ deben valer 0 si los vectores son \textit{linealmente independientes}.

        \parrafoDestacado[\tiny\red{\atencion}]{
          También podría poner todos los vectores uno abajo del otro y triangular de una, pero pintó hacerlo así.
        }
        Pero también me piden en el ejercicio que haga cosas con la \textit{transformación lineal}.
        Voy a tener que transformar el $(1,1,-4,5)$ así que también tengo que calcular cómo es la combineta:
        $$
          (1,1,-4,5) = \blue{a} \cdot (1,-1,1,-1) + \blue{b} \cdot  (4,0,-2,4) + \blue{c} \cdot   (1,1,-4,4) + \blue{d} \cdot (-4,2,9,-4) \llamada2
        $$
        Entonces tengo que resolver $\llamada1$ y $\llamada2$, i.e. esto:
        $$
          \matriz{cccc|c|c}{
            1 & 4 & 1 & -4  & 0 & 1 \\
            -1 & 0 & 1 & 2  & 0 & 1 \\
            1 & -2 & -4 & 9 & 0 & -4 \\
            -1 & 4 & 4 & -4 & 0 & 5
          }
          \Sii{triangular}[\magic]
          \matriz{cccc|c|c}{
            1 & 4 & 1 & -4  & 0 & 1 \\
            0 & 2 & 1 & -1  & 0 & 1 \\
            0 & 0 & -1 & 5  & 0 & -1 \\
            0 & 0 & 0 & 7  & 0 & -2
          }
          \entonces
          \llave{rcl}{
            \blue{a} & = & 0 \\
            \blue{b} & = & 0 \\
            \blue{c} & = & 0 \\
            \blue{d} & = & 0
          }
        $$
\end{enumerate}

\begin{aportes}
  \item \aporte{\dirRepo}{naD GarRaz \github}
\end{aportes}

% $$
%   \matriz{cccc}{
%     1 & 4 & 1 & -4 \\
%     -1 & 0 & 1 & 2 \\
%     1 & -2 & -4 & 9 \\
%     -1 & 4 & 1 & -4
%   }
%   \matriz{l}{
%     \blue{a}\\
%     \blue{b}\\
%     \blue{c}\\
%     \blue{d}
%   }
%   =
%   \matriz{l}{
%     0\\
%     0\\
%     0\\
%     0
%   }
% $$
% Resuelvo la matriz ampliada al cero:
% $$
%   \matriz{cccc|c}{
%     1 & 4 & 1 & -4  & 0 \\
%     -1 & 0 & 1 & 2  & 0 \\
%     1 & -2 & -4 & 9 & 0 \\
%     -1 & 4 & 1 & -4 & 0
%   }
%   \Sii{triangular}[\magic]
%   \matriz{cccc|c}{
%     1 & 0 & 0 & 0  & 0 \\
%     0 & 1 & 0 & 0  & 0 \\
%     0 & 0 & 1 & 0  & 0 \\
%     0 & 0 & 0 & 1  & 0
%   }
%   \entonces
%   \llave{rcl}{
%     \blue{a} & = & 0 \\
%     \blue{b} & = & 0 \\
%     \blue{c} & = & 0 \\
%     \blue{d} & = & 0
%   }
% $$
%   \llave{rcl}{
%     T(1,-1,1,-1) & = & (1, -7, 1, 9)\\
%     T(4,0,-2,4)  & = & (0, 8, -1, -6)\\
%     T(1,1,-4,4)  & = & (1, 0, -1, 0)\\
%     T(-4,2,9,-4) & = & (-2, 1, 3, 3)
%   }
%   \flecha{matricialmente}[esto es]
%   [T]_{BE}
%   \matriz{cccc}{
%     1 & 0 & 1 & -2 \\
%     0 & 8 & -1 & -6 \\
%     1 & 0 & -1 & 0 \\
%     -2 & 1 & 3 & 3
%   }
% $$
% Quiero $[T]_{EE}$:
% $$
%   [T]_{EE} = [T]_{BE} \cdot [C]_{EB},
% $$
% donde $[C]_{EB}$ es la inversa de $[C]_{BE}$, pero bueno, alguna cuenta hay que hacer. $[C]_{BE}$ tiene como columnas a los vectores de la
% base $B$, que en este caso son lo vectores que se usaron para transformar:
% $$
%   C_{BE} =
%   \matriz{cccc}{
%     1 & 4 & 1 & -4  \\
%     -1 & 0 & 1 & 2  \\
%     1 & -2 & -4 & 9 \\
%     -1 & 4 & 1 & -4
%   }
% $$
% $$
