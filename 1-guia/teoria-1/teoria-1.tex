\textit{Espacios Vectoriales: Palabras guías}

En un conjunto $A \distinto \vacio$
\begin{enumerate}[label=\tiny\purple{\faIcon{snowman}}]
  \item \textit{Operación: $(a * b) = c$}. Es una función $* : A \times A \to A$
        \begin{enumerate}[label=\roman*)]
          \item\label{item1} $*$ es \textit{asociativa} si $(a * b) * c = a * (b * c) \paratodo a, b, c \en A$.
          \item\label{item2} $*$ \textit{tiene elemento neutro $e$} si $e * a = a * e = a \paratodo a \en A$.
          \item\label{item3} si $*$ \textit{tiene elemento neutro $e$} todo elemento tiene \textit{inverso} para $*$
                si $\paratodo a \en A\quad a * a' = a' * a = e$.
          \item\label{item4} $*$ es \textit{conmutativa} si $a * b = b * a \paratodo a, b \en A$.
        \end{enumerate}

  \item \textit{Grupo: $(A, *)$} es un grupo si se satisfacen \ref{item1}, \ref{item2}y \ref{item3}. Si además se satisface \ref{item4}
        se tiene un \textit{grupo abeliano o conmutativo}.

  \item \textit{Anillo: $(A, +, \cdot)$}. Para ser anillo se debe cumplir:
        \begin{enumerate}[label=\roman*)]
          \item $(A, +)$ es un grupo abeliano o conmutativo.
          \item $\cdot$ es una operación asociativa y tiene elemento neutro.
          \item Vale distribuir:
                $
                  \llave{l}{
                    a \cdot (b+c) = a \cdot b + a \cdot c \\
                    (b+c) \cdot a = b \cdot a + c \cdot a
                  }
                $
        \end{enumerate}
        Si además de cumplir eso, $\cdot$ es conmutativa $(A, +, \cdot)$ es un \textit{anillo conmutativo}.

  \item \textit{Cuerpo $(K, + , \cdot)$:} Un conjunto $K$, $ + \text{ y } \cdot$ operaciones de $K$, es un cuerpo
        si $(K, +, \cdot)$ es un anillo conmutativo y todo elemento no nulo de $K$ tiene inverso.
        \begin{enumerate}[label=\roman*)]
          \item $(A, +) $ es un grupo abeliano o conmutativo,
          \item $(K - \set{0}, \cdot)$ es un grupo abeliano, y
          \item vale la propiedad distributiva de $\cdot$ con respecto a $+$.
        \end{enumerate}

  \item \textit{Acción $\accion$:} es una función $\accion : A \times B \to B$.

  \item \textit{$K-$espacio vectorial:} Sea $(K, +, \cdot)$ un cuerpo. Sea $V$ un conjunto no vacío, sea $+$ una operación en
        $V$ y sea $\accion$ una acción de $K$ en $V$. Se dice que $(V, +, \accion)$ es un $K-$\textit{espacio vectorial} si se
        cumplen las siguiente condiciones:

        \begin{enumerate}[label=\roman*)]
          \item $(V, +)$ es un grupo abeliano.

          \item La acción $\accion$: $K \times V \to V$ satisface:
                \begin{enumerate}[label=\alph*)]
                  \item $a \accion (v + w) = a \accion v + a \accion w  \paratodo a \en K; \paratodo v, w \en V$.
                  \item $(a + b) \accion v = a \accion v  + b \accion v$
                  \item $1 \accion v = v \paratodo v \en V$.
                  \item $(a \accion b) \accion v = a \accion (b \accion v) \paratodo a, b \en K; \paratodo v \en V$.
                \end{enumerate}

                Los elementos de $V$ son \textit{vectores} y los elementos de $K$ se llaman \textit{escalares}. La acción $\accion$
                se llama \textit{producto por escalares}.

                \faIcon{exclamation-triangle}Dejo de escribir a "$\accion$" en rojo, porque no hay problema cuando el punto "$\cdot$"
                actúa sobre un elemento de $K$ y uno de $V$ o entre  2 de $K$.
        \end{enumerate}

  \item \textit{Subespacios:} Subconjunto de un $K-$ espacio vectorial. Sea V un $K-$espacio vectoria y sea $S \subseteq V$. Entonces $S$ es
        un subespacio de $V$ si y solo si valen las siguientes condiciones:
        \begin{enumerate}[label=\roman*)]
          \item $0 \en S$
          \item $v, w \en S \entonces v + w \en S$
          \item $\lambda \en K, v \en S \entonces \lambda \cdot v \en S.$
        \end{enumerate}

  \item \textit{Suma directa de subespacios:} Sea $V$ un $K-$espacio vectorial y sean
        $S_1,\ldots,S_r$ existen únicos $s_i \en S_i, 1\leq i \leq r$, tales que
        $w = s_1 +\ldots + s_r$. En este caso se dice que $W$ es la \textit{suma directa de los
          subespacios} $S_1,\ldots,S_r$ y se nota:
        $$
          W = S_1 \sumaDirecta S_2 \sumaDirecta \cdots \sumaDirecta S_r,
        $$
        equivalentemente
        $$
          W = S_1 + \cdots + S_r  \text{ para cada } 1\leq j \leq r, \text{ vale }
          S_j \inter (S_1 + S_2 + \cdots + S_j + S_{j+1} + \cdots + S_r) = \set{0}.
        $$

  \item \hypertarget{teoria-1:combinacion-lineal}{\textit{\ul{Combinación lineal:}}}

        Sea $V$ un $K-$espacio vectorial, y sea $G = \set{v_1,\ldots,v_r} \subseteq V$.
        Una \textit{combinación lineal de $G$} es un elemento $v \en V$ tal que $v = \sumatoria{i=1}{r} \alpha_i \cdot v_i$ con
        $\alpha_i \en K$ para cada $1 \leq i \leq r$.

  \item\textit{\ul{Independencia lineal:}}
        Sea $V$ un $K-$espacio vectorial y sea $\set{v_\alpha}_{\alpha \en I}$ una familia de vectores de $V$. Se
        dice que  $\set{v_{\alpha}}_{\alpha \en I}$ es \textit{linealmente independiente} (l.i.) si
        $$
          \sumatoria{\alpha \en I}{} a_{\alpha} \cdot v_{\alpha} = 0 \entonces a_{\alpha} = 0 \paratodo \alpha \en I
        $$

  \item \textit{Bases y dimensión:}
        \begin{itemize}
          \item \textit{Escritura única:}
                Sea $V$ un $K-$e.v. y $A = \set{v_1,\ldots,v_n}$ un conjunto l.i. Entonces cualquier $w \en \ket{v_1,\ldots, v_n}$ si puede
                escribirse de una manera única como combineta de $A$.

          \item \textit{Definición base:}
                Sea $V$ un $K-$e.v. . $A = \set{v_1,\ldots,v_n}$ un conjunto. Se dice que es una \textit{base de $V$} si:
                \begin{itemize}
                  \item $A$ genera todo $V$.
                  \item $\ket{v_1,\ldots, v_n}$ son l.i.
                \end{itemize}

          \item Sea $V$ un $K-$e.v. de dimensión $n$, la base canónica de $E$ se define como $E = \set{e_1,\cdots,e_n}$ con
                $$
                  e_i =
                  \llave{ll}{
                    1 & i=j\\
                    0 & i\distinto j
                  }
                $$

          \item \textit{Dimensión:}
                Sea $V$ un $K-$e.v. . $B = \set{v_1,\ldots,v_n}$ una base de $V$. Entonces cualquier otra base $B'$ de $V$ tiene
                la misma cantidad de elementos. Esta cantidad es \textit{la dimensión de $V$}.
        \end{itemize}

  \item \textit{Espacio columna:} Si
        $A =
          \matriz{c|c|c}{
            A_1 & \cdots & A_n
          }
        $. El espacio columna de $A$ es $col(A) = \ket{A_1, \ldots, A_n}$

  \item \textit{Espacio fila:} Si
        $A =
          \matriz{c}{
            A_1 \\ \hline
            \vdots \\ \hline
            A_m
          }
        $. El espacio fila de $A$ es $fil(A) = \ket{A_1, \ldots, A_m}$

  \item \textit{Matrices}:
        \begin{itemize}
          \item \textit{Definición de matriz:}
                $$
                  A \en K^{m \times n} =
                  \set{
                    \matriz{ccc}{
                      a_{11} & \cdots & a_{1n}\\
                      \vdots & \ddots & \vdots\\
                      a_{m1} & \cdots & a_{mn}
                    }
                    / A_{ij} \en K \paratodo i,\,j \quad 1 \leq i \leq n, 1\leq j \leq m
                  }.
                $$

          \item \textit{Igualdad de matrices:}

                Dos matrices de la misma dimensión $A$ y $A'$ serán iguales:
                $$
                  A = A' \sisolosi A_{ij} = A'_{ij} \text{\quad para cada\quad} 1 \leq i \leq n,\, 1\leq j \leq m
                $$

          \item \hypertarget{teoria-1:operaciones-matrices}{\textit{Operaciones de matrices:}}

                \textit{Suma \blue{$A + A'$}, producto de un escalar por una matriz \green{$\alpha A$} y producto entre 2 matrices} \magenta{$A \cdot B$}.
                $$
                  \begin{array}{ccc}
                    \blue{(A + A')_{ij} \igual{def} A_{ij} + A'_{ij}}              & (1 \leq i \leq m, 1 \leq j \leq n)  \\
                    \green{(\alpha A)_{ij} \igual{def} \alpha A_{ij}}              & (1 \leq i \leq m, 1 \leq j \leq n)  \\
                    \magenta{C_{ij} \igual{def} \sumatoria{k = 1}{m} A_{ik}B_{kj}} & (1 \leq i \leq n, 1 \leq j \leq r).
                  \end{array}
                $$

          \item Sea $A \en \reales^{m \times n}$ y $B \en \reales^{n \times r}$
                $$
                  A\cdot B  =
                  \matriz{c|c|c}{
                    A \cdot
                    \matriz{c}{
                      b_{11}\\
                      \vdots\\
                      b_{m1}
                    }
                    &\cdots&
                    A \cdot
                    \matriz{c}{
                      b_{1r}\\
                      \vdots\\
                      b_{mr}
                    }
                  }
                $$
                Una factorización que se puede hacer pensando en esto último, $ A = C \cdot R$, donde:
                \begin{itemize}
                  \item $C$ tiene como columnas las columnas \textit{linealmente independientes} de $A$,
                  \item $R$ tiene como filas a \textit{combinaciones lineales} de las filas de $A$, $fil(A) = fil(R)$.
                \end{itemize}

          \item \textit{Inversa de una matriz:}
                $A \en K^{n \times n}$ es inversible si $\existe B \en K^{n \times n}$ tal que $A \cdot B = Id$.
        \end{itemize}
\end{enumerate}

\begin{enumerate}[label=\faIcon{gamepad}$_{\arabic*)}$]
  \item \textit{Cálculo de determinantes:}

        \begin{enumerate}[label=\tiny\faIcon{poo}]
          \item Dada $A = (a_{ij}) \en K^{n \times n}$ matriz cuadrada, definimos el determinante
                como
                $$
                  det(A) =
                  \llave{cc}{
                    a_{11} & n = 1\\
                    \sumatoria{j = 1}{n}(-1)^{i+j}a_{ij} \det(M_{ij}) & n > 1
                  }
                $$
                donde $M_{ij}$ es la matriz de $(n - 1) \times (n-1)$ que resulta de eliminar
                la fila $i$ y la columna $j$.
          \item Si $A\en K^{ 2 \times 2  }\big/
                  \det(A)=
                  \deter{cc}{
                    a_{11} & a_{12} \\
                    a_{21} & a_{22}
                  } = a_{11}a_{22}\red{-}a_{12}a_{21}  $

          \item Si $A\en \reales^{n \times n} (n\geq2)$, un ejemplo con $n=3$:
                $$
                  \det(A)=\deter{rrr}{
                    \red{ a_{11} } & \red {a_{12} } & \red{ a_{13} } \\
                    a_{21}                                 & a_{22}                                 & a_{23}                                 \\
                    a_{31}                                 & a_{32}                                 & a_{33}
                  } = \red{a_{11}}\cdot(-1)^{\red{1}+\red{1}} \deter{cc}{
                    a_{22} & a_{23} \\
                    a_{32} & a_{33}
                  } + \red{a_{12}}\cdot(-1)^{\red{1}+\red{2}} \deter{cc}{
                    a_{21} & a_{23} \\
                    a_{31} & a_{33}
                  } + \red{a_{13}}\cdot(-1)^{\red{1}+\red{3}} \deter{cc}{
                    a_{21} & a_{22} \\
                    a_{31} & a_{32}
                  }
                $$

          \item
                Y si pinta desarrollar por otra columna o fila:
                $$
                  \det(A)=\deter{rrr}{
                    a_{11} & \red{ a_{12} } & a_{13} \\
                    a_{21} & \red{ a_{22} } & a_{23} \\
                    a_{31} & \red{ a_{32} } & a_{33}
                  }	=	\red{a_{12}}\cdot(-1)^{\red{1}+\red{2}} \deter{cc}{
                    a_{21} & a_{23} \\
                    a_{31} & a_{33}
                  } + \red{a_{22}}\cdot(-1)^{\red{2}+\red{2}} \deter{cc}{
                    a_{11} & a_{13} \\
                    a_{31} & a_{33}
                  } + \red{a_{32}}\cdot(-1)^{\red{3}+\red{2}} \deter{cc}{
                    a_{11} & a_{13} \\
                    a_{21} & a_{23}
                  }
                $$

        \end{enumerate}

  \item \textit{Clasificación de un sistema a partir de su determinante:}
        \begin{enumerate}[label=\tiny\faIcon{poo}]
          \item Dado un sistema de ecuaciones:
                $$
                  \llave{ccc}{
                    a_{11}\red{x_1}+a_{12}\red{x_2}+\cdots+a_{1n}\red{x_n} & = & \blue{b_1},\\
                  a_{21}\red{x_1}+a_{22}\red{x_2}+\cdots+a_{2n}\red{x_n} & = & \blue{b_2},\\
                  \vdots & = & \vdots\\
                  a_{n1}\red{x_1}+a_{n2}\red{x_2}+\cdots+a_{nn}\red{x_n} & = & \blue{b_n}
                  }
                $$

          \item Se lo puede llevar a forma matricial así:
                $$
                  \matriz{cccc}{
                  a_{11}  & a_{12} & \cdots & a_{1n} \\
                  a_{21}  &a_{22} & \cdots & a_{2n}\\
                  \vdots&\vdots&\ddots&\vdots\\
                  a_{n1}& a_{n2} & \cdots & a_{nn}
                  }
                  \matriz{c}{
                    \red{x_1}\\
                    \red{x_2}\\
                    \vdots\\
                    \red{x_n}
                  }  = \matriz{c}{
                    \blue{b_1}\\
                    \blue{b_2}\\
                    \vdots\\
                    \blue{b_n}
                  }$$

          \item En notación compacta:
                $$
                  A \cdot \mathbf{\red{x}} = \mathbf{\blue{b}}
                  \flecha{sist. homogéneo}[$\mathbf{\blue{b}}=0$]
                  A \cdot \mathbf{\red{x}} = \blue{0}
                $$

          \item
                Dado un sistema:
                $$
                  A \cdot \mathbf{\red{x}} = \mathbf{\blue{b}}
                $$

                $$
                  \text{\faIcon{exclamation-triangle}}
                  \begin{array}{c}
                    \llave{lcl}{
                    \mathrm{si} & |A| \neq 0                                   & \flecha{seguro}[tengo] \boxed{\text{S.C.D.}} \to \boxed{\text{UNA SOLA SOLUCIÓN}} \to \boxed{\text{A ES INVERSIBLE}} \\
                    \mathrm{si} & |A| = 0                                      & \flecha{dos}[casos]
                      \llave{lcl}{
                    \mathrm{si} & A \cdot \mathbf{\red{x}} = \blue{0}          & \flecha{tengo}[entonces] \boxed{ \text{S.C.I.} }                                                                     \\
                    \mathrm{si} & A \cdot \mathbf{\red{x}} = \mathbf{\blue{b}} & \flecha{tengo dependiendo}[de \textbf{\blue{b}}]
                        \llave{c}{
                    \boxed{\text{S.C.I.}}                                                                                                                                                             \\
                    \otext                                                                                                                                                                            \\
                          \boxed{ \text{S.I.} }
                        }
                      }
                    }
                  \end{array}
                $$
        \end{enumerate}

  \item \textit{Clasificación de un sistema en general: Teorema de Rouché-Frobenius}

        \begin{enumerate}[label=\roman*)]
          \item Dado un sistema de $\blue{m}$ ecuaciones y $\red{n}$ incognitas:
                $$
                  \llave{ccl}{
                    a_{11}\red{x_1}+a_{12}\red{x_2}+\cdots+a_{1n}\red{x_n} &=& \blue{b_1},\\
                  a_{21}\red{x_1}+a_{22}\red{x_2}+\cdots+a_{2n}\red{x_n} &=& \blue{b_2},\\
                  \vdots  &= &\vdots\\
                  a_{m1}\red{x_1}+a_{m2}\red{x_2}+\cdots+a_{mn}\red{x_n} & = & \blue{b_m}
                  }
                $$

          \item Se lo puede llevar a forma matricial así:
                $$
                  \underbrace{
                  \matriz{cccc}{
                  a_{11}  & a_{12} & \cdots & a_{1n} \\
                  a_{21}  &a_{22} & \cdots & a_{2n}\\
                  \vdots&\vdots&\ddots&\vdots\\
                  a_{m1}& a_{m2} & \cdots & a_{mn}
                  }
                  }_{\text{matriz de coeficientes}}
                  \matriz{c}{
                    \red{x_1}	\\
                    \red{x_2}	\\
                    \vdots		\\
                    \red{x_n}
                  }	=
                  \matriz{c}{
                    \blue{b_1}\\
                    \blue{b_2}\\
                    \vdots\\
                    \blue{b_m}
                  }	\flecha{más}[compacto]
                  A \cdot \red{\mathbf{x}} = \blue{\mathbf{b}}
                $$

          \item Para resolver por triangulación:

                $$
                  \green{A^*}=
                  \underbrace{
                    \matriz{cccc|c}{
                      a_{11}  	& a	_{12} 	& \cdots 	& a_{1n}	 	& \blue{b_1} \\
                      a_{21}  	&	a_{22} 	& \cdots 	& a_{2n}		& \blue{b_2}\\
                      \vdots	&	\vdots	&	\ddots	&\vdots			& \vdots\\
                      a_{m1}		& a_{m2} 	& \cdots 	& 	a_{mn}		& \blue{b_m}
                    }
                  }_{\green{\text{matriz ampliada}}}
                  \flecha{más}[compacto]
                  \green{A^*} = (A|\blue{\mathbf{b}})
                $$

          \item
                \atencion
                $$
                  \flecha{para clasificar}[hay que calcular]
                  \llave{c}{
                    \rg( \green{A^*} ) \\
                    \rg(A)
                  }
                $$

                $$
                  \flecha{luego}[si]
                  \llaves{lcl}{
                    \rango( \green{A^*} ) > \rango(A) &\to& \mbox{S.I. $\to$ !`No hay solución!}\\
                    \\
                    \rango( \green{A^*} ) = \rango(A) &\to& \llave{lcl}{
                      \rango( \green{A^*} ) = \red{n} & \to & \mbox{S.C.D. $\to$ !`Única solución!}      \\
                      \rango( \green{A^*} ) < \red{n} & \to & \mbox{S.C.I. $\to$ !`$\infty$ soluciones!}
                    }
                  }
                $$
        \end{enumerate}
\end{enumerate}

