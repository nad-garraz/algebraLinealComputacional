\textit{Repaso CBC}
\begin{enumerate}[label=\faIcon{gamepad}$_{\arabic*)}$]
  \item \textit{Cálculo de determinantes:}

        \begin{enumerate}[label=\tiny\faIcon{poo}]
          \item Si $A\in \reales^{ 2 \times 2  }\bigg/\longrightarrow \det(A)=\deter{cc}{
                    a_{11} & a_{12} \\
                    a_{21} & a_{22}
                  } = a_{11}a_{22}\red{-}a_{12}a_{21}  $

          \item Si $A\in \reales^{n \times n} (n\geq2)$, un ejemplo con $n=3$:
                $$
                  \det(A)=\deter{rrr}{
                    \red{ a_{11} } & \red {a_{12} } & \red{ a_{13} } \\
                    a_{21}                                & a_{22}                                & a_{23}                                \\
                    a_{31}                                & a_{32}                                & a_{33}
                  } = \red{a_{11}}\cdot(-1)^{\red{1}+\red{1}} \deter{cc}{
                    a_{22} & a_{23} \\
                    a_{32} & a_{33}
                  } + \red{a_{12}}\cdot(-1)^{\red{1}+\red{2}} \deter{cc}{
                    a_{21} & a_{23} \\
                    a_{31} & a_{33}
                  } + \red{a_{13}}\cdot(-1)^{\red{1}+\red{3}} \deter{cc}{
                    a_{21} & a_{22} \\
                    a_{31} & a_{32}
                  }
                $$

          \item
                Y si pinta desarrollar por otra columna o fila:
                $$
                  \det(A)=\deter{rrr}{
                    a_{11} & \red{ a_{12} } & a_{13} \\
                    a_{21} & \red{ a_{22} } & a_{23} \\
                    a_{31} & \red{ a_{32} } & a_{33}
                  }	=	\red{a_{12}}\cdot(-1)^{\red{1}+\red{2}} \deter{cc}{
                    a_{21} & a_{23} \\
                    a_{31} & a_{33}
                  } + \red{a_{22}}\cdot(-1)^{\red{2}+\red{2}} \deter{cc}{
                    a_{11} & a_{13} \\
                    a_{31} & a_{33}
                  } + \red{a_{32}}\cdot(-1)^{\red{3}+\red{2}} \deter{cc}{
                    a_{11} & a_{13} \\
                    a_{21} & a_{23}
                  }
                $$
        \end{enumerate}

  \item \textit{Clasificación de un sistema a partir de su determinante:}
        \begin{enumerate}[label=\tiny\faIcon{poo}]
          \item Dado un sistema de ecuaciones:
                $$
                  \llave{ccc}{
                    a_{11}\red{x_1}+a_{12}\red{x_2}+\cdots+a_{1n}\red{x_n} & = & \blue{b_1},\\
                  a_{21}\red{x_1}+a_{22}\red{x_2}+\cdots+a_{2n}\red{x_n} & = & \blue{b_2},\\
                  \vdots & = & \vdots\\
                  a_{n1}\red{x_1}+a_{n2}\red{x_2}+\cdots+a_{nn}\red{x_n} & = & \blue{b_n}
                  }
                $$

          \item Se lo puede llevar a forma matricial así:
                $$
                  \matriz{cccc}{
                  a_{11}  & a_{12} & \cdots & a_{1n} \\
                  a_{21}  &a_{22} & \cdots & a_{2n}\\
                  \vdots&\vdots&\ddots&\vdots\\
                  a_{n1}& a_{n2} & \cdots & a_{nn}
                  }
                  \matriz{c}{
                    \red{x_1}\\
                    \red{x_2}\\
                    \vdots\\
                    \red{x_n}
                  }  = \matriz{c}{
                    \blue{b_1}\\
                    \blue{b_2}\\
                    \vdots\\
                    \blue{b_n}
                  }$$

          \item En notación compacta:
                $$
                  A \cdot \mathbf{\red{x}} = \mathbf{\blue{b}}
                  \flecha{sist. homogéneo}[$\mathbf{\blue{b}}=0$]
                  A \cdot \mathbf{\red{x}} = \blue{0}
                $$

          \item
                Dado un sistema:
                $$
                  A \cdot \mathbf{\red{x}} = \mathbf{\blue{b}}
                $$

                $$
                  \text{\faIcon{exclamation-triangle}}
                  \begin{array}{c}
                    \llave{lcl}{
                    \mathrm{si} & |A| \neq 0                                   & \flecha{seguro}[tengo] \boxed{\text{S.C.D.}} \to \boxed{\text{UNA SOLA SOLUCIÓN}} \to \boxed{\text{A ES INVERSIBLE}} \\
                    \mathrm{si} & |A| = 0                                      & \flecha{dos}[casos]
                      \llave{lcl}{
                    \mathrm{si} & A \cdot \mathbf{\red{x}} = \blue{0}          & \flecha{tengo}[entonces] \boxed{ \text{S.C.I.} }                                                                     \\
                    \mathrm{si} & A \cdot \mathbf{\red{x}} = \mathbf{\blue{b}} & \flecha{tengo dependiendo}[de \textbf{\blue{b}}]
                        \llave{c}{
                    \boxed{\text{S.C.I.}}                                                                                                                                                             \\
                    \otext                                                                                                                                                                            \\
                          \boxed{ \text{S.I.} }
                        }
                      }
                    }
                  \end{array}
                $$
        \end{enumerate}
\end{enumerate}

%\textiir{Cuerpo}
%\begin{enumerate}[label=\roman*)] 
% \item \K es un cuerpo si posee dos operaciones $+$ y $\dcot$ con los siguientes propiedades $\parartodo a \ytext b \en \K$
%         \begin{itemize}
%                 \item asociativa $(a+b)+c = a + (b+C)$
%                \item conmutaitva $a+b = b+a$
%                \item elemento neutro $a+0 = a = 0+ a$
%                \item elemento inverso $a+(-a) = 0$
%                \item distributiva $a\cdot (b+c) = a\cdot b + a\cdot c$
%                \item conmutativa
%         \end{itemize}
% \item def Sea(\K, +, \cdot) un cuerpo. Sea \V un conjunto no vavío, + una operación en \V y \cdot una acción de \K en \V
%         +: \V \to \V
%                \blue{\cdot} : \K \times \V \to \V
%
%                Entonces (V, +, \cdot) es un \K-espacio vectorial si se cumple:
%                \begin{enumerate}[label=\roman*)] 
%                        \item asociatividad +  \to u + (v + w) = (u+v) + w 
%                 \item conmutatividad
%                 \item elemento neutro + 
%                 \item    inverso + 
%                 \item compatibilidad de  \blue{\cdot} y \cdot  
%                         \to a \blue{\cdot} ( b \cdot v ) = (a \blue{\cdot} b) \cdot v
%                 \item elemento neutro $1\cdot v = v$
%                 \item distributiav de \cdot respecto a + en \V
%                 \item distributiva de \cdot respecto a + en \K:
%                 \end{enumerate}
%                 Ejemplos de \K espacio vectorial: \K, \K^n, \K^{m \times n} 
% \end{enumerate}
%
% \textit{Subespacios}
% def: Sea $V$ un $K$ e.v, un subconjunto $\S \subseteq \V$ no vacío se dice subespacio de $V$ si la suma y el producto por un escalar son una operación y 
% una acción en $S$ que lo convierte en un $K-e.v$.
%
% \textit{ejemplo}
%
%$$
%\begin{array}{c}
% S = \set{ (u,v,0) \en \reales^3, u,v \en \reales} \to es un e.v
% S = \set{ (u.v.1) \en \reales^3 , u, v \en \reales} ¿es subespacio? \to no
%\end{array}
% $$
%
% En general se tiene que si $(V,+,\cdot) $ es un $K$ ev y $S$ un subespacio de $V$ 
% \sii
% \begin{enumerate}[label=\arabic*)] 
%  \item  $0 \en S$ 
%  \item $+ : S \times S \to S$ ($+$ es cerrado por $S$)
%  \item  $\cdot : S \times S \to S$ (\cdot es cerrado por $S$)
%  \end{enumerate}
%
%
%  Determinar si $S$ es subespacio $S = \set{(v_1,v_2,v_3) \en \reales^3: v_1 - v_2 + 3v_3 = 0}$
%  \begin{itemize}
%        \item compruebo que dos vectores al sumarse caigan en $S$
%        \item compurebo que un escalar por un vector caiga en $S$
%  \end{itemize}
%
%  \textit{Generadores}
%
%  Dados $v_1,\ldots,v_n \en V$ una combinación lineal de $v_1,\ldots, v_n$ es un elemento 
%  $a_1v_1 + \cdots + a_nv_m \en V$ $a_1, \ldots, a_n \en K$
%
%
