\textit{Espacios Vectoriales: Palabras guías}

En un conjunto $A \distinto \vacio$
\begin{enumerate}[label=\tiny\purple{\faIcon{snowman}}]
  \item \textit{Operación: $(a * b) = c$}. Es una función $* : A \times A \to A$
        \begin{enumerate}[label=\roman*)]
          \item\label{item1} $*$ es \textit{asociativa} si $(a * b) * c = a * (b * c) \paratodo a, b, c \en A$.
          \item\label{item2} $*$ \textit{tiene elemento neutro $e$} si $e * a = a * e = a \paratodo a \en A$.
          \item\label{item3} si $*$ \textit{tiene elemento neutro $e$} todo elemento tiene \textit{inverso} para $*$
                si $\paratodo a \en A\quad a * a' = a' * a = e$.
          \item\label{item4} $*$ es \textit{conmutativa} si $a * b = b * a \paratodo a, b \en A$.
        \end{enumerate}

  \item \textit{Grupo: $(A, *)$} es un grupo si se satisfacen \ref{item1}, \ref{item2}y \ref{item3}. Si además se satisface \ref{item4}
        se tiene un \textit{grupo abeliano o conmutativo}.

  \item \textit{Anillo: $(A, +, \cdot)$}. Para ser anillo se debe cumplir:
        \begin{enumerate}[label=\roman*)]
          \item $(A, +)$ es un grupo abeliano o conmutativo.
          \item $\cdot$ es una operación asociativa y tiene elemento neutro.
          \item Vale distribuir:
                $
                  \llave{l}{
                    a \cdot (b+c) = a \cdot b + a \cdot c \\
                    (b+c) \cdot a = b \cdot a + c \cdot a
                  }
                $
        \end{enumerate}
        Si además de cumplir eso, $\cdot$ es conmutativa $(A, +, \cdot)$ es un \textit{anillo conmutativo}.

  \item \textit{Cuerpo $(K, + , \cdot)$:} Un conjunto $K$, $ + \text{ y } \cdot$ operaciones de $K$, es un cuerpo
        si $(K, +, \cdot)$ es un anillo conmutativo y todo elemento no nulo de $K$ tiene inverso.
        \begin{enumerate}[label=\roman*)]
          \item $(A, +) $ es un grupo abeliano o conmutativo,
          \item $(K - \set{0}, \cdot)$ es un grupo abeliano, y
          \item vale la propiedad distributiva de $\cdot$ con respecto a $+$.
        \end{enumerate}

  \item \textit{Acción $\accion$:} es una función $\accion : A \times B \to B$.

  \item \textit{$K-$espacio vectorial:} Sea $(K, +, \cdot)$ un cuerpo. Sea $V$ un conjunto no vacío, sea $+$ una operación en
        $V$ y sea $\accion$ una acción de $K$ en $V$. Se dice que $(V, +, \accion)$ es un $K-$\textit{espacio vectorial} si se
        cumplen las siguiente condiciones:

        \begin{enumerate}[label=\roman*)]
          \item $(V, +)$ es un grupo abeliano.

          \item La acción $\accion$: $K \times V \to V$ satisface:
                \begin{enumerate}[label=\alph*)]
                  \item $a \accion (v + w) = a \accion v + a \accion w  \paratodo a \en K; \paratodo v, w \en V$.
                  \item $(a + b) \accion v = a \accion v  + b \accion v$
                  \item $1 \accion v = v \paratodo v \en V$.
                  \item $(a \accion b) \accion v = a \accion (b \accion v) \paratodo a, b \en K; \paratodo v \en V$.
                \end{enumerate}

                Los elementos de $V$ son \textit{vectores} y los elementos de $K$ se llaman \textit{escalares}. La acción $\accion$
                se llama \textit{producto por escalares}.

                \faIcon{exclamation-triangle}Dejo de escribir a "$\accion$" en rojo, porque no hay problema cuando el punto "$\cdot$"
                actúa sobre un elemento de $K$ y uno de $V$ o entre  2 de $K$.
        \end{enumerate}

  \item \textit{Subespacios:} Subconjunto de un $K-$ espacio vectorial. Sea V un $K-$espacio vectoria y sea $S \subseteq V$. Entonces $S$ es
        un subespacio de $V$ si y solo si valen las siguientes condiciones:
        \begin{enumerate}[label=\roman*)]
          \item $0 \en S$
          \item $v, w \en S \entonces v + w \en S$
          \item $\lambda \en K, v \en S \entonces \lambda \cdot v \en S.$
        \end{enumerate}
\end{enumerate}

\bigskip

\textit{Repaso determinante:}

\bigskip

\begin{enumerate}[label=\faIcon{gamepad}$_{\arabic*)}$]
  \item \textit{Cálculo de determinantes:}

        \begin{enumerate}[label=\tiny\faIcon{poo}]
          \item Si $A\in \reales^{ 2 \times 2  }\bigg/\longrightarrow \det(A)=\deter{cc}{
                    a_{11} & a_{12} \\
                    a_{21} & a_{22}
                  } = a_{11}a_{22}\red{-}a_{12}a_{21}  $

          \item Si $A\in \reales^{n \times n} (n\geq2)$, un ejemplo con $n=3$:
                $$
                  \det(A)=\deter{rrr}{
                    \red{ a_{11} } & \red {a_{12} } & \red{ a_{13} } \\
                    a_{21}                                & a_{22}                                & a_{23}                                \\
                    a_{31}                                & a_{32}                                & a_{33}
                  } = \red{a_{11}}\cdot(-1)^{\red{1}+\red{1}} \deter{cc}{
                    a_{22} & a_{23} \\
                    a_{32} & a_{33}
                  } + \red{a_{12}}\cdot(-1)^{\red{1}+\red{2}} \deter{cc}{
                    a_{21} & a_{23} \\
                    a_{31} & a_{33}
                  } + \red{a_{13}}\cdot(-1)^{\red{1}+\red{3}} \deter{cc}{
                    a_{21} & a_{22} \\
                    a_{31} & a_{32}
                  }
                $$

          \item
                Y si pinta desarrollar por otra columna o fila:
                $$
                  \det(A)=\deter{rrr}{
                    a_{11} & \red{ a_{12} } & a_{13} \\
                    a_{21} & \red{ a_{22} } & a_{23} \\
                    a_{31} & \red{ a_{32} } & a_{33}
                  }	=	\red{a_{12}}\cdot(-1)^{\red{1}+\red{2}} \deter{cc}{
                    a_{21} & a_{23} \\
                    a_{31} & a_{33}
                  } + \red{a_{22}}\cdot(-1)^{\red{2}+\red{2}} \deter{cc}{
                    a_{11} & a_{13} \\
                    a_{31} & a_{33}
                  } + \red{a_{32}}\cdot(-1)^{\red{3}+\red{2}} \deter{cc}{
                    a_{11} & a_{13} \\
                    a_{21} & a_{23}
                  }
                $$
        \end{enumerate}

  \item \textit{Clasificación de un sistema a partir de su determinante:}
        \begin{enumerate}[label=\tiny\faIcon{poo}]
          \item Dado un sistema de ecuaciones:
                $$
                  \llave{ccc}{
                    a_{11}\red{x_1}+a_{12}\red{x_2}+\cdots+a_{1n}\red{x_n} & = & \blue{b_1},\\
                  a_{21}\red{x_1}+a_{22}\red{x_2}+\cdots+a_{2n}\red{x_n} & = & \blue{b_2},\\
                  \vdots & = & \vdots\\
                  a_{n1}\red{x_1}+a_{n2}\red{x_2}+\cdots+a_{nn}\red{x_n} & = & \blue{b_n}
                  }
                $$

          \item Se lo puede llevar a forma matricial así:
                $$
                  \matriz{cccc}{
                  a_{11}  & a_{12} & \cdots & a_{1n} \\
                  a_{21}  &a_{22} & \cdots & a_{2n}\\
                  \vdots&\vdots&\ddots&\vdots\\
                  a_{n1}& a_{n2} & \cdots & a_{nn}
                  }
                  \matriz{c}{
                    \red{x_1}\\
                    \red{x_2}\\
                    \vdots\\
                    \red{x_n}
                  }  = \matriz{c}{
                    \blue{b_1}\\
                    \blue{b_2}\\
                    \vdots\\
                    \blue{b_n}
                  }$$

          \item En notación compacta:
                $$
                  A \cdot \mathbf{\red{x}} = \mathbf{\blue{b}}
                  \flecha{sist. homogéneo}[$\mathbf{\blue{b}}=0$]
                  A \cdot \mathbf{\red{x}} = \blue{0}
                $$

          \item
                Dado un sistema:
                $$
                  A \cdot \mathbf{\red{x}} = \mathbf{\blue{b}}
                $$

                $$
                  \text{\faIcon{exclamation-triangle}}
                  \begin{array}{c}
                    \llave{lcl}{
                    \mathrm{si} & |A| \neq 0                                   & \flecha{seguro}[tengo] \boxed{\text{S.C.D.}} \to \boxed{\text{UNA SOLA SOLUCIÓN}} \to \boxed{\text{A ES INVERSIBLE}} \\
                    \mathrm{si} & |A| = 0                                      & \flecha{dos}[casos]
                      \llave{lcl}{
                    \mathrm{si} & A \cdot \mathbf{\red{x}} = \blue{0}          & \flecha{tengo}[entonces] \boxed{ \text{S.C.I.} }                                                                     \\
                    \mathrm{si} & A \cdot \mathbf{\red{x}} = \mathbf{\blue{b}} & \flecha{tengo dependiendo}[de \textbf{\blue{b}}]
                        \llave{c}{
                    \boxed{\text{S.C.I.}}                                                                                                                                                             \\
                    \otext                                                                                                                                                                            \\
                          \boxed{ \text{S.I.} }
                        }
                      }
                    }
                  \end{array}
                $$
        \end{enumerate}
\end{enumerate}

\newpage

\subsection*{Notas del labo}

\textit{Escribir $0.25$ en base $10$}:

Base 10 es obviamente nuestra base favorita:
$$
  \llave{rcl}{
    0.25 \cdot 10 & = & 2 + 0.5\\
    0.5  \cdot 10 & = & 5 + 0\\
    0 \cdot 10 & = & 0 + 0\\
    \cdots & &
  }
  \to
  (0.25)_{10} =
  2 \cdot 10^{-1} +
  5 \cdot 10^{-2} +
  0 \cdot 10^{-3} + 0\cdots
  = 0.25
$$

\textit{Escribir $0.25$ en base $2$}:

Acá va tomando impulso
$$
  \llave{rcl}{
    0.25 \cdot 2 & = & 0 + 0.5\\
    0.5  \cdot 2 & = & 1 + 0\\
    0 \cdot 2 & = & 0 + 0\\
    \cdots & &
  }
  \to
  (0.25)_2 =
  0 \cdot 2^{-1} +
  1 \cdot 2^{-2} +
  0 \cdot 2^{-3} + 0\cdots
  = 0.01
$$

\textit{Escribir $0.3$ en base $2$}:

Acá va tomando impulso
$$
  \llave{rcl}{
    0.3 \cdot 2 & = & \magenta{0} + 0.6\\
    0.6 \cdot 2 & = & \magenta{1} + 0.2\\ \rowcolor{Cerulean!10}
    0.2 \cdot 2 & = & \magenta{0} + 0.4\\
    0.4 \cdot 2 & = & \magenta{0} + 0.8\\
    0.8 \cdot 2 & = & \magenta{1} + 0.6\\
    0.6 \cdot 2 & = & \magenta{1} + 0.2\\ \rowcolor{Cerulean!10}
    0.2 \cdot 2 & = & \magenta{0} + 0.4\\
    0.4 \cdot 2 & = & \magenta{0} + 0.8\\
    0.8 \cdot 2 & = & \magenta{1} + 0.6\\
    \cdots & &
  }
  \scriptstyle
  \to
  (0.3)_{2} =
  \magenta{0} \cdot 2^{-1} +
  \magenta{1} \cdot 2^{-2} +
  \magenta{0} \cdot 2^{-3} +
  \magenta{0} \cdot 2^{-4} +
  \magenta{1} \cdot 2^{-5} +
  \magenta{1} \cdot 2^{-6} +
  \magenta{0} \cdot 2^{-7} +
  \magenta{0} \cdot 2^{-8} +
  \magenta{1} \cdot 2^{-9} +
  \magenta{1} \cdot 2^{-10} +
  \magenta{0} \cdot 2^{-11} +
  \magenta{0} \cdot 2^{-12}
  \cdots = 0.01\overline{0011}
$$
Para escribir al $0.3$ en base 2 voy a necesitar infinitos números en la \textit{mantisa}

\bigskip
\textit{Errores:}

Tengo que un \textit{número de máquina}, número posta que la máquina representa, con la notación \textit{mantisa}, \textit{exponente}:
$$
  \begin{array}{l}
    \text{En base 10} \to x = 0,a_1a_2a_3 \ldots a_m \cdot 10^{exp}  \text{ con } 0 \leq a_i \leq 9 (a_1 \distinto 0) \\
    \text{En base 2} \to x = 0,a_1a_2a_3 \ldots a_m \cdot 2^{exp}  \text{ con } 0 \leq a_i \leq 1  (a_1 \distinto 0)
  \end{array}
$$

Por ejemplo si $m = 3 \entonces x = 0,a_1a_2a_3 \cdot 2^{exp}$.
Para cada valor de $exp$ voy a tener un total de $\ua{1}{a_1} \cdot \ua{2}{a_2} \cdot \ua{2}{a_3} = 4$ posibles valores de máquina.
La separación entre 2 valores $x_1$ y $x_2$ consecutivos es de $2^m$, por eso para órdenes grandes la separación entre un número y otro
es mayor.

Si el número real, real que quiero es $x = 0.3$, la máquina no puede representarlo de forma exacta. Puedo acotar el error en forma absoluta como:
$$
  |x - x^*| \leq \frac{1}{2} \frac{1}{2^m}\cdot 2^{exp}
$$
Y en forma relativa como:
$$
  \frac{|x - x^*|}{|x|} \leq 5 \cdot 2^{-m}
$$
