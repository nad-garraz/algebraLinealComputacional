\begin{enunciado}{\ejercicio}
  Hallar un sistema de generadores para $S \inter T$ y para $S + T$ como subespacios de $V$, y
  determinar si la suma es directa en cada uno de los siguientes casos:
  \begin{enumerate}[label=(\alph*)]
    \item $V = \reales^3,\, S = \set{(x,y,z) : 3x - 2y + z = 0}$ y $T = \set{(x, y ,z) : x + z =0}$.
    \item $V = \reales^3,\, S = \set{(x,y,z) : 3x - 2y + z = 0,\, x-y = 0}$ y $T = \ket{(1,1,0),\, (5,7,3)}$.
    \item $V = \reales^3,\, S = \ket{(1,1,3), (1,3,5), (6,12,24)}$ y $T = \ket{(1,1,0),\, (3,2,1)}$.
    \item $V = \reales^{3 \times 3},\, S = \set{(x_{ij}) / x_{ij} = x_{ji} \paratodo i, j}$ y $T = \set{(x_{ij})\, /\,  x_{11} + x_{12} + x_{13} = 0}$.
    \item $V = \complejos^3,\, S = \ket{(i, 1, 3 3-i), (4,1 - i, 0)}$ y $T = \set{(x \en \complejos^3) : (1 - i)x_1 - 4x_2 + x_3 = 0}$.
  \end{enumerate}
\end{enunciado}

\begin{enumerate}[label=(\alph*)]
  \item Si los subespacios tienen intersección es una buena idea calcularla para armar el subespacio suma. Busco $S \inter T$,
        para eso pido que $(x, y, z)$ cumpla ambas ecuaciónes de los subespacios:
        $$
          \llave{rcl}{
            3x - 2 y  + z & = &  0 \\
            x + z & = &  0
          }
          \to
          \llave{rcl}{
            y = -z\\
            x = - z
          }
          \entonces
          S \inter T = \set{(-1, -1, 1)}
        $$
        Sé que $\dim(S \inter T) = 1$ por lo tanto
        $$
          \dim(S + T) = \dim(S) + \dim(T) - \dim(S \inter T) = 2 + 2 - 1 = 3.
        $$
        Es así que:
        $$\cajaResultado{
            S + T = \reales^3 \entonces B_{S+T} = \set{(1,0,0); (0,1,0); (0,0,1)}
          }.
        $$

        Peeeeero suponete que querés hacerlo de formás más mecánica, lo que podrías hacer es armar la base con algo de info:

        Sé que:
        $$
          \dim(S) = 2,\, \dim(T) = 2
        $$
        Por lo tanto para encontrar una base \textit{linda} de $S + T$, tengo que encontrar un conjunto de generadores, \textit{linealmente independientes}
        que tenga adentro a todo $S$ y a todo $T$. Saco un sistema de generadores de $S$ y uno de $T$:
        $$
          \begin{array}{c}
            S = \ket{(1, 0, -3); (0, 1, 2)} \\
            T = \ket{(-1, 0, 1); (0, 1, 0)}
          \end{array}
        $$
        Un sistema de generadores de $S + T = \ket{(1, 0, -3); (0, 1, 2); (-1, 0, 1); (0, 1, 0)}$. Esto \ul{no es una base}, porque tiene seguro
        algún vector l.d. con el resto. Entonces puedo sacar ese vector y ver si el resto son l.i.:
        $$
          \matriz{ccc}{
            1 & 0 & -3 \\
            0 & 1 & 2 \\
            -1 & 0 & 1 \\
            0 & 1 & 0
          }
          \triangulacion{
            F_3 + F_1 \to F3
          }
          \matriz{ccc}{
            1 & 0 & -3 \\
            0 & 1 & 2 \\
            0 & 0 & -2 \\
            0 & 1 & 0
          }
          \triangulacion{
            F_4 - F_2 \to F4
          }
          \matriz{ccc}{
            1 & 0 & -3 \\
            0 & 1 & 2 \\
            0 & 0 & -2 \\\rowcolor{red!10}
            0 & 0 & -2
          }
        $$
        Listo me quedo con los 3 vectores que sobrevivieron a la triangulación. Una base de $S+T$:
        $$
          \cajaResultado{
            S + T = \set{(1,0,-3), (0,1,2), (0,0,-2)}
          }
        $$
        A mí m gusta usar una base que tenga la info de la intersección y saber a qué subespacio pertenece cada vector, porque me da más control
        en caso de tener que hacer algo luego con esa base.

        Me armo una con un vector de $S$ y uno de $T$ sacados a ojo y también uso la intersección $\blue{(-1,-1,1)}$ que ya se calculó antes.
        Esto va a ser un subespacio suma, porque tiene a todo $S$ y a todo $T$:
        $$
          \cajaResultado{
            S + T = \{\oa{(1,0,-3)}{\en S}, \ua{\oa{\blue{(1,1, -1)}}{\en S}}{\en T}, \ua{(1,0,-1)}{\en T}\}
          }
        $$
        Son \textit{linealmente independientes}, sí. De no haberlo sido elegía otro vector hasta que alguno dé.
        Comprobalo con este código:
        \copyPaste
        \codigoPython{ej-7/codigo7-1.py}

  \item \hacer
  \item \hacer

  \item  $S$ es un subespacios describiendo matrices \textit{simétricas}, es decir que $A = A^T$ y T el subespacio con matrices de
        traza 0, es decir, $\sum t_{ii} = 0$. Escrito esto un poco más en extensión:
        $$
          S =
          \matriz{ccc}{
            s_{11} & s_{12} & s_{13}\\
            s_{12} & s_{22} & s_{23}\\
            s_{13} & s_{23} & s_{33}
          }
          \ytext
          T =
          \matriz{ccc}{
            -(t_{22} + t_{33}) & t_{12} & t_{13}\\
            t_{21} & t_{22} & t_{23}\\
            t_{31} & t_{32} & t_{33}
          }
          \entonces
          S \inter T =
          \matriz{ccc}{
            -(x_{22} + x_{33}) & x_{12} & x_{13}\\
            x_{12} & x_{22} & x_{23}\\
            x_{13} & x_{23} & x_{33}
          }
        $$
        En la última matriz tengo algo que cumple ambas condiciones de las descripciones por comprensión de los subespacios $S$ y $T$.
        El sistema de generadores buscado para la intersección:
        $$
          \cajaResultado{
            S \inter T =
            \ket{
              \matriz{ccc}{
                -1 & 0 & 0\\
                0 & 1 & 0\\
                0 & 0 & 0
              }
              ;\;
              \matriz{ccc}{
                -1 & 0 & 0\\
                0 & 0 & 0\\
                0 & 0 & 1
              }
              ;\;
              \matriz{ccc}{
                0 & 1 & 0\\
                1 & 0 & 0\\
                0 & 0 & 0
              }
              ;\;
              \matriz{ccc}{
                0 & 0 & 1\\
                0 & 0 & 0\\
                1 & 0 & 0
              }
              ;\;
              \matriz{ccc}{
                0 & 0 & 0\\
                0 & 0 & 1\\
                0 & 1 & 0
              }
            }
          }
        $$

        \bigskip

        La suma de estos subespacios tiene pinta de ser todo $\reales^{3 \times 3}$ a ver que onda la dimensión:
        $$
          \cajaResultado{
            \dim(S + T) = \dim(S) + \dim(T) - \dim(S \inter T) = 6 + 8 - 5 = 9
          }
        $$
        Tuqui.

  \item \hacer
\end{enumerate}

\begin{aportes}
  \item \aporte{\dirRepo}{naD GarRaz \github}
\end{aportes}
