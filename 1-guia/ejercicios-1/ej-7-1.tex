\begin{enunciado}{\ejercicio}
  Hallar un sistema de generadores para $S \inter T$ y para $S + T$ como subespacios de $V$, y
  determinar si la suma es directa en cada uno de los siguientes casos:
  \begin{enumerate}[label=(\alph*)]
    \item $V = \reales^3,\, S = \set{(x,y,z) : 3x - 2y + z = 0}$ y $T = \set{(x, y ,z) : x + z =0}$.
    \item $V = \reales^3,\, S = \set{(x,y,z) : 3x - 2y + z = 0,\, x-y = 0}$ y $T = \ket{(1,1,0),\, (5,7,3)}$.
    \item $V = \reales^3,\, S = \ket{(1,1,3), (1,3,5), (6,12,24)}$ y $T = \ket{(1,1,0),\, (3,2,1)}$.
    \item $V = \reales^{3 \times 3},\, S = \set{(x_{ij}) / x_{ij} = x_{ji} \paratodo i, j}$ y $T = \set{(x_{ij})\, /\,  x_{11} + x_{12} + x_{13} = 0}$.
    \item $V = \complejos^3,\, S = \ket{(i, 1, 3-i), (4,1 - i, 0)}$ y $T = \set{(x \en \complejos^3) : (1 - i)x_1 - 4x_2 + x_3 = 0}$.
  \end{enumerate}
\end{enunciado}

\begin{enumerate}[label=(\alph*)]
  \item Si los subespacios tienen intersección es una buena idea calcularla para armar el subespacio suma. Busco $S \inter T$,
        para eso pido que $(x, y, z)$ cumpla ambas ecuaciónes de los subespacios:
        $$
          \llave{rcl}{
            3x - 2 y  + z & = &  0 \\
            x + z & = &  0
          }
          \to
          \llave{rcl}{
            y = -z\\
            x = - z
          }
          \Entonces{meto en}[$(x,y,z)$]
          (x,y,z) = (-z, -z, z) = z\cdot(-1,-1,1)
        $$
        Obteniendo así una base de la intersección:
        $$
          S \inter T = \set{(-1, -1, 1)}
        $$
        La intersección está generada por un solo vector así que $\dim(S \inter T) = 1$ por lo tanto usando el teorema de la dimensión
        para la suma de subespacios:
        $$
          \dim(S + T) = \dim(S) + \dim(T) - \dim(S \inter T) = 2 + 2 - 1 = 3.
        $$
        Es así que:
        $$\cajaResultado{
            S + T = \reales^3 \entonces B_{S+T} = \set{(1,0,0); (0,1,0); (0,0,1)}
          }.
        $$

        Peeeeero suponete que querés hacerlo de formás más mecánica, lo que podrías hacer es armar la base con algo de info:

        Sé que:
        $$
          \dim(S) = 2,\, \dim(T) = 2
        $$
        Por lo tanto para encontrar una base \textit{linda} de $S + T$, tengo que encontrar un conjunto de generadores, \textit{linealmente independientes}
        que tenga adentro a todo $S$ y a todo $T$. Saco un sistema de generadores de $S$ y uno de $T$:
        $$
          S = \ket{(1, 0, -3); (0, 1, 2)}
          \ytext
          T = \ket{(-1, 0, 1); (0, 1, 0)}
        $$
        Un sistema de generadores de $S + T = \ket{(1, 0, -3); (0, 1, 2); (-1, 0, 1); (0, 1, 0)}$. Esto \ul{no es una base}, porque tiene seguro
        algún vector l.d. con el resto. Entonces puedo sacar ese vector y ver si el resto son l.i.:
        $$
          \matriz{ccc}{
            1 & 0 & -3 \\
            0 & 1 & 2 \\
            -1 & 0 & 1 \\
            0 & 1 & 0
          }
          \triangulacion{
            F_3 + F_1 \to F_3
          }
          \matriz{ccc}{
            1 & 0 & -3 \\
            0 & 1 & 2 \\
            0 & 0 & -2 \\
            0 & 1 & 0
          }
          \triangulacion{
            F_4 - F_2 \to F_4
          }
          \matriz{ccc}{
            1 & 0 & -3 \\
            0 & 1 & 2 \\
            0 & 0 & -2 \\\rowcolor{red!10}
            0 & 0 & -2
          }
        $$
        Listo me quedo con los 3 vectores que sobrevivieron a la triangulación. Una base de $S+T$:
        $$
          \cajaResultado{
            S + T = \set{(1,0,-3), (0,1,2), (0,0,-2)}
          }
        $$
        A mí me gusta usar una base que tenga la info de la intersección y saber a qué subespacio pertenece cada vector, porque me da más control
        en caso de tener que hacer algo luego con esa base. Onda, mirá el $(0,0,-2)$ de la base anterior, ese vector no está ni en $S$ ni en $T$!
        Da un poco de miedito, no {\tiny\marron{\poo}}?

        \medskip

        Por eso me armo una base con un vector de $S$ y uno de $T$ sacados a ojo y también uso la intersección
        $\blue{(-1,-1,1)}$ que ya se calculó antes. Esto va a ser un subespacio de $S + T$, porque tiene a todo $S$ y a todo $T$:
        $$
          \cajaResultado{
            S + T = \{\oa{(1,0,-3)}{\en S}, \ua{\oa{\blue{(1,1, -1)}}{\en S}}{\en T}, \ua{(1,0,-1)}{\en T}\}
          }
        $$
        Son \textit{linealmente independientes}, sí. De no haberlo sido elegía otro vector hasta que alguno dé.
        Comprobalo con este código:
        \copyPaste
        \codigoPython{ej-7/codigo7-1.py}

        Los subespacios \underline{no están} en suma directa.

  \item Se puede ver a ojo que $\dim(S) = 1$ y que $\dim{T} = 2$ es decir que podrían estar en suma directa, porque podría no haber intersección.
        Voy a intentar calcularla. Me armo un elemento genérico de $T$ y veo si cumple las ecuaciones de $S$:
        $$
          t_g = a\cdot(1,1,0) + b \cdot (5,7,3) = (\blue{a + 5b, a + 7b, 3b})
          \to
          \llave{l}{
            3 \cdot (\blue{a + 5b}) - 2 \cdot (\blue{a + 7b}) + \blue{3b} = 0 \Sii{$\llamada1$} a = 0\\
            (\blue{a + 5b}) - (\blue{a + 7b}) = 0 \sii b \igual{$\llamada1$} 0
          }
        $$

        Ese resultado me dice que la intersección es el $0$ o dicho de otra manera no tienen intersección:
        $$
          \cajaResultado{
            B_{S \sumaDirecta T} = \reales^3
          }
        $$
        Los subespacios $S$ y $T$ están en suma directa.

  \item  Lo primero que odio cuando veo este ejercicio es que tengo que ver si $S$ tiene generadores \textit{linealmente dependientes}
        y lo segundo que odio es que voy a tener que pasar a ecuaciones algo, para que sea fácil de calcular la intersección:
        $$
          \matriz{ccc}{
            1 & 1 & 3 \\
            1 & 3 & 5 \\
            6 & 12 & 24
          }
          \triangulacion{
            F_2 - F_1 \to F_2\\
            F_3 - 6F_1 \to F_3
          }
          \matriz{ccc}{
            1 & 1 & 3 \\
            0 & 2 & 2 \\ \rowcolor{red!10}
            0 & 6 & 6
          }
          \flecha{me quedo con}[estos generadores]
          S = \set{(1,1,3); (0,1,1)}
        $$
        Sé que \ul{seguro} va a haber una intersección entre $S$ y $T$, porque hay 4 vectores y estoy laburando en $\reales^3$.
        Busco las ecuaciones que generan a $S$:
        $$
          \matriz{cc|c}{
            1 & 0 & x_1 \\
            1 & 1 & x_2 \\
            3 & 1 & x_3
          }
          \triangulacion{
            F_2 - F_1 \to F_2\\
            F_3 - 3F_1 \to F_3
          }
          \matriz{cc|c}{
            1 & 0 & x_1 \\
            0 & 1 & x_2 - x_1 \\
            0 & 1 & x_3 - 3x_1
          }
          \triangulacion{
            F_3 - F_2 \to F_3
          }
          \matriz{cc|c}{
            1 & 0 & x_1 \\
            0 & 1 & x_2 - x_1 \\ \rowcolor{Cerulean!20}
            0 & 0 & x_3 - x_2 - 2x_1
          }
        $$
        Para que ese sistema sea compatible necesito que:
        $$
          x_3 - x_2 - 2x_1 = 0 \entonces S = \set{(x_1, x_2, x_3) / x_3 - x_2 - 2x_1 \igual{$\llamada1$} 0}
        $$

        Listo, ahora es cuestión de hacer como en el item anterior, agarro un genérico de $T$ y lo meto en la ecuación de $S$:
        $$
          t_g \igual{$\llamada2$} (a + 3b, a + 2b, b)
          \flecha{meto en}[$\llamada1$]
          \llave{l}{
            b - (a + 2b) - 2 (a + 3b) = 0
            \sii
            a = -\frac{8}{3}b
            \flecha{reemplazo}[en $\llamada2$]
            t_g = b \cdot (\frac{1}{3}, -\frac{2}{3}, 1)
          }
        $$

        Ese vector $t_g$, es un vector de $T$ que también cumple la ecuación de $S$ por lo tanto también está en $S$:
        $$
          B_{S \inter T} \igual{\red{!}} \set{1, -2, 3}
        $$

        Los subespacios \ul{no} están en suma directa. Y $S + T \igual{\red{!}} \reales^3$.

  \item  $S$ es un subespacios describiendo matrices \textit{simétricas}, es decir que $A = A^T$ y T el subespacio que
        cumpla que $t_{11} = -t_{12} - t_{13}$.
        Escrito esto un poco más en extensión:
        $$
          S =
          \matriz{ccc}{
            s_{11} & s_{12} & s_{13}\\
            s_{12} & s_{22} & s_{23}\\
            s_{13} & s_{23} & s_{33}
          }
          \ytext
          T =
          \matriz{ccc}{
            -(t_{22} + t_{33}) & t_{12} & t_{13}\\
            t_{21} & t_{22} & t_{23}\\
            t_{31} & t_{32} & t_{33}
          }
          \entonces
          S \inter T =
          \matriz{ccc}{
            -(x_{12} + x_{13}) & x_{12} & x_{13}\\
            x_{12} & x_{22} & x_{23}\\
            x_{13} & x_{23} & x_{33}
          }
        $$
        En la última matriz tengo algo que cumple ambas condiciones de las descripciones por comprensión de los subespacios $S$ y $T$.
        El sistema de generadores buscado para la intersección:
        $$
          \cajaResultado{
            S \inter T =
            \ket{
              \matriz{ccc}{
                -1 & 1 & 0\\
                1 & 0 & 0\\
                0 & 0 & 0
              }
              ;\;
              \matriz{ccc}{
                -1 & 0 & 1\\
                0 & 0 & 0\\
                1 & 0 & 0
              }
              ;\;
              \matriz{ccc}{
                0 & 0 & 0\\
                0 & 1 & 0\\
                0 & 0 & 0
              }
              ;\;
              \matriz{ccc}{
                0 & 0 & 0\\
                0 & 0 & 0\\
                0 & 0 & 1
              }
              ;\;
              \matriz{ccc}{
                0 & 0 & 0\\
                0 & 0 & 1\\
                0 & 1 & 0
              }
            }
          }
        $$

        \bigskip

        La suma de estos subespacios tiene pinta de ser todo $\reales^{3 \times 3}$ a ver que onda la dimensión:
        $$
          \cajaResultado{
            \dim(S + T) = \dim(S) + \dim(T) - \dim(S \inter T) = 6 + 8 - 5 = 9
            \quad
            \entonces
            \quad
            S+T = \reales^{3 \times 3}
          }
        $$

        No están en suma directa y la suma es todo el espacio de matrices de $3 \times 3$

  \item
        Armo un vector genético de $S$ para meter en ecuaciones de $T$:
        $$
          s_g = a \cdot (i, 1, 3 - i) + b \cdot (4,1 - i, 0)
          \begin{array}{lrll}
            \flecha{reemplazo} & (1 - i)(4b + ia) - 4(a + b) + 3a - ia & = & 0 \\
                               & b - i4b                               & = & 0
          \end{array}
        $$
        Por lo tanto con $b = 0$ la intersección queda:
        $$
          B_{S \inter T} = \set{(i, 1, 3-i)}
        $$
        Usando el teorema de la dimensión para suma de subespacios:
        $$
          \dim(S + T) = \dim(S) + \dim(T) - \dim(S \inter T) = 2 + 2 - 1 = 3\\
        $$
        Los espacios no están en suma directa y $S+T = \complejos^3$.

        \bigskip
        \red{  {\Large{CONSULTAR:}}
          ¿Qué onda el cuerpo? $\complejos \otext \reales$. En reales, debería extender, no???}
\end{enumerate}

\begin{aportes}
  \item \aporte{\dirRepo}{naD GarRaz \github}
\end{aportes}
