\begin{enunciado}{\ejercicio}
  Sea $S=\ket{(1,-1,2,1), (3,1,0,-1), (1,1,-1,-1))} \subseteq \reales^4$.

  \begin{enumerate}[label=(\alph*)]
    \item Determinar si $(2,1,3,5) \en S$.
    \item Determinar si $\set{x \en \reales^4 / x_1 - x_2 - x_3 = 0} \subseteq S$.
    \item Determinar si $S \subseteq \set{x \en \reales^4 / x_1 - x_2 - x_3 = 0}$.
  \end{enumerate}
\end{enunciado}

Antes de arrancar, siempre conviene verificar que los elementos de $S$ sean \textit{linealmente independientes}. Cuando nos dan un subespacio
expresado en sus generadores, conviene eliminar la información que sobra. Una forma de hacer esto es poner a los \textit{generadores} de $S$
como filas de una matriz y triangular. Al triangular lo que se está haciendo son operaciones lineales para ver si estos coinciden:
$$
  \matriz{cccc}{
    1 & -1 & 2 & 1\\
    3 & 1  & 0 & -1\\
    1 & 1  & -1  &-1
  }
  \triangulacion{
    F_2 - 3F_1 \to F_2\\
    F_3 - F_1 \to F_3\\
  }
  \matriz{cccc}{
    1 & -1 & 2 & 1\\ \rowcolor{blue!10!white}
    0 & 4  & -6 & -4\\\rowcolor{blue!10!white}
    0 & 2  & -3  &-4
  }
$$
Sobra un elemento. Puedo eliminar cualquiera de los 3 siempre y cuando los restantes sean \textit{linealmente independientes}.
Reescribo al subespacio como:
$$
  S = \set{(1,-1,2,1), (1,1,-1,-1)}
$$
Lo cual es una base de $S$.

\begin{enumerate}[label=(\alph*)]
  \item Para ver si $(2,1,3,5) \en S$ hay que realizar una combinación lineal de los elementos del subespacio $S$ igualada al vector en cuestión.
        $$
          (2,1,3,4) =
          a \cdot (1,-1,2,1) +
          b \cdot (1,1,-1,-1)
          \Sii{\red{!}}
          \matriz{cc|c}{
            1 & 1 & 2 \\
            -1 & 1 & 1 \\
            2 & -1 & 3 \\
            1 & -1 & 5
          }
          \equivalente
          \matriz{cc|c}{
            1 & 1 & 2 \\
            0 & 2 & 3 \\ \rowcolor{red!20!white}
            0 & 0 & 7 \\
            0 & 0 & 0
          }
        $$
        Eso es un absurdo, dado que $a \cdot 0  + b \cdot 0 \distinto 7 \paratodo a, b \en \reales$ {\tiny \rollingEyes}.
        Es así que:
        $$
          \cajaResultado{
            (2,1,3,4) \not\en S
          }
        $$

  \item Primero bautizo a ese subespacio como $T$:
        $$
          T = \set{x \en \reales^4 / x_1 - x_2 - x_3 = 0}.
        $$
        A mí me gusta atacar estos problemas primero viendo las \textit{dimensiones} de cada subespacio. Si $T$ es un subespacio
        que vive en $\reales^4$, y además está dado por comprensión con \underline{una sola ecuación}, entonces $\dim(T) = 3$ y como
        $\dim(S) = 2$ sé que:
        $$
          \cajaResultado{
            T \not\subseteq S
          }
        $$
        Si no lo ves directamente, siempre podés calcular los generadores de $T$ a partir de la ecuación, y vas a obtener algo así:
        $$
          T = \ket{(1, 0, 1, 0), (1,1,0,0), (0,0,0,1)}
        $$

  \item
        Este está medio regalado también, porque hay uno de los generadores de $S$ que no cumple la ecuación de $T$, así que ya puedo
        concluir que:
        $$
          \cajaResultado{
            S \not\subseteq T
          }
        $$
        Si no te parece que sea tan evidente, el procedimiento es armar una \textit{combinación lineal} de los elementos de $S$ y meter eso
        en la ecuación de $T$. Ese es el método para buscar $S \inter T$. Si esa intersección te da que $\dim(S \inter T) = 2$ entonces
        todo $S$ debería estar en $T$. Pero bueh, hacé las cuentas y te va a dar que $\dim(S \inter T) = 1$.
\end{enumerate}

\begin{aportes}
  \item \aporte{\dirRepo}{naD GarRaz \github}
\end{aportes}
