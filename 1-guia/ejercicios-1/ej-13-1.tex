\begin{enunciado}{\ejercicio}
  Sean $m, n$ y $r \en \naturales$.
  \begin{enumerate}[label=(\alph*)]
    \item Probar que si $A \en K^{m \times n}$ satisface que $Ax = 0  \paratodo x \en K^n$, entonces $A=0$.
          Deducir que si $A,B \en K^{m \times n}$ satisfacen que $Ax = Bx \paratodo x \en K^n$, entonces $A = B$.

    \item Probar que si $A \en K^{m \times n}, B \en K^{n \times r}$ con $B = (b_{ij})$ y, para $1 \leq j \leq r$,
          $B_j =
            \matriz{c}{
              b_{ij}\\
              \vdots\\
              b_{nj}
            }
          $
          es la columna $j-$ésima de $B$, entonces $AB = (AB_1 | \cdots | AB_r)$ (es decir, $AB_j$ es la columna $j-$ésima de $AB$).
  \end{enumerate}
\end{enunciado}

\begin{enumerate}[label=(\alph*)]
  \item Tengo $A \en K^{n \times n}$ entonces $Ax$:
        $$
          \matriz{ccc}{
            a_{11} & \dots & a_{1n} \\
            \vdots & \ddots & \vdots \\
            a_{m1} & \dots & a_{mn}
          }
          \cdot
          \matriz{c}{
            x_1 \\
            \vdots \\
            x_n
          }
          =
          x_1 \cdot
          \matriz{c}{
            a_{11} \\
            \vdots \\
            a_{m1}
          }
          +
          \dots
          +
          x_n \cdot
          \matriz{c}{
            a_{1n} \\
            \vdots \\
            a_{mn}
          }
          = 0
        $$

        Probando particularmente con la base canónica de $K^n$ $x \en K^n$ con $x \en B$, donde
        $$
          B = \set{(1, 0, \dots, 0); (0, 1, \dots, 0); \dots; (0, \dots, 1)}
        $$ muestro así que las columnas de $A$ son siempre nulas.

  \item Usando lo que hice en el anterior:
        $$
          Ax = Bx \sisolosi \blue{(A - B)} x = 0 \sisolosi \blue{C}x = 0
        $$
        Dado que $Cx = 0 \paratodo x \en K^n$ se muestra que $A = B$.

\end{enumerate}

\begin{aportes}
  \item \aporte{\dirRepo}{naD GarRaz \github}
\end{aportes}
