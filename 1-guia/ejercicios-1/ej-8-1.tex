\begin{enunciado}{\ejercicio}
  Determinar todos los $k \en \reales$ para los cuales:
  \begin{enumerate}[label=(\alph*)]
    \item $\ket{(-2, 1, 6), (3, 0, -8)} = \ket{(1,k,2k), (-1,-1, k^2-2), (1,1,k)}$.
    \item $S \inter T = \ket{(0,1,1)}$ siendo $S = \set{x \en \reales : x_1 + x_2 - x_3 = 0} \ytext T = \ket{(1,k,2), (-1,2,k)}$.
  \end{enumerate}
\end{enunciado}

\begin{enumerate}[label=(\alph*)]
  \item  Lo primero que quiero hacer es que los generadores sean \textit{linealmente independientes} y porque me gusta determinantes {\tiny\surprise}:
        $$
          \begin{array}{rrl}
            \deter{ccc}{
            1  & k    & 2k                   \\
            -1 & -1   & k^2 - 2              \\
            1  & 1    & k                    \\
            }
            \triangulacion{
              F_2 + F_3 \to F_3
            }
            \deter{ccc}{
            1  & k    & 2k                   \\
            -1 & -1   & k^2 - 2              \\ \rowcolor{Cerulean!10}
            0  & 0    & k^2 + k - 2
            }
               & =    &
            (-1)^6 \cdot ( k^2 + k - 2)
            \cdot
            \deter{cc}{
            1  & k                           \\
            -1 & -1
            }                                \\
               & =    &
            (k^2 + k - 2) \cdot (-1 + k) = 0 \\
               & \sii &
            k \en \set{-2, 1}
          \end{array}
        $$
        Así me saco el tema de las $k$ de encima, pero bueh todavía no se termina:
        $$
          \ket{(-2, 1, 6), (3, 0, -8)}
          \igual{?}
          \llave{lcl}{
            \ket{(1,-2,-4), (-1,-1,2), (1,1,-2)  }= \ket{(1,-2,-4), (1,1,-2)} & \text{ si } & k = -2\\
            \ket{(1,1,2), (-1,-1,-1), (1,1,1) }= \ket{(1,1,2), (1,1,1)} & \text{ si }&  k = 1
          }
        $$
        Para que los los subespacios sean iguales, por ejemplo podría ver si se intersectan en todos sus elementos. Voy a
        buscar la expresión por comprensión de $\ket{(-2, 1, 6), (3, 0, -8)}$:
        $$
          \begin{array}{rcl}
            \matriz{cc|c}{
            -2 & 3  & x_1                                     \\
            1  & 0  & x_2                                     \\
            6  & -8 & x_3                                     \\
            }
            \triangulacion{
              F_1 \leftrightarrow F_2
            }
            \matriz{cc|c}{
            1  & 0  & x_2                                     \\
            -2 & 3  & x_1                                     \\
            6  & -8 & x_3
            }
               &
            \triangulacion{
            F_2 + 2 F_1 \to F_2                               \\
              F_3 - 6 F_1 \to F_3
            }
               &
            \matriz{cc|c}{
            1  & 0  & x_2                                     \\
            0  & 3  & x_1 + 2 x_2                             \\
            0  & -8 & x_3 - 6 x_2
            }                                                 \\
               &
            \triangulacion{
              \frac{1}{3}F_2 \to F_2
            }
               &
            \matriz{cc|c}{
            1  & 0  & x_2                                     \\
            0  & 1  & \frac{1}{3}x_1 + \frac{2}{3} x_2        \\
            0  & -8 & x_3 - 6 x_2
            }                                                 \\
               &
            \triangulacion{
              F_3 + 8F_2 \to F_3
            }
               &
            \matriz{cc|c}{
            1  & 0  & x_2                                     \\
            0  & 1  & \frac{1}{3}x_1 + \frac{2}{3} x_2        \\ \rowcolor{red!10}
            0  & 0  & \frac{8}{3} x_1 - \frac{2}{3} x_2 + x_3
            }
          \end{array}
        $$
        Dado que ese sistema no puede dar un absurdo, porque el subespacio claramente no es $\vacio$ se debe cumplir:
        $$
          \ket{(-2, 1, 6), (3, 0, -8)} =
          \set{ x \en \reales^3 : 8 x_1 - 2 x_2 + 3x_3 =0 } \llamada1
        $$
        Encontrar la intersección ahora es fácil:

        \medskip

        \textit{Caso con $k = -2$}:
        $$
          (a + b,-2a + b,-4a -2 b)
          \flecha{meto en}[$\llamada1$]
          8(a + b) - 2 (-2a + b) + 3 (-4a - 2b) = 0 \paratodo a \text{ y } b \en \K
        $$
        Los subespacios son iguales con $k = -2$

        \medskip

        \textit{Caso con $k = 1$}:
        $$
          (a + b, a + b, 2a + b)
          \flecha{meto en}[$\llamada1$]
          8(a + b) - 2 (a + b) + 3 (2a + b) = 0
          \sii
          b = -\frac{4}{3}a
        $$
        Los subespacios tienen intersección pero no son iguales

        Concluímos que el único valor de $k$ para el cual los subespacios son iguales es:
        $$
          \cajaResultado{
            k  = -2
          }
        $$

  \item  \hacer
\end{enumerate}

\begin{aportes}
  \item \aporte{\dirRepo}{naD GarRaz \github}
\end{aportes}
