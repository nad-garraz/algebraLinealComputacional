\begin{enunciado}{\ejercicio}
  Determinar todos los $k \en \reales$ para los cuales:
  \begin{enumerate}[label=(\alph*)]
    \item $\ket{(-2, 1, 6), (3, 0, -8)} = \ket{(1,k,2k), (-1,-1, k^2-2), (1,1,k)}$.
    \item $S \inter T = \ket{(0,1,1)}$ siendo $S = \set{x \en \reales : x_1 + x_2 - x_3 = 0} \ytext T = \ket{(1,k,2), (-1,2,k)}$.
  \end{enumerate}
\end{enunciado}

\begin{enumerate}[label=(\alph*)]
  \item
        Hay una igualdad de subespacios. Para que estos sean iguales tienen que tener la misma dimensión. Dado que el subespacio
        $$
          \dim(\ket{(-2, 1, 6), (3, 0, -8)}) = 2,
        $$
        quiero que el subespacio de la derecha también tenga dimensión 2, peeero tiene 3 vectores, así
        que busco $k$ para que:
        $$
          \dim(\ket{(1,k,2k), (-1,-1, k^2-2), (1,1,k)}) = 2
        $$
        también. Dicho de otra manera, quiero que esos 3 vectores sean \textit{linealmente dependientes}:
        $$
          \begin{array}{rrl}
            \deter{ccc}{
            1  & k               & 2k          \\
            -1 & -1              & k^2 - 2     \\
            1  & 1               & k           \\
            }
            \triangulacion{
              F_2 + F_3 \to F_3
            }
            \deter{ccc}{
            1  & k               & 2k          \\
            -1 & -1              & k^2 - 2     \\ \rowcolor{Cerulean!10}
            0  & 0               & k^2 + k - 2
            }
               & =               &
            (-1)^6 \cdot ( k^2 + k - 2)
            \cdot
            \deter{cc}{
            1  & k                             \\
            -1 & -1
            }                                  \\
               & \igual{\red{!}} &
            (k^2 + k - 2) \cdot (-1 + k) = 0   \\
               & \sii            &
            k \en \set{-2, 1}
          \end{array}
        $$
        Ahora sé que la única forma de que la dimensión sea 2 es para los $k$ hallados.

        ¿Cómo queda el enunciado con los valores de $k$ hallados?:
        $$
          \ket{(-2, 1, 6), (3, 0, -8)}
          \igual{\red{?}}
          \llave{lccccl}{
            \text{ si } & k = -2 & \to & \ket{(1,-2,-4), (-1,-1,2), (1,1,-2)} & \igual{\red{!}} & \ket{(1,-2,-4), (1,1,-2)}\\
            \text{ si }&  k = 1 & \to &\ket{(1,1,2), (-1,-1,-1), (1,1,1)} & \igual{\red{!}} & \ket{(1,1,2), (1,1,1)}
          }
        $$
        Para que los los subespacios sean iguales, por ejemplo podría ver \textit{si se intersectan en todos sus elementos}. Voy a
        buscar la expresión por comprensión de $\ket{(-2, 1, 6), (3, 0, -8)}$, es decir la fórmula que deben satisfacer
        los elementos para pertenecer al subespacio:
        $$
          a \cdot (-2,1,6) + b \cdot (3,0,-8) = (x_1, x_2, x_3)
        $$
        Ese sistema es literalmente: ¿Cómo combino los elementos para formar algo del suespacio?, es un sistema que debe ser \underline{compatible},
        porque seguro que \textit{algo} tiene que salir de hacer una combinación lineal:
        $$
          \begin{array}{rcl}
            \matriz{cc|c}{
            -2 & 3  & x_1                                     \\
            1  & 0  & x_2                                     \\
            6  & -8 & x_3                                     \\
            }
            \triangulacion{
              F_1 \leftrightarrow F_2
            }
            \matriz{cc|c}{
            1  & 0  & x_2                                     \\
            -2 & 3  & x_1                                     \\
            6  & -8 & x_3
            }
               &
            \triangulacion{
            F_2 + 2 F_1 \to F_2                               \\
              F_3 - 6 F_1 \to F_3
            }
               &
            \matriz{cc|c}{
            1  & 0  & x_2                                     \\
            0  & 3  & x_1 + 2 x_2                             \\
            0  & -8 & x_3 - 6 x_2
            }                                                 \\
               &
            \triangulacion{
              \frac{1}{3}F_2 \to F_2
            }
               &
            \matriz{cc|c}{
            1  & 0  & x_2                                     \\
            0  & 1  & \frac{1}{3}x_1 + \frac{2}{3} x_2        \\
            0  & -8 & x_3 - 6 x_2
            }                                                 \\
               &
            \triangulacion{
              F_3 + 8F_2 \to F_3
            }
               &
            \matriz{cc|c}{
            1  & 0  & x_2                                     \\
            0  & 1  & \frac{1}{3}x_1 + \frac{2}{3} x_2        \\ \rowcolor{red!10}
            0  & 0  & \frac{8}{3} x_1 - \frac{2}{3} x_2 + x_3
            }
          \end{array}
        $$
        Como dije antes, este sistema debe ser \ul{compatible}, no puede dar un absurdo, porque el subespacio claramente no es $\vacio$. Debe cumplir:
        $$
          \ket{(-2, 1, 6), (3, 0, -8)}
          \igual{\red{!}}
          \set{ x \en \reales^3 : 8 x_1 - 2 x_2 + 3x_3 =0 } \llamada1
        $$
        Encontrar la intersección ahora con la ecuación del subespacio es fácil. Se arma un genérico y se mete en la ecuación:

        \medskip

        \textit{Caso con $k = -2$}:
        $$
          (a + b,-2a + b,-4a -2 b)
          \flecha{meto en}[$\llamada1$]
          8(a + b) - 2 (-2a + b) + 3 (-4a - 2b) = 0 \paratodo a \text{ y } b \en \K
        $$
        Los subespacios son iguales con $k = -2$

        \medskip

        \textit{Caso con $k = 1$}:
        $$
          (a + b, a + b, 2a + b)
          \flecha{meto en}[$\llamada1$]
          8(a + b) - 2 (a + b) + 3 (2a + b) = 0
          \sii
          b = -\frac{4}{3}a
        $$
        Los subespacios tienen intersección pero no son iguales

        Concluímos que el único valor de $k$ para el cual los subespacios son iguales es:
        $$
          \cajaResultado{
            k  = -2
          }
        $$

  \item  $$
          a \cdot (1,k,2) + b \cdot (-1,2,k) = (a - b, ak + 2b, 2a + bk) = (0,1,1)
          \flecha{\magic}
          \matriz{cc|c}{
            1 & -1 & 0 \\
            k & 2 & 1 \\
            2 & k & 1 \\
          }
        $$
        Para que el $(0,1,1) \en T$ ese sistema tiene que tener solución:
        $$
          \matriz{cc|c}{
            1 & -1 & 0 \\
            k & 2 & 1 \\
            2 & k & 1
          }
          \triangulacion{
            F_2 - kF_1 \flecha{$\magenta{k\distinto 0}$} F_2\\
            F_3 - 2F_1 \to F_3
          }
          \matriz{cc|c}{
            1 & -1 & 0 \\
            0 & k + 2 & 1 \\
            0 & k + 2 & 1
          }
          \triangulacion{
            F_3 - F_2 \to F_2
          }
          \matriz{cc|c}{
            1 & -1 & 0 \\
            0 & k + 2 & 1 \\
            0 & 0 & 0
          }
        $$
        Este sistema será compatible con $k \taa{\llamada1}{}{\distinto} -2$. No me quiero olvidar del caso $\magenta{k = 0}$:
        $$
          \matriz{cc|c}{
            1 & -1 & 0 \\
            \magenta{0} & 2 & 1 \\
            2 & \magenta{0} & 1
          }
          \triangulacion{
            F_3 - 2F_1 \to F_3
          }
          \matriz{cc|c}{
            1 & -1 & 0 \\
            \magenta{0} & 2 & 1 \\
            0 & 2 & 1
          }
          \triangulacion{
            F_3 - F_2 \to F_3
          }
          \matriz{cc|c}{
            1 & -1 & 0 \\
            \magenta{0} & 2 & 1 \\ \rowcolor{red!10}
            0 & 0 & 0
          }
        $$
        Entonces con $k = 0$ también es compatible, así que acá no pasó nada. Con $k = -2 \entonces (0,1,1) \not\en T$.

        Corroboro ahora que haya intersección entre $S$ y $T$. Hago un genérico de $T$:
        $$
          a \cdot (1,k,2) + b \cdot (-1,2,k) = (a - b, ak + 2b, 2a + bk)
        $$
        y lo reemplazo en ecuación de $S$:
        $$
          a-b + ak + 2b -2a - bk =  0
          \sii
          -a+b + ak - bk =  0
          \sii
          (-a + b)(1 - k) =  0
          \sii
          \llave{c}{
            a = b  \\
            \otext \\
            k = 1
          }
        $$
        Si $k = 1$ no tengo condiciones sobre $a \ytext  b$ es decir que la dimensión de la intersección sería todo $T$, eso es algo
        \ul{malo}, porque la dimensión debe ser 1.

        Cuando es $a = b$ no me importa el valor de $k$,  pero no olvidar $\llamada1$.
        Entonces, si quiero que $S \inter T = \ket{(0,1,1)}$ necesito:
        $$
          \cajaResultado{
            k \not\en \set{-2;\, 1}
          }
        $$
\end{enumerate}

\begin{aportes}
  \item \aporte{\dirRepo}{naD GarRaz \github}
\end{aportes}
