\begin{enunciado}{\ejercicio}
  \begin{enumerate}[label=(\alph*)]
    \item Determinar los valores $k \en \reales$ para que el siguiente sistema tenga solución
          única, infinitas soluciones, o no tenga solución:
          $$
            \llave{ccc}{
              x_1 + k x_2 - x_3 & = & 1\\
              -x_1 + x_2 + k^2x_3 & = & -1\\
              x_1 + kx_2 + (k - 2)x_3 & = & 2
            }
          $$

    \item
          Considerar el sistema homogéneo asociado y dar los valores de $k$ para los cuales
          admite solución no trivial. Para esos $k$, resolverlo.
  \end{enumerate}
\end{enunciado}

\begin{enumerate}[label=(\alph*)]
  \item
        No tengo ganas de triangular. Ejercicios con letras y matrices cuadradas, calculo determinante de la
        matriz de coeficiente:
        $$
          \begin{array}{ccl}
            \deter{ccc}{
            \blue{1}  & k     & -1    \\
            \blue{-1} & 1     & k^2   \\
            \blue{1}  & k     & k - 2
            }
                      & =     &
            \blue{1} \cdot
            \deter{cc}{
            1         & k^2           \\
            k         & k - 2
            }
            +
            (\blue{-1}) \cdot (-1)^3
            \deter{cc}{
            k         & -1            \\
            k         & k - 2
            }
            +
            (\blue{1}) \cdot (-1)^4
            \deter{cc}{
            k         & -1            \\
            1         & k^2
            }
            \\
                      & =     &
            k^2 - 1 = 0
            \sii
            k = 1 \otext k = -1
          \end{array}
        $$

        Por lo tanto sé que para que el sistema \underline{no tenga solución única}
        debe ocurrir que:
        $$
          \cajaResultado{
            k = 1 \otext k = -1
          }
        $$
        Ahora hay que probar a mano con cada valor de $k$
        para ver en cada caso si el sistema queda \textit{indeterminado} o \textit{incompatible}

        \medskip

        Si $\blue{k} = \blue{1}$:
        $$
          \matriz{ccc|c}{
            1  & \blue{1}     & -1 &1    \\
            -1 & 1     & \blue{1}^2  & -1 \\
            1  & \blue{1}     & \blue{1} - 2 &-2
          }
          \triangulacion{
            F_2 + F_1 \to F_2\\
            F_3 - F_1 \to F_3
          }
          \matriz{ccc|c}{
            1  & 1 & -1 & 1 \\
            0  & 2 & 0  & 0 \\\rowcolor{red!10}
            0  & 0 & 0  & -3
          }
        $$
        No hay solución con $k = 1$

        Si $\blue{k} = \blue{-1}$:
        $$
          \matriz{ccc|c}{
            1  & \blue{-1}     & -1 &1    \\
            -1 & 1     & (\blue{-1})^2  & -1 \\
            1  & \blue{-1}     & \blue{-1} - 2 &-2
          }
          \triangulacion{
            F_2 + F_1 \to F_2\\
            F_3 - F_1 \to F_3
          }
          \matriz{ccc|c}{
            1  & -1 & -1 & 1 \\\rowcolor{red!10}
            0  & 0 & 0  & 0 \\
            0  & -2 & -4  & -1
          }\llamada1
        $$
        Habrá infinitas soluciones con $k = -1$

  \item El sistema homogéneo asociado en el caso $\blue{k} = \blue{-1}$:
        $$
          \llave{ccc}{
            x_1 + \blue{-1} x_2 - x_3 & = & 0\\
            -x_1 + x_2 + (\blue{-1})^2x_3 & = & 0\\
            x_1 + \blue{-1} x_2 + (\blue{-1} - 2)x_3 & = & 0
          }
        $$

        Utilizando la triangulación de antes ($\llamada1$) el sistema quedaría así:
        $$
          \llave{ccl}{
            x_1 - x_2 - x_3 &=& 0 \\
            - 2x_2 - 4x_3 &=& 0
          }
          \sii
          x =
          \matriz{c}{
            x_1\\
            x_2\\
            x_3
          }
          =
          x_3 \cdot
          \matriz{c}{
            -1 \\
            -2 \\
            1
          }
        $$

\end{enumerate}

\begin{aportes}
  \item \aporte{\dirRepo}{naD GarRaz \github}
\end{aportes}
