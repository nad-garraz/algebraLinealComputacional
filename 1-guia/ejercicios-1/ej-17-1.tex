\begin{enunciado}{\ejercicio}
  Decidir cuáles de los siguientes subconjuntos son subespacios de $\reales^{n \times n}$ como $\reales-$espacio
  vectorial:
  \begin{enumerate}[label=(\alph*)]
    \begin{multicols}{2}
      \item $S_1 = \set{A \en \reales^{n \times n} : A \text{ es triangular inferior}}$
      \item $S_2 = \set{A \en \reales^{n \times n} : A \text{ es simétrica}}$
    \end{multicols}
  \end{enumerate}
\end{enunciado}

\begin{enumerate}[label=(\alph*)]
  \item Si $A$ es triangular inferior:
        $$
          [A]_{ij} =
          \llave{rcl}{
            0 & \text{si} & i < j \\
            a_{ij} & \multicolumn{2}{r}{\text{en otro caso}}
          }
        $$
        Es un subespacio cuyos elementos son matrices canónicas siempre triangulares inferiores de la pinta:
        $$
          S_1 = \set{E^{ij} \en \reales^{n \times n} \paratodo i > j} \ytext \dim(S_1) = \frac{n^2 + n}{2}
        $$
        La multiplicación y suma también resultará en matrices \textit{triangular inferiores}
        $$
          \alpha E^{ij} + \beta E^{kl} \en S_2
        $$

  \item También es un subespacio, una matriz simétrica:
        $$
          [A]_{ij} = [A]_{ji}  \paratodo i,\,j
        $$
        $S_2$ está generado por matrices de la forma:
        $$
          S_2^{ij} \en \reales^{n \times n}
          \text{ tal que }
          S_2^{ij} =
          \llave{rcl}{
            E^{ij} + E^{ij} & \text{si} & j \distinto i \\
            1 & \text{si} & i = j
          }
        $$
        Nuevamente la multiplicación por escalares y suma entre matrices simétricas tendrá como resultado a otra matriz simétrica:
        $$
          \alpha \cdot S_2^{ij} + \beta \cdot S_2^{kl} \en S_2
        $$
\end{enumerate}

\begin{aportes}
  \item \aporte{\dirRepo}{naD GarRaz \github}
\end{aportes}
