\begin{enunciado}{\ejercicio}
  Resolver los siguientes sistemas de ecuaciones lineales no homogéneos y los sistemas homogéneos asosciados en $\reales$ o en $\complejos$.
  Si la solución única, puede verificarse el resultado en \texttt{Python} utilizando el comando \texttt{np.lianlg.solve}.
  \begin{enumerate}[label=(\alph*)]
    \begin{multicols}{2}
      \item $
        \llave{ccc}{
          x_1 + x_2 - 2x_3 + x_4 & = & -2\\
          3x_1 - 2x_2 + x_3 + 5x_4 & = & 3\\
          x_1 - x_2 + x_3 + 2x_4 & = & 2
        }
      $

      \item $
        \llave{ccc}{
          x_1 + x_2 + x_3 - 2x_4 + x_5 & = & 1\\
          x_1 - 3x_2 + x_3 + x_4 + x_5 & = & 0\\
          3x_1 - 5x_2 + 3x_3 + 3x_5 & = & 0
        }
      $

      \item $
        \llave{ccc}{
          ix_1 - (1 + i)x_2 & = & -1\\
          x_1 - 2x_2 + x_3  & = & 0\\
          x_1 + 2ix_2 - x_3 & = & 2i
        }
      $

      \item $\llave{ccc}{
          2x_1 + (-1 + i)x_2 + x_4 & = & 2\\
          -x_1 + 3x_2 - 3ix_3 + 5x_4 & = & 1
        }
      $
    \end{multicols}
  \end{enumerate}
\end{enunciado}

%%
%% \codigoPython{ej-1/codigo1-1.py}
%%

\begin{enumerate}[label=(\alph*)]
  \item  Sistema con más incógnitas que ecuaciones, así que lo de la solución única, bien gracias. En forma matricial para hacer
        la gracia de triangular y coso:
        $$
          \begin{array}{ccc}
            \matriz{cccc|c}{
            1 & 1   & - 2          & 1            & -2           \\
            3 & - 2 & 1            & 5            & 3            \\
            1 & - 1 & 1            & 2            & 2
            }
              &
            \triangulacion{
            F_2 - 3F_1 \to F_2                                   \\
              F_3 - F_1 \to F_3
            }
              &
            \matriz{cccc|c}{
            1 & 1   & - 2          & 1            & -2           \\
            0 & -5  & 5            & 2            & 9            \\
            0 & -2  & 3            & 1            & 4
            }                                                    \\
              &
            \triangulacion{
            -\frac{1}{5} \cdot F_2 \to F_2                       \\
            }
              &
            \matriz{cccc|c}{
            1 & 1   & - 2          & 1            & -2           \\
            0 & 1   & -\frac{7}{5} & -\frac{2}{5} & -\frac{9}{5} \\
            0 & -2  & 3            & 1            & 4
            }                                                    \\
              &
            \triangulacion{
            F_3 + 2F_2 \to F_3                                   \\
            }
              &
            \matriz{cccc|c}{
            1 & 1   & - 2          & 1            & -2           \\
            0 & 1   & -\frac{7}{5} & -\frac{2}{5} & -\frac{9}{5} \\
            0 & 0   & \frac{1}{5}  & \frac{1}{5}  & \frac{2}{5}
            }
          \end{array}
        $$
        Cosa de lo más espantosa. Empiezo a escribir las ecuaciones:
        $$
          \llave{rclcl}{
            x_1 + x_2 - 2x_3 + x_4 & = & -2
            & \Sii{$\llamada2 \text{ y } \llamada1$} &
            x_1 = -2x_4 + 1 \\
            x_2 - \frac{7}{5}x_3 - \frac{2}{5}x_4 & = & -\frac{9}{5}
            & \Sii{$\llamada1$} &
            x_2 \igual{$\llamada2$} -x_4 +1\\
            \frac{1}{5}x_3 + \frac{1}{5}x_4 & = & \frac{2}{5}
            & \sii &
            x_3 \igual{$\llamada1$} -x_4 + 2
          }
        $$
        Yo estaba buscando algo de la pinta  $x^T = (x_1, x_2, x_3, x_4)$:
        $$
          \cajaResultado{
            x = \matriz{c}{
              x_1\\
              x_2\\
              x_3\\
              x_4
            }
            =
            \matriz{c}{
              -2x_4 + 1 \\
              -x_4 +1\\
              -x_4 + 2\\
              -x_4
            }
            =
            x_4 \cdot
            \matriz{c}{
              -2\\
              -1 \\
              -1\\
              1
            }
            +
            \matriz{c}{
              1\\
              1\\
              2\\
              0
            }
          }
        $$

  \item \hacer

  \item
        Hermosos y molestos números complejos. Acá probablemente se use mucho lo de $\times \ytext \div$
        por el conjugado mucho para sacar números con parte imaginaria del denominador, quiero decir:
        $$
          \frac{1}{z} \cdot \frac{\blue{\conj{z}}}{\blue{\conj{z}}} = \frac{\conj{z}}{|z|^2}
          \quad
          \Entonces{$z = a + ib$}
          \quad
          \frac{1}{z} = \frac{a -ib}{a^2 + b^2}
        $$

        $$
          \llave{ccc}{
            ix_1 - (1 + i)x_2 & = & -1\\
            x_1 - 2x_2 + x_3  & = & 0\\
            x_1 + 2ix_2 - x_3 & = & 2i
          }
        $$
        Escrito en forma matricial:
        $$
          \begin{array}{ccl}
            \matriz{ccc|c}{
            i & - 1 - i & 0  & -1 \\
            1 & - 2     & 1  & 0  \\
            1 & 2i      & -1 & 2i
            }
              &
            \triangulacion{
              \frac{1}{i} F_1 \to F_1
            }
              &
            \matriz{ccc|c}{
            1 & - 1 + i & 0  & i  \\
            1 & - 2     & 1  & 0  \\
            1 & 2i      & -1 & 2i
            }
            \\
            \\
              &
            \triangulacion{
            F_2 \to F_1           \\
              F_3 \to F_1
            }
              &
            \matriz{ccc|c}{
            1 & - 1 + i & 0  & i  \\
            0 & -1 - i  & 1  & -i \\
            0 & 1 + i   & -1 & i
            }
            \\
            \\
              &
            \triangulacion{
              F_2 + F_3 \to F_3
            }
              &
            \matriz{ccc|c}{
            1 & - 1 + i & 0  & i  \\
            0 & -1 - i  & 1  & -i \\\rowcolor{red!10}
            0 & 0       & 0  & 0
            }
          \end{array}
        $$
        Tuqui. Paso a sistema y resuelvo:
        $$
          \llave{ccccl}{
            x_1  + (- 1 + i)x_2  & = & i & \to & x_1 = i+ (1 - i)x_2\\
            -(1 + i)x_2  + x_3  & = & -i & \to & x_3 = -i + (1 + i)x_2
          }
        $$
        Yo estaba buscando algo de la pinta  $x^T = (x_1, x_2, x_3)$:
        $$
          \cajaResultado{
            x = \matriz{c}{
              x_1\\
              x_2\\
              x_3
            }
            =
            \matriz{c}{
              i+  (1-i) \cdot x_2 \\
              x_2\\
              -i + (1 + i)x_2
            }
            =
            x_2 \cdot
            \matriz{c}{
              1 - i\\
              1   \\
              1 + i
            }
            +
            \matriz{c}{
              i\\
              0 \\
              -i
            }
          }
        $$

  \item \hacer
\end{enumerate}

% Contribuciones
\begin{aportes}
  \item \aporte{\dirRepo}{naD GarRaz \github}
\end{aportes}
