\begin{enunciado}{\ejercicio}
  Decidir si los siguientes conjuntos son linealmente independientes sobre $K$. Cuando no lo sean,
  escribir a uno de ellos como combinación lineal de los otros.
  \begin{enumerate}[label=(\alph*)]
    \item $\set{(1,4,-1,3), (2,1,-3,1),(0,2,1,-5)} \en \reales^4$, para $K = \reales$.
    \item $\set{(1 - i, i), (2,-1 + i)} \en \complejos^2$, para $K = \complejos$.
  \end{enumerate}
\end{enunciado}

Un poco de teoría \rollingEyes:

\textit{\ul{Combinación lineal:}}

Sea $V$ un $K-$espacio vectorial, y sea $G = \set{v_1,\ldots,v_r} \subseteq V$.
Una \textit{combinación lineal de $G$} es un elemento $v \en V$ tal que $v = \sumatoria{i=1}{r} \alpha_i \cdot v_i$ con
$\alpha_i \en K$ para cada $1 \leq i \leq r$.

\textit{\ul{Independencia lineal:}}
Sea $V$ un $K-$espacio vectorial y sea $\set{v_\alpha}_{\alpha \en I}$ una familia de vectores de $V$. Se
dice que  $\set{v_{\alpha}}_{\alpha \en I}$ es \textit{linealmente independiente} (l.i.) si
$$
  \sumatoria{\alpha \en I}{} a_{\alpha} \cdot v_{\alpha} = 0 \entonces a_{\alpha} = 0 \paratodo \alpha \en I
$$

\bigskip

Todo muy lindo.

\begin{enumerate}[label=(\alph*)]
  \item
        $$
          \begin{array}{c}
            a \cdot (1,4,-1,3) + b \cdot (2,1,-3,1) + c\cdot (0,2,1,-5) = 0 \\
            \matriz{ccc}{
            1  & 2  & 0                                                     \\
            4  & 1  & 2                                                     \\
            -1 & -3 & 1                                                     \\
            3  & 1  & -5
            }
            \cdot
            \matriz{c}{
            a                                                               \\
            b                                                               \\
              c
            }
            =
            \matriz{c}{
            0                                                               \\
            0                                                               \\
            0                                                               \\
              0
            }
            \flecha{\magic}[triangulando]
            \matriz{ccc|c}{
            1  & 2  & 0  & 0                                                \\
            0  & 1  & -1 & 0                                                \\
            0  & 0  & 1  & 0                                                \\
            0  & 0  & 0  & 0
            }
            \to
            \cajaResultado{
              \llave{l}{
            a = 0                                                           \\
            b = 0                                                           \\
                c = 0
              }
            }
          \end{array}
        $$
        Los vectores son \textit{linealmente independientes}.

  \item Ahora los coeficientes $\alpha$ y $\beta \en \complejos$
        $$
          \begin{array}{c}
            \alpha \cdot (1-i,i) + \beta \cdot (2,-1 + i)= 0 \\
            \matriz{cc}{
            1-i & 2                                          \\
            i   & -1 + i                                     \\
            }
            \cdot
            \matriz{c}{
            \alpha                                           \\
              \beta
            }
            =
            \matriz{c}{
            0                                                \\
              0
            }                                                \\
            \triangulacion{
              \frac{1}{1-i} \cdot F_1 \to F_1}
            \matriz{cc|c}{
            1   & 1+i    & 0                                 \\
            i   & -1 + i & 0                                 \\
            }
            \triangulacion{
              F_2 - i\cdot F_1 \to F_2
            }
            \matriz{cc|c}{
            1   & 1+i    & 0                                 \\
            0   & 0      & 0                                 \\
            }
            \to
            \cajaResultado{
              \llave{l}{
            \alpha = (1+i)\beta                              \\
              }
            }
          \end{array}
        $$

        Y estos bichos \ul{no} serían \textit{linealmente independientes}.
\end{enumerate}

\begin{aportes}
  \item \aporte{\dirRepo}{naD GarRaz \github}
\end{aportes}
