\begin{enunciado}{\ejercicio}
  Decidir si los siguientes conjuntos son linealmente independientes sobre $K$. Cuando no lo sean,
  escribir a uno de ellos como combinación lineal de los otros.
  \begin{enumerate}[label=\alph*)]
    \item $\set{(1,4,-1,3), (2,1,-3,1),(0,2,1,-5)} \en \reales^4$, para $K = \reales$.
    \item $\set{(1 - i, i), (2,-1 + i)} \en \complejos^2$, para $K = \complejos$.
  \end{enumerate}
\end{enunciado}

Un poco de teoría \rollingEyes:

\textit{Combinación lineal:}

Sea $V$ un $K-$espacio vectorial, y sea $G = \set{v_1,\ldots,v_r} \subseteq V$.
Una \textit{combinación lineal de $G$} es un elemento $v \en V$ tal que $v = \sumatoria{i=1}{r} \alpha_i \cdot v_i$ con
$\alpha_i \en K$ para cada $1 \leq i \leq r$.

\textit{Independecia lineal:}
Sea $V$ un $K-$espacio vectorial y sea $\set{v_\alpha}_{\alpha \en I}$ una familia de vectores de $V$. Se
dice que  $\set{v_{\alpha}}_{\alpha \en I}$ es \textit{linealmente independiente} (l.i.) si
$$
  \sumatoria{\alpha \en I}{} a_{\alpha} \cdot v_{\alpha} = 0 \entonces a_{\alpha} = 0 \paratodo \alpha \en I
$$

\bigskip

Todo muy lindo.

\bigskip

$$
  a \cdot (1,4,-1,3) + b \cdot (2,1,-3,1) + c\cdot (0,2,1,-5) = 0
  \flecha{\magic}
  \matriz{ccc|c}{
    1 & 2 & 0 & 0 \\
    4 & 1 & 2 & 0 \\
    -1 & -3 & 1 & 0 \\
    3 & 1 & -5 & 0
  }
$$
