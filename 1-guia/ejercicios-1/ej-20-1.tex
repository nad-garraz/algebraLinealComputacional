\begin{enunciado}{\ejercicio}
  Sean $A, B, C, D \en K^{n \times n}$ y $M \en K^{2n \times 2n}$ la matriz de bloques
  $$
    M =
    \matriz{cc}{
      A & B \\
      C & D \\
    }.
  $$
  Probar que si $A$ es inversible, entonces
  \begin{enumerate}[label=(\alph*)]
    \item $M =
            \matriz{cc}{
              A & 0 \\
              C & I
            }
            \cdot
            \matriz{cc}{
              I & A^{-1} B \\
              0 & D - CA^{-1} B
            }.$

    \item $\det(M) = \det(AD - ACA^{-1}B).$ Concluir que si $AC = CA$, $\det(M) = \det(AD - CB)$.
  \end{enumerate}
\end{enunciado}

\begin{enumerate}[label=(\alph*)]
  \item\label{ej-20:item-a} La hipótesis es que $A$ es inversible, es decir que $\existe A^{-1}$ tal que $A \cdot A^{-1} = I$
        $$
          \matriz{cc}{
            A & 0 \\
            C & I
          }
          \cdot
          \matriz{cc}{
            I & A^{-1} B \\
            0 & D - C A^{-1} B
          }
          =
          \matriz{cc}{
            A \cdot I + 0 & A A^{-1} B + 0 \\
            C \cdot I + 0 & C A^{-1} B + D - C A^{-1} B
          }
          =
          \matriz{cc}{
            A & B \\
            C &  D
          }.
        $$

  \item Quiero ver que:
        $$
          \det(M) = \det(A D - A C A^{-1} B)
        $$
        por propiedad de determinantes $\llamada1$:
        $$
          \det(A D - A C A^{-1} B)
          \igual{\red{!}}
          \det(A) \cdot \det(D - C A^{-1} B)
        $$
        Por el apunte de bloques:
        Si la matriz está "diagonal" en bloques $\llamada2$:
        $$
          \deter{cc}{
            A & B \\
            0 & C
          } = \det(A) \det(C)
          \ytext
          \deter{cc}{
            A & 0 \\
            C & D
          }
          = \det(A) \det(D)
        $$
        Veamos si puedo armar la expresión que quiero partiendo del determinante de $M$ usando las propiedades.
        Escribo a $M$ usando la factorizacion del ítem \ref{ej-20:item-a}:
        $$
          \begin{array}{rcl}
            \det(M) & =                   &
            \det
            \parentesis{
              \matriz{cc}{
            A       & 0                              \\
            C       & I
              }
              \cdot
              \matriz{cc}{
            I       & A^{-1} B                       \\
            0       & D - C A^{-1} B
              }
            }                                        \\
                    & \igual{$\llamada1$} &
            \deter{cc}{
            A       & 0                              \\
            C       & I
            }
            \cdot
            \deter{cc}{
            I       & A^{-1} B                       \\
            0       & D - C A^{-1} B
            }                                        \\
                    & \igual{$\llamada2$} &
            \det(A I) \cdot \det(I (D - C A^{-1} B)) \\
                    & =                   &
            \det(A) \cdot \det(D - C A^{-1} B)       \\
                    & =                   &
            \det(AD - A C A^{-1} B)                  \\
          \end{array}
        $$
        Con esa última expresión, si $AC = CA$:
        $$
          \det(AD - \blue{A C} A^{-1} B)
          \igual{\red{!}}
          \det(AD - \blue{C A} A^{-1} B) =
          \det(AD - CB)
        $$

\end{enumerate}

\begin{aportes}
  \item \aporte{\dirRepo}{naD GarRaz \github}
  \item \aporte{https://github.com/juandelia03}{Juan D Elia \github}
\end{aportes}
