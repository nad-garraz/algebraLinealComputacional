\begin{enunciado}{\ejercicio}
  Sean $A, B, C, D \en K^{n \times n}$ y $M \en K^{2n \times 2n}$ la matriz de bloques
  $$
    M =
    \matriz{cc}{
      A & B \\
      C & D \\
    }.
  $$
  Probar que si $A$ es inversible, entonces
  \begin{enumerate}[label=(\alph*)]
    \item $M =
            \matriz{cc}{
              A & 0 \\
              C & I
            }
            \cdot
            \matriz{cc}{
              I & A^{-1} B \\
              0 & D - CA^{-1} B
            }.$

    \item $\det(M) = \det(AD - ACA^{-1}B).$ Concluir que si $AC = CA$, $\det(M) = \det(AD - CB)$.
  \end{enumerate}
\end{enunciado}

\begin{enumerate}[label=(\alph*)]
  \item La hipótesis es que $A$ es inversible, es decir que $\existe A^{-1}$ tal que $A \cdot A^{-1} = I$
        $$
          \matriz{cc}{
            A & 0 \\
            C & I
          }
          \cdot
          \matriz{cc}{
            I & A^{-1} B \\
            0 & D - C A^{-1} B
          }
          =
          \matriz{cc}{
            A \cdot I + 0 & A A^{-1} B + 0 \\
            C \cdot I + 0 & C A^{-1} B + D - C A^{-1} B
          }
          =
          \matriz{cc}{
            A & B \\
            C &  D
          }.
        $$

  \item Quiero ver que
  \[
  \det(M) 
  = \det(A D - A C A^{-1} B)
  \]
  por propiedad de determinantes:
  \[
  \det(A D - A C A^{-1} B) = \det(A) \det(D - C A^{-1} B)
  \]
  
  
  Por el apunte de bloques:
  
  Si la matriz tiene una estructura de la forma:
  
  \[
  \det \begin{bmatrix} A & B \\ 0 & C \end{bmatrix} = \det(A) \det(C)
  \]
  
  
  También se cumple que:
  
  \[
  \det \begin{bmatrix} A & 0 \\ C & D \end{bmatrix} = \det(A) \det(D)
  \]
  
  Veamos si puedo armar la expresion que quiero partiendo del determinante de M usando las propiedades. Escribo a M usando la factorizacion del punto a:
  
  \[
  \det(M) = \det \left( \begin{bmatrix} A & 0 \\ C & I \end{bmatrix}
  \begin{bmatrix} I & A^{-1} B \\ 0 & D - C A^{-1} B \end{bmatrix} \right)
  \] 
  
  Aplicando la propiedad de determinantes:
  
  \[
  = \det \begin{bmatrix} A & 0 \\ C & I \end{bmatrix} \cdot \det \begin{bmatrix} I & A^{-1} B \\ 0 & D - C A^{-1} B \end{bmatrix}
  \]
  
  Usando la propiedad antes mencionada:
  
  \[
  = \det(A I) \cdot \det(I (D - C A^{-1} B))
  = \det(A) \cdot \det(D - C A^{-1} B)
  \]
  
  Que es lo que queríamos demostrar.

\end{enumerate}

\begin{aportes}
  \item \aporte{\dirRepo}{naD GarRaz \github}
  \item \aporte{https://github.com/juandelia03}{Juan D Elia \github} 
\end{aportes}
