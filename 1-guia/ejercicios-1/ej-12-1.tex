\begin{enunciado}{\ejercicio}
  Sean $v_1, \ldots, v_k \en \reales^n$. Probar que $\set{v_1, \ldots, v_k}$ es linealmente independiente sobre
  $\reales$ si y solo si $\set{v_1,\ldots,v_k}$ es linealmente independiente sobre $\complejos$.
\end{enunciado}

Es un \textit{si solo si} así que sale doble implicación:

\begin{itemize}
  \item[(\red{$\Leftarrow$})] Para este lado sale un poco más fácil, por eso arranco por acá.

        Sé que por independencia lineal:
        $$
          \sumatoria{i = 1}{k} z_i \cdot v_i = 0
          \quad
          \text{ con }
          z_i \en \complejos \ytext z_i = 0 \text{ para }  1 \leq i \leq k
        $$
        Quiero probar que:
        $$
          \sumatoria{i = 1}{k} r_i \cdot v_i = 0
          \quad
          \text{ con }
          r_i \en \reales \ytext r_i = 0 \text{ para }  1 \leq i \leq k
        $$
        Es inmediato ver que en este caso a pesar de que los coeficientes $z_i$, valen todos 0,
        es decir que particularmente son reales también\red{!} Puedo tomar $r_i = z_i$ y listo, tengo la combinación lineal
        igualada a cero y todos los coeficientes son reales y nulos.

  \item[(\red{$\Rightarrow$})]
        Este es un poco más picante, porque no es \textit{obvio} que deba ocurrir ¿O no lo es para mí?:
        Sé que por independencia lineal:
        $$
          \sumatoria{i = 1}{k} r_i \cdot v_i \igual{$\llamada1$} 0
          \quad
          \text{ con }
          r_i \en \reales \ytext r_i = 0 \text{ para }  1 \leq i \leq k
        $$
        Quiero probar que:
        $$
          \sumatoria{i = 1}{k} z_i \cdot v_i = 0 \ \llamada2
          \quad
          \text{ con }
          z_i \en \complejos \ytext z_i = 0 \text{ para }  1 \leq i \leq k
        $$
        Laburo un poco $\llamada2$:
        $$
          \begin{array}{c}
            \sumatoria{i = 1}{k} z_i \cdot v_i =
            \blue{z_1} \cdot v_1 + \cdots + \blue{z_k} \cdot v_k = 0                     \\
            \Sii{\red{!!}}[$z_j = a_j + i b_j$]                                          \\
            \blue{(a_1 + i b_1)} \cdot v_1 + \cdots + \blue{(a_k + i b_k)} \cdot v_k = 0 \\
            \Sii{\red{!!}}[$v_j \en \reales^n$]                                          \\
            \ub{(a_1 \cdot v_1 + \cdots + a_k \cdot v_k)}{\llamada3} +
            i
            \ub{(b_1 \cdot v_1 + \cdots + b_k \cdot v_k)}{\llamada4} = 0 + i0
          \end{array}
        $$
        Para que esa igualdad se cumpla debe ocurrir que las combinetas en $\llamada3$ y $\llamada4$ sean 0. Para que esas combinaciones
        sean 0 sí o sí los coeficientes $a_i \ytext b_i$ deben ser todos nulos porque \ul{son reales y los $v_i$ son \textit{linealmente independientes} sobre
                $\reales \llamada1$}.
        Y como
        $$
          z_i = \ua{a_i}{=0} + i \ua{b_i}{=0}
          \entonces
          \sumatoria{i = 1}{k} \ua{z_i}{=0} \cdot v_i = 0 \entonces  \set{1,\ldots,v_k} \text{ son \textit{linealmente independientes} sobre } \complejos.
        $$

        \bigskip

        \textit{Nota que puede ser de interés:}

        Mirá que ese último \red{!!} es porque los $v_j \en \reales$, porque si estuvieran en $\complejos$, por ejemplo:
        $$
          \set{(i,1), (1,-i)}
        $$
        Esos $v_j \en \complejos$ si laburás con $K = \reales$ son {\large{MEGA}} \textit{linealmente independientes},
        peeeero si $K = \complejos$:
        $$
          i \cdot (i, 1) + 1 \cdot (1,-i) = 0
        $$
        todo lo contrario. Solo se llega a las expresiones $\llamada3$ y $\llamada4$ gracias a que $v_j \en \reales^n$.

        \textit{Fin de Nota que puede ser de interés:}
\end{itemize}

\begin{aportes}
  \item \aporte{\dirRepo}{naD GarRaz \github}
\end{aportes}
