\begin{enunciado}{\ejercicio}
  Sean $S$ y $T$ subespacios de un $K$-espacio vectorial $V$. Probar que $S\union T$ es un subespacio
  de $V$ si y solo si $S \subseteq T$ o $T \subseteq S$.
\end{enunciado}

Hay que probar que:
$$
  S \union T \text{ es subespacio }
  \sii
  S \subseteq T \o T \subseteq S
$$

Para demostrar este ejercicio voy a necesitar demostrar ambas implicaciones

$$ S \subseteq T \lor T \subseteq S \implies S \union T \text{ es subespacio } $$
\begin{enumerate}
  \item Caso $S = T \text{subespacio}\implies S \union T \text{ es subespacio }$ 
    
        \begin{align}
          & S = T \implies S \union T \text{ es subespacio }    \\
          & S = T \implies S \union S \text{ es subespacio }    \\
          & S = T \implies S \text{ es subespacio }             \\
          & \text{which is true}
        \end{align}

    \item Caso $S \subseteq T \land S \nsubseteq T \text{subespacio}\implies S \union T \text{ es subespacio }$ 

    \begin{align}
      & S \subseteq T \land S \nsubseteq T \implies S \union T \text{ es subespacio }    \\
      & S \subseteq T \land S \nsubseteq T \implies T \text{ es subespacio }             \\
      & \text{which is true}
    \end{align}

    \item Es lo mismo que con el anterior caso

    $ \implies $ Como probamos todos los casos posibles de combinación entre S y T que cumplía los requisitos,
    probamos que es verdadero la afirmación
    
    \\

    Ahora vamos con la vuelta

    $\Longleftarrow $ $S \union T \text{ es subespacio } \implies S \subseteq T \lor T \subseteq S \text{subespacio} $ 

    A probarlo por contrareciproco: \\

    Asumo que $ S \nsubseteq T \land S \nsubseteq T $

    Sea $t_{0}$ un vector tal que $t_{0} \in T \land t_{0} \notin S$
     y $s_{0}$ un vector tal que $s_{0} \in S \land s_{0} \notin T$

     Ahora, el conjunto $ S \union T $ se define como: $ \{ \forall v \in S \union T | v \in S \lor v \in T \} $ \red {Revisar que es correcto esta forma de escribirlo}

    Sumo $s_{0} + t_{0}$

    Esto, para que sea un subespacio debería estar dentro del mismo. Pero, por definición de $ S \union T $
    no es posible esto, al ser solo la unión. 

    $ S \union T $ une solo los 2 espacios vectoriales, pero no los relaciona de modo que 
    no existe ninguna combinación lineal que sea $ s_{0} + t_{0} \in S \union T$ a menos que consideremos que uno está
    contenido en otro. Entonces llegamos al absurdo, demostrando que si es subespacio vectorial, entonces alguno de los
    2 subespacios esta contenido dentro de otro

  \end{enumerate}
}
