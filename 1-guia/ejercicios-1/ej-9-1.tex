\begin{enunciado}{\ejercicio}
  Sean $S$ y $T$ subespacios de un $K$-espacio vectorial $V$. Probar que $S\union T$ es un subespacio
  de $V$ si y solo si $S \subseteq T$ o $T \subseteq S$.
\end{enunciado}

Si bien la tentación de pensar que
$$
  \cancel{S \union T = S + T}
$$ es grande, eso no es en general así. $S+T$ es el subespacio que tiene a todos los elementos de la pinta $s_i + t_j$.
Mientras que $S \union T$ es un conjunto con los elementos de $S$ y los elementos de $T$, no necesariamente está ahí el elemento $s_i + t_j$.

\medskip

\textit{
  Intuitivamente lo que este ejercicio sugiere, es que para que $S \union T$ sea un subespacio, $S$ no debería aportarle nada nuevo a $T$ y viceversa,
  para que al sumar 2 elementos de $S \union T$, eso seguro caiga dentro dentro del mismo conjunto, porque estarías sumando
  dos elementos de $S$ o dos de $T$. Ponele.
}

\begin{itemize}
  \item[(\red{$\Leftarrow$})]
        Esta sale más directa. Si:
        $$
          \begin{array}{c}
            S \subseteq T \entonces S \union T = T \\
            \otext                                 \\
            T \subseteq S \entonces S \union T = S
          \end{array}
        $$
        La explicación es análoga para las dos implicaciones. En el caso $S \subseteq T$,
        los elementos  de $S$ no van a aportar nueva información al subespacio.
        Si $T$ es un subespacio por hipótesis de enunciado y $S \union T = T$, listo.

  \item[(\red{$\Rightarrow$})]
        Vamos por el absurdo. Es decir que voy a probar \blue{$ p \entonces \neg q$} y voy a llegar a una contradicción.

        Supongamos que:
        $$
          T \not\subseteq S \text{ y que } S \not\subseteq T
          \entonces
          \existe s_1,\, t_1 \text{ tal que }
          \llamada1
          \llave{l}{
            s_1 \en S \ytext s_1 \not\en T\\
            t_1 \en T \ytext t_1 \not\en S
          }
        $$

        \underline{\red{¡}Por hipótesis $S \union T$ es un subespacio\red{!}} Por lo que:

        $$
          \llave{l}{
            s_1 \en S  \Entonces{$S \union T$ es}[subespacio] s_1 \en S \union T \\
            t_1 \en T  \Entonces{$S \union T$ es}[subespacio] t_1 \en S \union T
          }
          \entonces s_1 + t_1 \en S \union T,
        $$
        El subespacio $S \union T$ tiene elementos que pertenecen a $T$ o elementos que pertenecen a $S$, de forma tal que:
        $$
          s_1 + t_1 \taa{\llamada2}{}{\en} S \otext s_1 + t_1 \taa{\llamada3}{}{\en} T
        $$
        En el caso $\llamada2$ tengo:

        Dado que \underline{$S$ es un subespacio}, si $s_1 \en S
          \Entonces{también}
          -s_1 \en S $:
        $$
          s_1 + t_1 + (-s_1) \ua{\en}{\text{def. subespacio}} S \sisolosi s_1 + t_1 + (-s_1) = \red{t_1 \en S}
        $$
        Contradiciendo lo que se supuso en $\llamada1$. Llegar a la contradicción de que $s_1 \en T$ es análoga.

\end{itemize}

\begin{aportes}
  \item \aporte{https://github.com/fmancilla00}{Fede Mancilla \github}
  \item \aporte{https://github.com/misProyectosPropios}{Iñaki Frutos \github}
  \item \aporte{\dirRepo}{naD GarRaz \github}
\end{aportes}
