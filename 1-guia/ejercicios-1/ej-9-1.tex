\begin{enunciado}{\ejercicio}
  Sean $S$ y $T$ subespacios de un $K$-espacio vectorial $V$. Probar que $S\union T$ es un subespacio
  de $V$ si y solo si $S \subseteq T$ o $T \subseteq S$.
\end{enunciado}

\parrafoDestacado[\red{\atencion}]{
  Si bien la tentación de pensar que
  $$
    S \union T = S + T,
  $$
  \ul{eso no es en general así}.

  $S+T$ es el subespacio que tiene a todos los elementos de la pinta $s_i + t_j$.

  $S \union T$ es un conjunto con los elementos de $S$ y los elementos de $T$, no necesariamente está ahí el elemento $s_i + t_j$.

  Lo que si sabés es que:
  $$
    S \union T \subseteq S+T
  $$
}

\bigskip

\parrafoDestacado{
  \textit{
    Intuitivamente lo que este ejercicio sugiere, es que para que $S \union T$ sea un subespacio, $S$ no debería aportarle nada nuevo a $T$ y viceversa,
    para que al sumar 2 elementos de $S \union T$, eso seguro caiga dentro dentro del mismo conjunto, porque estarías sumando
    dos elementos de $S$ o dos de $T$. Ponele.
  }
}

\bigskip

Para probar la \textit{doble implicación}, pruebo la ida y la vuelta. Arranco por la que parece más fácil:

\begin{itemize}
  \item[(\red{$\Leftarrow$})]
        Esta sale más directa.

        Si:
        $$
          \begin{array}{c}
            S \subseteq T \entonces S \union T = T \\
            \otext                                 \\
            T \subseteq S \entonces S \union T = S
          \end{array}
        $$
        La explicación es análoga para las dos implicaciones. En el caso $S \subseteq T$,
        los elementos  de $S$ no van a aportar nueva información al subespacio.
        Si $T$ es un subespacio por hipótesis de enunciado y $S \union T = T$, listo.

  \item[(\red{$\Rightarrow$})]
        Está es medio \href{\chinito}{chino}:

        Vamos por el absurdo, en vez de laburar con $p \entonces q$, laburo con \blue{$ p \entonces \neg q$} llegando a una contradicción.

\medskip

        Supongamos que:
        $$
          T \not\subseteq S \text{ y que } S \not\subseteq T
          \entonces
          \existe s_1,\, t_1 \text{ tal que }
          \llamada1
          \llave{l}{
            s_1 \en S \ytext s_1 \not\en T\\
            t_1 \en T \ytext t_1 \not\en S
          }
        $$

        \underline{\red{¡}Por hipótesis $S \union T$ es un subespacio\red{!}} Por lo que:

        $$
          \llave{l}{
            s_1 \en S  \Entonces{$S \union T$ es}[subespacio] s_1 \en S \union T \\
            t_1 \en T  \Entonces{$S \union T$ es}[subespacio] t_1 \en S \union T
          }
          \entonces
          \ub{
            s_1 + t_1 \en S \union T
          }{\text{esto no tiene}\\\text{nada que ver}\\\text{con } S + T},
        $$
        El subespacio $S \union T$ tiene elementos que pertenecen a $T$ o elementos que pertenecen a $S$, de forma tal que
        que sumar 2 elementos \ul{cualesquiera} también están en $S \union T$, en \textit{particular} esos de $\llamada1$:
        $$
          \ub{s_1 + t_1}{\en S \union T} \taa{\llamada2}{}{\en} S ~\lor~ \ub{s_1 + t_1}{\en S \union T} \taa{\llamada3}{}{\en} T
        $$
        En el caso $\llamada2$ tengo:

        Dado que \underline{$S$ es un subespacio}, si $s_1 \en S$, entonces también $-s_1 \en S$, y nuevamente la suma de 2 elementos
        \ul{cualesquiera} de $S$ también estarán en $S$:
        $$
          \ob{s_1 + t_1 + (-s_1)}{\en S \union T} \ua{\en}{\text{def. subespacio}} S
          \sisolosi
          \red{t_1 \en S}
        $$
        Contradiciendo lo que se supuso en $\llamada1$, donde $t_1 \en T$, pero $t_1 \not\en{S}$. Llegar a la contradicción de que $s_1 \en T$ es análoga.

\end{itemize}

\begin{aportes}
  \item \aporte{https://github.com/fmancilla00}{Fede Mancilla \github}
  \item \aporte{https://github.com/misProyectosPropios}{Iñaki Frutos \github}
  \item \aporte{\dirRepo}{naD GarRaz \github}
\end{aportes}
