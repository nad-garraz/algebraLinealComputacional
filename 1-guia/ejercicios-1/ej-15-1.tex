\begin{enunciado}{\ejercicio}
  Dadas las bases de $\reales^3$, $B = \set{(1,1,0); (0,1,1); (1,0,1)}$ y $B'= \set{(-1,1,1);(2,0,1);(1,-1,3)}$
  \begin{enumerate}[label=(\alph*)]
    \item Calcular $[(1,1,0)]_B$
          \ytext  $[(1,1,0)]_B'$.

    \item Calcular la matiz de cambio de base $C(B,B')$.

    \item Comprobar que $C(B, B')[(1,1,0)]_B  = [(1,1,0)]_{B'}$.
  \end{enumerate}
\end{enunciado}

\begin{enumerate}[label=(\alph*)]
  \item
        Para calcular las coordenadas en una base $B$:
        $$
          (1,1,0) = a (1,1,0) + b(0,1,1) + c(1,0,1)
          \Sii{a ojímetro}
          \llave{l}{
            a = 1 \\
            b = 0 \\
            c = 0
          }
          \entonces
          \cajaResultado{
            [(1,1,0)]_B = (1,0,0)
          }
        $$

        En la base $B'$ voy a tener que hacer más cuentas:
        $$
          \begin{array}{c}
            (1,1,0) = a (-1,1,1) + b(2,0,1) + c(1,-1,3) \\
            \matriz{ccc|c}{
            -1 & 2 & 1            & 1                   \\
            1  & 0 & -1           & 1                   \\
            1  & 1 & 3            & 0
            }
            \Sii{\magic}
            \llave{ccc}{
            a  & = & \frac{1}{2}                        \\
            b  & = & 1                                  \\
            c  & = & -\frac{1}{2}
            }
            \entonces
            \cajaResultado{
              [(1,1,0)]_{B'} = (\frac{1}{2},2,-\frac{1}{2})
            }
          \end{array}
        $$

        \copyPaste
        \codigoPython{ej-15/codigo15-1.py}

  \item Quiero la matriz que tiene por columnas a las coordenas de los generadores de $B$ en la base $B'$:
        $$
          \begin{array}{c}
            C(B,B') =
            \matriz{ccc}{
            a_1           & a_2 & a_3                                                                                   \\
            b_1           & b_2 & b_3                                                                                   \\
            c_1           & c_2 & c_3
            }
            \Entonces{donde}
            \llave{ccc}{
            (1,1,0)       & =   & a_1 (-1,1,1) + b_1 (2,0,1) + c_1 (1,-1,3)                                             \\
            (0,1,1)       & =   & a_2 (-1,1,1) + b_2 (2,0,1) + c_2 (1,-1,3)                                             \\
            (1,0,1)       & =   & a_3 (-1,1,1) + b_3 (2,0,1) + c_3 (1,-1,3)
            }                                                                                                           \\
            \matriz{ccc|c|c|c}{
            -1            & 2   & 1                                         & 1            & 0            & 1           \\
            1             & 0   & -1                                        & 1            & 1            & 0           \\
            1             & 1   & 3                                         & 0            & 1            & 1
            }
            \Sii{\magic}
            \matriz{ccc|c|c|c}{
            1             & 0   & 0                                         & \frac{1}{2}  & \frac{7}{8}  & \frac{1}{8} \\
            0             & 1   & 0                                         & 1            & \frac{1}{2}  & \frac{1}{2} \\
            0             & 0   & 1                                         & -\frac{1}{2} & -\frac{1}{8} & \frac{1}{8}
            }
            \sii
            \llave{ccc}{
            (a_1,b_1,c_1) & =   & (\frac{1}{2} ,1,-\frac{1}{2})                                                         \\
            (a_2,b_2,c_2) & =   & (\frac{7}{8} ,\frac{1}{2},-\frac{1}{8})                                               \\
            (a_3,b_3,c_3) & =   & (\frac{1}{8} ,\frac{1}{2},\frac{1}{8})
            }
          \end{array}
        $$

        \copyPaste

        \codigoPython{ej-15/codigo15-2.py}

        Finalmente la matriz $C(B,B')$:
        $$
          \cajaResultado{
            C(B, B') =
            \matriz{ccc}{
              \frac{1}{2}  & \frac{7}{8}  & \frac{1}{8}\\
              1            & \frac{1}{2}  & \frac{1}{2}\\
              -\frac{1}{2} & -\frac{1}{8} & \frac{1}{8}
            }
          }
        $$

  \item Lo que hay que hacer es :
        $$
          C(B, B')
          \cdot
          \matriz{c}{
            1\\
            0\\
            0
          }
          \igual{\red{!}}
          \matriz{c}{
            \frac{1}{2}\\
            1\\
            -\frac{1}{2}
          }
        $$
        Tuqui. Da eso.
\end{enumerate}

\begin{aportes}
  \item \aporte{\dirRepo}{naD GarRaz \github}
\end{aportes}
