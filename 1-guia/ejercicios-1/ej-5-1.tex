\begin{enunciado}{\ejercicio}
  Encontrar un sistema de generadores para los siguientes espacios vectoriales:
  \begin{enumerate}[label=(\alph*)]
    \item $\set{(x,y,z) \en \reales^3 : x + y - z = 0;\; x - y = 0}$

    \item $\set{\mathbf{A} \en \complejos^{3 \times 3} : \mathbf{A} = - \mathbf{A}^t}$

    \item $\set{\mathbf{A} \en \reales^{3 \times 3} : tr(\mathbf{A}) = 0}$

    \item $\set{x \en \complejos^4 : x_1 + x_2 - i x_4 = 0; ix_1 + (1+i)x_2 - x_3 = 0}$
  \end{enumerate}
\end{enunciado}

\begin{enumerate}[label=(\alph*)]
  \item $\ket{1,1,2}$

  \item Describo a $\mathbf{A}$ y a $-\mathbf{A}^t$ como :
        $$
          \mathbf{A} = \set{a_{ij} \en \complejos : 1\leq i, j \leq 3}
          \ytext
          -\mathbf{A}^t = \set{- a_{ji} \en \complejos : 1\leq i, j \leq 3}
        $$
        O escrito en idioma humano:
        $$
          \mathbf{A} =
          \matriz{ccc}{
            a_{11} & a_{12} & a_{13} \\
            a_{21} & a_{12} & a_{23} \\
            a_{31} & a_{32} & a_{33}
          }
          \ytext
          \mathbf{A}^T =
          \matriz{ccc}{
          -a_{11} & -a_{21} & -a_{31} \\
          -a_{12} & -a_{22} & -a_{32} \\
          -a_{13} & -a_{23} & -a_{33}
          }
        $$
        Entonces los elementos de la diagonal \textit{no se mueven, solo cambian de signo}, mientas que los
        elementos fuera de la diagonal tienen esa reflexión respecto a la diagonal:
        $$
          a_{ij} \igual{?}  - a_{ji}
          \sii
          a_{ij} =
          \llave{rcl}{
            0 & \text{ si } & i = j \\
            -a_{ji}  & \text{ si } & i \distinto j
          }
        $$
        Estoy buscando algo de la pinta:
        $$
        \textstyle
          \matriz{ccc}{
            a_{11} & a_{12} & a_{13} \\
            a_{21} & a_{12} & a_{23} \\
            a_{31} & a_{32} & a_{33}
          }
          =
          \matriz{ccc}{
          0 & -a_{21} & -a_{31} \\
          a_{21} & 0 & -a_{32} \\
          a_{31} & a_{32} & 0
          }
          =
          a_{21} \cdot
          \matriz{ccc}{
            0 & -1 & 0 \\
            1 & 0 & 0 \\
            0 & 0 & 0
          }
          +
          a_{31} \cdot
          \matriz{ccc}{
            0 & 0 & -1 \\
            0 & 0 & 0 \\
            1 & 0 & 0
          }
          +
          a_{32} \cdot
          \matriz{ccc}{
            0 & 0 & 0 \\
            0 & 0 & -1 \\
            0 & 1 & 0
          }
        $$
        El conjunto de generadores buscado:
        $$
          \cajaResultado{
            \ket{
              \matriz{ccc}{
                0 & -1 & 0 \\
                1 & 0 & 0 \\
                0 & 0 & 0
              };\;
              \matriz{ccc}{
                0 & 0 & -1 \\
                0 & 0 & 0 \\
                1 & 0 & 0
              }
              ;\;
              \matriz{ccc}{
                0 & 0 & 0 \\
                0 & 0 & -1 \\
                0 & 1 & 0
              }
            }}
        $$

  \item Veo que tr(a) es la función que suma los elementos de la diagonal principal de una matriz.
  
  La matriz expandida es de la forma:

  $$
        \textstyle
   \matriz{ccc}{
     A_{11} & A_{2} & A_{13} \\
     A_{12} & A_{22} & A_{23} \\
     A_{13} & A_{23} & A_{33}
   }
   $$

   Entonces, la restricción que me impone este subespacio es: 

   $$ A_{11} + A_{22} + A_{33} = 0 $$

   Despejando de la ecuación nos queda:
$$ A_{11}  = -A_{22} - A_{33}$$
   Que de forma matricial queda


   $$
    \ket{
    \matriz{ccc}{
      A_{-A_{22} - A_{33}} & A_{12} & A_{13} \\
      A_{12} & A_{22} & A_{23} \\
      A_{13} & A_{23} & A_{33}
    }
   }
   $$

   \red{Los generadores serían separar cada uno de las variables independientes en distintas matrices y luego tomar
   sus coeficientes. En este caso no se hizo así porque no me entró en la hoja}




  \item \hacer
\end{enumerate}

\begin{aportes}
  \item \aporte{\dirRepo}{naD GarRaz \github}
\end{aportes}
