\begin{enunciado}{\ejercicio}
  Sean las siguiente matrices de $3 \times 3$:
  $$
    A =
    \matriz{ccc}{
      1 & 3 & 0 \\
      0 & 1 & 2 \\
      1 & 0 & 1
    }
    B =
    \matriz{ccc}{
      1 & 1 & 1 \\
      3 & 0 & 1 \\
      2 & 0 & 2
    }
    C =
    \matriz{ccc}{
      c_{11} & c_{12} & c_{13} \\
      c_{21} & c_{22} & c_{23} \\
      c_{31} & c_{32} & c_{33}
    }
  $$
  Y consideremos el producto $AB = C$ en bloques:
  $$
    \matriz{c|c}{
      A_{11} & A_{12} \\ \hline
      A_{21} & A_{22}
    }
    \matriz{c|c}{
      B_{11} & B_{12} \\ \hline
      B_{21} & B_{22}
    }
    =
    \matriz{c|c}{
      C_{11} & C_{12} \\ \hline
      C_{21} & C_{22}
    }
  $$
  Para cada una de las particiones en bloques mencionadas a continuación,
  indicar si es realizable el producto $C = AB$ en bloques. En caso de ser realizable, calcular cada
  bloque $C_{ij}$ indicando sus dimensiones.
  \begin{enumerate}[label=(\alph*)]
    \item
          $A_{11} = [a_{11}],\,
            A_{12} = [a_{12} \  a_{13}],\,
            A_{21} =
            \bloque{c}{
              a_{21}\\
              a_{31}
            },\,
            A_{22} =
            \bloque{cc}{
              a_{22} & a_{23}\\
              a_{32} & a_{33}
            }
          $

          $B_{11} = [b_{11}],\,
            B_{12} = [b_{12} \  b_{13}],\,
            B_{21} =
            \bloque{c}{
              b_{21}\\
              b_{31}
            },\,
            B_{22} =
            \bloque{cc}{
              b_{22} & b_{23}\\
              b_{32} & b_{33}
            }
          $

    \item
          $A_{11} = [a_{11} a_{12}],\,
            A_{12} = [a_{13}],\,
            A_{21} =
            \bloque{cc}{
              a_{21} & a_{22}\\
              a_{31} & a_{32}
            },\,
            A_{22} =
            \bloque{c}{
              a_{23}\\
              a_{33}
            }
          $

          $B_{11} = [b_{11}],\,
            B_{12} = [b_{12}\ b_{13}],\,
            B_{21} =
            \bloque{c}{
              b_{21}\\
              b_{31}
            },\,
            B_{22} =
            \bloque{cc}{
              b_{22} & b_{23}\\
              b_{32} & b_{33}
            }
          $

    \item
          $A_{11} =
            \bloque{c}{
              a_{11}\\
              a_{21}
            },\,
            A_{12} =
            \bloque{cc}{
              a_{12} & a_{13}\\
              a_{22} & a_{23}
            },\,
            A_{21} = [a_{31}],\,
            A_{22} = [a_{32} \ a_{33}]
          $

          $B_{11} = [b_{11}],\,
            B_{12} = [b_{12}\ b_{13}],\,
            B_{21} =
            \bloque{c}{
              b_{21}\\
              b_{31}
            },
            B_{22} =
            \bloque{cc}{
              b_{22} & b_{23}\\
              b_{32} & b_{33}
            }
          $
  \end{enumerate}
  ¿Qué otras particiones válidas son posibles?
\end{enunciado}

¡¿Que interesante, no?!, peero: {\tiny\textit{¿Qué es esta verga?}}. A esta altura ya está clarísimo que para
poder multiplicar dos matrices, se tiene que cumplir que \textit{la cantidad de columnas del primer factor sea igual
  a la cantidad de filas del segundo}:
$$
  M \cdot M'
  \text{  se puede hacer si }
  M \en K^{n \times \blue{m}}
  \ytext
  M' \en K^{\blue{m} \times l}
$$

Hay que prestar atención a eso y después hacer el producto y suma en bloques, es un \textit{parecido pero distinto}.

\begin{enumerate}[label=(\alph*)]
  \item
        $$
          A =
          \matriz{c|cc}{
            \yellow{1} & \green{3} & \green{0} \\ \hline
            \blue{0} & \purple{1} & \purple{2} \\
            \blue{1} & \purple{0} & \purple{1}
          }
          B =
          \matriz{c|cc}{
            \yellow{1} & \green{1} & \green{1} \\\hline
            \blue{3} & \purple{0} & \purple{1} \\
            \blue{2} & \purple{0} & \purple{2}
          }
        $$
        Multiplico $A \cdot B$ en bloques:
        \begin{enumerate}[label=\tiny\faIcon{calculator}$_{\arabic*)}$]
          \item Busco el bloque $C_{11}$ ¿Se podrá hacer el cálculo?:
                $$
                  \yellow{A_{11}} \cdot \yellow{B_{11}}
                  +
                  \green{A_{12} \cdot \blue{B_{21}}} =
                  \yellow{[\, 1 \, ]} \cdot \yellow{[1]}  +
                  \green{
                    \bloque{cc}{
                      3 & 0
                    }
                  }
                  \cdot
                  \blue{
                    \bloque{c}{
                      3\\
                      2
                    }
                  }
                  = 10 = C_{11} \en K^{1 \times 1}
                $$

          \item Busco el bloque $C_{12}$ ¿Se podrá hacer el cálculo? {\tiny \rollingEyes}:
                $$
                  \yellow{A_{11}} \cdot \green{B_{12}}
                  +
                  \green{A_{12} \cdot \purple{B_{22}}} =
                  \yellow{[\, 1 \, ]} \cdot \green{[\,1\ 1\,]}  +
                  \green{
                    \bloque{cc}{
                      3 & 0
                    }
                  }
                  \cdot
                  \purple{
                    \bloque{cc}{
                      0 & 1\\
                      0 & 2
                    }
                  }
                  =
                  \bloque{c c}{
                    1 & 4
                  }
                  =
                  C_{12} \en K^{1 \times 2}
                $$

          \item Busco el bloque $C_{21}$ ¿Se podrá hacer el cálculo? \rollingEyes:
                $$
                  \blue{A_{21}} \cdot \yellow{B_{11}}
                  +
                  \purple{A_{22} \cdot \blue{B_{21}}} =
                  \blue{
                    \bloque{c}{
                      0\\
                      1
                    }
                  }
                  \cdot
                  \yellow{
                    [\, 1\, ]
                  }
                  +
                  \purple{
                    \bloque{cc}{
                      1 & 2\\
                      0 & 1
                    }
                  }
                  \cdot
                  \blue{
                    \bloque{c}{
                      3\\
                      2
                    }
                  }
                  =
                  \bloque{c}{
                    0\\
                    1
                  }
                  +
                  \bloque{c}{
                    7\\
                    2
                  }
                  =
                  \bloque{c}{
                    7\\
                    3
                  }
                  =
                  C_{21} \en K^{2 \times 1}
                $$

          \item Busco el bloque $C_{22}$ ¿Se podrá hacer el cálculo? {\LARGE \rollingEyes}:
                $$
                  \blue{A_{21}} \cdot \green{B_{12}}
                  +
                  \purple{A_{22} \cdot \purple{B_{22}}}
                  =
                  \blue{
                    \bloque{c}{
                      0\\
                      1
                    }
                  }
                  \cdot
                  \green{
                    \bloque{cc}{
                      1 & 1
                    }
                  }
                  +
                  \purple{
                    \bloque{cc}{
                      1 & 2\\
                      0 & 1
                    }
                  }
                  \cdot
                  \purple{
                    \bloque{cc}{
                      0 & 1\\
                      0 & 2
                    }
                  }
                  =
                  \bloque{cc}{
                    0 & 0\\
                    1 & 1
                  }
                  +
                  \bloque{cc}{
                    0 & 5 \\
                    0 & 2
                  }
                  =
                  \bloque{cc}{
                    0 & 5 \\
                    1 & 3
                  }
                  =
                  C_{22} \en K^{2 \times 2}
                $$
        \end{enumerate}

        Si todavía no te volaste la tapa de los sesos esto queda así:
        $$
          \cajaResultado{
            A \cdot B =
            \matriz{c|cc}{
              \yellow{1} & \green{3} & \green{0} \\ \hline
              \blue{0} & \purple{1} & \purple{2} \\
              \blue{1} & \purple{0} & \purple{1}
            }
            \cdot
            \matriz{c|cc}{
              \yellow{1} & \green{1} & \green{1} \\\hline
              \blue{3} & \purple{0} & \purple{1} \\
              \blue{2} & \purple{0} & \purple{2}
            }
            =
            \matriz{c|cc}{
              10 & 1 & 4 \\\hline
              7 & 0 & 5  \\
              3 & 1 & 3
            }
          }
        $$
        Sí, multiplicar en bloques dio lo mismo que multiplicar como siempre. ¿Es magia? NO, es \magic \textit{matemagia} \magic.

  \item \hacer
  \item \hacer
\end{enumerate}

\begin{aportes}
  \item \aporte{\dirRepo}{naD GarRaz \github}
\end{aportes}
