\begin{enunciado}{\ejercicio}
  Encontrar los coeficientes de la parábola $y = ax^2 + bx + c$ que pasa por los puntos
  $(1, 1), (2, 2) \ytext (3, 0)$. Verificar el resultado obtenido usando \python.
  Graficar los puntos y la parábola aprovechando el siguiente código:

  \codigoPython{ej-4/codigo1-4-1.py}
\end{enunciado}

Hay que armar la matriz para luego resolverla:
$$
  \llave{l}{
    y(\blue{1}) = a \cdot \blue{1}^2 + b \cdot \blue{1} + c = \blue{1}\\
    y(\blue{2}) = a \cdot \blue{2}^2 + b \cdot \blue{2} + c = \blue{2}\\
    y(\blue{3}) = a \cdot \blue{3}^2 + b \cdot \blue{3} + c = \blue{0}
  }
$$
El sistema a resolver en forma matricial:
$$
  \matriz{ccc}{
    1 & 1 & 1 \\
    4 & 2 & 1 \\
    9 & 3 & 1 \\
  }
  \cdot
  \matriz{c}{
    a\\
    b\\
    c
  }
  =
  \matriz{c}{
    1\\
    2\\
    0
  }
$$
Ampliamos la matriz de coeficientes:
$$
  \matriz{ccc|c}{
    1 & 1 & 1 & 1\\
    4 & 2 & 1 & 2\\
    9 & 3 & 1 & 0
  }
  \Sii{\magic}
  \matriz{ccc|c}{
    1 & 1 & 1 & 1\\
    0 & 1 & \frac{3}{2} & 1\\
    0 & 0 & 1 & -3
  }
  \entonces
  \llave{ccc}{
    a & = & -\frac{3}{2} \\
    b & = & \frac{11}{2} \\
    c & = & -3
  }
$$
La parábola queda:
$$
  \cajaResultado{
    y = -\frac{3}{2}x^2 + \frac{11}{2}x - 3
  }
$$

\bigskip

\copyPaste

\codigoPython{ej-4/codigo1-4-2.py}

\begin{aportes}
  \item \aporte{\dirRepo}{naD GarRaz \github}
\end{aportes}
