\begin{enunciado}{\ejercicio}
  \begin{enumerate}[label=(\alph*)]
    \begin{multicols}{3}
      \item
      $
        A =
        \matriz{ccc}{
          1 & 1 & 1 \\
          0 & 1 & 1 \\
          0 & 0 & 1
        }
      $

      \item
      $
        A =
        \matriz{ccc}{
          \cos(\theta) & -\sin(\theta) & 0 \\
          \sin(\theta) & \cos(\theta) & 0 \\
          0 & 0 & 1
        }
      $

      \item
      $
        A =
        \matriz{cccc}{
          1 & 0 & -1 & 0 \\
          0 & 0 & 1  & 0 \\
          2 & 1 & -2 & 3 \\
          3 & 1 & -1 & 3
        }
      $

      \item
      $
        A =
        \matriz{ccccc}{
          2 & 1 & 3 & 1 & 2 \\
          0 & 5 & -1& 8 & 2 \\
          0 & 1 & 0 & 1 & 2 \\
          0 & 1 & 0 & 1 & 2 \\
          0 & 1 & 0 & 0 & 2
        }
      $

      \item
      $
        A =
        \matriz{cccc}{
          a_{11} & 0 & \cdots  & 0 \\
          0 & a_{22} & \cdots & 0  \\
          \vdots & \ddots & \ddots & \vdots  \\
          0 & 0 & \cdots & a_{nn}
        }
      $
    \end{multicols}
  \end{enumerate}
\end{enunciado}
\parrafoDestacado[\faIcon{calculator}]{
  Una matriz inversible tiene determinante distinto de 0 y rango completo
}
Voy a estar calculando determinantes con \textit{Laplace}, \hyperlink{teoria-1:calculo-determinante}{acá en la teoría está como hacerlo \click}
\begin{enumerate}[label=(\alph*)]
  \item En una matriz \textit{triangular superior o inferior}: El determinante es el producto de los elementos diagonales.
        $$
          \det(A) = 1 \distinto 0 \entonces A \text{ es inversible}
        $$
        Resolviendo el sistema para $X \en \reales^{3 \times 3}$:
        $$
          A \cdot \ua{X}{A^{-1}} = I_3
          \sii
          \matriz{ccc|c|c|c}{
            1 & 1 & 1 & 1 & 0 & 0 \\
            0 & 1 & 1 & 0 & 1 & 0 \\
            0 & 0 & 1 & 0 & 0 & 1
          }
          \Sii{\magic}
          X = A^{-1} =
          \matriz{ccc}{
            1 & -1 & 0 \\
            0 & 1 & -1 \\
            0 & 0 & 1
          }
        $$

  \item Desarrollando por la 3era fila:
        $$
          \det(A) =
          \deter{ccc}{
            \cos(\theta) & -\sin(\theta) & 0 \\
            \sin(\theta) & \cos(\theta)  & 0 \\ \rowcolor{red!5!white}
            0            & 0             & 1
          }
          =
          \deter{cc}{
            \cos(\theta) & -\sin(\theta) \\
            \sin(\theta) & \cos(\theta)
          }
          = \cos^2(\theta) + \sin^2(\theta) = 1 \distinto 0
          \entonces A \text{ es inversible}
        $$
        Esta inversa sale \text{pensándolo geométricamente}, dado que $A$ es la matriz que rota un punto al rededor del eje $z$ en \textit{sentido
          antihorario}, la inversa debería rotar en \textit{sentido horario}. Ese cambio de sentido en la rotación es \ul{cambiar el signo de $\theta$}:
        $$
          A^{-1}
          =
          \matriz{ccc}{
            \cos(\red{-\theta}) & -\sin(\red{-\theta}) & 0 \\
            \sin(\red{-\theta}) & \cos(\red{-\theta})  & 0 \\
            0            & 0             & 1
          }
          \igual{\red{!}}
          \matriz{ccc}{
            \cos(\theta) & \sin(\theta) & 0 \\
            -\sin(\theta) & \cos(\theta)  & 0 \\
            0            & 0             & 1
          }
        $$
        En \red{!} uso que el la paridad de las funciones:
        $$
          \llave{rcc}{
            \cos(-\theta) & = & \cos(\theta)\\
            \sin(-\theta) & = & -\sin(\theta)
          }
        $$

        Obviamente podés calcular la inversa como a vos te pinte.

  \item  Fijate que sumando
        Calculo determinante:
        $$
          A =
          \deter{cccc}{
            1 & 0 & -1 & 0 \\ \rowcolor{red!5!white}
            0 & 0 & 1  & 0 \\
            2 & 1 & -2 & 3 \\
            3 & 1 & -1 & 3
          }
          =
          (-1)
          \deter{ccc}{
            \rowcolor{red!5!white}
            1 & 0 & 0 \\
            2 & 1 & 3 \\
            3 & 1 & 3
          }
          =
          (-1)
          \deter{ccc}{
            1 & 3 \\
            1 & 3
          }
          =0
        $$
        Ya en el segundo paso se veía bien que las filas eran \textit{linealmente independientes}.
        Como el $|A| = 0$, no existe la inversa de $A$.

  \item
        Una matriz cuadrada triangulada tiene por determinante le producto de sus elementos diagonales.
        En este caso:
        $$
          \textstyle
          |A| = \productoria{i=1}{n} a_{ii} = 0
        $$
        La matriz $A$ no tiene inversa.

  \item Como en el ítem anterior:
        $$
          \textstyle
          |A| = \productoria{i=1}{n} a_{ii}
        $$
        Tengo que pedir que:
        $$
          \text{La matriz $A$ es inversible} \sisolosi a_{ii} \distinto 0 \paratodo i \en[1, n]
        $$
        La inversa de $A$ es fácil de calcular en este caso:
        $$
          A^{-1} =
          \matriz{cccc}{
            \frac{1}{a_{11}} & 0 & \cdots  & 0 \\
            0 & \frac{1}{a_{22}} & \cdots & 0  \\
            \vdots & \ddots & \ddots & \vdots  \\
            0 & 0 & \cdots & \frac{1}{a_{nn}}
          }
        $$
\end{enumerate}

\begin{aportes}
  \item \aporte{\dirRepo}{naD GarRaz \github}
\end{aportes}
