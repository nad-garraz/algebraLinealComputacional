\begin{enunciado}{\ejercicio}
  Extraer una base de $S$ de cada uno de los siguientes sistemas de generadores y hallar
  la dimensión de $S$. Extender la base de $S$ a una base del espacio vectorial correspondiente.
  \begin{enumerate}[label=(\alph*)]
    \item $S = \ket{ (1,1,2); (1,3,5); (1,1,4), (5,1,1)} \subseteq \reales^3, \, K = \reales$

    \item $S = \ket{
              \matriz{cc}{
                1 & 1 \\
                1 & 1
              },
              \matriz{cc}{
                0 & i \\
                1 & 1
              },
              \matriz{cc}{
                0 & i \\
                0 & 0
              },
              \matriz{cc}{
                1 & 1 \\
                0 & 0
              }
            } \subseteq \complejos^{2 \times 2}, \, K = \complejos$
  \end{enumerate}
\end{enunciado}

\begin{enumerate}[label=(\alph*)]
  \item Demasiados vectores en ese sistema de generadores, voy a quedarme solo con los \textit{linealmente independientes},
        así obteniendo una base del subespacio $S$:
        $$
          \matriz{ccc}{
            1 & 1 & 2 \\
            1 & 3 & 5 \\
            1 & 1 & 4\\
            5 & 1 & 1
          }
          \triangulacion{
            F_2 - F_1 \to F_2\\
            F_3 - F_1 \to F_3\\
            F_4 - 5F_1 \to F_4\\
          }
          \matriz{ccc}{
            1 & 1 & 2 \\
            0 & 2 & 3 \\
            0 & 0 & 2\\
            0 & -4 & -9
          }
          \triangulacion{
            F_4 + 2F_2 \to F_4\\
          }
          \matriz{ccc}{
            1 & 1 & 2 \\
            0 & 2 & 3 \\
            0 & 0 & 2\\ \rowcolor{red!10}
            0 & 0 & -3
          }
        $$
        Por lo tanto se tiene que una posible base para $S$:
        $$
          \cajaResultado{
            B_S = \set{(1,1,2); (0,2,3); (0,0,2)}
          }
        $$
        La base genera todo $\reales^3$.

  \item
        Ataco parecido, pero voy a desarrollar mejor la forma de triangular, porque a veces acá uno puedo
        entrar en la rosca de como \textit{estirar la matrices} para luego triangular. Planteo una combineta:
        $$
          a \cdot
          \matriz{cc}{
            1 & 1 \\
            1 & 1
          } +
          b \cdot
          \matriz{cc}{
            0 & i \\
            1 & 1
          } +
          c \cdot
          \matriz{cc}{
            0 & i \\
            0 & 0
          } +
          d \cdot
          \matriz{cc}{
            1 & 1 \\
            0 & 0
          }
          =
          \matriz{cc}{
            0 & 0 \\
            0 & 0
          }
          \sii
          \matriz{cc}{
            a+d & a+d + ib + ic \\
            a+b & a+b
          }
          =
          \matriz{cc}{
            0 & 0 \\
            0 & 0
          }
        $$
        Paso a sistema de ecuaciones y triangulo:
        $$
          \begin{array}{rcl}
            \llave{rcl}{
            a + d           & = & 0           \\
            a + d + ib + ic & = & 0           \\
            a + b           & = & 0           \\
            a + b           & = & 0
            }
            \equivalente
            \matriz{cccc|c}{
            1               & 0 & 0  & 1  & 0 \\
            1               & i & i  & 1  & 0 \\
            1               & 1 & 0  & 0  & 0 \\
            1               & 1 & 0  & 0  & 0
            }
                            &
            \triangulacion{
            F_2 - F_1 \to F_2                 \\
            F_3 - F_1 \to F_3                 \\
            F_4 - F_1 \to F_4                 \\
            }               &
            \matriz{cccc|c}{
            1               & 0 & 0  & 1  & 0 \\
            0               & i & i  & 0  & 0 \\
            0               & 1 & 0  & -1 & 0 \\\rowcolor{red!10}
            0               & 1 & 0  & -1 & 0
            }
            \\
                            &
            \triangulacion{
            iF_3 - F_2 \to F_3                \\
            }
                            &
            \matriz{cccc|c}{
            1               & 0 & 0  & 1  & 0 \\
            0               & i & i  & 0  & 0 \\
            0               & 0 & -i & -i & 0 \\\rowcolor{red!10}
            0               & 0 & 0  & 0  & 0
            }                                 \\
                            &
            \equivalente
                            &
            \llave{rcc}{
            a               & = & -d          \\
            b               & = & d           \\
            c               & = & -d
            }
          \end{array}
        $$
        En el sistema me queda \ul{solo una variable libre}. Necesito 3 matrices para formarla. También
        veo que me quedaron 3 \textit{filas no nulas} es decir que el \textit{rango fila} es 3, por lo que el \textit{rango columna}
        también es 3 y las columnas en esa matriz sería como haber puesto a las matrices {\tiny(\textit{estiradas})} y al triangular,
        eliminar una así opppppteniendo \href{\mindExplosion}{3 matrices l.i.}.

        Agarro ahora 3 matrices \textit{linealmente independientes}:

        Me quedo entonces con la base para $S$:
        $$
          \cajaResultado{
            B_S = \set{
              \matriz{cc}{
                1 & 1 \\
                1 & 1
              },
              \matriz{cc}{
                0 & i \\
                1 & 1
              },
              \matriz{cc}{
                0 & i \\
                0 & 0
              }
            }
          }
        $$
        La dimensión de $S = 3$ y $\complejos^{2 \times 2}$ con $K = \complejos$ tiene dimensión 4, así que necesito
        \blue{un elemento} \textit{linealmente independiente} para extender la base:

        Si no se encuentra a ojo, una forma mecánica para encontrar la \blue{matriz} es poner a las matrices de las bases en filas,
        triangular y ver ahí una que quede \textit{linealmente independiente}.
        $$
          \cajaResultado{
            B_S = \set{
              \matriz{cc}{
                1 & 1 \\
                1 & 1
              },
              \matriz{cc}{
                0 & i \\
                1 & 1
              },
              \matriz{cc}{
                0 & i \\
                0 & 0
              },
              \blue{
                \matriz{cc}{
                  0 & 0 \\
                  0 & 1
                }
              }
            }
          }
        $$
\end{enumerate}

\bigskip

\textit{Nota que podría ser de interés:}

Si estoy laburando en un espacio tipo $\complejos^2$ hay que prestarle \ul{mucha} atención al cuerpo $K$,
porque mirá las bases de este espacio según el cuerpo:
$$
  \begin{array}{ccl}
    K = \complejos & \to & B_{\complejos^2} = \set{(1,0); (0,1)}               \\
    K = \reales    & \to & B_{\complejos^2} = \set{(1,0); (0,1); (i,0); (0,i)}
  \end{array}
$$
Onda en uno la dimensión es 2 y en el otro 4 \faIcon{hands-wash}.

\textit{Fin Nota que podría ser de interés.}

\begin{aportes}
  \item \aporte{\dirRepo}{naD GarRaz \github}
  \item \aporte{\neverGonnaGiveYouUp}{Federico \youtube}
\end{aportes}
