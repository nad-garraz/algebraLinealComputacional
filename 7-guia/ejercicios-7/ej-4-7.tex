\begin{enunciado}{\ejercicio}
  Decidir para cada una de las siguientes matrices si los métodos de Jacobi y de Gauss-Seidel converge.
  $$
    A =
    \matriz{ccc}{
      1 & 1 & 0 \\
      -1 & 2 & 1 \\
      0 & 0 & 1 \\
    }
    \qquad
    B =
    \matriz{ccc}{
      1 & 0 & -1 \\
      -2 & 1 & 1 \\
      -1 & 0 & -1
    }
    \qquad
    C =
    \matriz{ccc}{
      3 & -1 & -4 \\
      -1 & 5 & 7 \\
      -4 & 7 & 14
    }
  $$
\end{enunciado}

\textit{Matriz $A$:}

$A$ es una matriz \textit{tridiagonal} es decir que averiguando el radio espectra de algún método ya
sé lo que pasa con el otro, debido a que para este tipo de matrices:
$$
  \rho(T_{GS}) \igual{$\llamada1$} \rho^2(T_J)
$$
Calculo $T_J = -D^{-1}(L+U)$:
$$
  T_J =
  -\matriz{ccc}{
    1 & 0 & 0 \\
    0 & \frac{1}{2} & 0 \\
    0 & 0 & 1
  }
  \matriz{ccc}{
    0 & 1 & 0 \\
    -1 & 0 & 1 \\
    0 & 0 & 0
  }
  =
  \matriz{ccc}{
    0 & -1 & 0\\
    \frac{1}{2} & 0 & \frac{-1}{2}\\
    0 & 0 & 0
  } $$
Calculo los autovalores:
$$
  \det(T_J - \lambda I) = -\lambda \cdot (\lambda^2 + \frac{1}{2}) = 0
  \sii
  \llave{rcl}{
    \lambda_1 & = & 0\\
    \lambda_2 & = & \frac{i}{\sqrt{2}}\\
    \lambda_3 & = & \frac{-i}{\sqrt{2}}
  }
  \entonces
  \rho(T_J) = \maximo_i(|\lambda_i|) = \frac{1}{\sqrt{2}} \Entonces{$\llamada1$} \rho(T_{GS}) = \frac{1}{2}
$$
Ambos métodos convergen dado que:
$$
  \cajaResultado{
    \rho(T_J) = \frac{1}{\sqrt{2}} < 1 \ytext \rho(T_{GS}) = \frac{1}{2} < 1.
  }
$$
En particular el método de Gauss-Seidel converge más rápido dado que el de Jacobi, dado que $\rho_{GS} < \rho{T_J}$

\bigskip

\textit{Matriz $B$:}

\hacer

\textit{Matriz $C$:}
C es una matriz \textit{simétrica y definida positiva}.
$$
  (x\ y\ z)
  \matriz{ccc}{
    3 & -1 & -4 \\
    -1 & 5 & 7 \\
    -4 & 7 & 14
  }
  \matriz{c}{
    x\\
    y\\
    z
  }
  =
$$
\hacer

