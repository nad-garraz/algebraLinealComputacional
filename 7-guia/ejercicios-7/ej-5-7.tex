\begin{enunciado}{\ejercicio}
  Decidir para cada uno de los siguientes sistemas, si los métodos de Jacobi y de
  Gauss-Seidel son convergentes. En caso afirmativo usarlos para resolver el sistema. Si ambos
  métodos convergen, determinar cuál converge más rápido
  ¿Es la matriz del sistema diagonal dominante? ¿Y simétrica y definida positiva?

  \begin{enumerate}[label=\alph*)]
    \begin{multicols}{2}
      \item
      $
        \matriz{ccc}{
          3 & 1 & 1\\
          2 & 6 & 1\\
          1 & 1 & 4
        }
        \matriz{c}{
          x_1 \\
          x_2\\
          x_3
        } =
        \matriz{c}{
          5 \\
          9\\
          6
        },
      $

      \item
      $
        \matriz{cccc}{
          5 & 7 & 6 & 5  \\
          7 & 10 & 8 & 7 \\
          6 & 8 & 10 & 9 \\
          5 & 7 & 9 & 10 \\
        }
        \matriz{c}{
          x_1 \\
          x_2 \\
          x_3 \\
          x_4
        } =
        \matriz{c}{
          23 \\
          32 \\
          33 \\
          31
        }
      $
    \end{multicols}
  \end{enumerate}
\end{enunciado}

\begin{enumerate}[label=\alph*)]
  \item Esta matriz es \textit{estrictamente diagonal dominante}:
        $$
          |a_{ii}| > \sumatoria{i \distinto j}{} |a_{ij}|
          \quad \paratodo i \en \naturales_{\leq n}
        $$
        Por lo tanto ambos métodos convergen.
        Para resolver el sistema encuentro las matrices de iteración.

        \bigskip

        \textit{Jacobi}:
        $$
          M_J = - D^{-1} \cdot (L + U) =
          -\matriz{ccc}{
            \frac{1}{3} & 0 & 0 \\
            0 & \frac{1}{6}  & 0 \\
            0 & 0 &\frac{1}{4}
          }
          \matriz{ccc}{
            0 & 1 & 1\\
            2 & 0 & 1\\
            1 & 1 & 0
          }
          =
          -\matriz{ccc}{
            0 & \frac{1}{3} & \frac{1}{3}\\
            \frac{1}{3} & 0 & \frac{1}{6}\\
            \frac{1}{4} & \frac{1}{4} & 0
          }
          \ytext
          \tilde{b} = D^{-1}\cdot b =
          \matriz{c}{
            \frac{5}{3} \\
            \frac{3}{2} \\
            \frac{3}{2}
          }
        $$
        Con todos esos ingredientes puedo empezar a iterar. Agarro
        a $x_0 = (1,1,1)^t$ como semilleeeta, porque a ojo se ve que la solución
        viene por este lado, \textit{¿Es esto hacer trampa?, no sé. Si te molesta andá al psicólogo}:
        $$
          x_1 = M_Jx_0 + \tilde{b} = (1,1,1)^t                                            \\
        $$
        Bueh, dado que $x_0 = x_1$ el método de Jacobi ya convergió con esa excelente semilla.

        \bigskip

        \textit{Gauss-Seidel:}
        $$
          \begin{array}{c}
            M_{GS} = - (L+D)^{-1} \cdot U =
            -\matriz{ccc}{
            \frac{1}{3}   & 0             & 0             \\
            -\frac{1}{9}  & \frac{1}{6}   & 0             \\
            -\frac{1}{18} & -\frac{1}{24} & \frac{1}{4}
            }
            \matriz{ccc}{
            0             & 1             & 1             \\
            0             & 0             & 1             \\
            0             & 0             & 0
            }
            =
            -\matriz{ccc}{
            0             & \frac{1}{3}   & \frac{1}{3}   \\
            0             & -\frac{1}{9}  & \frac{1}{18}  \\
            0             & -\frac{1}{18} & -\frac{7}{72}
            }                                             \\
            \ytext                                        \\
            \tilde{b} = (L + D)^{-1}\cdot b =
            \matriz{c}{
            \frac{5}{3}                                   \\
            \frac{17}{18}                                 \\
              \frac{61}{72}
            }
          \end{array}
        $$
        ¿Uso la misma semilla? La altero un poquito, $x_0 = (3/4,1,1/2)$:
        $$
          \llave{rcl}{
            x_1 & = & M_{GS} \cdot x_0 + \tilde{b} = (7/6,37/36,137/144)^t \approx (1.167, 1.028, 0.951)                                      \\
            x_2 & = & M_{GS} \cdot x_1 + \tilde{b} = (145/144,869/864,3445/3456)^t   \approx (1.006, 1.006, 0.997)                                         \\
            x_3 & = & M_{GS} \cdot x_2 + \tilde{b} = (1151/1152,20753/20736, 82945/82944)^t   \approx (0.999, 1.001, 1.000)                                         \\
          }
        $$
        Listo, suficiente para mí, \textit{quedo con una sensación entre espantado y maravillado!}

        Para ver la velocidad de convergencia, habría que ver los \textit{radios espectrales}:
        \hacer

  \item
        \textit{Jacobi:}
        \hacer
        \textit{Gauss-Seidel:}
        \hacer
\end{enumerate}

