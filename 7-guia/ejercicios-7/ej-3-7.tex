\begin{enunciado}{\ejercicio}
  Considerar el sistema $Ax = b$ para
  $
    A = \matriz{cc}{
      64 & -6 \\
      6 & -1
    }
  $ y $b = (1,2)^t$.
  \begin{enumerate}[label=(\alph*)]
    \item Demostrar que el método de Jacobi converge para todo dato inicial.

    \item Sea $J$ la matriz de iteración. Hallar las normas 1, $\infinito$ y 2 de $J$.

          ¿Contradice la convergencia del método?

    \item Hallar una norma $\normaBullet$ en la cual $\norma{J}$ sea $<1$.

          \textit{Sugerencia: Considerar una base de autovectores de $J$}.
  \end{enumerate}
\end{enunciado}

\begin{enumerate}[label=(\alph*)]
  \item Busco autovalores de $J$,
        \parrafoDestacado{
          Si bien es una matriz chica de $2 \times 2$ quiero practicar el método para calcular los autovalores sin calcular la inversa
          de $D$. Sea como sea, la \textit{sugerencia del final} dice que voy a terminar calculando $J$.
        }
        Escribiendo a la matriz como $A = L + D + U$ (\hyperlink{teoria-7:determinante}{ver acá el genérico \click)},
        Si quiero los autovalores de la matriz de iteración es $J = -D^{-1}(D + U)$ puedo hacer:
        $$
          J = -D^{-1}(L + U) \text{ tiene autovalor }\blue{\lambda}
          \sii
          \det\big(\blue{\lambda} D + (L + U)\big) = 0
        $$
        con
        $$
          A  =
          \ub{
            \matriz{cc}{
              0 & 0 \\
              6 & 0
            }
          }{ L }
          +
          \ub{
            \matriz{cc}{
              64 & 0 \\
              0 & -1
            }
          }{ D }
          +
          \ub{
            \matriz{cc}{
              0 & -6 \\
              0 & 0
            }
          }{ U }
        $$
        Calculo determinante:
        $$
          \det\big(\blue{\lambda} D + (L + U)\big) = 0
          \sii
          \deter{cc}{
            64 \blue{\lambda} & -6              \\
            6                 & -\blue{\lambda}
          } = 0
          \sii
          \llave{rcc}{
            \blue{\lambda_1} & = & \frac{3}{4} \\
            \blue{\lambda_2} & = & -\frac{3}{4}
          }
        $$
        Dado que el radio espectral $\rho(A) = \frac{3}{4} < 1$ el método converge para cualquier valor inicial.

  \item La matriz de iteración de Jacobi:
        $$
          J =
          \ub{
            \matriz{cc}{
              \frac{1}{64} & 0 \\
              0 & -1
            }
          }{
            D^{-1}
          }
          \ub{
            \matriz{cc}{
              0 & -6 \\
              6 & 0
            }
          }{
            (L + U)
          }
          =
          \matriz{cc}{
            0 & -\frac{3}{32} \\
            -6 & 0 \\
          }
          \flecha{autovectores}
          \llave{rcc}{
            E_{\blue{\lambda_1} = \frac{3}{4}} & = & \ket{(1, 8)} \\
            E_{\blue{\lambda_2} = -\frac{3}{4}} & = & \ket{(-1, 8)}
          }
        $$
        Recordar acá que:
        \parrafoDestacado[\atencion]{
          Una matriz de iteración $M_I$ converge si y solo si $\rho(M_I) \taa{\llamada1}{}< 1$.

          Por otro lado para cualquier norma $\normaBullet$, se tiene que $\rho(M_I) < \norma{M_I}$. Por lo tanto
          es \textit{\magenta{suficiente}} que $\norma{M_I} < 1$ para garantizar convergencia.

          Peeeeero, si $\norma{M_I} > 1$ no quiere decir que no converja el método, debido todo depende de $\llamada1$.
        }

        Las normas pedidas:
        \begin{enumerate}[label=(\blue{\faIcon{ruler}})]
          \item $\norma{J}_1 = 6$
          \item $\norma{J}_\infinito = 6$
          \item $\norma{J}_2 = \frac{3}{4}\quad \to $ acá está la papa.
        \end{enumerate}

  \item
        Hay que usar:
        $$
          \norma{A}_W = \norma{W^{-1} A W}_\infinito
        $$
        La matriz generada por los autovectores:
        $$
          C =
          \matriz{cc}{
            1 & -1 \\
            8 & 8
          }
        $$
        Y puedo hacer:
        $$
          J = CDC^{-1}
          \sii
          C^{-1} J C =
          \matriz{cc}{
            \frac{3}{4} & 0 \\
            0 & -\frac{3}{4}
          }
        $$
        Por lo tanto usando la norma $\normaBullet_C$
        $$
          \norma{J}_C =
          \norma{C^{-1}JC}_\infinito =
          \norma{D}_\infinito =
          \frac{3}{4} < 1
        $$
\end{enumerate}

\begin{aportes}
  \item \aporte{\dirRepo}{naD GarRaz \github}
\end{aportes}
