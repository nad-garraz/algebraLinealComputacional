\begin{enunciado}{\ejExtra}
  {\tiny [\violet{recu 5/12/2024}]}
  Se desea resolver el sistema $Ax = b$ para un $b \en \reales^3$ y
  $A = \matriz{ccc}{
      1 & 0 & \alpha \\
      1 & 1 & 0 \\
      -1 & 1 & 1 \\
    }$
  con $\alpha \en \reales$.
  \begin{enumerate}[label=(\alph*)]
    \item Determinar los valores de $\alpha$ para los cuales el método de Gauss-Seidel converge para cualquier vector inicial $x_0$.

    \item Probar que si $\alpha = 0$ el método de Jacobi converge en 3 pasos para cualquier $x_0$.

          \textit{Sugerencia:} analizar $B_J^3$, siendo $B_J$ la matriz que gobierna la iteración del método de Jacobi.
  \end{enumerate}
\end{enunciado}

\begin{enumerate}[label=(\alph*)]
  \item Busco $\alpha$ tal que $\rho(B_{GS}) < 1$. Interesante en este ejercicio es el \textit{ratoneo}, demostración
        de como las pocas ganas de invertir una matriz revelan una forma de encontrar lo deseado minimizando el
        esfuerzo y coso, en fin:
        $$
          A =
          \ub{
            \matriz{ccc}{
              0 & 0 & 0 \\
              1 & 0 & 0 \\
              -1 & 1 & 0
            }
          }{
            L
          }
          \ub{
            \matriz{ccc}{
              1 & 0 & 0 \\
              0 & 1 & 0 \\
              0 & 0 & 1
            }
          }{
            D
          }
          \ub{
            \matriz{ccc}{
              0 & 0 & \alpha \\
              0 & 0 & 0 \\
              0 & 0 & 0
            }
          }{
            U
          }
          \entonces
          T_{GS} =
          -(D + L)^{-1}
          \matriz{ccc}{
            0 & 0 & \alpha \\
            0 & 0 & 0 \\
            0 & 0 & 0
          }
        $$
        El producto $T_{GS}$ tiene ceros en la primera y segunda columna, mientras que en la tercera tiene
        a la \textit{primera columna} de $-(D + L)^{-1}$ multiplicada por $\alpha$.
        Es \textit{fácil} encontrar esa \textit{primera columna} sin invertir toda la matriz:
        $$
          (D + L)
          \matriz{c}{
            x_{11}\\
            x_{21}\\
            x_{31}
          }
          =
          \matriz{c}{
            1\\
            0 \\
            0
          }
          \sii
          \matriz{c}{
            x_{11}\\
            x_{21}\\
            x_{31}
          }
          =
          \matriz{c}{
            1 \\
            -1 \\
            2
          }
        $$
        Así tengo $T_{GS}$:
        $$
          T_{GS} =
          \matriz{ccc}{
            0 & 0 & \alpha \\
            0 & 0 & -\alpha \\
            0 & 0 & 2\alpha
          }
        $$
        Matriz que tiene un \textit{radio espectral}: $\rho(T_{GS}) = 2|\alpha|$.
        Por lo tanto para que el método de Gauss-Seidel converja para todo vector inicial $x_0$:
        $$
          \cajaResultado{
            |\alpha| < 1
          }
        $$
        Fin.

  \item Jacobi:
        $$
          A =
          \ub{
            \matriz{ccc}{
              0 & 0 & 0 \\
              1 & 0 & 0 \\
              -1 & 1 & 0
            }
          }{
            L
          }
          \ub{
            \matriz{ccc}{
              1 & 0 & 0 \\
              0 & 1 & 0 \\
              0 & 0 & 1
            }
          }{
            D
          }
          \ub{
            \matriz{ccc}{
              0 & 0 & \alpha \\
              0 & 0 & 0 \\
              0 & 0 & 0
            }
          }{
            U
          }
          \entonces
          T_J =
          -D^{-1} (L + U) =
          \matriz{ccc}{
            0 & 0 & 0 \\
            -1 & 0 & 0 \\
            1 & -1 & 0
          }
        $$
        Hago hasta la cuarta iteración, para comparar la tercera y cuarta:
        $$
          \begin{array}{rcl}
            x^{(1)} & = & \green{T_J x^{(0)} + b}                                                                                                    \\
            x^{(2)} & = & T_J x^{(1)} + b = T_J (\green{T_J x^{(0)} + b}) + b = \violet{T_J^2 x^{(0)} + T_Jb + b}                                    \\
            x^{(3)} & = & T_J x^{(2)} + b = T_J (\violet{T_J^2 x^{(0)} + T_Jb + b}) + b = \purple{T_J^3 x^{(0)} + T_J^2b + T_Jb + b}                 \\
            x^{(4)} & = & T_J x^{(3)} + b = T_J (\purple{T_J^3 x^{(0)} + T_J^2b + T_Jb + b}) + b = \blue{T_J^4 x^{(0)} + T_J^3b + T_J^2b + T_Jb + b}
          \end{array}
        $$
        Usando la sugerencia puedo ver que $x^{(3)} \ytext x^{(4)}$ son iguales:
        {\tiny
        $$
          \textstyle
          \begin{array}{c}
            T_J^2 =
            \matriz{ccc}{
            0  & 0  & 0 \\
            -1 & 0  & 0 \\
            1  & -1 & 0
            }
            \matriz{ccc}{
            0  & 0  & 0 \\
            -1 & 0  & 0 \\
            1  & -1 & 0
            }
            =
            \matriz{ccc}{
            0  & 0  & 0 \\
            0  & 0  & 0 \\
            1  & 0  & 0
            }
            \flecha{una}[más]
            T_J^3 =
            \matriz{ccc}{
            0  & 0  & 0 \\
            0  & 0  & 0 \\
            1  & 0  & 0
            }
            \matriz{ccc}{
            0  & 0  & 0 \\
            -1 & 0  & 0 \\
            1  & -1 & 0
            }
            =
            \matriz{ccc}{
            0  & 0  & 0 \\
            0  & 0  & 0 \\
            0  & 0  & 0
            }
          \end{array}
        $$
        }
        Por lo tanto:
        $$
          \begin{array}{cccccl}
                    & x^{(4)} & = & \blue{T_J^4 x^{(0)} + T_J^3b + T_J^2b + T_Jb + b} & = & \blue{T_J^2b + T_Jb + b}   \\
            \red{-} &         &   &                                                   &   &                            \\
                    & x^{(3)} & = & \purple{T_J^3 x^{(0)} + T_J^2b + T_Jb + b}        & = & \purple{T_J^2b + T_Jb + b} \\\hline
          \end{array}
          \to
          \cajaResultado{
            x^{(4)} = x^{(3)}
          }
        $$
        Por lo tanto converge en la tercera iteración.
\end{enumerate}

\begin{aportes}
  \item \aporte{\dirRepo}{naD GarRaz \github}
\end{aportes}
