\begin{enunciado}{\ejExtra}
  {\tiny[\violet{segundo cuatri 2023}]}
  Dada una matriz real $A$, notamos $A = D + L + U$, donde $D$ es diagonal, $L$ triangular inferior estricta
  y $U$ triangular superior estricta.
  \begin{enumerate}[label=(\alph*)]
    \item Probar que $x$ es solución de $Ax = b$ si y solo si $x$ satisface:
          $$
            \textstyle
            (I + \frac{1}{2}L)x = - (D - I + \frac{1}{2}L + U)x + b.
          $$

    \item Considerar el método iterativo derivado de la formulación anterior:
          $$
            \textstyle
            x^{n+1} = Bx^n + c,
          $$
          donde $B = -(I + \frac{1}{2}L)^{-1} (D - I + \frac{1}{2} L + U)$ y $c = (I + \frac{1}{2}L)^{-1}b.$ Probar que $\lambda$
          es un autovalor de $B$ si y solo si $\lambda$ es raíz de la ecuación:
          $$
            \textstyle
            \det\big(D - I + \frac{1}{2}L + U + \lambda(I + \frac{1}{2}L)\big) = 0.
          $$

    \item Para $a \en \reales$ se define
          $$
            \textstyle
            A =
            \matriz{ccc}{
              1 & a & 0 \\
              a & 1+a^2 & a \\
              0 & a & 1
            }.
          $$
          Probar que el método anterior converge para una matriz $A$ si y solo si $|a| < 1$.

    \item Probar que para que el método de Jacobi converja se debe cumplir la misma condición. Deducir
          de esto que la condición para que Gauss-Seidel converja es la misma ¿Qué método es preferible para la
          matriz $A$?
  \end{enumerate}
\end{enunciado}

\begin{enumerate}[label=(\alph*)]
  \item  Para que $x$ sea solución solo hay que hacer un par de cuentas y ver que queda una igualdad:
        $$
          \begin{array}{rcl}
            \textstyle
            (I + \frac{1}{2}L)x = - (D - I + \frac{1}{2}L + U)x + b
             & \sii &
            Ix + \frac{1}{2}Lx = -Dx + Ix - \frac{1}{2}Lx - Ux + b \\
             & \sii &
            Dx + Lx + Ux =  b                                      \\
             & \sii &
            (D + L + U)x =  b                                      \\
             & \sii &
            Ax =  b
          \end{array}
        $$

  \item  $B$ va a tener a $\lambda$ como autovalor si y solo si $|B - \lambda I| = 0$. Hay que acomodar ese determinante feo y llegar
        a esa expresión:
        {\footnotesize
        $$
          \begin{array}{rcl}
            \det\big(
            D - I + \frac{1}{2}L + U + \lambda(I + \frac{1}{2}L)
            \big) = 0
             & \sii          &
            \det\big(
            (\ub{-(I + \frac{1}{2}L)(-(I + \frac{1}{2}L)^{-1})}{I_n}
            (D - I + \frac{1}{2}L + U) + \lambda(I + \frac{1}{2}L)
            \big) = 0          \\
             & \sii          &
            \det\big(
            (-(I + \frac{1}{2}L)B
            + \lambda(I + \frac{1}{2}L)
            \big) = 0          \\
             & \sii          &
            \det\big(
            (-B + \lambda I)(I + \frac{1}{2}L)
            \big) = 0          \\
             & \Sii{\red{!}} &
            \det( -B + \lambda I)
            \cdot
            \ub{\det( I + \frac{1}{2}L)}
            {
              \distinto 0
            }
            = 0                \\
             & \sii          &
            \det( -B + \lambda I)
            = 0                \\
             & \sii          &
            \det( B - \lambda I)
            = 0                \\
          \end{array}
        $$
        }

  \item De la teórica sabemos que:
        $$
          \text{La sucesión } \set{B^k} \text{ converge }
          \sisolosi
          \rho(B) \igual{def} \maximo\set{ |\lambda| : \lambda \text{ autovalor de } B} < 1
        $$
        Por lo tanto quiero calcular los autovalores de la matriz de iteración $B = -(I + \frac{1}{2}L)^{-1} (D - I + \frac{1}{2} L + U)$,
        lo cual, dado lo visto en el ítem anterior, es lo mismo que calcular:
        $$
          \det\big(
          D - I + \frac{1}{2}L + U + \lambda(I + \frac{1}{2}L)
          \big) = 0
        $$
        Cálculo que \textit{no requiere invertir nada}, lo cual nos saca una sonrisa \rosa{\faIcon[regular]{smile}}. Junto los ingredientes para cocinar eso:
        $$
          A =
          \ub{
            \matriz{ccc}{
              1 & 0 & 0 \\
              0 & 1+a^2 & 0 \\
              0 & 0 & 1
            }
          }{
            D
          }
          +
          \ub{
            \matriz{ccc}{
              0 & 0 & 0 \\
              a & 0 & 0 \\
              0 & a & 0
            }
          }{
            L
          }
          +
          \ub{
            \matriz{ccc}{
              0 & a & 0 \\
              0 & 0 & a \\
              0 & 0 & 0
            }
          }{
            U
          }
        $$
        $$
          \textstyle
          D - I + \frac{1}{2}L + U + \lambda(I + \frac{1}{2}L) =
          \matriz{ccc}{
            0 & a & 0 \\
            \frac{a}{2} & a^2 & a \\
            0 & \frac{a}{2} & 0 \\
          }
          +
          \matriz{ccc}{
            \lambda & 0 & 0 \\
            \lambda\frac{a}{2} & \lambda & 0 \\
            0 & \lambda\frac{a}{2} & \lambda
          }
          =
          \matriz{ccc}{
            \lambda & a & 0 \\
            (\lambda + 1)\frac{a}{2} & a^2 + \lambda & a \\
            0 & (\lambda + 1)\frac{a}{2} & \lambda
          }
        $$
        Calculo el determinante de eso:
        $$
          \deter{ccc}{
            \lambda                  & a                        & 0       \\
            (\lambda + 1)\frac{a}{2} & a^2 + \lambda            & a       \\
            0                        & (\lambda + 1)\frac{a}{2} & \lambda
          }
          \igual{\red{!}}
          \lambda^3 - a^2\lambda \igual{\red{!}}
          \lambda \cdot (\lambda - a) \cdot (\lambda + a) = 0
          \sii
          \llave{rcc}{
            \lambda_1 & = & 0 \\
            \lambda_2 & = & a \\
            \lambda_3 & = & -a
          }
        $$
        Por lo tanto para que la matriz de iteración $B$ converga sin importar el vector inicial:
        $$
          \cajaResultado{
            |a| < 1
          }
        $$

  \item La matriz de iteración, $B_J$, para el método de Jacobi:
        $$
          B_J = -D^{-1}(L + U).
        $$
        Con la propiedad que se usó en el ejercicio anterior, para calcular los autovalores de esta $B_J$:
        $$
          \lambda D + L + U =
          \matriz{ccc}{
            \lambda & a & 0 \\
            a & \lambda(1+a^2) & a \\
            0 & a & \lambda
          }
        $$
        Calculo el determinante e igualo a cero:
        $$
          \textstyle
          \deter{ccc}{
            \lambda & a              & 0       \\
            a       & \lambda(1+a^2) & a       \\
            0       & a              & \lambda
          } = 0
          \Sii{\red{!}}
          \lambda \cdot \left(\lambda - \sqrt{\frac{2a^2}{1+a^2}}\right) \cdot \left(\lambda + \sqrt{\frac{2a^2}{1+a^2}}\right) = 0
          \sii
          \llave{rcc}{
            \lambda_1 & = & 0 \\
            \lambda_2 & = & \sqrt{\frac{2a^2}{1+a^2}} \\
            \lambda_3 & = & -\sqrt{\frac{2a^2}{1+a^2}}
          }
        $$
        Por lo tanto el radio espectral de $B$, $\rho(B)$ debe cumplir que:
        $$
          \textstyle
          \rho(B) < 1
          \sii
          \left|\sqrt{\frac{2a^2}{1+a^2}}\right| < 1
          \Sii{\red{!}}
          0 < \frac{2a^2}{1+a^2} < 1
          \Sii{\red{!}}
          0 < 2a^2 < 1 + a^2
          \Sii{\red{!}}
          -a^2 < a^2 < 1
          \Sii{\red{!}}
          \cajaResultado{
            |a| < 1
          }
        $$
        Misma condición que la matriz anterior.

        Lo que sigue ahora queda servido para usar la propiedad de la \textit{tridiagonalidad} que relaciona los
        \textit{radios espectrales} de los métodos de Jacobi y Gauss-Seidel.

        Dado que $A$ es tridiagonal para todo valor de $a$, sé que:
        $$
          \textstyle
          \rho^2(B_J) =
          \rho(B_{GS})
          \sii
          \left(
          \sqrt{\frac{2a^2}{1+a^2}}
          \right)^2 = |\frac{2a^2}{1+a^2}|
        $$
        Para que el método de Gauss-Seidel converja:
        $$
          \textstyle
          |\frac{2a^2}{1+a^2}| < 1
          \Sii{\red{!}}
          |a| < 1
        $$
        Por lo tanto tengo la misma condición que para el método de $Jacobi$.

        Con respecto a la velocidad de convergencia, hay que pensar que lo que se está haciendo, \textit{maomeno}, es multiplicar
        una matriz por sí misma una y otra vez, por lo tanto mientras más rápido se achique más rápido va a converger. Y dado que para
        cualquier norma subordinada:
        $$
          \textstyle
          \rho(B) = \limite{k}{\infinito} \sqrt[n]{\norma{B^k}},
        $$
        mientras más chico el $\rho(B)$ más rápido va a converger.
        $$
          \rho(B_J) < 1
          \ytext
          \rho(B_{GS}) < 1
          \ytext
          (\rho(B_J))^2 = \rho(B_{GS})
          \Sii{\red{!}}
          \cajaResultado{
            \rho(B_J) > \rho(B_{GS})
          }
        $$
        El método de Gauss-Seidel converge más rápido que el de Jacobi para esta matriz $A$.
        Comparo con $\rho(B) = |a|$:
        $$
          \rho(B_{GS}) = \frac{2a^2}{1 + a^2}
          \ytext
          \rho(B) = |a|
        $$
        Abro el módulo de $a$ planteo un caso cuando $a > 0$ y otro cuando $a < 0$:
        $$
          \textstyle
          |a| =
          \llave{cc}{
            \text{si } a > 0 &
            \llave{rcl}{
                \rho(B_{GS}) \mayor{\red{?}} \rho(B)
              &\sii &
              \frac{2a^2}{1 + a^2} > a\\
              & \sii      &
              2a^2 > a + a^3                                                                    \\
              & \sii      &
              0 > a \cdot (1 -2a + a^2) = a \cdot (a - 1)^2 \mayor{\red{!}} 0 \text{ ¡Absurdo!\faIcon{skull} } \\
              & \entonces &
              \cajaResultado{
                \rho(B_{GS}) \menor{\red{\checkmark}} \rho(B)
              }
            }\\
            \text{si } a < 0 &
            \llave{rcl}{
                \rho(B_{GS}) \mayor{\red{?}} \rho(B)
              &\sii&
              \frac{2a^2}{1 + a^2} > -a\\
              & \Sii{\red{!}}      &
              2a^2 \,\red{<}\, a + a^3\\                                                                    \\
              & \sii      &
              0 < a \cdot (1 -2a + a^2) = a \cdot (a - 1)^2 \menor{\red{!}} 0 \text{ ¡Absurdo!\faIcon{skull} } \\
              & \entonces &
              \cajaResultado{
                \rho(B_{GS}) \menor{\red{\checkmark}} \rho(B)
              }
            }
          }
        $$
        Es así que el método de Gauss-Seidel es el más rápido para converger de los tres para la matriz $A$,
        porque tiene el menor \textit{radio espectral}.
\end{enumerate}

\begin{aportes}
  \item \aporte{\dirRepo}{naD GarRaz \github}
\end{aportes}
