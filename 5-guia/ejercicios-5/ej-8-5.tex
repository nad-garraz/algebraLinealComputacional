\begin{enunciado}{\ejercicio}
  Determinar una descomposición en valores singulares de las siguientes matrices:
  \begin{enumerate}[label=(\alph*)]
    \begin{multicols}{2}
      \item
      $\matriz{ccc}{
          1 & -2 & 2 \\
          -1 & 2 & -2
        }$

      \item
      $\matriz{cc}{
          7 & 1 \\
          0 & 0  \\
          5 & 5
        }
      $
    \end{multicols}
  \end{enumerate}
\end{enunciado}

\begin{enumerate}[label=(\alph*)]
  \item
        Si
        $$
          A \igual{$\llamada1$}
          \matriz{ccc}{
            1 & -2 & 2 \\
            -1 & 2 & -2
          }
        $$
        llamo
        $$
          \hat{A} =
          \matriz{cc}{
            1 & -1  \\
            -2 & 2 \\
            2 & -2
          }
        $$
        que no es otra cosa la tonta $A^t$  de antes con un sombrero distinto, sigue teniendo todos los horrendos estereotipos
        de antes, \textit{¡Pero el sombrero es nuevo!}

        Voy a calcular la \textit{descompsición en valores singulares} de $\hat{A} = \hat{U} \hat{\Sigma} \hat{V}^t$ porque así me quedo con la versión de $2 \times 2$ para hacer menos cuentas.
        Una vez calculada esa la \textit{convierto} la descomposición a la de $A$.

        Calculo autovectores de
        $$
          \hat{H} = \hat{A}^t \hat{A} =
          \matriz{cc}{
            9 & -9 \\
            -9 & 9
          }
          \flecha{calculo autovalores}
          |H - \lambda I| = 0
          \sii
          \lambda \en \set{0,18}
          \text{ con }
          \llave{rcl}{
            E_{\lambda=18} & = & \ket{(\frac{1}{\sqrt{2}}, -\frac{1}{\sqrt{2}})}\\
            E_{\lambda=0} & = & \ket{(\frac{1}{\sqrt{2}}, \frac{1}{\sqrt{2}})}
          }
        $$
        Ya tengo:
        $$
          \hat{V} =
          \frac{1}{\sqrt{2}}
          \matriz{cc}{
            1 & 1 \\
            -1 & 1
          }
          \ytext
          \hat{\Sigma} =
          \matriz{cc}{
            3\sqrt{2} & 0 \\
            0 & 0 \\
            0 & 0 \\
          }.
        $$
        Necesito ahora encontrar $\hat{U}$, necesito una base ortonormal de $\reales^3$.
        Como solo tengo un $\sigma \distinto 0$ voy a poder encontrar 1 de los 3 con la fórmula:
        $$
          \hat{u}_1 =
          \frac{\hat{A}\hat{v}_1}{\hat{\sigma}_1}
          =\frac{1}{3}
          \matriz{c}{
            1\\
            -2\\
            2
          },
        $$
        el resto de los vectores puedo hacer \textit{Gram Schmidt} o lo que sea para encontrar
        2 vectores más:
        $$
          (x,y,z) \cdot
          \frac{1}{3}
          \matriz{c}{
            1\\
            -2\\
            2
          } =  0
          \sii
          x -2y + 2z = 0 \sii (x, y, z)
          \en
          \ket{
            \frac{1}{\sqrt{5}}
            \matriz{c}{
              2\\
              1\\
              0
            },
            \frac{1}{3\sqrt{5}}
            \matriz{c}{
              2\\
              -4\\
              -5
            }
          }.
        $$
        Listo tengo:
        $$
          \hat{U} =
          \matriz{ccc}{
            \frac{1}{3} & \frac{2}{\sqrt{5}} & \frac{2}{3\sqrt{5}} \\
            -\frac{2}{3} & \frac{1}{\sqrt{5}} &-\frac{4}{3\sqrt{5}} \\
            \frac{2}{3} &  0& -\frac{5}{3\sqrt{5}}
          }
        $$
        Listo:
        $$
          \hat{A} = \hat{U} \hat{\Sigma} \hat{V}^t
        $$
        Pero yo estoy buscando la \textit{descomposición en valores singulares} de $A$ transpongo:
        $$
          \llamada1 A = \hat{A}^t = \hat{V} \hat{\Sigma}^t \hat{U}^t =
          \frac{1}{\sqrt{2}}
          \matriz{cc}{
            1 & 1 \\
            -1 & 1
          }
          \matriz{ccc}{
            3\sqrt{2} & 0 & 0 \\
            0 & 0 & 0
          }
          \matriz{ccc}{
            \frac{1}{3} & -\frac{2}{3} &  \frac{2}{3}  \\
            \frac{2}{\sqrt{5}}  & \frac{1}{\sqrt{5}} & 0\\
            \frac{2}{3\sqrt{5}} &  -\frac{4}{3\sqrt{5}}& -\frac{5}{3\sqrt{5}}
          }
        $$
        Quedó entonces sin el \textit{sombrero}, la SVD de $A$:
        $$
          \cajaResultado{
            A = U \Sigma V^t =
            \frac{1}{\sqrt{2}}
            \matriz{cc}{
              1 & 1 \\
              -1 & 1
            }
            \matriz{ccc}{
              3\sqrt{2} & 0 & 0 \\
              0 & 0 & 0
            }
            \matriz{ccc}{
              \frac{1}{3} & -\frac{2}{3} &  \frac{2}{3}  \\
              \frac{2}{\sqrt{5}}  & \frac{1}{\sqrt{5}} & 0\\
              \frac{2}{3\sqrt{5}} &  -\frac{4}{3\sqrt{5}}& -\frac{5}{3\sqrt{5}}
            }
          }
        $$

  \item \hacer
\end{enumerate}
