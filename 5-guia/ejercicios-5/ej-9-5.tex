\begin{enunciado}{\ejercicio}
  Sea
  $$
    A =
    \matriz{cc}{
      2 & 14 \\
      8 & -19 \\
      20 & -10
    }
  $$
  Probar que para todo $v \en \reales^2$ se tiene $\norma{Av}_2 \geq 15 \norma{v}_2$.
\end{enunciado}

\textit{Let's calculate los singular values:}
$$
  \purple{H} =
  A^* \cdot A =
  \matriz{ccc}{
    2 & 8 & 20 \\
    14 & -19 & -10
  }
  \cdot
  \matriz{cc}{
    2 & 14 \\
    8 & -19 \\
    20 & -10
  }
  =
  \matriz{cc}{
    480 & -296\\
    -296  &657
  }
$$
\parrafoDestacado{
  \textit{¿Por qué esos números feos?}
}

Calculo \textit{autovalores} de $\purple{H}$:
$$
  \det(H - \lambda I) = 0
  \sii
  \llave{l}{
    \lambda_1 \approx 259.55\\
    \lambda_2 \approx 877.45
  }
$$
Los valores singulares sería:
$$
  \llave{l}{
    \sigma_1  = \sqrt{\lambda_1} \approx 29.62 \\
    \sigma_2  = \sqrt{\lambda_2} \approx 16.11
  }
$$
Para todo $v \en \reales^2$ con $\norma{v}_2 = 1$ se va a cumplir que:
$$
  \sigma_2 \leq \norma{Av}_2 \leq \sigma_1
  \equivalente
  16.11 \leq \norma{Av}_2 \leq 29.62
  \sii
  \cajaResultado{
    15 \leq \norma{Av}_2 \leq 30
  }
  \quad \paratodo v \en \reales^2,\, \norma{v}_2 = 1
$$

$$
  \begin{tikzpicture}[
      scale = 1.5,
      every pin/.style = {font=\tiny,
          fill=white,
          draw= black},
      every node/.style = {font=\footnotesize}]
    \begin{axis}[
        enlargelimits = false,
        xmin=-50, xmax=50,
        ymin=-25, ymax=25,
        zmin=-25, ymax=25,
        view={110}{20},
        3d box,
        xtick={-25, 0, 25},
        ytick={-25, 0, 25},
        ztick={-25, 0, 25},
        axis equal,
        xlabel={$x_1$},
        ylabel={$x_2$},
        zlabel={$x_3$},
        title={Scatter para 200 $\blue{\bm{x}} / \norma{\blue{\bm{x}}}_2 = 1$ y para 200 $\purple{Ax}$},
        legend style={
            cells={anchor=center},
            font =\tiny,
            anchor=center,
            at = {(0.5,-0.25)},
            legend columns=2
          },
        legend entries =
          {
            $\set{\blue{x}: x \en \reales^2, \norma{x}_2 = 1}$,
            $\set{\purple{Ax}: x \en \reales^2, \norma{x}_2 = 1}$,
          },
        cycle list name=color list,
        every axis plot/.append style={only marks, mark size=2pt},
        xminorticks=true,
        yminorticks=true,
        grid=both,
      ]
      \addplot3[mark size = 0.3pt, Cerulean, opacity=1 ] table {./ejercicios-5/dataFiles/9-ej-data/circulo.data};
      \addplot3[mark size = 0.1pt, Cerulean, opacity=0.2] table [x expr=-50] {./ejercicios-5/dataFiles/9-ej-data/circulo.data};
      \addplot3[mark size = 0.1pt, Cerulean, opacity=0.2] table [y expr=-50] {./ejercicios-5/dataFiles/9-ej-data/circulo.data};
      \addplot3[mark size = 0.1pt, Cerulean, opacity=0.2] table [z expr=-25] {./ejercicios-5/dataFiles/9-ej-data/circulo.data};

      \addplot3[mark size = 0.3pt, purple, opacity=1] table {./ejercicios-5/dataFiles/9-ej-data/matrizDotCirculo.data};
      \addplot3[mark size = 0.1pt, purple, opacity=0.2] table [x expr=-50] {./ejercicios-5/dataFiles/9-ej-data/matrizDotCirculo.data};
      \addplot3[mark size = 0.1pt, purple, opacity=0.2] table [y expr=-50] {./ejercicios-5/dataFiles/9-ej-data/matrizDotCirculo.data};
      \addplot3[mark size = 0.1pt, purple, opacity=0.2] table [z expr=-25] {./ejercicios-5/dataFiles/9-ej-data/matrizDotCirculo.data};

    \end{axis}
  \end{tikzpicture}
$$

\begin{aportes}
  \item \aporte{\dirRepo}{naD GarRaz \github}
\end{aportes}
