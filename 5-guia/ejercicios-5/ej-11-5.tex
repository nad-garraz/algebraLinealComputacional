\begin{enunciado}{\ejercicio}
  Sea $A \en \complejos^{m \times n}$, demostrar que los valores singulares de la matriz
  $\matriz{c}{
      I_n\\
      A
    }$
  son $\sqrt{1 + \sigma_i^2}$ donde $I_n$ es la matriz identidad de $\complejos^{n \times n}$ y $\sigma_i$ es
  el $i$-ésimo valor singular de $A$.
\end{enunciado}
Calculo:
$$
  \blue{G} =
  (I_n\ A^*) \cdot
  \matriz{c}{
    I_n\\
    A
  }
  =
  I_n + \ub{A^* A}{\en \complejos^{n \times n}} = I_n  + \purple{H}
$$
Calculo los autovalores de $I_n + \purple{H}$:
$$
  |I_n + \purple{H} - \lambda \cdot I_n| =
  |\purple{H} - \ub{(\lambda - 1)}{\mu}\cdot I_n| =
  |\purple{H} - \mu \cdot I_n|  = 0
  \sii
  \mu \text{ autovalores de } \purple{H} = A^*A
$$
Ahora identificando bien cada cosa:

Si $\mu_i$ es un autovalor de $\purple{H}$, entonces los valores singulares de $A$:
$$
  \sigma_i = \sqrt{\mu_i} = \sqrt{\lambda_i - 1}\ \llamada1
$$
es un valor singular de $A$.

Y si tengo que $\lambda_i$ es un autovalor de $\blue{G}$, entonces los valores singulares de
$
  \matriz{c}{
    I_n\\
    A
  }
$:
$$
  \cajaResultado{
    \blue{\varsigma_i} = \sqrt{\lambda_i} \igual{$\llamada1$} \sqrt{1 + \sigma_i^2}
  }
$$

\begin{aportes}
  \item \aporte{\dirRepo}{naD GarRaz \github}
\end{aportes}
