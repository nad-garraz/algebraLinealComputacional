\begin{enunciado}{\ejercicio}
  Caracterizar geométricamente y graficar la imagen de la esfera unitaria
  $$
    S_2 = \set{\bm{x} \en \reales^3 : \norma{\bm{x}}_2 = 1}
  $$
  por la transformación $T(\bm{x} ) = A\bm{x}$, con
  $$
    A =
    \matriz{cc}{
      1 & \frac{-2}{\sqrt{5}}  \\
      2 & \frac{1}{\sqrt{5}}
    }
    \matriz{ccc}{
      2 & 0 & 0  \\
      0 & 1 & 0
    }
    \matriz{ccc}{
      2 & 1 & 2  \\
      2 & -2 & -1 \\
      1 & 2 & -2
    }.
  $$
\end{enunciado}

Acomodo la expresión de $A$, para que sea una \textit{descomposición en valores singulares}:
$$
  A =
  U \Sigma V^t
  \igual{\red{!}}
  \matriz{cc}{
    \frac{1}{\sqrt{5}} & \frac{-2}{5}  \\
    \frac{2}{\sqrt{5}} & \frac{1}{5}
  }
  \matriz{ccc}{
    6\sqrt{5} & 0 & 0  \\
    0 & 3\sqrt{5} & 0
  }
  \matriz{ccc}{
    \frac{2}{3} & \frac{1}{3} & \frac{2}{3}  \\
    \frac{2}{3} & \frac{-2}{3} & \frac{-1}{3} \\
    \frac{1}{3} & \frac{2}{3} & \frac{-2}{3}
  }
$$
Y la expresión reducida:
$$
  A =
  \hat{U} \hat{\Sigma} \hat{V}^t
  \igual{\red{!}}
  \matriz{cc}{
    \frac{1}{\sqrt{5}} & \frac{-2}{5}  \\
    \frac{2}{\sqrt{5}} & \frac{1}{5}
  }
  \matriz{cc}{
    6\sqrt{5} & 0   \\
    0 & 3\sqrt{5}
  }
  \matriz{ccc}{
    \frac{2}{3} & \frac{1}{3} & \frac{2}{3}  \\
    \frac{2}{3} & \frac{-2}{3} & \frac{-1}{3}
  }
  =
  \frac{1}{5}
  \matriz{ccc}{
    20 - 4\sqrt{5} & 20 + 4\sqrt{5} & 20 + 2 \sqrt{5} \\
    40 + 2\sqrt{5} & 20 - 2\sqrt{5} & 40 + \sqrt{5}
  }
$$

$$
  \begin{tikzpicture}[
      scale = 2,
      every pin/.style = {font=\tiny,
          fill=white,
          draw= black},
      every node/.style = {font=\tiny}]
    \begin{axis}[
        % enlargelimits = false,
        xmin=-50, xmax=50,
        ymin=-70, ymax=70,
        zmin=-10, zmax=15,
        view={120}{30},
        % view={0}{90},
        % 3d box,
        xtick={-75, -60, -40, -20, 0, 20, 30, 40, 60, 75},
        ytick={-50, -30, -10, 0, 10, 30, 50, 60},
        ztick={-30, -20, -10, 0, 10, 20, 30},
        xminorticks=true,
        yminorticks=true,
        axis equal,
        xlabel={$x_1$},
        ylabel={$x_2$},
        zlabel={$x_3$},
        title={Scatter para 500 $\blue{\bm{x}} \en \reales^3 / \norma{\blue{\bm{x}}}_2 = 1$ y para 500 $\purple{Ax}\en \reales^2$  },
        legend style={
            cells={anchor=center},
            font =\tiny,
            anchor=center,
            at = {(0.5,-0.25)},
            legend columns=2,
            opacity=1
          },
        legend entries =
          {
            $\set{\blue{x}: x \en \reales^3, \norma{x}_2 = 1}$,
            $\set{\purple{Ax \en \reales^2}: x \en \reales^3, \norma{x}_2 = 1}$,
          },
        cycle list name=color list,
        every axis plot/.append style={only marks, mark size=1pt},
        grid=both
      ]
      \addplot3[mark size = 0.1pt, Cerulean, opacity=0.2] table[x expr=-69]  {./ejercicios-5/dataFiles/17-ej-data/circulo.data};
      \addplot3[mark size = 0.1pt, Cerulean, opacity=0.2] table [y expr=-70] {./ejercicios-5/dataFiles/17-ej-data/circulo.data};
      \addplot3[mark size = 0.1pt, Cerulean, opacity=0.3] table [z expr=-33] {./ejercicios-5/dataFiles/17-ej-data/circulo.data};
      \addplot3[mark size = 0.4pt, Cerulean, opacity = 1] table {./ejercicios-5/dataFiles/17-ej-data/circulo.data};

      \addplot3[mark size = 0.1pt, purple, opacity=0.2] table [x expr=-69] {./ejercicios-5/dataFiles/17-ej-data/matrizDotCirculo.data};
      \addplot3[mark size = 0.1pt, purple, opacity=0.2] table [y expr=-70] {./ejercicios-5/dataFiles/17-ej-data/matrizDotCirculo.data};
      \addplot3[mark size = 0.1pt, purple, opacity=0.5] table [z expr=-33] {./ejercicios-5/dataFiles/17-ej-data/matrizDotCirculo.data};
      \addplot3[mark size = 0.4pt, purple, opacity=1] table {./ejercicios-5/dataFiles/17-ej-data/matrizDotCirculo.data};

    \end{axis}
  \end{tikzpicture}
$$

Si bien la perspectiva puede ser engañoza, la transformación de la \blue{esfera unitaria} da como
resultado una \purple{superficie elíptica en dos dimensiones} que está ubicada en el plano $x_3 = 0$, como se puede ver
en las proyecciones a los planos catesianos.
$$
  A
  =
$$
