\begin{enunciado}{\ejercicio}
  Sea $A =
    \matriz{ccc}{
      4 & \alpha + 2 & 2 \\
      \alpha^2 & 4 & 2 \\
      2 & 2 & 1
    }$
  \begin{enumerate}[label=(\alph*)]
    \item\label{ej-6:itema} Hallar los valores de $\alpha \en \reales$ para que $A$ sea simétrica y $\lambda = 0$ sea
          autovalor de $A$.

    \item Para el valor de $\alpha$ hallado en \ref{ej-6:itema}, diagonalizar ortonormalmente la matriz $A$.
  \end{enumerate}
\end{enunciado}

\begin{enumerate}[label=(\alph*)]
  \item Quiero que $A$ sea simétrica:
        $$
          A = A^t
          \sii
          \alpha \en \set{-1, 2}
        $$
        $$
          A_{\alpha = 2} =
          \matriz{ccc}{
            4 & 4 & 2 \\
            4 & 4 & 2 \\
            2 & 2 & 1
          }
          \quad
          \ytext
          \quad
          A_{\alpha = -1} =
          \matriz{ccc}{
            4 & 1 & 2 \\
            1 & 4 & 2 \\
            2 & 2 & 1
          }
        $$
        Noto que si $\alpha = 2$ la matriz queda con filas \textit{linealmente dependientes},
        por lo tanto cuando $\alpha = 2$ tengo autovalor $\lambda = 0$. Podría triangular la matriz con
        $\alpha = -1$, para ver si hay alguna fila \textit{linealmente dependiente}, pero no hay ganas.

  \item Dado que $A$ es una matriz simétrica, es \textit{ortonormalmente diagonalizable}. Hay que diagonalizar
        asegurando que la base de \textit{autovectores} sea una BON. El procedimientos puede hacerse como cualquier
        diagonalización, pero acá voy a \textit{explotar \faIcon{bomb}} el hecho de que la \textit{base de autovectores}
        va a ser ortogonal para distintos autovalores.

        Busco autovectores de $\lambda = 0$, que equivale a buscar elementos del núcleo de la matriz $A$ a ojo:
        $$
          \begin{array}{rcl}
            (A - \blue{\lambda}I)v_{_{(\lambda = 0)}} = 0
             & \sisolosi                         &
            v_{_{(\lambda = 0)}} \en \set{(1, -1, 0), (0,1, -2)} \\
             & \Sii{ortonormalizo}[Gram-Schmidt] &
            v_{_{(\lambda = 0)}} \en
            \set{
              \frac{1}{\sqrt{2}}(1, -1, 0),
              \frac{1}{\sqrt{2}}(\frac{1}{3},\frac{1}{3}, -\frac{4}{3})
            }                                                    \\
          \end{array}
        $$
        Como estoy en $\reales^3$ no hay muchas opciones para el vector restante, tiene que ser ortogonal a esos dos:
        $$
          \llave{rcl}{
            \frac{1}{\sqrt{2}}(1, -1, 0)\cdot (x,y,z)    & = & 0\\
            \frac{1}{\sqrt{2}}(\frac{1}{3}, \frac{1}{3}, -\frac{4}{3}) \cdot (x,y,z)    & = & 0
          }
          \Sii{\red{!}}
          (x,y,z)  = \frac{1}{3}(2,2,1)
        $$

        Ahora quiero ver a que autovalor corresponde:
        $$
          A v=
          \matriz{ccc}{
            4 & 4 & 2 \\
            4 & 4 & 2 \\
            2 & 2 & 1
          }
          \matriz{c}{
            2  \\
            2  \\
            1
          }
          =
          \matriz{c}{
            18  \\
            18  \\
            9
          }
          = \blue{9}
          \matriz{c}{
            2  \\
            2  \\
            1
          }
        $$
        Tengo así la siguiente \textit{base ortonormal} para diagonalizar la matriz:
        $$
          \textstyle
          \text{BON } =
          \big\{
          \ub{
            \frac{1}{\sqrt{2}}(1, -1, 0),
            \frac{1}{\sqrt{2}}(\frac{1}{3},\frac{1}{3}, -\frac{4}{3})
          }{
            E_{_{(\lambda = 0)}}
          },
          \ub{
            \frac{1}{3}(2,2,1)
          }{
            E_{_{(\lambda = 9)}}
          }
          \big\}
        $$
        Y ahora queda fácil, porque la inversa de la matriz de autovectores $C$ es $C^t$,
        dado que es una \textit{matriz ortogonal} (o \textit{matriz unitaria} si $\en \complejos$):
        $$
          A =
          \matriz{ccc}{
            \frac{1}{\sqrt{2}}  &   \frac{1}{3\sqrt{2}}  & \frac{2}{3} \\
            -\frac{1}{\sqrt{2}} & \frac{1}{3\sqrt{2}}    & \frac{2}{3} \\
            0                   & -\frac{2\sqrt{2}}{3}   & \frac{1}{3}
          }
          \matriz{ccc}{
            0 & 0 & 0 \\
            0 & 0 & 0 \\
            0 & 0 & 9 \\
          }
          \matriz{ccc}{
            \frac{1}{\sqrt{2}} & -\frac{1}{\sqrt{2}} & 0 \\
            \frac{1}{3\sqrt2} & \frac{1}{3\sqrt2} & -\frac{2\sqrt{2}}{3} \\
            \frac{2}{3} & \frac{2}{3} & \frac{1}{3}
          }
        $$
\end{enumerate}

\begin{aportes}
  \item \aporte{\dirRepo}{naD GarRaz \github}
\end{aportes}
