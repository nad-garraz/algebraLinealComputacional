\begin{enunciado}{\ejercicio}
  Sea
  $$
    A =
    \matriz{ccc}{
      1 & 4 & 0 \\
      -4 & -1 & 0 \\
      0 & 0 & 2
    }
  $$
  \begin{enumerate}[label=(\alph*)]
    \item Hallar la matriz de rango 2 que mejor aproxima a $A$ en norma 2.
    \item Hallar la matriz de rango 1 que mejor aproxima a $A$ en norma 2.
  \end{enumerate}
\end{enunciado}

\begin{enumerate}[label=(\alph*)]
  \item Tengo que calcular la \textit{descomposición en valores singulares}:
        $$
          H = A^tA
          =
          \matriz{ccc}{
            17 &  8 &  0 \\
            8 & 17 & 0 \\
            0 & 0 & 4
          }
        $$
        Busco autovalores de $H$:
        $$
          |H - \lambda I| = 0
          \sii
          \lambda \en \set{4, 9, 25}
          \text{ y autovectores }
          Hv_{\lambda} = \lambda v_{\lambda}
          \sii
          \llave{l}{
            E_{\lambda = 25} = \ket{\frac{1}{\sqrt{2}}(1,1,0)}\\
            E_{\lambda = 9} = \ket{\frac{1}{\sqrt{2}}(1,-1,0)}\\
            E_{\lambda = 4} = \ket{(0,0,1)}
          }
        $$
        Los autovalores de una matriz simétrica resultaron todos distintos, por lo tanto los autovectores resultaron ortogonales.
        Por lo tanto tengo a la matriz $V$ y $\Sigma$:
        $$
          V =
          \frac{1}{\sqrt{2}}
          \matriz{ccc}{
            1 & 1 & 0 \\
            1 & -1 & 0 \\
            0 & 0 & \sqrt{2}
          }
          \ytext
          \Sigma =
          \matriz{ccc}{
            5 & 0 & 0 \\
            0 & 3 & 0 \\
            0 & 0 & 2
          }
        $$
        Ahora necesito la $U$, que la consigo con una BON:
        $$
          \textstyle
          \set{u_1, u_2, u_3} =
          \set{\frac{Av_1}{\sigma_1},\frac{Av_2}{\sigma_2},\frac{Av_3}{\sigma_3}} =
          \set{
            \frac{1}{\sqrt{2}}(1, -1, 0),
            \frac{1}{\sqrt{2}}(-1, -1, 0),
            (0, 0, 1)
          }
          \entonces
          U =
          \frac{1}{\sqrt{2}}
          \matriz{ccc}{
            1 & 1 & 0 \\
            1 & -1 & 0 \\
            0 & 0 & \sqrt{2}
          }
        $$
        Por lo tanto la descomposición queda:
        $$
          \cajaResultado{
            A =
            U \Sigma V^t=
            \frac{1}{\sqrt{2}}
            \matriz{ccc}{
              1 & -1 & 0 \\
              -1 & -1 & 0 \\
              0 & 0 & \sqrt{2}
            }
            \matriz{ccc}{
              5 & 0 & 0 \\
              0 & 3 & 0 \\
              0 & 0 & 2
            }
            \frac{1}{\sqrt{2}}
            \matriz{ccc}{
              1 & 1 & 0 \\
              1 & -1 & 0 \\
              0 & 0 & \sqrt{2}
            }
          }
        $$
        La matriz de rango 2 que mejor aproxima a $A$:
        $$
          \cajaResultado{
            B =
            U \Sigma V^t=
            \frac{1}{\sqrt{2}}
            \matriz{ccc}{
              1 & -1 & 0 \\
              -1 & -1 & 0 \\
              0 & 0 & \sqrt{2}
            }
            \matriz{ccc}{
              5 & 0 & 0 \\
              0 & 3 & 0 \\
              0 & 0 & 0
            }
            \frac{1}{\sqrt{2}}
            \matriz{ccc}{
              1 & 1 & 0 \\
              1 & -1 & 0 \\
              0 & 0 & \sqrt{2}
            }
            =
            \matriz{ccc}{
              1 & 4 & 0 \\
              -4 & -1 & 0 \\
              0 & 0 & 0
            }
          }
        $$

  \item
        La de rango 1:
        $$
          \cajaResultado{
            B =
            U \Sigma V^t=
            \frac{1}{\sqrt{2}}
            \matriz{ccc}{
              1 & -1 & 0 \\
              -1 & -1 & 0 \\
              0 & 0 & \sqrt{2}
            }
            \matriz{ccc}{
              5 & 0 & 0 \\
              0 & 0 & 0 \\
              0 & 0 & 0
            }
            \frac{1}{\sqrt{2}}
            \matriz{ccc}{
              1 & 1 & 0 \\
              1 & -1 & 0 \\
              0 & 0 & \sqrt{2}
            }
            =
            \matriz{ccc}{
              5 & 5 & 0 \\
              -5 & -5 & 0 \\
              0 & 0 & 0
            }
          }
        $$
\end{enumerate}

\begin{aportes}
  \item \aporte{\dirRepo}{naD GarRaz \github}
\end{aportes}
