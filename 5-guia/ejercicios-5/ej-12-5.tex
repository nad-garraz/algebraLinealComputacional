\begin{enunciado}{\ejercicio}
  Sea $A \en \complejos^{n \times n}$ y $\sigma > 0$. Demostrar que $\sigma$ es valor singular de A
  si y solo si la matriz
  $\matriz{cc}{
      A^* & -\sigma I_n \\
      -\sigma I_n & A
    }$
  es singular, donde $I_n$ es la matriz identidad de $\complejos^{n \times n}$.
\end{enunciado}

\begin{itemize}
  \item[$(\red{\Rightarrow})$] Sé que $\sigma$ es un \textit{valor singular} de $A$.
        Entonces sé que una matriz $A$ tiene su descomposición SVD:
        $$
          A^*A v = \lambda v
          \Sii{def}
          A^*A v = \sigma^2 v
          \sii
          (A^*A - \sigma^2 I_n) v
          \Sii{$v \distinto 0$}
          |A^*A - \sigma^2 I_n| = 0
          \sii \sigma \text{ es valor singular de } A
        $$
        La expresión $|A^*A - \sigma^2 I_n|$ es igual al determinante de la matriz
        $\matriz{cc}{
            A^* & -\sigma I_n \\
            -\sigma I_n & A
          }$

  \item[$(\red{\Leftarrow})$]
        La matriz es singular, lo que quiere decir que su determinante es cero:
        $$
          \det
          \matriz{cc}{
            A^* & -\sigma I_n \\
            -\sigma I_n & A
          } =
          \det(A^*A - \sigma^2I_n) = 0.
        $$
        Esta última expresión es la ecuación del polinomio característico de la matriz $A^*A$ en la variable $\sigma^2$,
        las raíces del polinomio son los autovalores de $A^*A$ y por definición la raíz de esos autovalores
        $\sqrt{\sigma^2} \igual{$\oa{}{\sigma \geq 0}$} \sigma$ son los valores singulares de $A$.
\end{itemize}

\begin{aportes}
  \item \aporte{\dirRepo}{naD GarRaz \github}
\end{aportes}
