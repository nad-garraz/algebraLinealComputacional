\begin{enunciado}{\ejercicio}
  Sea $A \en \complejos^{n \times n}$ y $\sigma > 0$. Demostrar que $\sigma$ es valor singular de A
  si y solo si la matriz
  $\matriz{cc}{
      A^* & -\sigma I_n \\
      -\sigma I_n & A
    }$
  es singular, donde $I_n$ es la matriz identidad de $\complejos^{n \times n}$.
\end{enunciado}

\begin{itemize}
  \item[$(\red{\Rightarrow})$] Sé que $\sigma$ es un \textit{valor singular} de $A$. Calculo el determinante:
        $$
          \det
          \matriz{cc}{
            A^* & -\sigma I_n \\
            -\sigma I_n & A
          } =
          \det(A^*A - \sigma^2I_n).
        $$
        Y si
        $\sigma > 0$ es un \text{valor singular} de $A$, entonces $\sigma^2 = \lambda$ con $\lambda$ autovalor de $A^*A$
        $$
          \det(A^*A - \ua{\lambda}{\sigma^2}I_n) = 0
        $$
        Entonces la matriz
        $
          \matriz{cc}{
            A^* & -\sigma I_n \\
            -\sigma I_n & A
          }$ tiene determinante nulo, es decir que es singular si $\sigma$ es un valor singular de $A$.

  \item[$(\red{\Leftarrow})$]
        ¿Es lo mismo que el otro pero en reversa?
        Sé que $\det\matriz{cc}{
            A^* & -\sigma I_n \\
            -\sigma I_n & A
          } =
          \det(\ub{A^*A - \sigma^2I_n}{\text{ecuación}\\\text{característica}})=
          0$
        La ecuación característica da 0 para los autovalores de $A^*A$, por lo tanto $\sqrt{\sigma^2} \igual{$\oa{}{\sigma>0}$} \sigma$ tiene que ser
        un \textit{valor singular} de $A$.

          {\huge \red{CONSULTAR}, esta demo con gusto a mal}
\end{itemize}

\begin{aportes}
  \item \aporte{\dirRepo}{naD GarRaz \github}
\end{aportes}
