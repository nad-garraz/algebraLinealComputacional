\begin{enunciado}{\ejercicio}
  Probar que $A \en K^{n \times n}$ es hermitiana y definida positiva si y solo si $A$ es unitariamente semejante
  a una matriz diagonal real con elementos de la diagonal positivos.
\end{enunciado}

Hay que probar una doble implicación.

Para una matriz $A \en K^{n \times n}$ con autovector $v$ asociado a un autovalor $\lambda$:
\begin{itemize}
  \item[($\red{\Rightarrow}$)]
        $$
          \begin{array}{l}
            A v = \lambda v
            \Sii{$\times v^*$}[$\to$]
            v^* A v = \lambda v^*v
            \sii
            v^* A v \igual{$\llamada1$} \lambda \norma{v}_2^2 \\
            A v = \lambda v
            \Sii{$^*$}
            v^* A^* = \conj{\lambda} v^*
            \Sii{$\times v$}[$\ot$]
            v^* A^* v = \conj{\lambda} v^*v
            \sii
            v^* A^* v \igual{$\llamada2$} \conj{\lambda} \norma{v}_2^2
          \end{array}
        $$
        Como $A = A^*$ el miembro izquierdo en $\llamada1$ y $\llamada2$ es igual.
        Por lo tanto $\lambda = \conj{\lambda}
          \sii
          \lambda \en \reales$.

        Ahora si $A$ es una matriz definida positiva:
        $$
          Av = \lambda v
          \Sii{$\times v$}[$\to$]
          \ub{v^*Av}{>0 \text{ si } v\distinto 0} = \lambda v^*v = \lambda \cdot \norma{v}_2^2 > 0
          ,~\paratodo v \distinto 0
          \sii \lambda > 0
        $$
        Hasta acá, tengo \textit{autovalores reales y positivos}, tengo que ver que los autovectores tienen que ser ortogonales.
        Dado 2 autovectores $v_1$ y $v_2$ asociados a \ul{distintos} autovalores $\lambda_1$ y $\lambda_2$ respectivamente:
        $$
          \begin{array}{l}
            Av_1 = \lambda_1 v_1
            \ytext
            Av_2 = \lambda_2 v_2
            \sii
            \llave{rcl}{
            v_2^* A v_1 \igual{\red{!}} (Av_2)^* v_1 = \lambda_2v_2^* \cdot v_1 & \igual{$\llamada3$} & \lambda_1 v_2^* \cdot v_1 \\
            v_1^* A v_2 = \lambda_2 v_1^* \cdot v_2                             & \igual{$\llamada4$} & \lambda_2 v_1^* \cdot v_2
            }
          \end{array}
        $$
        Restando $\llamada3$ y $\llamada4$ y teniendo en cuenta que le producto interno entre 2 vectores es conmutativo:
        $$
          0 \igual{\red{!}}
          \ub{
            (\lambda_1 - \lambda_2)
          }{
            \distinto 0
          }
          (v_1^* \cdot v_2)
          \sii
          v_1 \perp v_2
        $$
        Para el caso en que tenga autovalores de multiplicidad mayor a 1, puedo hacer Gram-Schmidt para conseguir los vectores ortogonales.

  \item[($\red{\Leftarrow}$)]
        En este caso:
        $$
          A = UDU^*
          \ytext
          A^* = (UDU^*)^*
          \igual{\red{!}}
          UDU^*
          \sisolosi
          A = A^*
        $$
        En el \red{!} use que los elementos diagonales de $D$ son reales. La matriz $A$ es hermiticana dado que es
        igual a su autoadjunta o a su transpuesta conjugada, como más te guste decirle.

        Por otro lado la \ul{matriz diagonal $D$ tiene todos sus elementos positivos}, es una matriz definida positiva:
        $$
          w^* A w = w^* U D U^* w
          \sisolosi
          w^* A w =
          \ub{
            (U^* w)^*
          }{
            \omega^*
          } D
          \ub{
            U^* w
          }{
            \omega
          } = \omega^* D \omega > 0 \paratodo \omega \distinto 0
        $$
        Eso último es:
        $$
          (\omega_1^*\ \cdots \ \omega_n^*)
          \matriz{ccc}{
            d_{11} & & 0\\
            & \ddots & \\
            0&  & d_{nn}
          }
          \matriz{c}{
            \omega_1 \\
            \vdots\\
            \omega_n
          }
          = 
          d_{11} w^2_1 +
          \cdots + 
          d_{nn} w^2_n > 0
        $$
        Ya que $d_{ii} > 0$ por hipótesis.
\end{itemize}
