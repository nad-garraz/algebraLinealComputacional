\begin{enunciado}{\ejercicio}
  Mostrar que $A \en \complejos^{n \times n}$ tiene un valor singular nulo si y solo si tiene un autovalor nulo.
\end{enunciado}
\begin{itemize}
  \item[$(\red{\Leftarrow})$]
        Si $A$ tiene un autovalor $\lambda_i = 0$ tiene $\nucleo(A) \distinto 0$ y existe $Av = 0$ para algún $v$.
        Entonces $A^*A$:
        $$
          A^* A v = 0
        $$
        Por lo tanto $A^*A$ tiene un autovalor nulo y como $\sigma_i^2 = \lambda_i$ hay un valor singular nulo.

  \item[$(\red{\Rightarrow})$]
        Si $A$ es cuadrada, su descomposición \textit{en valores singulares} es el producto de matrices cuadradas:
        $$
          A = U \Sigma V^*
          \flecha{calculo}[determinante]
          |A| =
          |U \Sigma V^*| =
          \ua{|U|}{\distinto 0} \cdot \ua{|\Sigma|}{= 0} \cdot \ua{|V^*|}{\distinto 0} = 0
        $$
        Porque sigma tiene la forma:
        $$
          [\Sigma]_{ij} =
          \llave{rcl}{
            \sigma_i & si & i = j\\
            0 & si & i \distinto j
          }
        $$
        Y si uno de los $\sigma_i = 0$, bueh, $\det(A) = 0$. Por lo tanto
        $$
          \nucleo(A) \distinto \set{0}
        $$
        Entonces existe un $v$ tal que:
        $$
          Av = 0 \sii Av = \ua{0}{\lambda_i} \cdot v
        $$
\end{itemize}

\begin{aportes}
  \item \aporte{\dirRepo}{naD GarRaz \github}
\end{aportes}
