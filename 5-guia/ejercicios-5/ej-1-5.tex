\begin{enunciado}{\ejercicio}
  Dada la matriz
  $$
    A =
    \matriz{ccc}{
      13 & 8 & 8 \\
      -1 & 7 & -2 \\
      -1 & -2 & 7
    }
  $$
  \begin{enumerate}[label=(\alph*)]
    \item Hallar una descomposición de Schur $A = UTU^*$, con $U$ unitaria y $T$ triangular superior
          con los autovalores de la matriz $A$ en la diagonal.

    \item Descomponer a la matriz $T$ hallada en el ítem anterior como suma de una matriz diagonal $D$ y una matriz triangular
          superior $S$ con ceros en la diagonal. Probar que $S^j = 0$ para todo $j \geq 2$.
  \end{enumerate}
\end{enunciado}

\begin{enumerate}[label=(\alph*)]
  \item Busco \textit{autovalores} y \textit{autovectores} de $A$:
        $$
          |A - \lambda I| = 0
          \sii
          \lambda = 9
          \quad \text{ con } \quad
          E_{\lambda = 9} = \ket{\textstyle\frac{1}{\sqrt{5}}(-2,1,0), \frac{1}{\sqrt{5}}(-2, 0, 1)}
        $$
        Solo salieron 2 autovectores del único autovalor $\lambda = 9$. Esto nos dice que la matriz no es diagonalizable. Pero
        nadie nos pidió que diagonalicemos, así que ahora para encontrar la \textit{descomposición de Schur}
        expando a una \textit{base ortonormal} de $\reales^3$:
        $$
          \text{BON} =
          \set{
            \textstyle \frac{1}{\sqrt{5}}(-2,1,0), \frac{1}{5}(-2, -4, 5), \blue{\frac{1}{3}(1, 2, 2)}
          }
        $$
        Donde usé Gram Schmidt para calcular el autovector $(-\frac{2}{5}, -\frac{4}{5}, 1)$ y también para calcular
        un \blue{vector extra para formar todo $\reales^3$}.

        Entonces tengo ya la base para encontrar la matriz unitaria $U_1$:
        $$
          U_1 =
          \matriz{ccc}{
            -\frac{2}{\sqrt{5}} & -\frac{2}{3\sqrt{5}} & \frac{1}{3} \\
            \frac{1}{\sqrt{5}} & -\frac{4}{3\sqrt{5}} & \frac{2}{3} \\
            0 & \frac{5}{3\sqrt{5}} & \frac{2}{3}
          }
        $$
        Ahora calculo:
        $$
          \begin{array}{rcl}
            U_1^t A U_1
                                 & =                    &
            \matriz{ccc}{
            -\frac{2}{\sqrt{5}}  & \frac{1}{\sqrt{5}}   & 0                    \\
            -\frac{2}{3\sqrt{5}} & -\frac{4}{3\sqrt{5}} & \frac{5}{3\sqrt{5}}  \\
            \frac{1}{3}          & \frac{2}{3}          & \frac{2}{3}
            }
            \matriz{ccc}{
            13                   & 8                    & 8                    \\
            -1                   & 7                    & -2                   \\
            -1                   & -2                   & 7
            }
            \matriz{ccc}{
            -\frac{2}{\sqrt{5}}  & -\frac{2}{3\sqrt{5}} & \frac{1}{3}          \\
            \frac{1}{\sqrt{5}}   & -\frac{4}{3\sqrt{5}} & \frac{2}{3}          \\
            0                    & \frac{5}{3\sqrt{5}}  & \frac{2}{3}
            }
            \\
                                 & =                    &
            \frac{1}{3\sqrt{5}}
            \matriz{ccc}{
            -6                   & 3                    & 0                    \\
            -2                   & -4                   & 5                    \\
            \sqrt{5}             & 2\sqrt{5}            & 2\sqrt{5}
            }
            \matriz{ccc}{
            13                   & 8                    & 8                    \\
            -1                   & 7                    & -2                   \\
            -1                   & -2                   & 7
            }
            \frac{1}{3\sqrt{5}}
            \matriz{ccc}{
            -6                   & -2                   & \sqrt{5}             \\
            3                    & -4                   & 2\sqrt{5}            \\
            0                    & 5                    & 2\sqrt{5}
            }
            =
            \ub{
              \matriz{ccc}{
            9                    & 0                    & \frac{27}{5}\sqrt{5} \\
            0                    & 9                    & -\frac{9}{5}\sqrt{5} \\
            0                    & 0                    & 9
              }
            } { T }                                                            \\
          \end{array}
        $$
        La matriz resultante quedó triangular superior.
        \parrafoDestacado{
          Las \ul{dos primeras} columnas de la matriz $T$ están regaladas, porque son autovectores, entonces
          $U^t A v_i = \lambda_i e_i$. La tercera columna es parte de la arquitectura que sostiene al infierno.
        }
        Por lo tanto se tiene que:
        $$
          U_1^t A U_1 = T
          \sii
          A = U_1 T U_1^t
        $$
        $$
          \cajaResultado{
            A =
            \frac{1}{3\sqrt{5}}
            \matriz{ccc}{
              -6                   & -2                   & \sqrt{5}             \\
              3                    & -4                   & 2\sqrt{5}            \\
              0                    & 5                    & 2\sqrt{5}
            }
            \matriz{ccc}{
              9                    & 0                    & \frac{27}{5}\sqrt{5} \\
              0                    & 9                    & -\frac{9}{5}\sqrt{5} \\
              0                    & 0                    & 9
            }
            \frac{1}{3\sqrt{5}}
            \matriz{ccc}{
              -6                   & 3                    & 0                    \\
              -2                   & -4                   & 5                    \\
              \sqrt{5}             & 2\sqrt{5}            & 2\sqrt{5}
            }
          }
        $$

  \item Descompongo:
        $$
          T =
          \matriz{ccc}{
            9                    & 0                    & \frac{27}{5}\sqrt{5} \\
            0                    & 9                    & -\frac{9}{5}\sqrt{5} \\
            0                    & 0                    & 9
          }
          =
          \matriz{ccc}{
            9                    & 0                    & 0 \\
            0                    & 9                    & 0 \\
            0                    & 0                    & 9
          }
          +
          \ub{
            \matriz{ccc}{
              0                    & 0                    & \frac{27}{5}\sqrt{5} \\
              0                    & 0                    & -\frac{9}{5}\sqrt{5} \\
              0                    & 0                    & 0
            }
          }{S}
        $$
        Ahora tengo que ver que $S^j = 0 \paratodo j \geq 2$:
        $$
          S^2 =
          \matriz{ccc}{
            0                    & 0                    & \frac{27}{5}\sqrt{5} \\
            0                    & 0                    & -\frac{9}{5}\sqrt{5} \\
            0                    & 0                    & 0
          }
          \matriz{ccc}{
            0                    & 0                    & \frac{27}{5}\sqrt{5} \\
            0                    & 0                    & -\frac{9}{5}\sqrt{5} \\
            0                    & 0                    & 0
          }
          =
          \matriz{ccc}{
            0                    & 0                    & 0 \\
            0                    & 0                    & 0 \\
            0                    & 0                    & 0
          }
        $$
        Que ejercicio raro, me deja pensando ¿Para que \textit{verga} hice esto?
\end{enumerate}

\begin{aportes}
  \item \aporte{\dirRepo}{naD GarRaz \github}
\end{aportes}
