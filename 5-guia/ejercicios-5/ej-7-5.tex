\begin{enunciado}{\ejercicio}
  Considerar la matriz:
  $$
    A = \matriz{cc}{
      4 & 0 \\
      3 & 5
    }
  $$
  \begin{enumerate}[label=(\alph*)]
    \item Calcular una descomposición en valores singulares de $A$.
    \item Dibujar el círculo unitario en $\reales^2$ y la elipse $\set{Ax : x \en \reales^2,\ \norma{x}_2 = 1}$,
          señalando los valores singulares y los vectores singulares a izquierda y a derecha.

    \item Calcular $\norma{A}_2$ y $\condicion_2(A)$.

    \item Calcular $A^{-1}$ usando la descomposición hallada.
  \end{enumerate}
\end{enunciado}

\begin{enumerate}[label=(\alph*)]
  \item
        Quieron encontrar la \textit{descomposición en valores singulares}:
        $$
          A = \blue{U} \green{\Sigma} \purple{V}^*
        $$

        Voy a calcular $A^* \cdot A$ para calcular sus \textit{jugosos autovalores}. Como la matriz \ul{es cuadrada},
        no me preocupo por pensar si es mejor hacer $A \cdot A^*$ o al revés, porque van a tener el mismo tamaño:
        $$
          \violet{H} = A^* \cdot A =
          \matriz{cc}{
            4 & 3 \\
            0 & 5 \\
          }
          \matriz{cc}{
            4 & 0 \\
            3 & 5 \\
          }
          =
          \matriz{cc}{
            25 & 15 \\
            15 & 25
          }
          \flecha{calculo}[autovalores]
          \det(\violet{H} - \lambda I) = 0
          \sii
          \lambda \en \set{10, 40}
        $$
        Ahora puedo decir que los \textit{valores singulares} son:
        $$
          \sigma_i = \sqrt{\lambda_i}
          \Entonces{de mayor}[a menor]
          \set{\sigma_1,  \sigma_2} = \set{2\sqrt{10}, \sqrt{10}}
          \Entonces{matriz}
          \cajaResultado{
            \green{\Sigma} =
            \sqrt{10}\matriz{cc}{
              2 & 0 \\
              0 & 1
            }
          }
        $$
        Calculo autovectores de $\violet{H}$ y los normalizo para obtener una \textit{base ortonormal} una BON:
        $$
          \textstyle
          \violet{H} v_\lambda = \lambda v_\lambda
          \sii
          \llave{rcl}{
            E_{_{\lambda = 40}} &=& \set{(\frac{1}{\sqrt{2}},\frac{1}{\sqrt{2}})}\\
            &\ytext&\\
            E_{_{\lambda = 10}} &=& \set{(\frac{1}{\sqrt{2}}, -\frac{1}{\sqrt{2}})}\\
          }
          \entonces
          \text{BON} = \set{(\frac{1}{\sqrt{2}},\frac{1}{\sqrt{2}}),(\frac{1}{\sqrt{2}}, -\frac{1}{\sqrt{2}})}
        $$
        \parrafoDestacado{
          Siempre en una matriz unitaria como $\violet{H}$ los autovectores asociados a autovalores de \ul{distinto valor}
          son perpendiculares.
        }
        Estoy en condiciones de armar la matriz $\purple{V}$, matriz que tiene a los $\purple{v_i}$
        autovectores de $\violet{H}$ normalizados como columnas, es decir la BON recién calculada:
        $$
          \cajaResultado{
            \purple{V} =
            \matriz{cc}{
              \frac{1}{\sqrt{2}} & \frac{1}{\sqrt{2}}\\
              \frac{1}{\sqrt{2}} & -\frac{1}{\sqrt{2}}
            }
            =
            \frac{1}{\sqrt{2}}
            \matriz{cc}{
              1 & 1\\
              1 & -1
            }
          }
        $$
        Falta menos. Ahora voy a buscar la $\blue{U}$, que tiene como columnas a los:
        $$
          \textstyle
          \blue{u_i} = \frac{A \purple{v_i}}{\sigma{i}}
          \quad \text{ con } \quad
          \sigma_i \distinto 0
          \Entonces{armo}[BON]
          \set{\blue{u_1}, \blue{u_2}} =
          \set{\frac{A \purple{v_1}}{\sigma_1}, \frac{A \purple{v_2}}{\sigma_2}}
          \igual{\red{!}}
          \set{\frac{1}{\sqrt{5}}(1,2),\frac{1}{\sqrt{5}}(2,-1)}
        $$
        Entonces tengo:
        $$
          \cajaResultado{
            \blue{U} =
            \matriz{cc}{
              \frac{2}{\sqrt{5}} & \frac{1}{\sqrt{5}}\\
              \frac{1}{\sqrt{5}} & -\frac{2}{\sqrt{5}}
            }
            =
            \frac{1}{\sqrt{5}}
            \matriz{cc}{
              1 & 2\\
              2 & -1
            }
          }
        $$
        Finalmente:
        $$
          \cajaResultado{
            A = \blue{U} \green{\Sigma} \violet{V}^*
            =
            \frac{1}{\sqrt{5}}
            \matriz{cc}{
              1 & 2\\
              2 & -1
            }
            \sqrt{10}\matriz{cc}{
              2 & 0 \\
              0 & 1
            }
            \frac{1}{\sqrt{2}}
            \matriz{cc}{
              1 & 1\\
              1 & -1
            }
            \igual{$\red{\checkmark}$}
            \matriz{cc}{
              4 & 0\\
              3 & 5
            }
          }
        $$

  \item\label{ej-7:item-b}
        $$
          \begin{tikzpicture}[scale = 1.3, every pin/.style={font=\tiny, fill=white, draw= black}, every node/.style = {font=\footnotesize}]
            \begin{axis}[
                ymax=7,
                ymin=-6,
                xmax=2,
                xmin=2,
                axis equal,
                xlabel={$x_1$},
                ylabel={$x_2$},
                title={Scatter para 200 $\blue{\bm{x}} / \norma{\blue{\bm{x}}}_2 = 1$ y para 200 $\purple{Ax}$},
                legend style={
                    cells={anchor=center},
                    font =\tiny,
                    anchor=center,
                    at = {(0.5,-0.25)},
                    legend columns=2
                  },
                legend entries =
                  {
                    $\set{\blue{x}: x \en \reales^2, \norma{x}_2 = 1}$,
                    $\set{\purple{Ax}: x \en \reales^2, \norma{x}_2 = 1}$,
                  },
                cycle list name=color list,
                every axis plot/.append style={only marks, mark size=2pt},
                xminorticks=true,
                yminorticks=true,
                grid=both,
              ]
              \addplot[mark size = 0.5pt, Cerulean] table {./ejercicios-5/dataFiles/7-ej-data/item-circulo.data};
              \addplot[mark size = 0.5pt, purple] table {./ejercicios-5/dataFiles/7-ej-data/item-matrizDotCirculo.data};

              \draw[-latex, thick, violet] (axis cs:0,0) --
              node [midway, pin=above left:{$A\violet{v_1} = \sigma_1\blue{u_1}$}]{}
              (axis cs:{({2*sqrt(2)*1},{2*sqrt(2)*2})});
              \draw[-latex, thick, violet] (axis cs:0,0) --
              node [midway, pin=right:{$A\violet{v_2} = \sigma_2\blue{u_2}$}]{}
              (axis cs:{{(sqrt(2)*2},-{sqrt(2)})});

              \node [coordinate, pin=right:{$\maximo[\norma{x}_2 = 1](\frac{\norma{Ax}_2}{\norma{x}_2})$}]
              at (axis cs:{({2*sqrt(2)*1},{2*sqrt(2)*2})}) {};
            \end{axis}
          \end{tikzpicture}
        $$
  \item\label{ej-7:item-c}
        La definición de norma subordinada:
        $$
          \norma{A}_2 = \maximo[\norma{x}_2 = 1]\left(\frac{\norma{Ax}_2}{\norma{x}_2}\right)
        $$
        Y viendo el gráfico:
        $$
          \norma{A}_2 = \norma{\sigma_1 u_1}_2 = |\sigma_1| \cdot \ub{\norma{u_1}_2}{=1} = \sigma_1 \quad \llamada1
        $$

        Por otro lado la definición de condición:
        $$
          \condicion_2(A) = \norma{A}_2 \cdot \norma{A^{-1}}_2
        $$
        Ya tengo $\norma{A}_2$, ahora quiero encontrar $\norma{A^{-1}}$:
        $$
          A =
          \blue{U} \green{\Sigma} \violet{V}^*
          \Sii{invierto}[$\green{\Sigma} \en \reales^{2\times 2}$]
          A^{-1}=
          (\violet{V}^*)^{-1} \green{\Sigma}^{-1} \blue{U}^{-1}
          \igual{\red{!}}
          \violet{V} \green{\Sigma}^{-1} \blue{U}^* =
          \violet{V}
          \matriz{cc}{
            \frac{1}{\sigma_1} & 0 \\
            0 &\frac{1}{\sigma_2}
          }
          \blue{U}^*
        $$
        Por lo tanto
        $$
          \norma{A^{-1}} = \frac{1}{\sigma_2} \quad \llamada2
          \Entonces{finalmente}[$\llamada1 \llamada2$]
          \cajaResultado{
            \condicion_2(A) =
            \norma{A}_2 \cdot \norma{A^{-1}}_2 =
            \frac{\sigma_1}{\sigma_2} = 2
          }
        $$
  \item Usando el cálculo del ítem \ref{ej-7:item-c}:
        $$
          \cajaResultado{
            \textstyle
            A^{-1}
            \igual{\red{!}}
            \violet{V} \green{\Sigma}^{-1} \blue{U}^* =
            \frac{1}{\sqrt{2}}
            \matriz{cc}{
              1 & 1\\
              1 & -1
            }
            \frac{1}{\sqrt{10}}\matriz{cc}{
              \frac{1}{2} & 0 \\
              0 & 1
            }
            \frac{1}{\sqrt{5}}
            \matriz{cc}{
              1 & 2\\
              2 & -1
            }
            \igual{$\red{\checkmark}$}
            \frac{1}{20}
            \matriz{cc}{
              5 & 0\\
              -3 & 4
            }
          }
        $$
        \parrafoDestacado[\red{\atencion}]{
          Si bien esto es \underline{una descompsición} de $A^{-1}$

          ¡\ul{No es una \textit{descomposición en valores sigulares}}!

          Se puede sacar info de esa expresión, pero ya que la diagonal de $\green{\Sigma}$ no esté ordenada
          en orden decreciente es suficiente para justificar que no es una SVD.
        }
        Pero moviendo las columnas se encuentra la \textit{descomposición en valores singulares}, mirá:
        {\small
        $$
          \textstyle
          \begin{array}{rcl}
            A^{-1}
            \igual{\red{!}}
            \violet{V} \green{\Sigma}^{-1} \blue{U}^*
                        & \igual{$\red{!!!}$} &
            \frac{1}{\sqrt{2}}
            \ob{
              \matriz{cc}{
            1           & 1                     \\
            1           & -1
              }
              \yellow{
                \matriz{cc}{
            0           & 1                     \\
            1           & 0
                }
              }
            }{\text{permuto columnas}}
            \frac{1}{\sqrt{10}}
            \ob{
              \yellow{
                \matriz{cc}{
            0           & 1                     \\
            1           & 0
                }
              }
              \matriz{cc}{
            \frac{1}{2} & 0                     \\
            0           & 1
              }
              \yellow{
                \matriz{cc}{
            0           & 1                     \\
            1           & 0
                }
              }
            }{
              \text{permuto filas y columnas}
            }
            \frac{1}{\sqrt{5}}
            \ob{
              \yellow{
                \matriz{cc}{
            0           & 1                     \\
            1           & 0
                }
              }
              \matriz{cc}{
            1           & 2                     \\
            2           & -1
              }
            }{\text{permuto filas}}
            \\
            \\
                        & =                   &
            \frac{1}{\sqrt{2}}
            \matriz{cc}{
            1           & 1                     \\
            -1          & 1
            }
            \frac{1}{\sqrt{10}}
            \matriz{cc}{
            1           & 0                     \\
            0           & \frac{1}{2}
            }
            \frac{1}{\sqrt{5}}
            \matriz{cc}{
            2           & -1                    \\
            1           & 2
            }
            \igual{\red{\checkmark}}
            \frac{1}{20}
            \matriz{cc}{
            5           & 0                     \\
            -3          & 4
            }
          \end{array}
        $$
        }
        Notar que esa matriz
          {\tiny
            $\yellow{
                \matriz{cc}{
                  0&1 \\
                  1&0
                }
              }$
          } es involutiva, es su propia inversa.
        Es así que la \textit{descomposición en valores singulares} de $A^{-1}$ que nadie pidió pero todos queremos:
        $$
          \cajaResultado{
            \textstyle
            A^{-1}
            \igual{\red{!}}
            \violet{\tilde{U}} \green{\tilde{\Sigma}} \blue{\tilde{V}}^*
            =
            \frac{1}{\sqrt{2}}
            \matriz{cc}{
              1           & 1                     \\
              -1          & 1
            }
            \frac{1}{\sqrt{10}}
            \matriz{cc}{
              1           & 0                     \\
              0           & \frac{1}{2}
            }
            \frac{1}{\sqrt{5}}
            \matriz{cc}{
              2           & -1                    \\
              1           & 2
            }
          }
        $$

        Acá te hago el gráfico del ítem \ref{ej-7:item-b} pero para $A^{-1}$:
        $$
          \begin{tikzpicture}[scale = 1.3, every pin/.style={font=\tiny, fill=white, draw= black}, every node/.style = {font=\footnotesize}]
            \begin{axis}[
                axis equal,
                xlabel={$x_1$},
                ylabel={$x_2$},
                title={Scatter para 200 $\blue{\bm{x}} / \norma{\blue{\bm{x}}}_2 = 1$ y para 200 $\purple{A^{-1}x}$},
                legend style={
                    cells={anchor=center},
                    font =\tiny,
                    anchor=center,
                    at = {(0.5,-0.25)},
                    legend columns=2
                  },
                legend entries =
                  {
                    $\set{\blue{x}: x \en \reales^2, \norma{x}_2 = 1}$,
                    $\set{\purple{A^{-1}x}: x \en \reales^2, \norma{x}_2 = 1}$,
                  },
                cycle list name=color list,
                every axis plot/.append style={only marks, mark size=2pt},
                xminorticks=true,
                yminorticks=true,
                grid=both,
              ]
              \addplot[mark size = 0.5pt, Cerulean] table {./ejercicios-5/dataFiles/7-ej-data/item-circulo.data};
              \addplot[mark size = 0.5pt, purple] table {./ejercicios-5/dataFiles/7-ej-data/item-matrizInvDotCirculo.data};

              \draw[-latex, thick, violet] (axis cs:0,0) --
              node [midway, pin=below:{$A^{-1}\violet{\tilde{v}_1} = \tilde{\sigma}_1\blue{\tilde{u}_1}$}]{}
              (axis cs:{({1/sqrt(20)},-{1/sqrt(20)})});
              \draw[-latex, thick, violet] (axis cs:0,0) --
              node [midway, pin=above:{$A^{-1}\violet{\tilde{v}_2} = \tilde{\sigma}_2\blue{\tilde{u}_2}$}]{}
              (axis cs:{({1/sqrt(80)},{1/sqrt(80)})});

              \node [coordinate, pin=above right:{$\maximo[\norma{x}_2 = 1](\frac{\norma{\purple{A^{-1}x}}_2}{\norma{x}_2})$}]
              at (axis cs:{({1/sqrt(20)},-{1/sqrt(20)})}) {};
            \end{axis}
          \end{tikzpicture}
        $$

\end{enumerate}

\begin{aportes}
  \item \aporte{\dirRepo}{naD GarRaz \github}
\end{aportes}
