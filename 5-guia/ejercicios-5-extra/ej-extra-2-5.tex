\begin{enunciado}{\ejExtra} {\tiny\violet{[segundo parcial 8/7/2023]}} Sea $A \en \reales^{3 \times 2}$ dada por
  $$
    A =
    \matriz{cc}{
      1 & 0 \\
      -2 & 1 \\
      0 & 1
    }
  $$
  \begin{enumerate}[label=(\alph*)]
    \item Encontrar la descomposición SVD de la matriz $A$.
    \item Construir una matriz $B$ de dimensión apropiada que satisfaga a la vez: $B^tB = A^tA$ y
          $\imagen(B) = \ket{(1,1,0,0), (1,1,1,1)}$.
  \end{enumerate}
\end{enunciado}

\begin{enumerate}[label=(\alph*)]
  \item La descomposiciónserá algo como:
        $$
          A = \ua{U}{\scriptscriptstyle \en \reales^{3 \times 3}} \
          \oa{\Sigma}{\scriptscriptstyle \en \reales^{3 \times 2}} \
          \ua{V^t}{\scriptscriptstyle \en \reales^{2 \times 2}}
        $$
        Primero busco los valores singulares $\sigma_i$ para formarme $\Sigma$:
        $$
          |A^tA - \lambda I_n| = 0
          \sii
          \deter{cc}{
            5 - \lambda & -2         \\
            -2          & 2 -\lambda
          }
          = 0
          \sii
          \llave{rcl}{
            \lambda_1 & = & 6 \entonces \sigma_1 = \sqrt{6} \\
            \lambda_2 & = & 1 \entonces \sigma_2 = 1
          }
        $$
        $$
          \textstyle
          E_{\lambda = 6} = \ket{\frac{1}{\sqrt{5}}(-2,1)}
          \ytext
          E_{\lambda = 1} = \ket{\frac{1}{\sqrt{5}}(1,2)}
        $$
        Listo ya tengo $\Sigma$ y $V$ los autovectores ortonormalizados de $A^tA$ me forman las columnas de $V$:
        $$
          \Sigma =
          \matriz{cc}{
            \sqrt{6} & 0 \\
            0 & 1 \\
            0 & 0
          }
          \ytext
          V =
          \frac{1}{\sqrt{5}}
          \matriz{cc}{
            -2 & 1 \\
            1 & 2
          }
        $$
        Ahora falta encontrar $U$ solo voy a poder encontrar 2, luego completo con un vector perpendicular a ambos:
        $$
          A v_i =
          U \Sigma V^t v_i =
          U \Sigma e_i =
          U \sigma_i e_i =
          \sigma_i u_i
          \entonces
          BON_U = \set{\frac{1}{\sqrt{30}}(-2,5,1), \frac{1}{\sqrt{5}}(1, 0, 2), \frac{1}{\sqrt{6}}(2,1,-1)}
        $$
        La descomposición queda como:
        $$
          A =
          \ub{
            \matriz{ccc}{
              -\frac{2}{\sqrt{30}} & \frac{1}{\sqrt{5}}    &\frac{2}{\sqrt{6}}  \\
              \frac{5}{\sqrt{30}} &  0   &  \frac{1}{\sqrt{6}}  \\
              \frac{1}{\sqrt{30}} &  \frac{2}{\sqrt{5}}   &  -\frac{1}{\sqrt{6}}
            }
          }{
            U
          }
          \ub{
            \matriz{cc}{
              \sqrt{6} & 0 \\
              0 & 1 \\
              0 & 0
            }
          }{
            \Sigma
          }
          \ub{
            \frac{1}{\sqrt{5}}
            \matriz{cc}{
              -2 & 1 \\
              1 & 2
            }
          }{
            V^t
          }
        $$

  \item
        Tiro data para pensar y armar la matriz:
        \begin{itemize}
          \item $\imagen(B) = \ket{(1,1,0,0), (1,1,1,1)}$ lo voy a usar para las dos primeras columnas de $U$.
          \item $B$ manda cosas de $\reales^2 \to \reales^4$, como la $\dim(\imagen(B)) = 2$ entonces $\dim(\nucleo(B)) = 0$,
                esto implica que ninguna columna de $V$, $v_i \en \nucleo(B)$
          \item $B^tB = A^tA$ la matriz $B \en \reales^{4 \times 2}$ y además tienen los mismos valores singulares que $A$.
        \end{itemize}
        $$
          B =
          \matriz{cccc}{
            \frac{1}{\sqrt{2}} & \frac{1}{2} & * & * \\
            \frac{1}{\sqrt{2}} & \frac{1}{2} & * & * \\
            0 & \frac{1}{2} & * & * \\
            0 & \frac{1}{2} & * & * \\
          }
          \matriz{cc}{
            \sqrt{6} & 0 \\
            0  & 1 \\
            0  & 0 \\
            0  & 0
          }
          \matriz{cc}{
            1 & 0 \\
            0 & 1
          }
        $$
        Con eso estoy cumpliendo los requerimientos del enunciado. Falta expandir las columnas de $U$ a una base ortonormal. Sale a ojo,
        si no se ve, entonces Gram-Schmidt:
        $$
          B =
          \matriz{cccc}{
            \frac{1}{\sqrt{2}} & \frac{1}{2} & \frac{-1}{\sqrt{2}} &0 \\
            \frac{1}{\sqrt{2}} & \frac{1}{2} & \frac{1}{\sqrt{2}} &0 \\
            0 & \frac{1}{2} & 0 & \frac{-1}{\sqrt{2}} \\
            0 & \frac{1}{2} & 0 & \frac{1}{\sqrt{2}}  \\
          }
          \matriz{cc}{
            \sqrt{6} & 0 \\
            0  & 1 \\
            0  & 0 \\
            0  & 0
          }
          \matriz{cc}{
            1 & 0 \\
            0 & 1
          }
        $$
\end{enumerate}

\begin{aportes}
  \item \aporte{\dirRepo}{naD GarRaz \github}
\end{aportes}
