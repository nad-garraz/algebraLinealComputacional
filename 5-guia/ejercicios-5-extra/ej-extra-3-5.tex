\begin{enunciado}{\ejExtra}
  \begin{enumerate}[label=(\alph*)]
    \item Hallar, si existe, una matriz $A$ de coeficientes reales y del tamaño adecuado tal que
          $$
            A^tA
            \matriz{c}{
              3\\
              4
            }
            =
            \matriz{c}{
              \frac{3}{4}\\
              1
            },
            \quad
            \norma{A}_2 = 5,
            \ytext
            \matriz{c}{
              4\\
              1\\
              8
            } \en \nucleo(A^t)
          $$

    \item Graficar la imagen de la esfera unitaria $S_2 = \set{x \en \reales^2}$ por la transformación lineal
          $T(x) = Ax$.
  \end{enumerate}
\end{enunciado}

\begin{enumerate}[label=\alph*)]
  \item
        \begin{itemize}
          \item Todo indicaría que $A \en \reales^{3 \times 2}$, ya que
                $A$ come \yellow{\simpleicon{burgerking}} $ \en \reales^2$
                y $A^t$ come \yellow{\simpleicon{mcdonalds}} $ \en \reales^3$.

          \item Si $\norma{A}_2 = 5 = \blue{\sigma_1}$.

          \item
                Con el dato del autovector de $A^tA$, si $A = U \Sigma V^t$:
                $$
                  \begin{array}{c}
                    A^tA =
                    (U \Sigma V^t)^t(U \Sigma V^t) =
                    V \Sigma^t \ub{U^tU}{I} \Sigma V^t =
                    V
                    \matriz{cc}{
                    \sigma_1^2 & 0          \\
                    0          & \sigma_2^2
                    } V^t                   \\
                    \\
                    A^tA
                    \matriz{c}{
                    3                       \\
                      4
                    }
                    =
                    \matriz{c}{
                    \frac{3}{4}             \\
                      1
                    }
                    =
                    \ua{
                      \magenta{\frac{1}{4}}
                    }{
                      \blue{\sigma^2_2}
                    }
                    \matriz{c}{
                    3                       \\
                      4
                    }
                  \end{array}
                $$
                Por lo que se ve que
                $\matriz{c}{
                    3                       \\
                    4
                  }$
                es un autovector de $A^tA$ de
                $\lambda = \magenta{\frac{1}{4}}$, por lo tanto $\sqrt{\magenta{\frac{1}{4}}} = \frac{1}{2} = \sigma_2 $
                conseguí un \textit{valor singular}de $A$.
                Ese autovector será una de la filas de la matriz $V$.

          \item Si $(4,1,8) \en \nucleo(A^t)$:

                Es un vector de la matriz $U$ que siempre se multiplica donde hay ceros en $\Sigma$
        \end{itemize}
        $$
          A = U \Sigma V^t =
          \matriz{ccc}{
            0                       & \frac{\sqrt{65}}{9}     & \frac{4}{9} \\
            -\frac{8}{\sqrt{65}}    & -\frac{4}{9\sqrt{65}}   & \frac{1}{9}  \\
            \frac{1}{\sqrt{65}}     & -\frac{32}{9\sqrt{65}}  & \frac{8}{9}
          }
          \matriz{ccc}{
            5 & 0           \\
            0 & \frac{1}{2} \\
            0 & 0
          }
          \matriz{cc}{
            \frac{4}{5} & -\frac{3}{5}     \\
            \frac{3}{5} & \frac{4}{5}     \\
          }
        $$

  \item \hacer
\end{enumerate}
