\begin{enunciado}{\ejExtra}
	\begin{enumerate}[label=(\alph*)]
		\item Hallar, si existe, una matriz $A$ de coeficientes reales y del tamaño adecuado tal que
		      $$
			      A^tA
			      \matriz{c}{
				      3\\
				      4
			      }
			      =
			      \matriz{c}{
				      \frac{3}{4}\\
				      1
			      },
			      \quad
			      \norma{A}_2 = 5,
			      \ytext
			      \matriz{c}{
				      4\\
				      1\\
				      8
			      } \en \nucleo(A^t)
		      $$

		\item Graficar la imagen de la esfera unitaria $S_2 = \set{x \en \reales^2}$ por la transformación lineal
		      $T(x) = Ax$.
	\end{enumerate}
\end{enunciado}

\begin{enumerate}[label=\alph*)]
	\item
	      \begin{itemize}
		      \item Si $\norma{A}_2 = 5 = \blue{\sigma_1}$.
		      \item
		            Con el dato del autovector de $A^tA$:
		            $$
			            A^tA
			            \matriz{c}{
				            3\\
				            4
			            }
			            =
			            \matriz{c}{
				            \frac{3}{4}\\
				            1
			            }
			            =
			            \ua{
				            \frac{1}{4}
			            }{
				            \blue{\sigma^2_2}
			            }
			            \matriz{c}{
				            3\\
				            4
			            }
		            $$

		      \item Si $(4,1,8) \en \nucleo(A^t)$:

		            Es un vector de la matriz $U$ que siempre se multiplica donde hay ceros en $\Sigma$
	      \end{itemize}
	      $$
		      A = U \Sigma V^t =
		      \matriz{ccc}{
			      0                       & \frac{\sqrt{65}}{9}     & \frac{4}{9} \\
			      -\frac{8}{\sqrt{65}}    & -\frac{4}{9\sqrt{65}}   & \frac{1}{9}  \\
			      \frac{1}{\sqrt{65}}     & -\frac{32}{9\sqrt{65}}  & \frac{8}{9}
		      }
		      \matriz{ccc}{
			      5 & 0           \\
			      0 & \frac{1}{2} \\
			      0 & 0
		      }
		      \matriz{cc}{
			      \frac{4}{5} & -\frac{3}{5}     \\
			      \frac{3}{5} & \frac{4}{5}     \\
		      }
	      $$

	\item \hacer
\end{enumerate}
