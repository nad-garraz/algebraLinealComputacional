\begin{enunciado}{\ejExtra} {\tiny[\violet{segundo parcial 8/7/2023}]}
  \begin{enumerate}[label=(\alph*)]
    \item Probar que el producto de matrices ortogonales es una matriz ortogonal.

    \item Sean dos matrices $A, B \en \reales^{n \times n}$. Probar que $A$ y $B$ tienen
          los mismos valores singulares si y solo si existen $P$ y $Q$ matrices ortogonales tales que $A = PBQ$.

    \item Sea $\set{c_1, c_2, c_3}$ una base ortonormal de $\reales^3$. Hallar la matriz singular (en términos de $c_1, c_2, c_3$)
          que mejor aproxima a la matriz $C$ en norma 2, siendo
          $$
            C =
            \matriz{c|c|c}{
              &&\\
              2c_1 & -5c_2 & 3c_3\\
              &&
            }
          $$
  \end{enumerate}
\end{enunciado}

\hyperlink{teoria-5:matrices}{Acá algunas cosas de matrices ortogonales y otras \click}

\begin{enumerate}[label=(\alph*)]
  \item Si $Q$ y $P$ son dos matrices ortogonales:
        $$
          Q^t Q = I
          \ytext
          P^t P = I
          \Entonces{$\llamada1$}
          (QP)^t (QP) =
          P^t\ub{Q^t Q}{I} P = P^t P = I
        $$
        por lo tanto el producto de dos matrices ortogonales es una matriz ortogonal.

  \item Hay que demostrar la doble implicación:
        \begin{itemize}
          \item[$\red{(\Rightarrow)}$]
                $$
                  \llave{l}{
                    A = U_A \blue{\Sigma} V_A^t\\
                    B = U_B \blue{\Sigma} V_B^t
                    \sii
                    \blue{U_B^t B V_B} = \blue{\Sigma}
                  }
                  \entonces
                  A = \ub{U_A \blue{U_B^t}}{P} \blue{B} \ub{\blue{V_B} V_A^t}{Q}
                  \igual{$\llamada1$}
                  PBQ
                $$

          \item[$\red{(\Leftarrow)}$]
                La matriz $B$ como cualquier hija de vecino, tiene una \textit{descomposición en valores singulares}:
                $$
                  B = U \Sigma V^t
                $$
                Mientras que
                $$
                  A^tA =
                  (PBQ)^t(PBQ) =
                  Q^tB^tP^t PBQ =
                  Q^tB^tBQ
                $$
                Dado que $Q^t = Q^{-1}$ queda que la matrices $A^tA$ y $B^tB$ son semejantes, es decir que tienen los mismos autovalores.
                Dado que los valores singulares $A$ y $B$ son los $\sigma_i = \sqrt{\lambda_i}$, se concluye que $A$ y $B$ tienen mismos
                valores singulares.
        \end{itemize}

  \item 
\end{enumerate}
