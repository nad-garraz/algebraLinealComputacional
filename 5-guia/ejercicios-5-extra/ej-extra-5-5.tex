\begin{enunciado}{\ejExtra} \fechaEjercicio{final 07/2023}
  \begin{enumerate}[label=\alph*)]
    \item
          Sea $A \en \reales^{n \times n}$. Probar que $A$ tiene todos sus valores singulares iguales si y solo si
          es múltiplo de una matriz ortogonal.
          (\textit{Sugerencia: Recordar que el producto de dos matrices ortogonales es una matriz ortogonal y que el 'si y solo si' son dos implicaciones}).

    \item Sea $B = (-2)Q$, con
          $Q =
            \matriz{c|c|c}{
              \quad& \quad& \quad\\
              q_1 & q_2 & q_3 \\
              \quad& \quad& \quad
            }$ una matriz ortogonal. Hallar dos descomposiciones en valores sigulares distintas de $B$.
          (Observación: dos descomposiciones
          $C =
            U_1 \Sigma_1 V_1^t =
            U_2 \Sigma_2 V_2^t
          $ son iguales si y solo si
          $U_1 = U_2$, $\Sigma_1 = \Sigma_2 $ y $ V_1 = V_2$).

    \item Calcular una matriz singular que sea más cercana a $B$ en norma 2.
  \end{enumerate}
\end{enunciado}

\begin{enumerate}[label=\alph*)]
  \item
        \begin{itemize}
          \item[$(\red{\Rightarrow})$]
                $$
                  A = U \cdot \Sigma \cdot V^t
                  \igual{HIP} U  \cdot \sigma I_n \cdot V^t
                  = \sigma \ub{U \cdot V^t}{Q}
                  \igual{sug.} \sigma Q
                  \entonces
                  \cajaResultado{
                    A = \sigma Q
                  }
                $$
                Donde $Q$ es una matriz ortogonal ya que $(U  V^t ) \cdot (U  V^t)^t = I_n$.

          \item[$(\red{\Leftarrow})$] Como $A = \blue{k} Q$ y la matriz ortogonal tiene determinante no nulo,
                entonces no hay ningún valor singular nulo. Los valores singulares los saco de los
                autovalores de $A^tA$:
                $$
                  A^tA = \blue{k}^2Q^tQ = \blue{k}^2 I_n
                  \Entonces{autovalores}
                  |\blue{k}^2 I_n - \lambda I_n| = 0
                  \sii
                  \lambda = \blue{k}^2 \text{ con multiplicidad } n
                $$
                Por lo tanto todos los $n$ valores singulares de $A$ son:
                $$
                  \cajaResultado{
                    \sigma = \sqrt{\blue{k}^2} = |\blue{k}|
                  }
                $$
        \end{itemize}

  \item  Del ítem anterior tengo que los valores singulares de $B$ son todos $2$, luego para calcular $V$ busco los autovectores de
        $B^t B$ los $v_i$, y por último $U$ tiene como columnas a $Bv_i = \sigma u_i$, Las cosas quedan \textit{lindas}, porque:
        $$
          B^t B = 4Q^tQ = 4I_n
        $$
        Con esas cosas puedo calcular todos los ingredientes para cocinar dos SVD para la matriz B:
        $$
          B = U_1 \Sigma_1 V_1^t =
          \matriz{c|c|c}{
            \quad& \quad& \quad\\
            -q_1 & -q_2 & -q_3 \\
            \quad& \quad& \quad
          }
          \matriz{ccc}{
            2 & 0 & 0 \\
            0 & 2 & 0 \\
            0 & 0 & 2
          }
          \matriz{ccc}{
            1 & 0 & 0 \\
            0 & 1 & 0 \\
            0 & 0 & 1
          }
        $$
        y la otra sale como:
        $$
          B = U_2 \Sigma_2 V_2^t =
          \matriz{c|c|c}{
            \quad& \quad& \quad\\
            q_1 & q_2 & q_3 \\
            \quad& \quad& \quad
          }
          \matriz{ccc}{
            2 & 0 & 0 \\
            0 & 2 & 0 \\
            0 & 0 & 2
          }
          \matriz{ccc}{
            -1 & 0 & 0 \\
            0 & -1 & 0 \\
            0 & 0 & -1
          }
        $$

  \item La matriz \ul{singular más cercana a $B$}, $\tilde{B}$ va a ser aquella que \textit{transforme} un vector $x$ y cuya diferencia sea mínima entre
        $$
          \norma{Bx - \tilde{B}x}_2
        $$
        Esa matriz se consigue eliminando el menor de los valores singulares, ya que ese valor es el que menos perturba en la transformación, pero como acá son todos iguales,
        saco solo el último, así generando una fila de ceros que me dé la \ul{singularidad}.
        $$
          B = U \Sigma V^t =
          \matriz{c|c|c}{
            \quad& \quad& \quad\\
            -q_1 & -q_2 & -q_3 \\
            \quad& \quad& \quad
          }
          \matriz{ccc}{
            2 & 0 & 0 \\
            0 & 2 & 0 \\
            0 & 0 & \magenta{0}
          }
          \matriz{ccc}{
            1 & 0 & 0 \\
            0 & 1 & 0 \\
            0 & 0 & 1
          }
          =
          (-2)
          \matriz{c|c|c}{
            \quad& \quad& \quad\\
            q_1 & q_2 & 0 \\
            \quad& \quad& \quad
          }
        $$
\end{enumerate}

\begin{aportes}
  \item \aporte{\dirRepo}{naD GarRaz \github}
\end{aportes}
