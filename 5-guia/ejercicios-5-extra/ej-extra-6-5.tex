\begin{enunciado}{\ejExtra} \fechaEjercicio{final 21/07/25}
  Sea $A \en \reales^{n \times n}$ una matriz tal qeu $A^t = A = A^{-1}$.
  \begin{enumerate}[label=\alph*)]
    \item ¿Cuánto vale el determinante de $A$? ¿Es $A$ diagonalizable?
    \item ¿Cuáles son sus posibles autovalores?
    \item Calcular la matriz $\Sigma$ de la factorización SVD de $A$. Justificar.
    \item Calcular los autovalores de la siguiente matriz.
          $$
            B =
            \matriz{cccc}{
              5/10 & -5/10 & -1/10 & -7/10 \\
              -5/10 & 5/10 & -1/10 & -7/10 \\
              -1/10 & -1/10 & 98/100 & -14/100 \\
              -7/10 & -7/10 & -14/100 & 2/100
            }
          $$
          \textit{Sugerencia: usar los items anteriores.}
  \end{enumerate}
\end{enunciado}

Una matriz real que cumple  $A^t = A = A^{-1}$ es una matriz ortogonal.
\begin{enumerate}[label=\alph*)]
  \item
        $$
          A^2 = A \cdot A^{-1} = I_n
          \entonces \det(A^2) = (\det(A))^2 = \det(I_n) \sii \cajaResultado{\det(A) = \pm 1}
        $$
        Dado que $A$ es simétrica, se puede diagonalizar. En particular tiene una diagonalización ortogonal.

  \item Como $A$ es diagonalizable y además cumple que $A = A^{-1}$:
        $$
          \textstyle
          A = C D C^{-1}
          \sii
          A v = \lambda v
          \Sii{$\times A$}[$\to$]
          v = \lambda A^{-1}v
          \Sii{$\lambda \distinto 0$}
          A^{-1} v = \frac{1}{\lambda} v
        $$
        Entonces la matriz $A$ tiene un autovector $v$ con autovalor $\lambda$, entonces $A^{-1}$ también lo tiene como autovector pero
        asociado al autovalor $\frac{1}{\lambda}$, peeeeeero como:
        $$
          A = A^{-1}
          \sii
          \lambda_i = \frac{1}{\lambda_i}
          \sii
          \lambda_i^2 = 1
          \sii
          \cajaResultado{
            \lambda_i = \pm1
          }
        $$

  \item  Para calcular $\Sigma$ calculos los autovalores de $M = A^t \cdot A$. Teniendo en cuenta que los autovalores de $A$ y de $A^t$ son los mismos
        y como se vio antes los autovalores son $1$ o $-1$:
        $$
          M = A^tA = CD^2C^{-1} = I
          \sii
          D = I_n
          \sii
          \lambda_{ii} = 1
        $$
        Los autovalores de $M$, $\lambda_M$ son todos iguales a 1. Los valores singulares $\sigma_i = \sqrt{\lambda_M} = 1$:
        $$
          \Sigma = I_n
        $$
        Como a es una matriz cuadrada también lo es $\Sigma$.

  \item Como indica la \textit{sugerencia}:

        La matriz $B$ es una matriz ortogonal, sus columnas o filas forman una base ortonormal de $\reales^4$, es simétrica y $A^t \cdot A = I_n$
        Por lo tanto sus autovalores valen $1$ o $-1$. Dado que $B$ es semejante a una matriz $D$ con los autovalores en la diagonal, es decir
        $$
          B = C D C^{-1} \quad \text{con} \quad
          D =
          \matriz{cccc}{
            \lambda_1 & 0 & 0 & 0 \\
            0 & \lambda_2 & 0 & 0 \\
            0 & 0 & \lambda_3 & 0 \\
            0 & 0 & 0 & \lambda_4
          }
        $$
        la \textit{relación de semejanza de matrices} tiene entre otras propiedades que las trazas de las matrices coinciden. La traza
        de $B$, $\traza(B) = 2$, por lo tanto:
        $$
          \llave{rcl}{
            \lambda_1 & = &  1 \\
            \lambda_2 & = &  1 \\
            \lambda_3 & = &  1 \\
            \lambda_4 & = &  -1
          }
        $$
\end{enumerate}

\begin{aportes}
  \item \aporte{\dirRepo}{naD GarRaz \github}
\end{aportes}
