\begin{enunciado}{\ejExtra} \fechaEjercicio{final 21/07/25}
  Sea $A \en \reales^{n \times n}$ una matriz tal qeu $A^t = A = A^{-1}$.
  \begin{enumerate}[label=\alph*)]
    \item ¿Cuánto vale el determinante de $A$? ¿Es $A$ diagonalizable?
    \item ¿Cuáles son sus posibles autovalores?
    \item Calcular la matriz $\Sigma$ de la factorización SVD de $A$. Justificar.
    \item Calcular los autovalores de la siguiente matriz.
          $$
            B =
            \matriz{cccc}{
              5/10 & -5/10 & -1/10 & -7/10 \\
              -5/10 & 5/10 & -1/10 & -7/10 \\
              -1/10 & -1/10 & 98/100 & -14/100 \\
              -7/10 & -7/10 & -14/100 & 2/100
            }
          $$
          \textit{Sugerencia: usar los items anteriores.}
  \end{enumerate}
\end{enunciado}

\begin{enumerate}[label=\alph*)]
  \item
        $$
          A^2 = A \cdot A^-1 = I_n
          \entonces \det(A^2) = (\det(A))^2 = \det(I_n) \sii \cajaResultado{\det(A) = \pm 1}
        $$

    \item
        $$
        $$
\end{enumerate}

\begin{aportes}
  \item \aporte{\dirRepo}{naD GarRaz \github}
\end{aportes}
