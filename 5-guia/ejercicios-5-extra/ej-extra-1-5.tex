\begin{enunciado}{\ejExtra}
  {\tiny[segundo recu 5/12/2024]} Sean $A, B \en \reales^{n \times n}$.
  \begin{enumerate}[label=(\alph*)]
    \item
          Probar que $A^t A = B^t B$ si y solo si existe una matriz ortogonal $U \en \reales^{n \times n}$ tal que
          $B = UA$.

    \item
          Sea $A = QR$ la factorización $QR$ de $A$. Probar que $A$ y $R$ tienen los mismos valores singulares.

    \item
          Sea $
            A = \frac{1}{\sqrt{2}}
            \matriz{cc}{
              1 & -1 \\
              1 & 1
            }
            \matriz{cc}{
              0 & 0 \\
              0 & 7
            }
          $
  \end{enumerate}
\end{enunciado}

\begin{enumerate}[label=(\alph*)]
  \item
        \begin{itemize}
          \item[$\red{(\Rightarrow)}$]
                Multiplico por una $I_n = \blue{U}^* \blue{U}$ con $\blue{U}$ una matriz ortogonal. Luego acomodo y nombro a la matriz adecuada como $B$:
                $$
                  A^tA =
                  A^t \blue{U}^t \blue{U} A =
                  (\blue{U}A)^t (\blue{U} A) =
                  B^t B
                $$

          \item[$\red{(\Leftarrow)}$]
                Parto de $B = UA$, con $\blue{U}$ una matriz ortogonal:
                $$
                  B = \blue{U}A \sii B^t = (\blue{U}A)^t
                  \entonces
                  B^tB =
                  (\blue{U}A)^t \blue{U}A =
                  A^t\blue{U}^t \blue{U}A =
                  A^t A
                $$
        \end{itemize}

  \item Los valores singulares de una matriz $A$:
        $$
          \sigma_i = \sqrt{\lambda_i} \quad \text{ con $\lambda_i$ tal que }\quad |A^tA - \lambda_iI| = 0
        $$
        Usando el resultado del punto anterior y recordando que la $Q$ en la descomposición $QR$
        tiene como columnas una base ortonormal, es decir que $Q$ es una matriz ortogonal, de forma tal que:
        $$
          Q^t Q = I_n
          \entonces
          A^tA =
          (Q R)^t (Q R) \igual{\red{!}}
          R^t R
        $$.
        Por lo tanto los valores singulares de $A$ y $R$ seran los mismos.

  \item
        Esto de la \textit{descomposición en valores sigulares} es mucho más sencillo cuando la matriz es cuadrada, porque hay menos cosas que contemplar.
        Ya sé los tamaños de la matrices y no tengo que pensar que conviene hacer:
        $$
          C = \ua{U}{\en \reales^{2 \times 2}}\ \oa{\Sigma}{\en \reales^{2 \times 2}}\ \ua{V^t}{\en \reales^{2 \times 2}}
        $$
        Nos dan una $A$ que está casi en SVD. ¿Se ve?, voy a empezar a permutar para dejar bien ordenados los valores singulares:
        {
        \small
        $$
          A =
          \frac{1}{\sqrt{2}}
          \matriz{cc}{
            1 & -1 \\
            1 & 1
          }
          \matriz{cc}{
            0 & 0 \\
            0 & 7
          }
          =
          \frac{1}{\sqrt{2}}
          \matriz{cc}{
            1 & -1 \\
            1 & 1
          }
          \ub{
            \green{
              \matriz{cc}{
                0 & 1 \\
                1 & 0
              }
            }
            \green{
              \matriz{cc}{
                0 & 1 \\
                1 & 0
              }
            }
          }{
            I_2
          }
          \matriz{cc}{
            0 & 0 \\
            0 & 7
          }
          =
          \frac{1}{\sqrt{2}}
          \matriz{cc}{
            -1 & 1 \\
            1 & 1
          }
          \matriz{cc}{
            0 & 7 \\
            0 & 0
          }
        $$
        }
        Ahora permuto para mover ese 7 para la izquierda:
        {
        \small
        $$
          A =
          \frac{1}{\sqrt{2}}
          \matriz{cc}{
            -1 & 1 \\
            1 & 1
          }
          \matriz{cc}{
            0 & 7 \\
            0 & 0
          }
          \ub{
            \green{
              \matriz{cc}{
                0 & 1 \\
                1 & 0
              }
            }
            \green{
              \matriz{cc}{
                0 & 1 \\
                1 & 0
              }
            }
          }{
            I_2
          }
          =
          \ub{
            \frac{1}{\sqrt{2}}
            \matriz{cc}{
              -1 & 1 \\
              1 & 1
            }
          }{
            U
          }
          \ub{
            \matriz{cc}{
              7 & 0 \\
              0 & 0
            }
          }{
            \Sigma
          }
          \ub{
            \matriz{cc}{
              0 & 1 \\
              1 & 0
            }
          }{
            V^t
          }
        $$
        }
        !Magia! ¿Y para qué me sirve eso? No sé, pero tenía ganas de hacerlo. ¡Nah, mentira! \red{Se terminó el ejercicio}. Esa última matriz
        es la $C$ que cumple lo pedido.

        Lo que viene a continuación es la forma menos \textit{hacker} de hacerlo, básicamente como lo encaré yo antes de darme cuenta
        \red{\faIcon[regular]{sad-cry}} que esa SVD cumplía todo lo pedido.

        Entendiendo como funciona la \hyperlink{teoria-5:svd}{\textit{descomposición en valores singulares} (mirá acá estos resultados \click)} sé que:
        $$
          \llave{rcc}{
            \nucleo(A) & = & \ket{(0,1)}\\
            \imagen(A)  & = & \ket{(-1,1)}\\
            \norma{A}_2 & = & 7
          }
          \ytext
          C^t
          \matriz{c}{
            -1 \\
            1
          }
          \igual{$\llamada1$}
          \matriz{c}{
            0 \\
            7\sqrt{2}
          }
        $$
        \begin{itemize}
          \item $\norma{C}_2 = \norma{A}_2 = \sigma_1 = 7$
          \item $
                  \dim(\nucleo(C)) = 1
                  \entonces
                  \sigma_2 = 0
                $
          \item $V$ tiene como columnas a una \textit{base ortonormal}
                $
                  B_V = \set{v_1, v_2}
                $ donde $v_2 \en \nucleo(C)$
                Quizás te estés preguntando: ¿Y $v_1$? {\tiny Me importa poco.}

          \item $U$ tiene como columnas a una \textit{base ortonormal}
                $
                  B_U = \set{u_1, u_2}
                $ donde $u_1 \en \imagen(C)$
                Quizás te estés preguntando: ¿Y $u_2$? {\tiny Me importa otro poco.}
        \end{itemize}
        El dato de $\llamada1$ me dice que $(0, 7\sqrt{2})$ es una combinación de las columnas de $V$.
        No sé si es la mejor forma de encararlo, pero me lo imagino así:
        $$
          C = U \Sigma V^t
          \Sii{\red{!}}
          C^t = V \Sigma U^t
        $$
        Por lo tanto los generadores de la $\imagen(C^t)$ son las columnas de $V$ ahora. Como $\dim(\imagen(C^t)) = 1$
        están diciendo que $(0,7\sqrt{2}) \en \ket{(0,1)}$ lo cual es cierto!

        Con toda esa data se puede encontrar una matriz $C$ sin mucha rosca dado que al matriz es cuadrada y estamos
        en $\reales^{2 \times 2}$ a tener en cuenta:
        $$
          C =
          \ub{
            \frac{1}{\sqrt{2}}
            \matriz{cc}{
              -1 & 1 \\
              1 & 1
            }
          }{
            U
          }
          \ub{
            \matriz{cc}{
              7 & 0 \\
              0 & 0
            }
          }{
            \Sigma
          }
          \ub{
            \matriz{cc}{
              0 & 1 \\
              1 & 0
            }
          }{
            V^t
          }
        $$
\end{enumerate}
