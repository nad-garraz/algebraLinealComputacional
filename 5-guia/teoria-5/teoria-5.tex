\begin{enumerate}[label=\tiny\purple{\faIcon{snowman}}]
  \item \hypertarget{teoria-5:matrices}{\textit{Recuerdo nomenclatura de matrices}}

        \begin{minipage}{0.5\textwidth}
          \textit{Nombres usados en matrices en $\reales$:}
          \begin{itemize}
            \item Ortogonal: $Q \en \reales^{n \times n}$  con $Q^tQ = I$
                  \begin{enumerate}[label={\tiny\faIcon{atom}}]
                    \item Las columnas de $Q$ forman una BON de $\reales^n$
                    \item Es ortogonalmente diagonalizable.
                    \item Preserva la norma en la multiplicación.
                    \item $\det(Q) = \pm1$
                  \end{enumerate}

            \item Simétrica: $A \en \reales^{n \times n}$  con $A^t = A$
                  \begin{enumerate}[label={\tiny\faIcon{atom}}]
                    \item $A$ tiene autovalores reales.
                    \item Es ortogonalmente diagonalizable.
                  \end{enumerate}

          \end{itemize}
        \end{minipage}
        \begin{minipage}{0.5\textwidth}
          \textit{Nombres usados en matrices en $\complejos$:}
          \begin{itemize}
            \item Unitaria: $U \en \complejos^{n \times n}$  con $U^*U = I$
                  \begin{enumerate}[label={\tiny\faIcon{atom}}]
                    \item Las columnas de $U$ forman una BON de $\complejos^n$
                    \item Es ortogonalmente diagonalizable.
                    \item Preserva la norma en la multiplicación.
                    \item $\det(U) = \pm1$
                  \end{enumerate}

            \item Hermitiana: $A \en \complejos^{n \times n}$  con $A^* = A$
                  \begin{enumerate}[label={\tiny\faIcon{atom}}]
                    \item $A$ tiene autovalores reales.
                    \item Es ortogonalmente diagonalizable.
                  \end{enumerate}

          \end{itemize}
        \end{minipage}

  \item \hypertarget{teoria-5:svd}{\textit{Descomposición en valores singulares}:}

        Tengo una matriz $A \en K^{m \times n}$
        $$
          A = U \Sigma V^*
        $$
        Con $U$ y $V$ matrices unitarias, por lo tanto \ul{cuadradas, simétricas}
        $\llaves{c}{
            U^*U = I\\
            V^*V = I
          }$,
        y $\Sigma \en K^{m \times n}$ el mismo tamaño que $A$.
        \begin{itemize}
          \item Para obtener $\Sigma$ calulo los valores $\sigma_i = \sqrt{\lambda_i}$ donde:
                $$
                  A^*A v_i = \lambda_i v_i
                $$
                y luego ordeno los elementos diagonales $[\Sigma]_{ii} = \sigma_i$ de mayor a menor. Completo con \textit{fila o columas de ceros},
                hasta llegar a la dimesión correcta.

          \item Para obtener la matriz $V$ pongo a los $v_i$ calculados previamente como columnas en orden correspondiente a su $\sigma_i$.

          \item Para calcular $U$:
                $$
                  Av_i = U\Sigma V^*v_i =  U\Sigma e_i = U\sigma_i e_i = \sigma_i u_i
                  \sii
                  \cajaResultado{
                    Av_i = \sigma_i u_i
                  }
                $$
        \end{itemize}
        De ese último resultado se desprende info de la matriz $A$. Como $A \en K^{m \times n}$ con $(m > n)$ tiene rango $r < n$:
        $$
          \nucleo(A) = \ket{v_{r+1}, \ldots, v_n}
          \ytext
          \imagen(A) = \ket{u_1, \ldots, u_r}
        $$

  \item \textit{Pseudo-inversa:}

        Si se tiene una $A \en K^{m \times n}$
        $$
          A = U \Sigma V^*
          \Entonces{pseudo}[inversa]
          A^\dagger = V \Sigma^\dagger U^*,
        $$
        con la $\Sigma^\dagger$ que sería como $\Sigma^t$ invirtiendo los elementos diagonales no nulos
        $[\Sigma^\dagger]_{ii} = \frac{1}{\sigma_{ii}}$, dejando los ceros donde están.
        Propiedades de esta cosa dignas de ser mencionadas:
        \begin{itemize}
          \item Si bien en general, $\magenta{A A^\dagger} \distinto I_m$
                y los mismo con $\magenta{A^\dagger A} \distinto I_n$, tenemos este simpático resultado:
                $$
                  \magenta{A A^\dagger} A = A
                  \ytext
                  \magenta{A^\dagger A} A^\dagger = A^\dagger
                $$

        \end{itemize}
\end{enumerate}
