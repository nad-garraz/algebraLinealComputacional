\begin{enumerate}[label=\tiny\purple{\faIcon{snowman}}]
  \item \textit{Procesos de Markov:}

        Sucesión de vectores $\bm{v}_k$ con $k \en \naturales_0$
        $$
          \set{\bm{v}_0, \bm{v}_1, \ldots}
          \quad
          \text{con}
          \quad
          \bm{v}_{k+1} = M \bm{v}_k.
        $$
        $M$ es una matriz de \textit{Markov} si es una matriz estocástica \underline{por columnas}, es decir:
        \begin{itemize}
          \item Todas los elementos $m_{ij}$ de la matriz $M$ son \underline{no negativos}.

          \item Cada columna de $M$ suma 1:
                $$
                  \left[\sumatoria{i = 1}{n} m_{i\blue{j}}\right]_{\blue{j}} = 1
                  \quad
                  \paratodo \blue{j} \en \naturales_{\leq n}
                $$

          \item $M$ tiene por lo menos un autovalor $\lambda = 1$.

          \item Los autovalores de $M$ cumplen que $|\lambda| \leq 1$.
        \end{itemize}

  \item \textit{Vector estocástico o de probabilidad:}

        Sea un $\bm{v} = (v_1, \ldots, v_n)$ cumple que sus coordenadas son \underline{no negativas} y suman 1.
        Las coordenada $j-$ésima corresponde \ul{la probabilidad de estar en el estado $j-$ésimo o la proporción
          de la población que se encuentra en ese estado}.
\end{enumerate}
